\newpage \newpage
\section{Tests}
Eine der Fragen, die sich bei der Rekonstruktion von Attraktoren aus experimentellen
Zeitreihen stellte, war ob ob es sich hierbei wirklich um ein deterministisches System
handelt.  Man k"onnte hier m"oglicherweise versuchen, "uber die Dimension des
rekonstruierten Attraktors zu argumentieren. Die Dimension eines deterministischen,
nichtlinearen Systems hat immer einen endlichen Wert. Dagegen spannen die Rekonstruktionen
stochastischer Signale immer den ganzen Phasenraum auf. Die berechnete Dimension
konvergiert nicht mit steigender Einbettungsdimension. K"onnen also "uber die
Dimensionsberechnungen deterministische von stochastischen Systemen unterschieden werden?

Die Antwort ist leider \naja(Nein). Wie \autor(A. R. Osborne) und \autor(A. Provencale)
nachwiesen, k"onnen auch stochastische Systeme mit Leistungsspektren
$P(\omega)\propto\omega^{-\alpha}$, gegen eine endliche Korrelationsdimension
konvergieren\cite{Osborne89a}. F"ur $1<\alpha<3$ erhielten sie $D_2=2/(\alpha-1)$.  Dies
liegt an zeitlichen Korrelationen aufeinanderfolgender Werte in der Zeitreihe. Diese
k"onnen zwar durch die Methode von \autor(Theiler) (siehe Abschnitt \ref{corrdimtheiler})
vermieden werden, andererseits existieren noch andere Effekte, die eine Konvergenz der
Korrelationsdimension bei wei"sem oder farbigem Rauschen bewirken k"onnen.

\comment{
  \epsfigfour{surrogate/noise/noise}{surrogate/noise/fourier}{surrogate/noise/corrint}{surrogate/noise/corrdim}
  {Links oben Zeitreihe $1/f^2$ Rauschen. Rechts oben Fourierspektrum. Links unten
    Korrelationsintegral. Rechts unten Korrelationsdimension, s"attigt bei ca.\ $5,0\dots
    5,5$ }{einsfnoise}{-0.2cm} }

\subsection{Statitische Hypothesentests}
Es sind noch weitere M"oglichkeiten vorgeschlagen worden, Zeitreihen hinsichtlich eines
zugrundeliegenden deterministischen Systems zu untersuchen (beispielsweise der
Determinismustest von \autor(Kaplan) und \autor(Glass) \cite{kaplan-glass}).  Diese sind
jedoch in der Anwendung oft sehr beschr"ankt.  Die umfassendste und mathematisch
fundierteste M"oglichkeit diesem Problem zu begegnen, ergibt sich im Feld statistischer
Hypothesentests. Hiermit sind wir zugleich nicht mehr beschr"ankt auf die Frage nach dem
Determinismus, sondern haben ein Grundger"ust, um Fragen der verschiedensten Art an die
vorliegende Zeitreihen zu stellen. Beispielweise
\begin{itemize}
\item Sind die Daten nicht-gau"sverteilt ?
\item Gibt es zeitliche Korrelationen in der Zeitreihe ?
\item Existiert eine nichtlineare Struktur ?
\item Sind die Daten durch eine chaotische Dynamik erzeugt ?
\end{itemize}
Um eine dieser Frage zu beantworten, wird nun eine \begriff(Nullhypothese) $\nullhyp$
aufgestellt, welcher eine Verneinung eben dieser Frage entspricht.  Die Nullhypothese kann
nun weder bewiesen noch widerlegt werden. Man versucht hingegen, die Nullhypothese
abzulehnen, d.h. zu zeigen, da"s es unwahrscheinlich ist, da"s die Daten mit der Hypothese
in Einklang stehen.

Um genauer zu sein: der Nullhypothese $\nullhyp$ wird ein Proze"s bzw.\ eine Klasse von
Prozessen $\process$ zugeordnet, die mit $\nullhyp$ in Einklang stehen. Bei der ersten
Fragen w"are dies beispielsweise die Menge aller Prozesse, die gau"sverteilte Daten
erzeugen. Die Frage ist nun, ob die realen Daten durch einen Proze"s aus $\process$
erzeugt worden sein k"onnen. Hierzu wird eine \begriff(Teststatistik) $T$ berechnet. Liegt
der $T$-Wert der realen Daten au"serhalb des Bereichs, den man f"ur Prozesse aus
$\process$ erwarten kann, wird die Nullhypothese abgelehnt. Liegt der Wert innerhalb des
\begriff(Annahme-) oder \begriff(Akzeptanzbereichs) der Nullhypothese wird sie angenommen.
Man sagt hier auch, der Test h"atte versagt, die Nullhypothese abzulehnen, da das i.allg.\ 
das Ergebnis ist, das man haben m"ochte. Da die Teststatistik dazu dienen soll, die realen
Daten von mit der Nullhypothese konsistenten Prozessen $\process$ zu \naja(unterscheiden),
bezeichnet man $T$ auch als \begriff(Diskriminator).

\subsubsection{Fehler 1. und 2. Art}
Bei der Annahme oder Ablehnung einer Nullhypothese k"onnen jeweils Fehler auftreten. Der
erste Fehler ist, da"s die Nullhypothese abgelehnt wird, obwohl sie eigentlich zutrifft.
Man spricht hier von einem \begriff(Fehler 1. Art). Die Wahrscheinlichkeit $\alpha$, mit
der Fehler 1. Art auftreten, kann frei bestimmt werden. Dies geschieht, indem als
Annahmebereich des Tests das $(1-\alpha)$\begriff(-Konfidenzintervall) der Teststatistik
f"ur die betrachteten Prozesse gew"ahlt wird. Das $(1-\alpha)$-Konfidenzintervall ist der
Bereich der $T$-Werte, f"ur den mit Wahrscheinlichkeit $1-\alpha$ Realisierungen von
Prozessen aus $\process$ in diesem Bereich liegen\korrektur(einfacher formulieren).  Man
spricht bei einem Test mit vorgegebenem $\alpha$ auch von einem \begriff(Niveau
$\alpha$-Test) bzw.\ von einem \begriff(Test zum Signifikanzniveau $\alpha$). Anstatt die
Signifikanz von vorneherein festzulegen, wird ab und zu auch der $p$-Wert eines Tests
angegeben. Dies ist der kleinste Wert von $\alpha$, f"ur den die Nullhypothese gerade noch
abgelehnt w"urde.

Bei Annahme der Nullhypothese, wenn sie tats"achlich falsch ist, spricht man von einem
Fehler 2. Art.  Die Wahrscheinlichkeit f"ur das Auftreten solcher Fehler wird mit $\beta$
bezeichnet. Die komplement"are Wahrscheinlichkeit $1-\beta$ gibt an, wie \naja(gut) der
Test in der Lage ist, die Nullhypothese f"ur mit ihr inkonsistenten Daten abzulehnen. Man
bezeichnet $1-\beta$ daher auch als die \begriff(G"ute) des Tests.  Da die G"ute eines
Tests davon abh"angt, wie nicht-konsistent die wirklichen Daten mit der Nullhypothese
sind, kann $\beta$ im Gegensatz zu $\alpha$ nicht vorgegeben werden.  Allerdings h"angt
die G"ute des Tests von $\alpha$ ab -- je h"oher die Signifikanz $\alpha$ des Test ist,
desto geringer ist die G"ute $1-\beta$. Es ist andererseits nicht sinnvoll, um $1-\beta$
gro"s zu machen $\alpha$ sehr klein zu machen. Man w"urde sich die h"ohere G"ute des Tests
mit einer geringeren Signifikanz, d.h.\ Aussagekraft, des Test erkaufen.

\subsubsection{Einfache Nullhypothesen}
Bei der Konstruktion eines Tests ist zu beachten, da"s zwei verschiedene Typen von
Nullhypothesen existieren: \begriff(einfache) und \begriff(zusammengesetzte). Bei
einfachen Nullhypothesen besteht die Menge der mit $\nullhyp$ konsistenten Prozesse
$\process$ aus nur einem Element. Ein Beispiel f"ur eine solche Nullhypothese w"are die,
da"s die Daten Gau"sverteilt mit Mittelwert $\mu_0=0$ und Varianz $\sigma_0^2=1$ sind. Als Diskriminator
$T$ k"onnten wir f"ur diese Nullhypothese ein h"oheres Moment der Verteilung w"ahlen, sagen
wir 
\eqnl[discrim1]{T=\frac{1}{N}\sum_{i=1}^N x_i^4} 
Prinzipiell h"atten wir jede beliebige Funktion der $N$ Argumente $X=(x_1,\dots,x_n) $ 
w"ahlen k"onnen. Allerdings h"angt die G"ute des Tests stark von der Teststatistik $T$ ab\footnotemark.
\footnotetext{In der Tat ist das hier gew"ahlte $T$ nicht die optimale Wahl, da nicht-gau"sf"ormige 
Verteilungen existieren, die das gleiche vierte Moment wie eine Gau"sverteilung besitzen. Dies ist f"ur die
 folgenden Betrachtungen jedoch ohne Belang.}


Wir berechnen nun eine Anzahl $B$ von sogenannten \begriff(Surrogatdaten) oder kurz
\begriff(Surrogaten) $\{X_k, k=1,\dots,B\}$. Zur Erzeugung der Surrogatdaten nehmen wir
einen Gau"sproze"s $G(0,1)$, der $N$ unabh"angige, gau"sverteilte Zufallszahlen mit
Mittelwert $\hat\mu\simeq 0$ und empirischer Varianz $\hat\sigma^2\simeq 1$
erzeugt\footnotemark. Wir berechnen nun $T$ f"ur die Surrogatdaten als auch f"ur die
realen Daten. Die $T$-Werte seien mit $\{T_k,k=1,\dots,B\}$ bzw.\ $T_R$ bezeichnet. Wollen
dir die Nullhypothese auf dem $\alpha$-Signifikanzniveau ablehnen, mu"s $T_R$ unter den
$(B+1)/alpha/2$ kleinsten oder den $(B+1)\alpha/2$ gr"o"sten Werten der sortierten Liste
sein, die sowohl die $T_k$ als auch $T_R$ enth"alt\footnotemark. Zu beachten ist, da"s
$B+1$ mindestens gleich $2/\alpha$ sein mu"s. Im allgemeinen wird als ein Vielfaches von
$2/alpha$ gew"ahlt. F"ur ein "ubliches Signifikanzniveau von $\alpha=0,05$ w"are $B=39$,
die minimale Anzahl an verwendeten Surrogaten.  \footnotetext{Ein Gau"sproze"s
  $G(\mu,\sigma)$ erzeugt Zufallszahlen mit Erwartungswert $\mu$ und Standardabweichung
  $\sigma$. Bei der Realisierung $X$ eines solchen Prozesses k"onnen der Mittelwert
  $\hat\mu$ und die empirische Standardabweichung$\hat\sigma$ der erzeugten Daten hiervon abweichen.
  Die Werte f"ur die Realisierung eines solchen Prozesses sollen daher durch ein Dach
  "uber der Variablen unterschieden werden. F"ur sehr gro"se $N$ konvergieren $\hat\mu$ und
  $\hat\sigma$ gegen Erwartungswert $\mu$ und Standardabweichung $\sigma$.}  \footnotetext{Bei sehr
  gro"sem $B$ oder bekannter $T$-Verteilung h"atten wir auch den $\alpha$-Konfidenzbereich
  $[T_{\alpha,\tmin},T_{\alpha,\tmax}]$ berechnen k"onnen. Liegt $T_R$ au"serhalb des
  Kondidenzbereichs, kann die Nullhypothese abgelehnt werden.}

\subsubsection{Zusammengesetzte Nullhypothesen}
Dieses Beispiel ist nun -- au"ser zu Demonstrationszwecken -- reichlich uninteressant. Bei
vorliegenden Daten wollen wir die Nullhypothese pr"ufen, ob die Daten allgemein
gau"sverteilt mit unbekanntem Mittelwert $\mu$ und Varianz $\sigma^2$ sind. Dies ist
jedoch eine zusammengesetzte Nullhypothese. Die mit $\nullhyp$ konsistenten Prozesse, sind
alle Gau"sprozesse mit beliebigem $\mu$ und $\sigma^2$. Es w"are nun offensichtlich nicht
sinnvoll und auch nicht praktikabel, die Teststatistik \eqnref{discrim1} f"ur alle
m"oglichen Gau"sprozesse $G(\mu,\sigma)$ zu berechnen. Es existieren nun drei verschiedene
Ans"atze
\begin{itemize}
\item Wir benutzen eine \begriff(zentrale) Teststatistik. Bei einer zentralen
  Teststatistik ist die $T$-Verteilung f"ur alle Elemente von $\process$ gleich.
\item Wir berechnen $\hat\mu_R$ und $\hat\sigma_R$ f"ur die realen Daten und betrachten f"ur die
  Erzeugung der Surrogatdaten nur Gau"sprozesse mit $\mu=\hat\mu_R$ und $\sigma=\hat\sigma_R$,
  d.h.\ wir nehmen nur eine Teilmenge von $\process_0\subset\process$ mit
  $\process_0=\{G(\mu,\sigma),\mu=\hat\mu_R\land\sigma=\hat\sigma_R\}$. Man spricht hier von \begriff(typischen
  Realisierungen).
\item Wir benutzen nur Surrogatdaten f"ur das berechnete $\hat\mu$ und $\hat\sigma$ exakt gleich
  Mittelwert und Varianz der Originaldaten sind. Dieser als \begriff(eingeschr"ankte
  Realisierung) bezeichnete Ansatz, weicht von dem vorherigen leicht, aber doch
  signifikant ab.
\end{itemize}

\paragraph{Zentrale Teststatistiken}
Um aus der Teststatistik \eqnref{discrim1} eine zentrale Teststattistik zu machen definieren wir $T$
 nun folgenderma"sen
\eqnl[discrim2]{T'=\frac{1}{N}\sum_{i=1}^N \left( \frac{x_i-\hat\mu}{\hat\sigma} \right)^4} 
Durch die in der Gleichung vorgenommene Umskalierung hat den Effekt, da"s die Teststatistik
f"ur alle Gau"sprozesse $G(\mu,\sigma)$ die gleiche Verteilung aufweist. 


\paragraph{Typische Realisierungen}


\paragraph{Eingeschr"ankte Realisierungen}



\begin{itemize}
\item Nullhypothesen: einfache und zusammengesetzte
\item Teststatistik bei zusammengesetzten: pivotal
\item andere M"oglichkeit: typische Realisierung, eingeschr"ankte (gezwungene, gez"ugelt) Realisierungen 
\item Realisierungen und Kofidenzintervalle "uber Monte-Karlo-Methoden
\item Beispiel der Gau"sverteilung: a) 1,0 Test b) pivotal c) typische Realisierung d) eingeschr"ankte Realisierung
\item Beispiel typische Realisierungen: ARMA-prozesse
\item Eingeschr"ankte Realisierungen: FT-Surrogate, AAFT-Surrogate
\end{itemize}


Bei Nullhyptothesen sind zwei Klassen zu unterscheiden: einfache und zusammengesetzte
Nullhypothesen.  Eine einfacher Nullhypothese

Bei zusammengesetzten Nullhypothesen, h"angt diese noch von einem oder mehreren freien
Parametern ab. Ein Beispiel f"ur eine solche, w"are die Nullhypothese, da"s die Daten
gau"sverteilt sind. Die freien Parameter w"aren hier der Mittelwert $\bar x$ und die
Varianz $\sigma_x$. Ein einfache Nullhypothese w"are dagegen, da"s die Daten gau"sverteilt
mit Mittelwert $0$ und Varianz $1$ sind. Ein Test f"ur einfache Nullhypothese ist
offensichtlich weitaus einfacher als f"ur eine zusammengesetzte. Jedoch sind die meisten
Nullhypothesen, die von Interesse sind, zusammengesetzt. Wie man f"ur diese vern"unftige
Tests und Modelle entwirft wird uns im folgenden besch"aftigen.






Eine bessere Methode \cite{prichard-theiler3}


\comment{
\subsection{Statistische Testverfahren}

\subsection{Anwendung in der Zeitreihenanalyse}

\subsection{Modelle}

\subsubsection{ARMA-Modelle}

\subsubsection{Surrogatdaten}

\paragraph{Phasenrandomisierung}

\paragraph{Amplitudenangepa"ste Phasenrandomisierung }

\subsection{Diskriminatoren}

\subsection{Beispiele}

\subsubsection{Wei"ses und farbiges Rauschen}

\subsubsection{Rauschfreie und verrauschte deterministische Systeme}
}







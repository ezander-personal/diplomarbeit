\chapter{Medizinische Fachbegriffe} 

Die meisten der hier erl"auterten medizinischen Fachbegriffe stammen aus ``Pschyrembel --
Klinisches W"orterbuch'' \cite{Pschyrembel}.  F"ur Fr"uhgeborene spezifische "Anderungen
oder Erweiterungen der Definitionenen sind aus \autor(Poets) (1993) \cite{Poets93} und
\autor(Hoch) und \autor(Bergmann) (1996) \cite{Hoch96} erg"anzt worden.

\begin{description}
\item[Apnoe:] Atemstillstand.
\item[Atmung, periodische:] Atmung mit Abwechselnd auftretenden mehreren tiefen Atemz"ugen
  und darauffolgender kurzer apnoischer Pause.
\item[Bradykardie:] langsame Schlagfolge des Herzens mit einer Pulsfrequenz unter 60/min,
  bei Fr"uhgeborenen schon ab unter 90-100/min.
\item[Epidemiologie:] Wissenschaftszweig, der sich mit der Verteilung von "ubertragbaren
  und nicht"ubertragbaren Krankheiten und deren physikalischen, chemischen, psychischen
  und sozialen Determinanten und Folgen in der Bev"olkerung befa"st.
\item[Gestationsalter:] Schwangerschaftsdauer, Reifezeichen des Neugeborenen.
\item[Hypox"amie:] niedriger Sauerstoffpartialdruck im arterielle Blut ($\mathrm{pO_2}<70$
  mmHg). Bei Neugeborenen ein Abfall auf unter 40 -- 45 mmHg bzw.\  unter 20\% des Basalwertes. 
\item[idiopathisch:] ohne erkennbare Ursache entstanden, Ursache nicht nachgewiesen.
\item[Konzeptionsalter:] Lebensalter mit Beginn der Empf"angnis.
\item[Neonatologie:] Teilgebiet der Kinderheilkunde, das sich mit Diagnose und Therapie von
  Erkrankungen des Neugeborenen befa"st.
\item[Pathophysiologie:] Lehre von den krankhaften Lebensvorg"angen im menschlichen
  Organismus.
\item[pathologisch:] krankhaft.
\item[Pr"avalenz:] Anzahl der Erkrankungsf"alle einer best.\  Erkrankung bzw.\
  H"aufigkeit eines best.\  Merkmals zu einem best.\  Zeitpunkt oder innerhalb einer
  best.\  Zeitperiode.
\item[Pulsoxymetrie:] transkutane (unblutige) Messung der arteriellen
  Sauerstoffs"attigung.
\item[QRS-Komplex:] Phase der Erregungsausbreitung in den Herzkammern.
\item[REM-Schlaf:] Abk"urzung f"ur engl.\  Rapid Eye Movement. Schlafphase mit raschen
  Augenbewegungen und erh"ohter Herz- und Atemfrequenz.
\item[thorakal:] zum Brustkorb geh"orig.
\item[Thorax:] Brustkorb.
\item[transkutan:] durch die Haut hindurch.
\item[Zyanose:] blau-rote F"arbung von Haut und Schleimh"auten infolge einer Abnahme des
  Sauerstoffgehalts im Blut.
\end{description}


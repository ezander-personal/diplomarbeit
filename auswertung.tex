\section{Auswertung der Zeitreihen} 

\subsection{Fourier-Analyse}
Die gemessen Zeitreihen wurden zuerst einer Fourier-Analyse
unterzogen, wobei hier nur die Leistungsspektren $P(f)$ (d.h. der Betrag der Amplituden der
Fourier-Transfor\-mier\-ten) von Interesse sind. Die Leistungsspektren der beiden
ausgew"ahlten Zeitreihen zeigen \psref{medfourierr} (oben) und \psref{medfourierp}
(oben). Im Spektrum der regelm"a"sigen Atmung sind vier Peaks gut zu unterscheiden, von
denen allerdings nur die ersten drei durchgehend in den Leistungsspektren der
regelm"a"sigen Atmung auftauchen. Der 
erste liegt bei ca.\ 0,8 Hz und entspricht der mittleren Atemfrequenz des Kindes. Die
Atemfrequenzen Fr"uhgeborener liegen in der Regel bei ca.\ 
0,6-0,8 Hz, und k"onnen in Ausnahmef"allen auch bis 1,2 Hz reichen, ohne da"s dies als
pathologische Indikation zu werten ist. Die Atemfrequenz dieser Patienten liegt damit
deutlich h"oher als die von erwachsenen Individuen. Die beiden weiteren Peaks (bei 2,2 bzw. 4,3 Hz)
r"uhren von der Herzt"atigkeit her, was durch Vergleich mit den Spektren des gleichzeitig
aufgenommenen EKGs best"atigt wird (siehe \psref{medfourierr} unten). Auch die Herzfrequenz
Fr"uhgeborener liegt h"oher als die Erwachsener, wobei eine Frequenz von ca.\ 2 Hz als normal
anzusehen ist. Die deutlich zu erkennenden Oberfrequenzen werden vornehmlich hervorgerufen
durch die steilen Flanken des sogenannten QRS-Komplexes, eines im EKG-Signal auftretenden sehr
starken Ausschlags.

Bei der periodischen Atmung liegen die Peaks an etwa der gleichen Stelle. Im dargestellten
Spektrum (\psref{medfourierp} oben) erkennt man Peaks bei 0,7 und 2,1 Hz sowie
andeutungsweise bei 4,3 Hz.  Die Auspr"agung ist jedoch weniger deutlich als bei der
regelm"a"sigen Atmung. Auch hier k"onnen die Peaks mit Frequenzen "uber 2 Hz mit
entsprechenden Peaks des Leistungsspektrums des EKGs identifiziert werden.

Da die Peaks bei Frequenzen "uber 2 Hz im allgemeinen von der Herzt"atigkeit herr"uhren, ist es
h"aufige Praxis, auf die Zeitreihen Tiefpa"sfilter mit Grenzfrequenzen von 2 bis 2,5 Hz
anzuwenden. Da hierbei jedoch nicht klar ist, ob durch die Filterung eventuell f"ur die
Atmungsdynamik wesentliche Informationen verlorengehen, findet diese Vorgehensweise hier
keine Anwendung. 


\epsfigdouble
{anwendung/fourier/fr_r_3}
{anwendung/herzfreq/hfr_r}
{Oben das Leistungsspektrum der ausgew"ahlten Zeitreihe des thorakalen Impedanzsignals bei regelm"a"siger Atmung.
  Darunter zum Vergleich das entsprechende Spektrum  des gleichzeitig aufgenommen EKGs.
}
{medfourierr}{-0.5cm}

\epsfigdouble
{anwendung/fourier/fr_p_2}
{anwendung/herzfreq/hfr_p}
{Oben das Leistungsspektrum der ausgew"ahlten Zeitreihe des thorakalen Impedanzsignals bei
  periodischer Atmung.
  Darunter zum Vergleich das entsprechende Spektrum  des gleichzeitig aufgenommen EKGs.
}
{medfourierp}{-0.5cm}


\subsection{Bestimmung der Verz"ogerungszeit}
Die Verz"ogerungszeiten f"ur die Phasenraumrekonstruktionen wurden "uber das Verfahren der
Redundanzanalyse (siehe Seite \pageref{chapredundancy}ff) bestimmt. Die Redundanz $R(t)$
f"ur die beiden ausgew"ahlten Zeitreihen zeigt \psref{medredund}.  Die "uber das erste
lokale Minimum der Redundanz bestimmten Verz"ogerungszeiten $\delay_R$
lagen bei 0,27 -- 0,42 s f"ur regelm"a"sige und bei 0,24 -- 0,36 s f"ur periodische Atmung,
wobei der Unterschied nicht signifikant ist. Eine "Ubersicht "uber die gemessenen Werte
zeigt die untenstehende Tabelle. Da die Verz"ogerungszeiten eine recht geringe Streuung
aufweisen, wurde f"ur die meisten Auswertungen ein Standardwert von $\delay_R=0,32$ s
benutzt. 

\begin{center}
\begin{tabular}{|l||c|c|c|}
  \hline
  Atmungstyp & $\delay_{R,\tmin}/$s & $\delay_{R,\tmax}/$s & $\bar \delay_{R}/$s \\
  \hline
  regelm"a"sig   &  0,27 &  0,42   &  0,33$\pm$0,04 \\
  periodisch        &  0,24 &  0,36   &  0,31$\pm$0,03 \\
  \hline
\end{tabular}
\end{center}

\epsfigdouble
{anwendung/redund/mut_r_3}
{anwendung/redund/mut_p_1}
{Redundanzanalyse f"ur die Zeitreihen mit regelm"a"siger (oben) bzw.\  periodischer
  (unten) Atmung. Die Verz"ogerungszeiten
  liegen bei $\delay_R=0,33 s$ f"ur regelm"a"siges bzw.\ bei $\delay_R=0,27 s$ f"ur
  periodisches Atmen. 
}
{medredund}{-0.5cm}

\clearpage
\subsection{Versuch einer Phasenraumrekonstruktion}
Mit den so gewonnenen Verz"ogerungszeiten wurden Darstellungen der Phasenraumrekonstruktionen
erzeugt. Es sei darauf hingewiesen, da"s diese Abbildungen {\em keine} g"ultigen
Rekonstruktionen eventuell zugrunde liegender seltsamer Attraktoren darstellen, da die
verwendete Einbettungsdimension $\embed=2$ mit Sicherheit nicht ausreichend ist. Sie sollen
nur Hinweise auf die  Struktur der Dynamik und Anhaltspunkte f"ur die weitere
Verfahrensweise geben.

In der Rekonstruktion des regelm"a"sigen Atemsignals erkennt man deutlich eine toroidale
Struktur (siehe \psref{medrekonst} links oben). Diese ist allerdings gest"ort durch Rauschen
sowie ein gewisses \naja(Hin-und-her-Driften) der Trajektorien entlang der
Hauptdiagonalen; die Rekonstruktion erweckt den Eindruck eines Grenzzyklus, der entlang dieser Diagonalen hin
und her geschoben wird. Dies kann als Hinweis auf
eine Instationarit"at der Zeitreihe aufgefa"st werden. Die Rekonstruktion des periodischen Atmens
zeigt keine klare Struktur sondern eher ein \naja(Kn"auel) durcheinander laufernder
Trajektorien (siehe \psref{medrekonst} rechts oben).  Um das sichtbar starke Rauschen zu
unterdr"ucken wurde f"ur beide Signale eine Singular Value Decomposition durchgef"uhrt und
die erste Komponente der SVD-Rekonstruktion wieder eingebettet (siehe \psref{medrekonst}
unten). Die Trajektorien scheinen ein wenig gegl"attet zu sein, lassen jedoch auch nicht
mehr von der Struktur der Dynamik erahnen. 

\afterpage{
\epsfigfour
{anwendung/rects/rct_r_3}
{anwendung/rects/rct_p_1}
{anwendung/rects/rctsf_r_3}
{anwendung/rects/rctsf_p_1}
{Oben dargestellt sind die Rekonstruktionen eines regelm"a"sigen (links) bzw.\  periodischen Atemsignals (rechts)
  zur Einbettungsdimension $\embed=2$ und zur Verz"ogerungszeit $\delay_R=3,2$ s; unten
 die entsprechenden Rekonstruktionen der "uber SVD gefilterten Signale.
}
{medrekonst}{-0.5cm}
}

Die Instationarit"at der Zeitreihen l"a"st sich auch an "Anderungen der Verteilungen
$n(x)$ der Me"swerte "uber verschiedene Zeitr"aume beobachten, wobei die Gr"o"se $n(x)$
besagt wieviele Me"swerte in ein enges Intervall um den Me"swert $x$ fallen.
\psref{medinstat} zeigt jeweils die Verteilungen sowohl f"ur die gesamte als auch getrennt
f"ur die erste und zweite H"alfte der Zeitreihe. Zwischen den Verteilungen f"ur beide
H"alften bestehen signifikante Unterschiede. Sowohl die Lage des Maximums
als auch die Form der Verteilung variieren zwischen den beiden H"alften der Zeitreihen 
deutlich. Dies l"a"st sich auch quantitativ "uber die Mittelwerte und die
Streuungen der Verteilungen ausdr"ucken. Bei der Zeitreihe des regelm"a"sigen Atemsignals
betr"agt der Unterschied der Mittelwerte beider H"alften $1,2$ Prozent und der Unterschied
der Streuungen $9,5$ Prozent, wobei die Prozentangaben jeweils auf die Streuung
der gesamten Zeitreihe bezogen sind. Bei
der Zeitreihe aus periodischer Atmung betragen die entsprechenden Unterschiede $5,5$
Prozent bei den Mittelwerten und $55,8$ Prozent bei den Streuungen. Zum Vergleich: Bei
einer Zeitreihe des Lorenz-System mit vergleichbarer L"ange betrugen die gemessenen
Unterschiede nur $0,5$ bzw. $0,4$ Prozent.

Es sei an dieser Stelle noch darauf hingewiesen, da"s die bei diesen beiden
Zeitreihen beobachteten Instationarit"aten noch relativ gering sind und bei anderen
Zeitreihen weit gravierender ausfallen.

\afterpage{
\epsfigdouble
{anwendung/instat/cmphsregel}
{anwendung/instat/cmphsperiod}
{ Verteilungen $n$ der Me"swerte $x$ f"ur das regelm"a"sige Atmen (oben) und das periodische 
  Atmen (unten). Die Kurve f"ur die gesamte Zeitreihe (4 min) ist jeweils durchgezogen, f"ur die beiden 
  H"alften der Zeitreihe (je 2 min) jeweils gestrichelt dargestellt.
}
{medinstat}{-0.5cm}
}

\subsection{Bestimmung der Korrelationsintegrale und -dimensionen}
F"ur die Atemsignale wurden Korrelationsintegrale bis zu einer Einbettungsdimension
$\embed=8$ berechnet. Zwar sind bei dem gegebenen Datenumfang maximal
Korrelationsdimensionen bis $\corrdim\simeq 4$ berechenbar, was auch nur eine
Einbettungsdimension von $d=4$ erfordern w"urde, jedoch l"a"st die Berechnung  bis $d=8$
erkennen, ob die berechneten Integrale bzw.\  Dimensionen konvergieren oder nicht.

Die zu den Parameterwerten $d_\tmax=8$, $\delay=0,32$ s und $W=60$ berechneten
Korrelationsintegrale zeigt \psref{medcorrint}. In dem Korrelationsintegral der
regelm"a"sigen Atmung scheint ein Skalierungsbereich zwischen $\ln r$-Werten von  $-4$ und 
$-3$ zu liegen (siehe \psref{medcorrint} oben), bei der periodischen Atmung k"onnte man einen Skalierungsbereich zwischen
$-4,5$ und $-3$ vermuten (siehe \psref{medcorrint} unten).
Ob die Steigung des Korrelationsintegrals in diesen Bereichen wirklich konstant ist,
l"a"st sich jedoch besser an den entsprechenden Slopeplots erkennen erkennen. F"ur die
regelm"a"sige Atmung existiert (entgegen dem bei den Korrelationsintegralen gewonnenen Eindruck)
kein gr"o"serer Bereich von $r$-Werten
mit konstantem $\corrdim(r)$; weiterhin ist keine Konvergenz der Steigungen festzustellen
(siehe \psref{medcorrslp} oben). F"ur die
periodische Atmung scheint bei etwa $\ln r=-4$ Konvergenz aufzutreten (siehe
\psref{medcorrslp} unten). Dies ist jedoch eher
als Artefakt des bei h"oheren Einbettungsdimensionen auftretenden \begriff(Randeffekts)
zu deuten. Auch die "uber Takens' Sch"atzverfahren zum Maximalabstand $\ln\rmax=-2,5$ bestimmten
Dimensionen zeigen keine Konvergenz (siehe \psref{medcorrdim}). 


\epsfigdouble{anwendung/corrint/ci_r_3}{anwendung/corrint/ci_p_1}
{
Korrelationsintegrale der beiden ausgew"ahlten Zeitreihen f"ur regelm"a"sige (oben) und
periodische (unten) Atmung mit den Parameterwerten $\delay=0,32$ s und $W=60$ f"ur die
Einbettungsdimensionen $d=1,\dots,8$ (jeweils von oben nach unten).
}
{medcorrint}{-0.5cm}

\epsfigdouble{anwendung/corrint/cs_r_3}{anwendung/corrint/cs_p_1}
{
Slopekurven zu den Korrelationsintegralen aus \psref{medcorrint} f"ur die 
Einbettungsdimensionen $d=1,\dots,8$ (jeweils von unten nach oben).
}
{medcorrslp}{-0.5cm}

\epsfigdouble{anwendung/corrint/td_r_3}{anwendung/corrint/td_p_1}
{
"Uber Takens' Sch"atzverfahren berechnete Korrelationsdimension  zu den
Korrelationsintegralen aus \psref{medcorrint} mit dem Parameterwert $\ln\rmax=-2,5 $.
}
{medcorrdim}{-0.5cm}



Die Berechnungen wurden auch mit Variationen der Parameterwerte sowie mit verschiedenen
auf die Zeitreihen angewandten Filtertechniken
ausgef"uhrt. Die nachfolgend aufgez"ahlten Parametereinstellungen und Filtermethoden
wurden auf verschiedene Weisen kombiniert, f"uhrten jedoch zu keiner
Verbesserung der Ergebnisse.
\begin{description}
\item[Filterung:] Es wurden Zeitreihen mit und ohne SVD-Filterung verwendet, wobei f"ur
  die Singular Value Decomposition Einbettungsdimensionen zwischen $d=5$ und $d=30$
  gew"ahlt wurden. Weiterhin wurden Tiefpa"sfilterungen mit Grenzfrequenz bei 2,5 Hz
  durchgef"uhrt, um Beeinflussungen durch die Herzdynamik zu minimieren.
\item[Stationarit"at:] Biologische Zeitreihen sind, wie gesehen, oftmals instation"ar und weisen von
  der eigentlichen Dynamik unabh"angige langreichweitige Schwankungen (Bias) auf. 
  Um diese herauszufiltern wurden Fenster "uber die Zeitreihe gelegt, in denen dieser
  Grundwert berechnet und nachfolgend von den Signalwerten subtrahiert wurde.
\item[Verz"ogerungszeiten:] Kleine Verz"ogerungszeiten bringen bei der Berechnung der
  Korrelationsintegrale oftmals besseres Ergebnisse als die berechneten
  Verz"ogerungszeiten. Es wurden Verz"ogerungszeiten bis hinunter zu $\delay=0,1$ s
  getestet.
\item[Autokorrelationszeit:] Um Effekte durch Autokorrelationen auszuschlie"sen, wurde
  der Parameter f"ur die Autokorrelationsl"ange $W$ von 1 bis zu Vielfachen der
  Verz"ogerungszeit ausgetestet.
\end{description}
Da eine endliche fraktale Dimension Voraussetzung f"ur die Existenz eines seltsamen
Attraktors ist und die Korrelationsanalyse (die Korrelationsdimension ist eine untere
Absch"atzung f"ur die fraktale Dimension) keine konvergenten Ergebnisse erbrachte, k"onnen wir somit
nicht auf die Existenz eines solchen schlie"sen. 

F"ur den Atemtypus der regelm"a"sigen Atmung ist noch eine andere (sehr vorsichtige)
Deutung m"oglich. Es k"onnte sich hierbei um eine Grenzzyklus handeln, der jedoch stark
verrauscht ist. Dies st"unde sowohl im Einklang mit dem gemessenen Fourier-Spektrum, das
relativ klar hervortretende Peaks aufweist, als auch mit 2 bzw.\ 3-dimensionalen
Rekonstruktionen, die eine toroidale Struktur zeigen. Zudem weisen nicht-chaotische Systeme
einen verk"urzten Skalierungsbereich im Korrelationsintegral auf: Das Korrelationsintegral
skaliert nur "uber einen Bereich der Ordnung \order{N} statt der Ordnung \order{N^2} wie
im chaotischen Fall \cite{Theiler}.

\subsection{Vorhersagbarkeit von Apnoen}
Da die im vorigen Abschnitt berechneten Korrelationsintegrale keinen Skalierungsbereich
aufwiesen und auch keine Konvergenz der Slopekurven vorlag, konnte die Existenz eines
niedrigdimensionalen Attraktors mit bestimmter Korrelationsdimension $\corrdim$ f"ur die
Atmungsdynamik nicht nachgewiesen werden. Betrachtet man den Wert der f"ur eine bestimmte
Einbettungsdimension $d$ und eine bestimmte Verz"ogerungszeit $\delay$ berechneten
Korrelationsdimension $\corrdim$ jedoch nicht als wirkliche Dimension des zugrunde
liegenden dynamischen Systems sondern als ein Ma"s, da"s die relative Komplexit"at der
Dynamik in dem durch die Zeitreihe gegebenen Zeitabschnitt beschreibt, k"onnen die f"ur
verschiedene zeitlich versetzte Zeitreihen berechneten Werte "Anderungen dieses
Komplexit"atsmasses anzeigen und so eventuell eine Vorhersage von Atemstillst"anden
erm"oglichen.

Zum Test dieser M"oglichkeit wurde eine Zeitreihe, in der zwei Apnoen kurz
hintereinander auftraten,
verwendet (siehe \psref{medtestapnoe} oben). Die L"ange dieser Zeitreihe betrug 9 Minuten
(statt der "ublicherweise benutzen 4 min"utigen Zeitreihen), damit auch in  gr"o"seren
Zeitabst"anden vor Auftreten der Apnoen die Korrelationsdimensionen berechnet werden
konnten. Aus dieser Zeitreihe wurden St"ucke von jeweils 2 Minuten L"ange
extrahiert, wobei die Anfangszeiten dieser Zeitreihen jeweils um 15 Sekunden verschoben
wurden. Jede dieser Zeitreihen erstreckt sich somit "uber einen Zeitraum von $t_{i,a}=i
15\sek$ bis $t_{i,e}=i 15\sek + 120\sek$. Es wurden nun die
entsprechenden Korrelationsintegrale zur Einbettungsdimension $d=5$ berechnet und die Korrelationsdimensionen $D_{2,i}$ "uber Takens' 
Sch"atzverfahren mit dem Maximalabstand $\ln\rmax=-2,5$
bestimmt. Die Berechnung der Korrelationsdimension geschieht immer "uber einen bestimmten
Zeitraum, so da"s eine Auftragung der Korrelationsdimension "uber der Zeit nicht ganz
trivial ist.
Da zur Berechnung der Korrelationsdimension die Zeitreihe jedoch komplett vorliegen
mu"s, ist es sinnvoll bei dieser Auftragung jeweils das zeitliche Ende der Reihe als Wert auf
der Zeitachse zu w"ahlen, d.h. man betrachtet die Auftragung von $D_{2,i}$ "uber
$t_{i,e}$. Das Ergebnis einer solchen Berechnung zeigt \psref{medtestapnoe} (unten).

\epsfigdouble
{anwendung/apnoe/apn2}
{anwendung/apnoe/dimensionen}
{
Oben die verwendete Zeitreihe mit Apnoen bei $t=390\sek$ und $t=460\sek$. Unten die
zeitabh"angige Korrelationsdimension $D_2(t)$ f"ur $t\geq 120\sek$. F"ur $t<120\sek$ lie"s 
sich die Berechnung aufgrund der L"ange der Zeitreihen von 2 min nicht durchf"uhren.
}
{medtestapnoe}{-0.5cm}

Die Werte der Korrelationsdimension f"ur $t\leq390\sek$ schwanken in einem Bereich
zwischen ca.\ $4,0$ und ca.\ $4,3$, wobei in diesen Schwankungen kein bestimmter Trend auszumachen ist.
Ferner liegen die Werte in einem Bereich, der auch bei anderen Zeitreihen, in denen
keine Apnoen auftreten, typisch ist. Erst bei $t=405\sek$ zeigt sich ein deutlicher Abfall
der Korrelationsdimension auf ca.\ $2,1$, die sich nachfolgend langsam wieder zu h"oheren
Werten hin entwickelt.  Da der Atemstillstand jedoch schon bei $t=400\sek$ eintritt, ist
die Korrelationsdimension folglich f"ur die Vorhersage von Apnoen nicht zu gebrauchen.


\subsection{Determinismustest "uber AAFT-Surrogate}

Trotz der negativen Ergebnisse bei der Dimensionsanalyse k"onnen die f"ur bestimmte
Einbettungsdimensionen $d$ und f"ur einen bestimmten Maximalabstand $\rmax$ berechneten Werte der
Korrelationsdimension zum Vergleich mit entsprechenden Werten von Surrogatdaten im Rahmen
eines Determinismustests herangezogen werden. Da die Dichteverteilung der Me"swerte eine
deutlich nicht-gau"ssche Verteilung zeigen (siehe \psref{medinstat}) wurden als Surrogatdaten
AAFT-Surrogate verwendet. Die Anzahl der erstellten Datenreihen betrug in beiden F"allen
(d.h. f"ur regelm"a"sige bzw.\ periodische Atmung) 39. F"ur diese wurden mit den gleichen
Parametern ($\embed_\tmax=8, \delay=0,32, W=60)$ wie f"ur die Originalzeitreihen
Korrelationsintegrale berechnet. Aus diesen wurde "uber Takens' Sch"atzverfahren die
Korrelationsdimension $\corrdim$ in Abh"angigkeit von der Einbettungsdimension bestimmt. Die
Maximalabst"ande lagen bei $\rmax=-2,5$. Den Vergleich der Dimensionsberechnungen zwischen
Original- und Surrogatdaten zeigt \psref{medsurrodim}. Deutlich zu erkennen ist, da"s sich 
f"ur beide Atemtypen bei Einbettungsdimensionen $d\geq4$ die Korrelationsdimension der
Originaldaten stark von den Korrelationsdimensionen der Surrogatdaten unterscheidet.
Aus der Verteilung der Korrelationdimensionen wurde gem"a"s \eqnref{eqnsigni} die
Signifikanz $\mathcal S$ der Abweichungen berechnet (siehe \psref{medsurrosigni}).
Au"ser f"ur $d=7$ und $d=8$ bei der regelm"a"sigen Atmung und $d=1,\dots,3$ bei der
periodischen Atmung liegt die Signifikanz sehr hoch ($\mathcal{S}>5$). 
Die entsprechenden $p$-Werte (siehe \eqnref{eqnpvalue}) liegen in diesen F"allen unter
$10^{-7}$. 

Da die $p$-Werte angeben, mit welcher Wahrscheinlichkeit die Ablehnung der Nullhypothese
falsch ist, k"onnen wir mit "uber $99,9$ prozentiger Sicherheit davon ausgehen, da"s es
sich bei der Atmungsdynamik um kein durch ein ARMA-Modell beschreibbares stochastisches
System handelt. Selbst bei Zugrundelegung des vorher festlegegten Signifikanzniveaus
$\alpha=0,05$, das ja f"ur die minimale Anzahl der berechneten Surrogatzeitreihen
ausschlaggebend ist, liegt die Wahrscheinlichkeit, da"s der Atmung ein deterministischer
Charakter zugesprochen werden kann, bei mindestens $95$ Prozent.

\epsfigdouble{anwendung/surrogat/regel/cmpsurcdim}{anwendung/surrogat/period/cmpsurcdim}
{
Vergleich der berechneten Korrelationdimensionen zwischen Original- \gpmarkb\  und
Surrogatdaten \gpmarka\  f"ur regelm"a"sige Atmung (oben) und periodische Atmung (unten). 
}
{medsurrodim}{-0.5cm}

\epsfigdouble{anwendung/surrogat/regel/signi}{anwendung/surrogat/period/signi}
{
Signifikanz $\mathcal{S}$ der Abweichung der Korrelationsdimension der Originaldaten vom
Mittelwert der Korrelationsdimensionen der Surrogatdaten f"ur regelm"a"sige Atmung (oben)
und periodische Atmung (unten).
}
{medsurrosigni}{-0.5cm}

\comment{
{anwendung/surrogat/regel/pvalue}
{anwendung/surrogat/period/pvalue}
}


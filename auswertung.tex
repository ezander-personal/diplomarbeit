\section{Auswertung} 

\subsection{Fourier-Analyse}
Die gemessen Zeitreihen wurden zuerst einer Fourier-Analyse
unterzogen. Die Fourier-Spektren von zwei ausgew"ahlten Zeitreihen zeigt \psref{medfourier}
(oben). Im Spektrum der regelm"a"sigen Atmung sind drei Peaks gut zu unterscheiden. Der
erste liegt bei ca.\ 0,8 Hz und, entspricht der mittleren Atemfrequenz des Kindes. Die
Atemfrequenzen Neugeborener liegen deutlich "uber denen Erwachsener bei ca.\ 
0,6-0,8 Hz, und k"onnen in Ausnahmef"allen auch bis 1,2 Hz reichen, ohne da"s dies als
pathologische Indikation zu werten w"are. Die beiden weiteren Peaks (bei 2,2 bzw. 4,3 Hz)
r"uhren von der Herzt"atigkeit her, was durch Vergleich mit den Spektren des gleichzeitig
aufgenommenen EKGs best"atigt wird (siehe \psref{medfourier} unten). Auch die Herzfrequenz
Neugeborener liegt h"oher als die Erwachsener, wobei eine Fequenz von ca.\ 2 Hz als normal
anzusehen ist. Die deutlich zu erkennenden Oberfrequenzen werden vornehmlich hervorgerufen
durch die steilen Flanken des QRS-Komplexes.

Da die Peaks bei Frequenzen "uber 2 Hz i.allg. von der Herzt"atigkeit herr"uhren, ist es
h"aufige Praxis, auf die Zeitreihen Tiefpa"sfilter mit Grenzfrequenzen von 2 -- 2,5 Hz
anzuwenden. Da hierbei jedoch nicht klar ist, ob durch die Filterung eventuell f"ur die
Atmungsdynamik wesentliche Informationen verlorengehen, findet diese Vorgehensweise hier
keine Anwendung. 

Bei der periodischen Atmung liegen die Peaks an etwa der gleichen Stelle. Im dargestellten
Spektrum (\psref{medfourier} links) erkennt man Peaks bei 0,7 und 2,1 Hz sowie
andeutungsweise bei 4,3 Hz.  Die Auspr"agung ist jedoch weniger deutlich als bei der
regelm"a"sigen Atmung.


\epsfigfour
{anwendung/fourier/fr_r_3}
{anwendung/fourier/fr_p_2}
{anwendung/herzfreq/hfr_r}
{anwendung/herzfreq/hfr_p}
{Oben die Fourier-Spektren einer Zeitreihe des thorakalen Impedanzsignals bei regelm"a"siger (links)
  bzw. periodischer Atmung (rechts).  Darunter zum Vergleich die entsprechenden Spektren
  des gleichzeitig aufgenommen EKGs.
}
{medfourier}{-0.5cm}


\subsection{Bestimmung der Verz"ogerungszeit durch Redundanzanalyse}
Die Verz"ogerungszeiten f"ur die Phasenraumrekonstruktionen wurden "uber das Verfahren der 
Redundanzanalyse bestimmt (siehe \psref{medredund}).  Die bestimmten Verz"ogerungszeiten
lagen bei 0,27--0,42 s f"ur regelm"a"sige und bei 0,24 -- 0,36 s f"ur periodische Atmung,
wobei der Unterschied nicht signifikant ist. Eine "Ubersicht "uber die gemessenen Werte
zeigt die untenstehende Tabelle. Da die Verz"ogerunsgzeiten eine recht geringe Streuung
aufweisen, wurde f"ur die meisten Auswertungen ein Standardwert von $\delay_r=0,32$ s
benutzt. 

\begin{center}
\begin{tabular}{|l||c|c|c|}
  \hline
  Atmungstyp & $\delay_{R,\tmin}/$s & $\delay_{R,\tmax}/$s & $\bar \delay_{R}/$s \\
  \hline
  regelm"a"sig   &  0,27 &  0,42   &  0,33$\pm$0,04 \\
  periodisch        &  0,24 &  0,36   &  0,31$\pm$0,03 \\
  \hline
\end{tabular}
\end{center}

\epsfigdouble
{anwendung/redund/mut_r_3}
{anwendung/redund/mut_p_1}
{Redundanzanalyse f"ur die Zeitreihen aus \psref{medfourier}. Die Verz"ogerungszeiten
  liegen bei $\delay_R=0,33 s$ f"ur regelm"a"siges bzw.\ bei $\delay_R=0,27 s$ f"ur
  periodisches Atmen. 
}
{medredund}{-0.5cm}

\subsection{Versuch einer Phasenraumrekonstruktion}
Mit den so gewonnenen Verz"ogerungszeiten wurden Bilder der Phasenraumrekonstruktionen
erzeugt. Es sei darauf hingewiesen, da"s diese Bilder {\em keine} g"ultigen
Rekonstruktionen eventuell zugrundliegender seltsamer Attraktoren darstellen. Sie sollen
nur die Struktur der Dynamik veranschaulichen, und Anhaltspunkte f"ur die weitere
Verfahrensweise geben.

In der Rekonstruktion des regelm"a"sigen Atemsignals erkennt man deutlich eine toroidale
Struktur (siehe \psref{medrekonst} links oben). Diese ist allerdings gest"ort durch Rauschen
sowie das \naja(Hin-und-her-Driften) der Trajektorien entlang der Hauptdiagonalen, was auf
eine Instationarit"at der Zeitreihe hinweist. Die Rekonstruktion des periodischen Atmens
zeigt keine klare Struktur sondern eher ein \naja(Kn"auel) durcheinander laufernder
Trajektorien (siehe \psref{medrekonst} rechts oben).  Um das sichtbar starke Rauschen zu
unterdr"ucken wurde, f"ur beide Signale eine Singular Value Decomposition durchgef"uhrt und
die erste Komponente der SVD-Rekonstruktion wiedereingebettet (siehe \psref{medrekonst}
unten).

Die Instationarit"at der Zeitreihen l"a"st sich auch an "Anderungen Dichteverteilungen der
Me"swerte "uber verschiedene Zeitr"aume beobachten. \psref{medinstat} zeigt jeweils die
Verteilungen sowohl f"ur die gesamte als auch getrennt f"ur die erste und zweite H"alfte
der Zeitreihe. Zwischen den Verteilungen f"ur beide H"alften bestehen siginifikante
Unterschiede. Es sei an dieser Stelle nur darauf hingewiesen, da"s die bei diesen
Zeitreihen beobachteten Instationarit"aten noch relativ gering sind, und bei anderen weit
gravierender ausfallen.

\epsfigfour
{anwendung/rects/rct_r_3}
{anwendung/rects/rct_p_1}
{anwendung/rects/rctsf_r_3}
{anwendung/rects/rctsf_p_1}
{Rekonstruktionen eines regelm"a"sigen (links) bzw.\  periodischen Atemsignals (rechts)
  zur Einbettungsdimension $\embed=2$ und Verz"ogerungszeit $\delay_R=3,2$ s. Unten
  dargestellt sind die entsprechenden Rekonstruktionen der "uber SVD gefilterten Signale.
}
{medrekonst}{-0.5cm}

\epsfigdouble
{anwendung/instat/cmphsregel}
{anwendung/instat/cmphsperiod}
{ Dichteverteilungen der Me"swerte f"ur das regelm"a"sige (links) und das periodische
  Atemsignal (rechts). Die Kurve f"ur das gesamte Signal ist durchgezogen, f"ur die beiden 
  H"alften des Signals gestrichelt dargestellt.
}
{medinstat}{-0.5cm}


\subsection{Bestimmung der Korrelationsintegrale und -dimensionen}
F"ur die Atemsignale wurden Korrelationsintegrale bis zu einer Einbettungsdimension
$\embed=8$ berechnet. Zwar sind bei dem gegebenen Datenumfang maximal
Korrelationsdimensionen bis $D_2\simeq 4$ berechenbar, was auch nur eine
Einbettungsdimension von $d=4$ erforden w"urde, jedoch l"a"st die Berechnung  bis $d=8$
erkennen, ob die berechneten Integrale bzw.\  Dimensionen konvergieren oder nicht.

Die zu den Parameter $d_\tmax=8$, $\delay=0,32$ s und $W=60$ berechneten
Korrelationsintegrale zeigt \psref{medcorrint} (oben). Ob vern"unftige Skalierungsbereiche
vorliegen l"a"st sich jedoch besser an den Steigungen des Korrelationsintegrals erkennen
(siehe \psref{medcorrint} mitte). F"ur die regelm"a"sige Atmung ist weder ein Bereich
konstanter Steigung noch eine Konvergenz der Steigungen festzustellen. F"ur die
periodische Atmung scheint bei etwa $\ln r=-4$ Konvergenz aufzutreten. Dies ist jedoch eher
als Artefakt des bei h"ohreren Einbettungsdimensionen auftretenden \begriff(Randeffekts)
zu deuten. Auch die "uber Takens' Sch"atzer zum Maximalabstand $\ln\rmax=-2,5$ bestimmten
Dimensionen zeigen keine Konvergenz (siehe \psref{medcorrint} unten).

\epsfigsix
{anwendung/corrint/ci_r_3}
{anwendung/corrint/ci_p_1}
{anwendung/corrint/cs_r_3}
{anwendung/corrint/cs_p_1}
{anwendung/corrint/td_r_3}
{anwendung/corrint/td_p_1}
{
Korrelationsintegrale, logarithmische Ableitung und Korrelationsdimension ("uber Takens'
Sch"atzer) f"ur regelm"a"sige (links) und periodische Atmung (rechts).
}
{medcorrint}{-0.5cm}

Die Berechnungen wurden auch mit Variationen der Parameter sowie der Filtertechniken
ausgef"uhrt. Die nachfolgend aufgez"ahlten Parametereinstellungen und Transformationen der
Zeitreihen wurden auf verschiedene Weisen kombiniert, f"uhrten jedoch zu keiner
Verbesserung der Ergebnisse.
\begin{description}
\item[Filterung:] Es wurden Zeitreihen mit oder ohne SVD-Filterung verwendet, wobei f"ur
  die Singular Value Decomposition Einbettungsdimensionen zwischen $d=5$ und $d=30$
  gew"ahlt wurden. Weiterhin wurden Tiefpa"sfilterungen mit Grenzfrequenz bei 2,5 Hz
  durchgef"uhrt, um Beeinflussungen durch die Herzdynamik zu minimieren.
\item[Stationarit"at:] Biologische Zeitreihen sind oftmals instation"ar, und weisen von
  der eigentlichen Dynamik unabh"angige langreichweitige Schwankungen (Bias) auf. 
  Um diese herauszufiltern wurden Fenster "uber die Zeitreihe gelegt, in denen dieser
  Grundwert berechnet und nachfolgend von den Signalwerten subtrahiert wurde.
\item[Verzo"ogerungszeiten:] Kleine Verz"ogerungszeiten bringen bei der Berechnung der
  Korrelationsintegrale oftmals besseres Ergebnisse als die berechneten
  Verz"ogerungszeiten. Es wurden Verz"ogerungszeiten bis hinunter zu $\delay=0,1$ s
  getestet.
\item[Autokorrelationszeit:] Um Effekte durch Autokorrelationen auszuschlie"sen, wurde
  der Parameter f"ur die Autokorrelationsl"ange $W$ von 1 bis zu Vielfachen der
  Verz"ogerungszeit ausgetestet.
\end{description}
Da eine endliche fraktale Dimension Voraussetzung f"ur die Existenz eines seltsamen
Atraktors ist und die Korrelationsanalyse (die Korrelationsdimension ist eine untere
Absch"atzung f"ur die fraktale Dimension) keine konvergenten Ergebnisse erbrachte, k"onnen wir somit
nicht auf die Existenz eines solchen schlie"sen. 

F"ur den Atemtypus der regelm"a"sigen Atmung ist noch eine andere (sehr vorsichtige)
Deutung m"oglich. Es k"onnte sich hierbei um eine Grenzzyklus handeln, der jedoch stark
verrauscht ist. Dies st"ande sowohl in Einklang mit dem gemessenen Fourier-Spektrum, das
relativ klar hervortretende Peaks aufweist, als auch mit 2 bzw.\ 3-dimensionalen
Rekonstruktionen, die eine toroidale Struktur zeigen. Zudem weisen nicht-chaotische Systeme
einen verk"urzten Skalierungsbereich im Korrelationsintegral auf: Das Korrelationsintegral
skaliert nur "uber einen Bereich der Ordnung \order{N} statt der Ordnung \order{N^2}, wie
im chaotischen Fall \cite{Theiler}.

\subsection{Determinismustest "uber AAFT-Surrogate}

Trotz der negativen Ergebnisse bei der Dimensionsanalyse k"onnen die f"ur bestimmte
Einbettungsdimensionen und Maximalabstand $\rmax$ berechneten Werte der
Korrelationsdimension zum Vergleich mit entsprechenden Werten von Surrogatdaten im Rahmen
eines Determinismustest herangezogen werden. Da die Dichteverteilung der Me"swerte eine
deutlich nicht gau"ssche Verteilung zeigen (siehe \psref{medinstat}) wurden als Surrogatdaten
AAFT-Surrogate verwendet. Die Anzahl der erstellten Datenreihen betrug in beiden F"allen
(d.h. f"ur regelm"a"sige bzw.\ periodische Atmung) 39. F"ur diese wurden mit den gleichen
Parametern ($\embed_\tmax=8, \delay=0,32, W=60)$, wie f"ur die Originalzeitreihen
Korrelationsintegrale berechnet. Aus diesen wurde "uber Takens' Sch"atzer die
Korrelationsdimension $D_2$ in Abh"angigkeit von der Einbettungsdimension bestimmt. Die
Maximalabst"ande lagen bei $\rmax=-2,5$. Den Vergleich der Dimensionsberechnungen zwischen
Original- und Surrogatdaten zeigt \psref{medsurro} (oben).  Die Signifikanz $\mathcal S$
der Abweichung ist in den meisten F"allen sehr hoch (siehe  \psref{medsurro} mitte). Der mit
der Signifikanz verbundene $p$-Wert liegt, au"ser bei der periodischen Atmung f"ur
Einbettungsdimensionen $d\leq3$, bei Werten deutlich unter 0,05. Wir k"onnen daher die
Nullhypothese, da"s es sich bei der Atmung um ein durch ein ARMA-Modell beschreibbares
stochastisches System handelt, auf einem 0,05-Signifikanzniveau ablehnen. Wie man schon
intuitiv vermuten w"urde, handelt es sich der Atmung also um ein determinsitisches System.

\epsfigsix
{anwendung/surrogat/regel/cmpsurcdim}
{anwendung/surrogat/period/cmpsurcdim}
{anwendung/surrogat/regel/signi}
{anwendung/surrogat/period/signi}
{anwendung/surrogat/regel/pvalue}
{anwendung/surrogat/period/pvalue}
{
Surrogatdatenanalyse f"ur regelm"a"sige (links) und periodische Atmung (rechts). 
Vergleich der berechneten Korrelationdimensionen zwischen Original- und Surrogatdaten
(oben), Signifikanzniveau der Abweichung (mitte) und $p$-Wert (unten).
}
{medsurro}{-0.5cm}






\section{Zur Atmung von Fr"uhgeborenen}


\subsection{Problemstellung}

Die Fortschritte auf dem Gebiet der intensivmedizinischen Betreuung w"ahrend der letzten
20 Jahre erm"oglichen zunehmend das "Uberleben kleiner und sehr kleiner Fr"uhgeborener\footnote{Als
  \begriff(Fr"uhgeborene) bezeichnet man S"auglinge mit einem Gestationsalter unter 37
  Wochen. Die Angabe \naja(klein) bzw.  \naja(sehr klein) bezieht sich auf das
  Geburtsgewicht von unter 1500 g bzw.\ unter 1000 g.}.
In der auf die Zeit der Intensivbehandlung folgenden Zeit der
Stabilisierung und ``Aufzucht'' sind Schwestern und "Arzte mit verschiedenen durch die
Unreife bzw.  Instabilit"at dieser Patienten bedingten Problemen konfrontiert. Hierunter
erfordert die Instabilit"at der Atmung unter allen Reifealtern vordringliches Augenmerk:
Beim eigenst"andig atmenden Fr"uhgeborenen kann es ohne vorausgehende Anzeichen
zum Atemstillstand (\begriff(Apnoe)) kommen.  Dieser f"uhrt zum Sauerstoffmangel (\begriff(Hypox"amie))und
konsekutiv zu einem Abfall der Herzfrequenz (\begriff(Bradykardie)) auf kritische, die
Kreislaufversorgung gef"ahrdende Verh"altnisse.  Neben der Gefahr eines akuten
Herz-Kreislaufversagens bringen h"aufige Atemstillst"ande die Gef"ahrdung des sich
entwickelnden Gehirns durch latenten Sauerstoffmangel mit sich.  Die Kinder werden durch
Ableitung des EKG\footnote{Man spricht hier von einer \begriff(Ableitung) des EKG, da die 
  interessierenden Aktionspotentiale des Herzens nicht direkt gemessen sondern nur
  indirekt durch Anbringung von Elektroden auf der K"orperoberfl"ache \naja(abgeleitet)
  werden k"onnen.} und des atemsynchronen thorakalen Impedanzsignals (TI) fortlaufend
"uberwacht. Besonders gef"ahrdete Patienten werden zus"atzlich durch Messung der
Sauerstoffs"attigung des Blutes "uberwacht (\begriff(Pulsoxymetrie)).

\subsection{Atemrhythmen Fr"uhgeborener}

Heute wird vom Konzept eines lokalisierten Rhythmusgenerators der Atmung abgesehen.
Stattdessen ist experimentell ein verteiltes Netzwerk atemkompetenter Nervenzellen belegt.
Zur Zeit wird von sechs jeweils verschiedenen Atemphasen zugeordneten Nervenzellgruppen im
Hirnstamm ausgegangen. Ihre gemeinsame Endstrecke ist der das Zwerchfell stimulierende
\begriff(Nervus Phrenicus). Die f"ur die Atmung zust"andigen \comment{respiratorisch
  kompetenten} Gruppen unterliegen ihrerseits einer Anzahl peripherer (Lungendehnung,
$\mathrm{O}_2$-Gehalt des Blutes) und zentraler  (Wachheit) Einfl"usse, welche sie
integrativ verarbeiten. \comment{(\autor(Richter))}

F"ur die hier untersuchten Fr"uhgeborenen, werden im wesentlichen die
folgenden Atemrhythmen unterschieden, wobei diese Einteilung ph"anomenologisch erfolgt:
\begriff(regelm"a"siges) Atmen einerseits und \begriff(periodisches) Atmen andererseits.
Die Atemrhythmen k"onnen grob den verschiedenen Schlafphasen zugeordnet werden: W"ahrend
des REM-Schlafs tritt ausschlie"slich periodische, w"ahrend des Non-REM-Schlafs
haupts"achlich regelm"a"sige, aber auch periodische Atmung auf.

Der Begriff \begriff(regelm"a"sige Atmung) beschreibt das Auftreten regelm"a"siger
Atemz"uge gleicher Atemtiefe (siehe \psref{medrhythm} oben). F"ur den Terminus
\begriff(periodische Atmung) folgen wir der Definition von \autor(Kryger); siehe \cite{Hoch96}. F"ur ihn sind
die Kriterien dieses Atmungsmusters erf"ullt, wenn eine Folge von Atmungspausen von
jeweils mehr als 3 Sekunden L"ange, unterbrochen von regul"aren Atmungsperioden mit einer
Dauer von bis zu 20 Sekunden, vorliegt (siehe \psref{medrhythm} unten). Ein Zusammenhang
zwischen periodischer Atmung und dem Auftreten von Fr"uhgeborenenapnoen scheint
nicht gegeben \cite{Hoch96}.

\epsfigdouble
{anwendung/zeitreihen/regeln}
{anwendung/zeitreihen/periodn}
{Beispiele f"ur die verschiedenen Atemrhythmen: regelm"a"siges Atmen (oben) bzw.\  
  periodisches Atmen (unten). Dargestellt ist das atemkorrelierte thorakale
  Impedanzsignal.  }{medrhythm}{-0.5cm}

"Uber die Definition einer Fr"uhgeborenenapnoe bestehen einige unterschiedliche
Auffassungen. In vielen Ver"offentlichungen wird als Indikation f"ur eine Apnoe neben dem Stillstand der
Atmung  ein Abfall der Herzfrequenz und des Sauerstoffpartialdrucks des Blutes als
Kriterium angegeben \cite{Poets93}. Weitere Kennzeichen sind eine Zyanose oder eine Bl"asse
des Patienten. Da das Vorliegen einer Hypox"amie oder Bradykardie aus den vorliegenden
Daten (thorakale Impedanz) nicht hervorgeht, verwenden wir die Definition von
\autor(Glotzbach) (siehe \cite{Hoch96}), nach der bei einem Atemstillstand von mindestens 15 Sekunden eine 
Apnoe vorliegt.  Ein Beispiel einer ca.\ 20 Sekunden dauernden Apnoe w"ahrend einer Phase
regelm"a"siger Atmung zeigt \psref{medapnoe}.

\epsfigdouble
{anwendung/zeitreihen/reg_apnoe}
{anwendung/zeitreihen/reg_apnoe2}
{ Apnoe w"ahrend einer Phase regelm"a"siger Atmung. Unten
  ein vergr"o"serter Ausschnitt der oben dargestellten Zeitreihe. Aufgrund des eingeschr"ankten Me"sbereichs ist
  das Signal des der Apnoe vorausgehenden tiefen Atemzugs nach oben und unten abgeschnitten.
}{medapnoe}{-0.5cm}


\subsection{Stand der Forschung}

Mit erheblichem Aufwand wird versucht, die Bedingungen, welche zu Atemstillst"anden
f"uhren, zu identifizieren.  Nur wenige Arbeiten gehen bisher der Frage nach dem
dynamischen Charakter der Fr"uhgeborenenatmung nach. \autor(Pilgrim \etal) konnten
ausgehend von Messungen der Korrelationsdimension ($\corrdim<3$) f"ur die periodische
Atmung einen deterministischen Charakter dieses Atemrhythmus belegen \cite{Pilgram94}. Weitere Aussagen zum
dynamischen Charakter der Atmung im allgemeinen finden sich vor allem in
tierexperimentellen Arbeiten. \autor(Sammon) konnte in Versuchen mit narkotisierten Ratten
nachweisen, da"s durch Stimulation des Nervus Vagus eine Erh"ohung der Dimension der
Atmungsdynamik stattfindet \cite{Sammon91}.  \autor(Eldridge) konnte -- ebenfalls an einem
Experiment mit Ratten -- zeigen, da"s die experimentell gefundene Abh"angigkeit der durch
eine St"orung des Atemzyklus hervorgerufenen Reaktion von der Phasenlage durch
Modellierung mittels eines periodisch erregten \begriff(Van-der-Pol-Oszillators)
nachgebildet werden konnte \cite{Eldridge89}.  Insgesamt fehlen bisher f"ur das
menschliche Neugeborene deutliche Hinweise auf einen deterministischen Charakter des
Atemrhythmus. Von Interesse ist dies vor allem im Hinblick auf die m"ogliche
Vorhersagbarkeit von St"orungen der Atmungsaktivit"at.  Dar"uber hinaus spielt diese Frage auch f"ur das
Verst"andnis des Zusammenhangs zwischen der lebensnotwendigen Stabilit"at der Atmung
einerseits und den vielen Unterbrechungen des Atemrhythmus (Schlucken, Innehalten,
Sprechen etc.) andererseits ein Rolle.

\subsection{Signale, Daten und Patienten}

F"ur die vorliegende Arbeit standen Aufzeichnungen des thorakalen Impedanzsignals von acht
Fr"uhgeborenen zur Verf"ugung. Diese entstammen der Routine"uberwachung. Die Messung der
thorakalen Impedanz registriert den atemabh"angig schwankenden Wechselstromwiderstand des
Brustkorbes anhand eines Me"sstroms (30 kHz, 3 $\mu$A), welcher "uber zwei Elektroden
eingebracht wird. Sie bestimmt somit den Anteil des leitenden zum nichtleitenden Material
im Volumenleiter Thorax. Da nichtleitendes Atemgas periodisch ein und ausstr"omt, variiert
der gemessene Wechselstromwiderstand im wesentlichen mit der Atmung. Das Signal ist allerdings
wegen der Herzaktion und der sich aus dieser ergebenden "Anderung des leitenden Volumens
im Brustkorb auch an das Herzkreislaufsystem gekoppelt. 

Die Sampling Rate betrug 100 Hz bei einer Aufl"osung von 12 Bit. Die Me"swerte wurden
auf einen Bereich ganzer Zahlen von -2048 bis +2047 abgebildet, wobei das
Abbildungsverh"altnis und die Nullage jeweils dem Patienten angepa"st wurden. Die
Aufzeichnung erfolgte zusammen mit dem EKG "uber 24 Stunden. Nach Abschlu"s der Messungen
wurden 13 Zeitreihen mit regelm"a"siger und 12 mit periodischer Atmung ausgew"ahlt. Die
L"ange der ausgew"ahlten Zeitreihen betrug 4 min, was einem Datenumfang von je 24.000 Punkten
entspricht.  In 4 dieser Zeitreihen traten Apnoen auf.

Die bei den folgenden Auswertungen abgebildeten Darstellungen beruhen, soweit nicht anders 
angegeben, auf den Zeitreihen aus \psref{medrhythm}. Die Auswertungen selber wurden f"ur
alle vorliegenden Zeitreihen durchgef"uhrt, zeigten jedoch i.a.\    keine signifikanten
Unterschiede zu den beiden Beispielzeitreihen, so da"s auf deren separate Darstellung
verzichtet wurde.


\section{Zur Atmung von Fr"uh- und Reifgeborenen}


\subsection{Problemstellung}

Die Fortschritte auf dem Gebiet der intensivmedizinischen Betreuung w"ahrend der letzten
20 Jahre erm"oglichen das "Uberleben kleiner ($<$1500 gr. Geburtsgewicht) und sehr kleiner
($<$1000 gr. Geburtsgewicht) Fr"uhgeborener und schwer erkrankter reifer Neugeborener. In
der auf die Zeit der Intensivbehandlung folgenden Zeit der Stabilisierung und ``Aufzucht''
sind Schwestern und "Arzte mit verschiedenen durch die Unreife bzw. Instablilit"at dieser
Patienten bedingten Problemen konfrontiert. Hierunter erfordert die Instabilit"at der
Atmung unter allen Reifealtern vordringliches Augenmerk: beim eigenst"andig atmenden
Fr"uh- und Reifgeborenen kann es ohne vorausgehende Zeichen zum Atemstillstand kommen.
Dieser f"uhrt zum Sauerstoffmangel (\begriff(Hypox"amie))und konsekutiv zu einem Abfall
der Herzfrequenz (\begriff(Bradykardie)) auf kritische, die Kreislaufversorgung gef"ahrdende
Verh"altnisse. Neben der Gefahr eines akuten Herz-Kreislaufversagens bringen h"aufige
Atemstillst"ande die Gef"ahrdung des sich entwickelnden Gehirns durch latenten
Sauerstoffmangel mit sich. Die Kinder werden durch Ableitung des EKG und des
atemsynchronen thorakalen Impedanzsignals (TI) fortlaufend "uberwacht. Besonders gef"ahrdete
Patienten werden zus"atzlich durch Messung der Sauerstoffs"attigung des Blutes "uberwacht
(\begriff(Pulsoxymetrie)).

\subsection{Generation des Atemrhythus und Atemrhythmen Neugeborener}

Heute wird vom Konzept eines lokalisierten Rhythmusgenerators der Atmung abgesehen. Statt
dessen ist experimentell ein verteiltes Netzwerk atemkompetenter Nervenzellen belegt. Zur
Zeit wird von sechs jeweils verschiedenen Atemphasen zugeordneten Nervenzellgruppen im
Hirnstamm ausgegangen. Ihre gemeinsame Endstrecke ist der das Zwerchfell stimulierende
\begriff(Nervus Phrenicus). Die respiratorisch kompetenten Gruppen unterliegen ihrerseits
ein Anzahl peripherer (Lungendehnung, $\mathrm{O}_2$-Gehalt des Blutes) und zentraler
Einfl"u"se (Wachheit), welche sie integrativ verarbeiten (\autor(Richter)). F"ur die
Gruppe der Neugeborenen, welche hier interessiert, werden im wesentlichen die folgenden
Atemrhythmen unterschieden, wobei diese Einteilung im wesentlichen ph"anonemologisch
erfolgt: Schlafassoziiert: REM \begriff(periodisches) Atmen, NON-REM
\begriff(regelm"a"siges) Atmen, aber auch \begriff(periodisches) Atmen.

\subsection{Stand der Forschung}

Mit erheblichem Aufwand wird versucht, die Bedingungen, welche zu Atemstillst"anden
f"uhren, zu identifizieren. Artikel von \autor(Poets) hier einarbeiten.  Nur wenige
Arbeiten gehen bisher der Frage nach dem dynamischen Charakter der Fr"uhgeborenenatmung
nach. \autor(Pilgrim \etal) geben f"ur das periodische Atmung zur"uckhaltend einen
deterministischen Charakter mit niedrigdimensionalem Attraktor an. Eine Aussage zum
dynamischen Charakter findet sich in tierexperimentellen Arbeiten von \autor(Sammon) und
\autor(Eldridge). Erstgenannter konnte f"ur den vagalen Einflu"s auf die Ruheatmung von
narkotisierten Ratten den Dimensionswechsel eines niedrigdimensionalen Attraktors
nachweisen. In einer sp"ateren theoretischen Arbeit formulierte \autor(Sammon) dann den
Atemzyklus als heteroklines, periodisches Orbit. \autor(Eldridge) konnte - ebenfalls an
einem Experiment mit Ratten - zeigen, da"s die experimentell gefundene Abh"angigkeit der
Reaktion auf eine St"orung des Atemzyklus von der Phasenlage durch Modellierung mittels
eines periodisch erregten \begriff(Van-der-Pol-Oszillators) genau nachgebildet werden
konnte.  Insgesamt fehlen bisher f"ur das menschliche Neugeborene deutliche Hinweise auf
einen deterministischen Charakter des Atemrhythmus. Von Interesse ist es deswegen, weil
ein deterministisches System auch mit nicht-statistischen Mitteln vorhergesagt werden
kann. Dar"uber hinaus spielt diese Frage auch in den Zusammenhang zwischen Atmung als
\begriff(conditio sine qua non) einerseits und den vielen Unterbrechungen des Atemrhythmus bei
S"augetieren (Schlucken, Innehalten, Sprechen etc.) andererseits ein Rolle.

\subsection{Signal Daten und Patienten}

F"ur die vorliegende Arbeit standen Aufzeichnungen des thorakalen Impedanzsignals von 8
Fr"uhgeborenen zur Verf"ugung. Diese entstammen der Routine"uberwachung. Die Messung der
thorakalen Impedanz registriert den atemabh"angig schwankenden Wechselstromwiderstand des
Brustkorbes anhand eines Me"sstroms, welcher "uber zwei Elektroden eingebracht wird (30 Khz,
3 uA). Sie bestimmt somit den Anteil des leitenden zum nichtleitenden Material im
Volumenleiter Thorax. Da nichtleitendes Atemgas periodisch ein und ausstr"omt, variiert
der gemessene Wechselstromwiderstand im wesentlichen mit der Atmung. Das Signal ist aber
wegen der Herzaktion und der sich aus dieser ergebenden "Anderung des leitenden Volumens im
Brustkorb auch an das Herzkreislaufsystem gekoppelt. Die Samplingrate betrug 100Hz. Die
Aufzeichnung erfolgte zusammen mit dem EKG "uber 24 Stunden. Nach Abschlu"s der Messung
wurden x Zeitreihen mit regelm"a"sigem, x Zeitreihen mit periodischem ausgew"ahlt. Apnoen.

\subsection{Ergebnisse}
Hallo \cite{Poets93}, \cite{Hoch96}. 


\begin{appendix}
\chapter{Medizinische Fachbegriffe} 
Die meisten der hier erl"auterten Fachbegriffe stammen aus ``Pschyrembel -- Klinisches
W"orterbuch'' \cite{Pschyrembel}. F"ur Fr"uhgeborene spezifische "Anderungen oder
Erweiterungen der Definitionenen sind aus \autor(Poets) (1993) \cite{Poets93} und
\autor(Hoch) und \autor(Bergmann) (1996) \cite{Hoch96} erg"anzt worden.

\begin{description}
\item[Apnoe:] Atemstillstand.
\item[Atmung, periodische:] Atmung mit Abwechselnd auftretenden mehreren tiefen Atemz"ugen
  und darauffolgender kurzer apnoischer Pause.
\item[Bradykardie:] langsame Schlagfolge des Herzens mit einer Pulsfrequenz unter 60/min,
  bei Fr"uhgeborenen schon ab unter 90-100/min.
\item[Epidemiologie:] Wissenschaftszweig, der sich mit der Verteilung von "ubertragbaren
  und nicht"ubertragbaren Krankheiten und deren physikalischen, chemischen, psychischen
  und sozialen Determinanten und Folgen in der Bev"olkerung befa"st.
\item[Gestationsalter:] Schwangerschaftsdauer, Reifezeichen des Neugeborenen.
\item[Hypox"amie:] niedriger Sauerstoffpartialdruck im arterielle Blut ($\mathrm{pO_2}<70$
  mmHg). Bei Neugeborenen ein Abfall auf unter 40 -- 45 mmHg bzw.\  unter 20\% des Basalwertes. 
\item[idiopathisch:] ohne erkennbare Ursache entstanden, Ursache nicht nachgewiesen.
\item[Konzeptionsalter:] Lebensalter mit Beginn der Empf"angnis.
\item[Neonatologie:] Teilgebiet der Kinderheilkunde, das sich mit Diagnose und Therapie von
  Erkrankungen des Neugeborenen befa"st.
\item[Pathophysiologie:] Lehre von den krankhaften Lebensvorg"angen im menschlichen
  Organismus.
\item[pathologisch:] krankhaft.
\item[Pulsoxymetrie:] transkutane (unblutige) Messung der arteriellen Sauerstoffs"attigung.
\item[thorakal:] zum Brustkorb geh"orig.
\item[Thorax:] Brustkorb.
\item[transkutan:] durch die Haut hindurch.
\end{description}
   
\end{appendix}

   
      


\addchap{Einleitung}

Komplexes Verhalten tritt in nahezu allen Bereichen der Natur in Erscheinung. W"ahrend die
Physik diesen Ph"anomenen fr"uher relativ ratlos gegen"uberstand, konnten in den letzten
Jahren durch die Erkenntnisse der Chaostheorie auf diesem Gebiet gro"se Fortschritte
erzielt werden. Die Theorie nichtlinearer dynamischer Systeme leistete hier Wesentliches,
indem sie die oftmals auftretenden komplexen Dynamiken mathematisch-physikalisch erfa"sbar
und charakterisierbar machte.  Hierbei ist besonders das Konzept \begriff(seltsamer Attraktoren)
hervorzuheben.  Seltsame Attraktoren bilden die Menge der im Langzeitverhalten
erreichbaren Phasenraumpunkte dissipativer chaotischer Systeme, und besitzen im allgemeinen
eine sehr verwickelte fraktale Struktur. Einher geht ein Verlust der Langzeitvorhersage
durch die Eigenschaft dieser Systeme, kleine St"orungen exponentiell zu verst"arken.  Die
Erkenntnis, da"s solche Strukturen auch bei deterministischen Systemen mit wenigen
Freiheitsgraden auftreten k"onnen, hat in vielen Gebieten der Naturwissenschaften, auch
au"serhalb der Physik, die Frage aufgeworfen, ob die beobachteten Verhaltensweisen und
Strukturen durch diese Konzepte zu erkl"aren seien.


In den empirischen Naturwissenschaften liegen h"aufig nur unzureichende Daten "uber den
genauen Zustand des beobachteten Systems vor, oftmals in Form \begriff(skalarer
Zeitreihen) einer gemessenen Zustandsvariable.  Eine entscheidende Voraussetzung f"ur die
Charakterisierung dieser Zeitreihen mittels der Methoden der nichtlinearen Dynamik ist
die M"oglichkeit, aus diesen wieder Attraktoren rekonstruieren zu k"onnen.  Dies geschieht
"uber das Konzept der sogenannten \begriff(Verz"ogerungskoordinaten), welches durch eine
mathematische Theorie abgesichert werden konnte. Die Vorraussetzungen der mathematischen
Theorie sind in der experimentellen Praxis jedoch nicht erf"ullbar. Neben der st"andigen
Anwesenheit von Rauschen ist vor allem die zeitliche Begrenztheit der zur Verf"ugung
stehenden Zeitreihen ein Problem. Aufgrund dieser Einschr"ankungen ist eine gute Wahl der
Einbettungsparameter vonn"oten.  Desweiteren m"ussen M"oglichkeiten zur Verminderung des
Rauschens gefunden werden. (Kapitel~1.2)

F"ur die Charakterisierung seltsamer Attraktoren steht eine Vielzahl verschiedener Ma"szahlen zur
Verf"ugung. Die drei am h"aufigsten herangezogenen sind die fraktale Dimension, die
Kolmogorov-Entropie und die Lyapunov-Exponenten. Diese beschreiben respektive die Anzahl
der effektiven Freiheitsgrade, die Rate der Informationsproduktion sowie die exponentielle
Separation der Trajektorien des Systems. Da die Komplexit"at eines Systems meistens mit
der Anzahl seiner Freiheitsgrade korreliert ist, wendet sich diese Arbeit besonders den
verschiedenen Begriffen fraktaler Dimension zu.  Im Bereich der nichtlinearen
Zeitreihenanalyse wird zumeist die Korrelationsdimension verwendet, die durch die schnelle
und stabile Berechenbarkeit durch den Grassberger-Procaccia-Algorithmus wesentliche
Bedeutung erlangt hat. Probleme stellen sich hierbei haupts"achlich durch die endliche
Datenmenge, wei"ses und Digitalisierungsrauschen und sogenannte Randeffekte.  (Kapitel~1.3)

Damit die Verwendung der beschriebenen Methoden "uberhaupt einen Sinn macht, ist es
notwendig zu wissen, da"s der beobachteten Zeitreihe ein deterministisches System zugrunde
liegt.  Zur L"osung des Problems existieren eine Vielzahl von Vorschl"agen, von denen hier
der Methode der Surrogatdaten auf Basis statistischer Hypothesentests der Vorzug gegeben
wird.  Diese bieten ein breites Fundament, um Fragen nach der Struktur und
zugrundeliegendem Determinismus experimenteller Zeitreihen beantworten zu k"onnen. Durch
die Verf"ugbarkeit hoher Rechenleistungen in modernen Computern lassen sich eine Vielzahl
verschiedener Hypothesen durch Vergleich der Originalzeitreihen mit sogenannten
Surrogatdaten testen (Monte-Carlo-Methoden). (Kapitel~1.4)

Die Ergebnisse dieser Arbeit sollen auf ein konkretes Beispiel angewandt werden. Das Ziel
hierbei ist zu kl"aren ob die vorgestellten Methoden aus der Zeitreihenanalyse bei der
Untersuchung der Atmungdynamik Fr"uhgeborener von Nutzen sein k"onnen.
Konkret geht es hierbei um folgende Fragen
\begin{itemize}
\item Ist die den verschiedenen Atmungstypen zugrundeliegende Dynamik durch ein
  deterministisches System beschreibbar ?
\item K"onnen f"ur die verschiedenen Atmungstypen, regelm"a"siges und periodisches Atmen, die
  fraktale Dimension berechnet, und  die Existenz eines seltsamen Attraktors belegt werden ?
\item Lassen sich signifikante "Anderungen der Dimension vor Auftreten von
  Atemstillst"an\-den (Apnoen) feststellen ?
\end{itemize}
Eine Bejahung dieser Fragen k"onnte zu einer m"oglichen Fr"uherkennung von
Atemstillst"anden f"uhren, was f"ur die Fr"uhgeborenenmedizin von gro"sem Nutzen w"are.
(Kapitel~2)






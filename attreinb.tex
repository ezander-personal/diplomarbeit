\clearpage
\chapter{Grundlagen der Zeitreihenanalyse}


%%%%%%%%%%%%%%%%%%%%%%%%%%%%%%%%%%%%%%%%%%%%%%%%%%%%%%%%%%%%%%%%%%%%%%%%%%%%%%%%%%%%%%%%%%%%%%%%%%%%
%% Dynamische Systeme
%%%%%%%%%%%%%%%%%%%%%%%%%%%%%%%%%%%%%%%%%%%%%%%%%%%%%%%%%%%%%%%%%%%%%%%%%%%%%%%%%%%%%%%%%%%%%%%%%%%%

\section{Seltsame Attraktoren}
\label{chapdynsystems}

Das Konzept seltsamer Attraktoren hat in den letzten beiden Jahrzehnten eine st"andig
wachsende Bedeutung bei der Erkl"arung von Ph"anomenen aus den unterschiedlichsten
Forschungsgebieten gefunden. Gemeinsam ist vielen dieser Ph"anomene, da"s die untersuchten
Systeme prinzipiell durch gew"ohnliche Differentialgleichungen der Form
\eqnl[evolut]{\dot\x = \vec F(\x)}
beschrieben werden k"onnen. Die Geschwindigkeit $\dot\x$, mit der sich der Zustand des
Systems entwickelt, und damit auch die ganze zuk"unftige Entwicklung\footnote{Unter
  hinl"anglich allgemeinen Forderungen an das Vektorfeld $\vec F$, n"amlich der
  Lipschitz-Stetigkeit.}, ist determiniert durch den aktuellen Zustand $\x$. Man sagt
auch, das Vektorfeld $\vec F$ erzeuge einen Flu"s $\flow^t$:
\eqn{\x(t) = \flow^t(\x_0) .}
Der Flu"s beschreibt die zeitliche Entwicklung des Systems in Abh"angigkeit vom
Anfangszustand $\x_0$.  Die Bewegung findet in einem Raum statt, der als
\begriff(Phasenraum) $\phase$ bezeichnet wird. Jedem Freiheitsgrad des Systems entspricht
im allgemeinen eine Koordinate des Phasenraumes\footnote{Bei Hamilton-Systemen wird die
  Anzahl der Freiheitsgrade $f$ als die Anzahl der konjugierten Orte und Impulse
  definiert. In diesem Fall ist die Dimension des Phasenraumes gleich $2f$.}. Die Anzahl
der Freiheitsgrade und damit die Dimension des Phasenraumes ist in vielen F"allen endlich,
kann jedoch bei \naja(ausgedehnten) Systemen\footnote{Gemeint sind hier beispielsweise
  hydrodynamische Systeme deren, Zustand durch Dichte- , Temperatur- und
  Geschwindigkeitsfelder beschrieben wird. Die hier auftretenden partiellen
  Differentialgleichungen k"onnen oftmals durch eine Modenentwicklung (z.B. nach
  \autor(Ritz-Galerkin)) auf eine \eqnref{evolut} entsprechende Form gebracht werden.}
auch abz"ahlbar oder "uberabz"ahlbar unendlich werden.

Bei realen Systemen ist es die Regel, da"s Energie durch Prozesse wie Reibung
verlorengeht. Bei solchen \begriff(dissipativen Systemen) kann man die Beobachtung
machen, da"s sich die Dynamik nach einer transienten Phase auf eine 
niedrigdimensionale Untermannigfaltigkeit $\M\subset\phase$ des Phasenraumes
reduziert. Die Energiedissipation eines Systems ist gleichbedeutend mit einer Kontraktion
des Phasenraumvolumens $V$ beliebiger Teilmengen $\set B\subset\phase$ unter dem Flu"s $\flow^t$:
\eqn{\abl{}{t}V(\flow^t(\set B))\lt 0.}
Dies kann auch "uber das erzeugende Vektorfeld $\vec F$ ausgedr"uckt werden. Die Divergenz
des Vektorfeldes mu"s in den erreichbaren Teilen des Phasenraumes negativ sein:
\eqn{\mathop{\mathrm{div}}\vec F \lt 0.}
Hierf"ur ist es nicht notwendig, da"s die Phasenraumvolumina in alle Richtungen gestaucht
werden. Es k"onnen auch bestimmte Richtungen expandieren, solange die Kontraktion in
die anderen Richtungen "uberwiegt.

Bei vielen Systemen klingt die transiente Phase der Bewegung recht schnell ab. In einem solchen
Fall interessiert man sich meist f"ur das \begriff(Langzeitverhalten) des Systems, und
beschr"ankt sich bei der Untersuchung auf die \begriff(Grenzmenge)\footnote{Die positive
  Grenzmenge $\limit^+$ eines Systems ist definiert als die Menge aller Punkte $\y\in\phase$, f"ur
  die Folgen $t_0\lt t_1\lt t_2\dots$ existieren, so da"s
  $\lim\limits_{i\to\infty}\flow^{t_i}(\x)=\y$ f"ur mindestens ein $\x\in\phase$.}
$\limit^+$, d.h.\ den Bereich des Phasenraumes, in dem sich die Dynamik im Grenzfall
unendlich langer Zeiten abspielt.

Lange herrschte die Auffassung, da"s sich die einzig m"oglichen Grenzmengen aus
Punkt\-attraktoren, Grenzzyklen oder attraktiven $k$-Tori ($k\geq2$) zusammensetzen.
Diesen ist gemeinsam, da"s sie periodisch (in den ersten beiden F"allen) bzw.\ periodisch
oder quasiperiodisch (im letzten Fall) sind. Sie besitzen ein diskretes
Fourier-Spektrum.  Die Entstehung von Turbulenz wurde verstanden durch
sukzessive Generation immer neuer Fourier-Moden.  Das komplexe Verhalten turbulenter
Systeme w"urde also als "Uberlagerung sehr vieler inkommensurabler Frequenzen betrachtet
und w"are somit qualitativ nicht verschieden vom Verhalten einfacher Systeme. Bei der
Untersuchung eines (stark vereinfachten) Modells der Konvektion von Luft in der
Atmosph"are stie"s \autor(Lorenz) 1963 \cite{Lorenz63} jedoch auf ein System, das
keinerlei Periodizit"at und somit ein kontinuierliches Fourier-Spektrum aufwies
(siehe~\psref{attrlor}). Diese \begriff(deterministischen nichtperiodischen Fl"usse)
erhielten sp"ater von \autor(Ruelle) und \autor(Takens) \cite{Ruelle71a} den
einpr"agsamen Namen \begriff(seltsame Attraktoren).


\epsfigdouble{attraktoren/lornew}{attraktoren/fournew}
{Links eine typische Trajektorie des Lorenz-Systems. \comment{ zu den 
Standardparameterwerten $\sigma =10$, $r=28$, $b=8/3$} Rechts das Leistungsspektrum $P(\omega)$ der 
$x$-Komponente.}{attrlor}{-0.2cm}

Eine mathematisch exakte Definition seltsamer Attraktoren ist ein schwieriges
Problem. Eine endg"ultige Fassung, auf die sich alle geeinigt h"atten, ist bis heute nicht
gelungen \cite{Pawelzik91}. Die Schwierigkeit liegt darin, die Definition sowohl
mathematisch exakt zu halten, als auch so umfassend, da"s alle wesentlichen Eigenschaften
erfa"st werden. Ein grundlegenderes Problem ist jedoch, da"s selbst "uber die
Eigenschaften seltsamer Attraktoren keine Einigkeit zwischen allen beteiligten Wissenschaftlern
besteht. Wir wollen uns hier deshalb auf einige f"ur diese 
Arbeit wichtige Eigenschaften beschr"anken:
\begin{itemize}
\item Seltsame Attraktoren sind \begriff(mischend).  Ein Attraktor $\attr$ hei"st genau
  dann mischend, wenn es f"ur beliebige, bez"uglich $\attr$ offene\footnote{Eine Menge
    $\A$ hei"st \begriff(offen bez"uglich) einer Menge $\M$, wenn $\A$ ganz in $\M$ liegt
    und es zu jedem $x\in\A$ eine reelle Zahl $\eps>0$ gibt, so da"s der Schnitt der
    $\eps$-Umgebung $\U_\eps(x)$ mit der Menge $\M$ ganz in $\A$ liegt: $\U_\eps(x) \cap
    \M \subset \A$.  }, nichtleere Mengen $\set I,\set J\subset\,\attr$ mindestens einen
  Punkt $\x\in\set I$ gibt, so da"s $\flow^t(\x)\in\set J$ f"ur mindestens ein $t>0$.

\comment{Eine Menge $\A$ hei"st \begriff(offen bez"uglich) einer Menge $\M$, wenn es eine
  offene Menge $\tilde\A$ gibt, so da"s $\A=\tilde\A\cap\M$. Der Begriff \naja(offen
  bez"uglich) stellt eine Verallgemeinerung der Vorstellung dar, da"s sich alle Punkte
  einer offenen Menge \naja(innerhalb) derselben befinden.}

Eine wichtige Folgerung daraus ist, da"s fast alle Punkte
des Attraktors unter dem Flu"s $\flow^t$ jedem anderen Attraktorpunkt beliebig nahe
kommen. Diese Eigenschaft ist notwendig f"ur die Definition ergodischer Ma"se
auf dem Attraktor. 

\item Zwei zu einem gegebenen Zeitpunkt $t_0$ eng beieinander liegende Punkte auf dem Attraktor
werden unter dem Flu"s $\flow^t$ mit der Zeit exponentiell voneinander getrennt. Man spricht von einer 
\begriff(exponentiellen Separation) der Trajektorien oder auch \begriff(sensibler
Abh"angigkeit von den Anfangsbedingungen). Dies findet Ausdruck in der Existenz positiver
Lyapunov-Exponenten.

\item \begriff(Periodische Orbits) liegen dicht auf dem Attraktor. Aufgrund der
  exponentiellen Separation sind diese Orbits jedoch alle instabil. Der Tr"ager eines
  seltsamen Attraktors ist darstellbar als Abschlu"s seiner instabilen periodischen Orbits
  (IPOs).
  
  Diese Tatsache kann im Rahmen der Zeitreihenanalyse zur effizienten Bestimmung von
  charakteristischen Gr"o"sen wie Lyapunov-Exponenten und fraktalen Dimensionen ausgenutzt
  werden. F"ur die Berechnung mu"s nicht die ganze zur Verf"ugung stehende Datenmenge
  herangezogen werden, da oftmals Mittelungen "uber die IPOs ausreichend sind.  Ein
  Verfahren zur Extraktion instabiler periodischer Orbits aus experimentellen Zeitreihen
  findet sich in \cite{Pawelzik91,Pawelzik91a}.
\end{itemize}
Die aufgez"ahlten Eigenschaften sind nicht voneinander unabh"angig. \autor(J. Banks \etal)
konnten zeigen, da"s die exponentielle Separation der Trajektorien aus den beiden anderen
Eigenschaften gefolgert werden kann \cite{Banks92}. Da die Eigenschaften des Durchmischens
als auch das Dichtliegen periodischer Orbits topologische Invarianten sind, gilt
dies somit abenfalls f"ur die exponentielle Separation der Trajektorien. Unter
topologischen Abbildungen bleiben also die dynamischen und geometrischen Eigenschaften
seltsamer Attraktoren unber"uhrt.

Oft werden seltsame Attraktoren auch "uber ihre fraktale Struktur definiert
\cite{Peitgen92}. Diesem Ansatz soll hier nicht gefolgt werden, da er nur geometrische,
jedoch keine dynamischen Aspekte des Attraktors beschreibt. Es existieren Attraktoren, die
seltsam (nach obiger Definition), aber nicht fraktal sind (beispielsweise der
\naja(Attraktor) von \begriff(Arnolds Katzenabbildung)\footnote{Arnolds Katzenabbildung
  ist eine Abbildung eines 2-Torus auf sich selbst \cite{Arnold68}. Die Dynamik ist
  chaotisch. Da das System volumenerhaltend ist, ist der ganze Torus der Attraktor
  \cite{Eckmann-ruelle}.  }). Demgegen"uber ist der
\begriff(Feigenbaum-Attraktor)\footnote{Die logistische Abbildung ist definiert durch
  $f_\mu(x)=\mu x(1-x)$ mit $\mu\in[0,4]$. Sie hat attraktive periodische Punkte der
  Periode $2^n$, wobei $n$ gegen unendlich l"auft f"ur $\mu$ gegen $\mu_\infty=3,57\dots$
  . Der f"ur $\mu=\mu_\infty$ entstehende Attraktor wird als Feigenbaum-Attraktor
  bezeichnet. Er besitzt eine fraktale Struktur, ist jedoch nicht chaotisch, da keine
  sensitive Abh"angigkeit von den Anfangsbedingungen besteht \cite{Eckmann-ruelle}.}
fraktal, aber nicht seltsam.  Die meisten Attraktoren (Lorenz-Attraktor,
R"ossler-Attraktor \dots) sind jedoch sowohl seltsam als auch fraktal.


\section{Rekonstruktionsverfahren}

Die Theorie seltsamer Attraktoren kann oftmals gute Erkl"arungsans"atze f"ur das
Ver\-st"andnis dynamischer Systeme liefern. So kann beispielsweise das Lorenz-System als
Modell f"ur bestimmte Konvektionszellen (sogenannte Rayleigh-B\'enard-Zellen) verwendet
werden\footnote{\autor(Lorenz) benutzte zur Aufstellung seiner Gleichungen das
  \autor(Rayleigh)sche Modell der Str"omungsdynamik in bestimmten rechteckigen
  Fl"ussigkeitszellen. Er betrachtete die Amplituden spezieller L"osungen der
  Modellgleichungen als zeitabh"angig und setzte diese wieder in das Modell ein. Durch
  Streichung bestimmter Terme gelangte er so zu einem System von Differentialgleichungen
  f"ur die zeitabh"angigen Amplituden \cite{Peitgen92}.}. Durch Variation eines Parameters kann der
"Ubergang dieser Zellen von laminarem zu turbulentem Verhalten studiert werden. Die vielen
N"aherungen, die f"ur dieses System gemacht werden, erlauben zwar keine direkten
Voraussagen "uber reale Konvektionszellen, die Dynamik kann jedoch prinzipiell verstanden
werden.

Anders liegt der Fall bei biologischen, medizinischen oder auch komplizierteren
hydrodynamischen Systemen. Hier tauchen eine Reihe von Problemen auf:
\begin{itemize}
\item Die relevanten Phasenraumvariablen sind oft nicht bekannt. Gerade bei
medizinischen und biologischen Systemen ist dies sehr oft der Fall.
\item Die Anzahl der Freiheitsgrade ausgedehnter Systeme (z.B. meteorologischer oder
hydrodynamischer) ist in der Regel abz"ahlbar oder gar "uberabz"ahlbar unendlich. 
\item Selbst bei Kenntnis aller Phasenraumvariablen sind die f"ur die Dynamik zust"andigen
Evolutionsgleichungen unbekannt.
\end{itemize}
Daten, die in experimentellen Situationen gewonnen werden, beschr"anken sich so meistens
auf Me"sreihen einer oder weniger Gr"o"sen, von denen angenommen wird, da"s  sie die
Dynamik des Systems charakterisieren. Stellt sich nun heraus, da"s die so gewonnene
Zeitreihe einen nicht trivialen\footnote{\begriff(Trivial) bedeutet in diesem
  Zusammenhang, da"s sich die Zeitreihe nicht als "Uberlagerung weniger Frequenzen
  darstellen l"a"st, d.h.\  sie hat kein diskretes
Fourier-Spektrum.} zeitlichen Verlauf aufweist, ergeben sich daraus einige Fragen:
\begin{itemize}
\item Liegt der Zeitreihe ein deterministisches System zugrunde, oder ist das erratische
Verhalten eine Folge additiven Rauschens?
\item L"a"st sich das System durch einen seltsamen Attraktor beschreiben? Wenn ja, wie
k"onnen wir diesen aus den vorhandenen Daten rekonstruieren?
\item Wie k"onnen f"ur den rekonstruierten Attraktor charakteristische Gr"o"sen wie
fraktale Dimensionen oder Lyapunov-Exponenten bestimmt werden?
\end{itemize}
Wir wollen uns zuerst mit der zweiten Frage besch"aftigen, wobei wir im folgenden annehmen
(zumindest als Arbeitshypothese), dem System liege ein seltsamer Attraktor zugrunde.
Die erste und die dritte Frage werden wir in sp"ateren Abschnitten angehen.

\subsection{Verz"ogerungskoordinaten (MOD)}

Den ersten Ansatz zur L"osung des Problems der Attraktorrekonstruktion  lieferten
\linebreak \autor(Packard \etal) 1980
mit dem Konzept der \begriff(Verz"ogerungskoordinaten), auch kurz als
MOD (engl.: method of delays) bezeichnet \cite{Packard80}. Um das Verfahren 
zu veranschaulichen, verwenden wir ein System dessen Dynamik uns bereits bekannt ist. An
diesem soll eine einzige Observable\footnote{D.h.\   eine beliebige glatte Funktion
$v:\phase\to\R$ der Phasenraumvariablen.} (numerisch) gemessen werden, die uns als Zeitreihe
dient. Hieraus soll die Dynamik wieder rekonstruiert und mit der urspr"unglichen Dynamik
verglichen werden.

%Als dynamisches System w"ahlen wir 
Das R"ossler-System \cite{Roessler76} ist durch das folgende System von
Differentialgleichungen bestimmt
\eqna{
\dot x &=& -z-x \nonumber \\
\dot y &=& x + a y \nonumber \\
\dot z &=& b + z(x-c) 
}
Das System ist in der dritten Gleichung nichtlinear und wird f"ur den Parametersatz
$a=0.38$, $b=0.3$, $c=4.5$ chaotisch. Der Attraktor besitzt eine fraktale
Struktur\footnote{Durch Analyse der Gleichungen kann man bei diesem System sehr sch"on den Chaos
  erzeugenden \metapher(Streck-und-Falt)-Proze"s erkennen \cite{Peitgen92}.  Die Struktur
  entspricht lokal dem kartesischen Produkt einer 2-dimensionalen Mannigfaltigkeit und einer
  Cantor-Menge.}.

Aus den Differentialgleichungen wird durch numerische Integration ein diskreter
Orbit $\folge(\x,1,N)$\footnotemark erzeugt (siehe \psref{rekroe} oben).  Als Observable $v$ dient die $x$-Komponente der
Punkte $\x$: $v(\x)=x$, so da"s wir eine Zeitreihe $v_i = v(\x(i\sample))$ erhalten. 
Die Zeitreihe wurde zur Verdeutlichung ihres diskreten Charakters durch Punkte
dargestellt (siehe \psref{rekroe} unten).  \footnotetext{Verwendet wird ein Runge-Kutta-Verfahren vierter Ordnung mit
  Schrittweite $\sample=0.009$. Nach einer Transienzzeit von $1000\sample$, nach der sich
  das System dem Attraktor hinreichend gen"ahert hat, wird der Orbit in diskreten
  Schritten $\sample$ aufgezeichnet.  Die Anzahl der berechneten Orbitpunkte betr"agt
  $N=20000$.}


\epsfigdouble{rekonstruktion/roenew}{rekonstruktion/timenew}
{Oben eine Trajektorie des R"ossler-Attraktors aus einer numerischen
Integration. Unten die gemessene Observable $v$.}{rekroe}{-0.2cm}

Die Informationen "uber die Werte der Koordinaten $y$ und $z$ stehen nun nicht mehr zur
Verf"u\-gung. Nach der Idee von \autor(Packard \etal) kann jedoch der Zustand eines
$n$-dimensio\-na\-len Systems zu einer gegebenen Zeit durch jeden Satz von $n$
unabh"angigen, sonst aber beliebigen Koordinaten spezifiziert werden \cite{Packard80}.
Der Zustand des R"ossler-Systems zu einer Zeit $t_i$ sollte also statt durch
$(x_i,y_i,z_i)$ ebenso durch $(x_i,\dot x_i,\ddot x_i)$ oder $(x_i,x_{i+1},x_{i+2})$
beschrieben werden k"onnen.

Wir wollen nun anhand einer einfachen "Uberlegung 
plausibel machen, da"s in den Verz"ogerungskoordinaten
tats"achlich die gleiche Information steckt wie in den originalen Phasenraumkoordinaten.
Wenn die Zeit $\sample$ zwischen zwei Messungen klein ist, gilt n"aherungsweise:
\eqnal[recidea1]{\dot x_i &\simeq& (x_{i+1} - x_i) / \sample \nonumber \\
\ddot x_i &\simeq& (x_{i+2} - 2x_{i+1} + x_i) / \sample^2 ;}
andererseits gilt f"ur die erste und zweite Zeitableitung
\eqnal[recidea2]{ 
\dot x_i &=& f_1(x_i, y_i, z_i) \nonumber \\
\ddot x_i &=& \pabl{F_1}{x} F_1(x_i,y_i,z_i) + \pabl{F_1}{y} F_2(x_i,y_i,z_i) +
\pabl{F_1}{z} F_3(x_i,y_i,z_i) ,} 
wobei die $F_j$ die Komponenten des erzeugenden Vektorfeldes sind (siehe \eqnref{evolut}).
Unter im betrachteten Zusammenhang recht allgemeinen Bedingungen an die $F_j$,
lassen sich $x_i$, $\dot x_i$ und $\ddot x_i$ wieder nach den Phasenraumvariablen $x_i$,
$y_i$ und $z_i$ aufl"osen.  Da andererseits nach \eqnref{recidea1} die Ableitungen durch die
Verz"ogerungskoordinaten bestimmt sind, k"onnen wir den kompletten Systemzustand
$x_i,y_i,z_i$ aus den verz"ogerten Werten der Zeitreihe $x_{i},x_{i+1},x_{i+2}$
\metapher(zur"uckholen). Wenn diese \naja("Aquivalenz) zwischen Original- und
Verz"ogerungskoordinaten gegeben ist, spricht man von einer \begriff(Einbettung) des
Attraktors (n"aheres dazu sp"ater), und bezeichnet den Raum, in den die Zeitreihe
eingebettet wird, (in diesem Fall der $\R^3$) als \begriff(Einbettungsraum).


Die vorstehenden "Uberlegungen k"onnen auch auf beliebige Observable $v$ sowie auf h"oherdimensionale
Einbettungsr"aume $\R^\embed$ erweitert werden. Weiterhin k"onnen gr"o"sere
Zeit\-ab\-st"ande $k\sample$ zwischen den Verz"ogerungskoordinaten gew"ahlt werden.  Man
erh"alt so als Rekonstruktionsvektoren die Folge $(v_{i},
v_{i+k},\dots,v_{i+(\embed-1)k})$. Dieses Verfahren entspricht einer
Abbildung $\diffeo_{k,\embed,v}:\phase\to\R^\embed$ aus dem Original- in den
Rekonstruktionsphasenraum, welche durch
\eqn{\diffeo_{k,\embed,v}(\x) = (v(\x),v(\flow^{k\sample}(\x)), \dots, v(\flow^{(\embed-1)k\sample}(\x)))}
gegeben ist. Man bezeichnet diese Abbildung  als \begriff(Verz"ogerungskoordinatenabbildung).

Wir wollen das Verfahren nun bei dem oben erw"ahnten R"ossler-System anwenden.  Da wir die
Anzahl der Freiheitsgrade des R"ossler-Systems kennen, lassen wir es bei der uns bekannten
und \naja(ausreichenden) Einbettungsdimension $\embed=3$. F"ur die Verz"ogerung\footnote{
  Die in Einheiten der \begriff(Sampling Time) $\sample$ gemessene
  \begriff(Verz"ogerungszeit) $\delay=k\sample$ wird als \begriff(Verz"ogerung) $k$
  bezeichnet. Da Verz"ogerung und Verz"ogerungszeit {\em immer} "uber diese Relation
  eindeutig verkn"upft sind ($\sample$ ist konstant), wird im folgenden je nach Kontext
  der besser geeignete der beiden Begriffe benutzt.} w"ahlen wir $k=30$.

%\afterpage
{\epsfigsingle{rekonstruktion/recnew}
{Rekonstruktion des R"ossler-Attraktors aus der Zeitreihe in \psref{rekroe} (unten) zur
  Einbettungsdimension $\embed=3$ mit Verz"ogerung $k=30$. Die Rekonstruktionspunkte
  wurden zur besseren Darstellung durch gerade Linien verbunden.}{rekrek}{-0.5cm}}

Die Rekonstruktion in \psref{rekrek} "ahnelt einer verzerrten Kopie des
Originalattraktors in \psref{rekroe}. Insofern leistet die
Verz"ogerungskoordinatenabbildung schon gute Dienste. Die rein visuelle "Ahnlichkeit
reicht aber f"ur eine genauere Analyse der Dynamik nicht aus. Es stellen sich mehrere
Fragen, die noch zu beantworten sind
\begin{myitemize}
\item Sind die Dynamik des rekonstruierten und des Originalattraktors zueinander
konjugiert? Mit anderen Worten: Ist die Verz"ogerungskoordinatenabbildung $\diffeo_{k,\embed,v}$
ein Diffeomorphismus?
\item Sind fraktale Dimensionen und Lyapunov-Exponenten unter der Rekonstruktion durch 
die Verz"ogerungskoordinatenabbildung invariant?
\item Wie sind die Einbettungsparameter $d$ und $k$ zu w"ahlen, wenn das urspr"ungliche
System nicht bekannt ist?
\end{myitemize}
Diese Fragen sollen in den n"achsten Abschnitten beantwortet werden.




%%%%%%%%%%%%%%%%%%%%%%%%%%%%%%%%%%%%%%%%%%%%%%%%%%%%%%%%%%%%%%%%%%%%%%%%%%%%%%%%%%%%%%%%%%%%%%%%%%%%
%% Einbettungen
%%%%%%%%%%%%%%%%%%%%%%%%%%%%%%%%%%%%%%%%%%%%%%%%%%%%%%%%%%%%%%%%%%%%%%%%%%%%%%%%%%%%%%%%%%%%%%%%%%%%

\subsubsection{Einbettungen}

Bevor wir uns mit den Einbettungstheoremen besch"aftigen, soll erst einmal der Begriff der
\begriff(Einbettung) selbst gekl"art werden. Bei einer Einbettung handelt es sich immer um
eine Abbildung (einer Menge) aus einem Phasenraum in einen anderen Phasenraum. Nun sollen unter
dieser Abbildung (und auch unter der Umkehrabbildung) keine Punkte kollabieren, d.h. es
sollen keine verschiedenen Originalpunkte auf den selben Bildpunkt abgebildet werden.
Solche Abbildungen, die die topologischen Eigenschaften von Punktmengen invariant lassen
bezeichnet man als \begriff(Hom"oomorphismen). Weiterhin sollen durch die Einbettung auch
keine Tangentenrichtungen kollabieren, was beispielsweise f"ur die Bestimmung von
Lyapunov-Exponenten von Bedeutung ist.  Zus"atzlich wird also die stetige
Differenzierbarkeit der Abbildung und der Umkehrabbildung gefordert. Eine
Einbettung mu"s somit ein $\sm^1$-Diffeomorphismus sein.




Wir wollen uns nun mit dem Problem besch"aftigen, unter
welchen Voraussetzungen die Verz"ogerungskoordinatenabbildung $\diffeo_{k,\embed,v}$ eine
Einbettung im obigen Sinne ist. Den ersten Ansatz zur Beantwortung dieser Frage lieferte
\autor(Takens) 1980 \cite{Takens80}.  \comment{Sein erstes Theorem soll hier vollst"andig zitiert
werden}
\comment{
\begin{theorem}
Sei $\M$ eine kompakte Mannigfaltigkeit der Dimension $\mandim$. F"ur Paare $(\flow,v)$,
wobei  \linebreak[4] $\flow:\M\to\M$ ein  glatter Diffeomorphismus und $v:\M\to\R$ eine
glatte Funktion ist, ist es eine generische Eigenschaft, da"s die 
Abbildung $\diffeo_{(\flow,v)} : \M \to\R^{2\mandim+1}$, definiert durch 
\eqnl[takmod]{\diffeo_{(\flow,v)}(\x) = (v(\x),v(\flow^1(\x)), \dots, v(\flow^{2\mandim}(\x))),}
eine Einbettung ist; \metapher(Glatt) bedeutet hier mindestens $\sm^2$.
Hierbei werden zus"atzlich folgende Voraussetzungen an den Flu"s $\flow$ gestellt:
\begin{myitemize}
\item Wenn $\x$ periodischer Punkt der Periode $k\le 2\mandim+1$ ist, sind alle Eigenwerte
von $\mathrm{D}\flow^k(\x)$ paarweise verschieden und verschieden von 1.
\item F"ur verschiedene Fixpunkte $\x^*$ von $\flow$, sind auch die $v(\x^*)$
verschieden\footnote{Der Satz gilt entsprechend f"ur durch $\sm^2$-Vektorfelder erzeugte Fl"usse,
wobei sich die beiden Voraussetzungen leicht auf das Vektorfeld "ubertragen lassen.}.
\end{myitemize}
\end{theorem}
}

\begin{theorem}
  Sei $\M$ eine kompakte Mannigfaltigkeit der Dimension $\mandim$. F"ur Paare $(\vec
  F,v,\delay)$, wobei $\vec F$ ein glattes Vektorfeld, $v:\M\to\R$ eine glatte Funktion
  und $\delay>0$ eine reelle Zahl ist\footnotemark, ist es eine generische Eigenschaft, da"s die
  Abbildung $\diffeo_{(\vec F,v,\delay)}: \M \to\R^{2\mandim+1}$, definiert durch
  \eqnl[takmod]{\diffeo_{(\vec F,v,\delay)}(\x) = (v(\x),v(\flow^\delay(\x)), \dots,
    v(\flow^{2\mandim\delay}(\x))),} eine Einbettung ist, wobei $\flow^t$ der durch $\vec
  F$ erzeugte Flu"s ist; \metapher(Glatt) bedeutet hier mindestens $\sm^2$.  "Uber das
  Vektorfeld $\vec F$ werden folgende zus"atzliche Annahmen gemacht:
\begin{myitemize}
\item Wenn $\vec F(\x)=0$ ist, dann sind alle Eigenwerte von $\mathrm{D}\flow^\delay(\x)$ paarweise verschieden und verschieden von 1.
\item Kein periodischer Orbit von $\vec F$ hat eine Periode $n\delay\, (n\in\N)$ mit $n\leq2\mandim+1$.
\end{myitemize}
\end{theorem}
\footnotetext{\autor(Takens) formulierte und bewies das Theorem f"ur die Verz"ogerungszeit 
  $\delay=1$. Da die Zeit jedoch immer entsprechend umskaliert werden kann,
  erschien es mir sinnvoll, das Theorem gleich f"ur beliebige Verz"ogerungszeiten
  $\delay>0$ zu formulieren.}

Die Abbildung $\diffeo_{(\vec F,v,\delay)}$ im obigen Theorem entspricht der
Verz"ogerungskoordinatenabbildung $\diffeo_{k,\embed,v}$.  Da wir "uber das Vektorfeld
$\vec F$ keine Kenntnis haben, k"onnen wir die zus"atzlichen Annahmen des Theorems nicht
verifizieren. Nach \autor(Takens) sind diese jedoch auch unter generischen Bedingungen
erf"ullt. 

Das Theorem versichert uns also, da"s f"ur generische Vektorfelder $\vec F$,
Me"sfunktionen $v$ und Verz"ogerungszeiten $\delay>0$, die
Verz"ogerungskoordinatenabbildung $\diffeo_{(\vec F,v,\delay)}$ eine Einbettung ist. Da im
Experiment nur in diskreten Zeitschritten $\sample$ gemessen werden kann, steht uns jedoch
nicht, wie vorausgesetzt, eine kontinuierliche Funktion $v(t)$, sondern nur die diskrete
Me"sreihe $v_i=v(i\sample)$ zur Verf"ugung. Wir m"ochten nun wissen, ob auch die
Grenzmenge der diskreten Folge konjugiert zu der des Originalsystems ist. Dies wird durch
ein Korollar zu Takens' viertem Theorem beantwortet.

\begin{corollar}
Sei $\M$ eine kompakte Mannigfaltigkeit der Dimension $\mandim$. Wir betrachten Viertupel,
bestehend aus einem Vektorfeld $\vec F$, einer Funktion $v$, einem Punkt $\x$ und einer
positiven reellen Zahl $\delay$. F"ur generische $(\vec F, v, \x, \delay)$ ist die
positive Grenzmenge $\limitp(\x)$ \naja(diffeomorph) zu der Grenzmenge der folgenden
Sequenz im $\R^{2m+1}$
\eqn{ \left\{ \left( v(\flow^{i\delay}(\x)),v(\flow^{(i+1)\delay}(\x)), \dots
,v(\flow^{(i+2m)\delay}(\x))    \right) \right\}^\infty_{i=0} }
\naja(Diffeomorph) hei"st hier: es gibt eine glatte Einbettung von $\M$ nach $\R^{2m+1}$,
die $\limitp(\x)$ bijektiv auf die Grenzmenge dieser Punktfolge abbildet.
\end{corollar}

Wir k"onnen nun schlie"sen, da"s f"ur generische Verz"ogerungen $k$ und Me"sfunktionen
$v$ die Verz"ogerungskoordinatenabbildung $\diffeo_{k,\embed,v}$ eine Einbettung des
Attraktors liefert, sofern nur $\embed\geq2m+1$ ist\footnote{Zur Bestimmung von $m$ siehe
  Abschnitt \ref{chapparams}~.}.
Der Begriff \naja(generisch) ist jedoch relativ schwach und sagt nichts "uber die
Wahrscheinlichkeit aus, da"s dies tats"achlich der Fall ist, wenngleich man dies gerne
meinen m"ochte. 

Dies soll genauer erl"autert werden.  Der Audruck \naja(generisch) beschreibt die H"aufigkeit
des Auftretens bestimmter Eigenschaften bei Elementen einer Menge $\A$.
Eine \label{generisch} Eigenschaft hei"st bereits dann generisch auf $\set A$, wenn
eine \begriff(residuale Teilmenge)\footnote{Eine Menge $\set R\subset\set A$ hei"st
\begriff(residuale) Teilmenge von $\set A$, wenn $\set R$ abz"ahlbarer Durchschnitt
offener, dichter Teilmengen von $\set A$ ist. Residuale Mengen sind selbst wieder dicht, jedoch nicht notwendig offen. }
 $\set R$ von $\set A$ existiert, so
da"s alle Elemente von $\set R$ diese Eigenschaft aufweisen \cite{Liebert91}.
Zu jedem Element aus $\set A$ findet man also in
jeder endlichen, beliebig kleinen Umgebung ein Element aus $\set R$, das diese
Eigenschaft aufweist. Dies sagt jedoch nichts "uber die
Wahrscheinlichkeit, ein Element dieser Menge zuf"allig zu treffen. Es existieren Beispiele,
in denen diese Wahrscheinlichkeit sogar null wird. So ist beispielsweise
die Menge $\Omega_{\text{stab}}$ der Parameterwerte $\omega\in[0,2\pi]$, f"ur die die
eindimensionale Kreisabbildung
\eqn{g_{\omega,k}(x)=x+\omega+k\sin(x) \nonumber}
stabile Orbits besitzt, eine residuale Teilmenge von $[0,2\pi]$. F"ur $k\to0$ verschwindet
das Lebesgue-Ma"s von $\Omega_{\text{stab}}$, die Wahrscheinlichkeit, ein
$\omega$ aus $[0,2\pi]$ zuf"allig so zu w"ahlen, da"s $g_{\omega,k}$ stabile Orbits besitzt (d.h.\
$\omega\in\Omega_{\text{stab}}$), geht demnach gegen null \cite{Sauer91}.


F"ur den Experimentator ist die Zusicherung aus Takens' Korrolar somit nicht ausreichend,
da nach den obigen Ausf"uh\-rungen, Generizit"at nichts "uber die Wahrscheinlichkeit, da"s
hier wirklich eine Einbettung vorliegt, aussagt. Wir m"ochten sicher sein, da"s die
Verz"ogerungskoordinatenabbildung mit der Wahrscheinlichkeit eins eine Einbettung ist.

Um auszudr"ucken, eine Eigenschaft treffe mit Wahrscheinlichkeit eins auf die Elemente
einer Menge $\set A$ zu, sagen wir, die Eigenschaft sei \begriff(pr"avalent) auf $\set A$. 
Da die Definition diese Begriffs auch f"ur "uberabz"ahlbar dimensionale Mengen sinnvoll
sein soll, kann er nicht "uber das verschwindende Lebesgue-Ma"s der Komplement"armenge
definiert werden, da ein solches hier nicht existiert. Die folgende Definition ist
entnommen aus \autor(Sauer \etal) \cite{Sauer91}:

\begin{definition}
Eine Borel-Teilmenge $\set A$ eines normierten Vektorraumes $\set V$ ist \begriff(pr"avalent), wenn
es einen endlich dimensionalen Untervektorraum $\set E$ aus $\set V$ gibt, so da"s f"ur alle $v$ aus
$\set V$ gilt, $v+e\in \set A$ f"ur fast alle $e$ aus $\set E$.
\end{definition}
Den Unterraum $\set E$ bezeichnet man als \begriff(Testraum) (engl.: probe space). Die Pr"avalenz
einer Eigenschaft  kann man sich nun folgenderma"sen vorstellen. Sei irgendein
Punkt $v$ aus $V$ vorgegeben, dann kann man von da aus in jede beliebige Richtung aus $E$
\naja(wandern) und trifft mit Wahrscheinlichkeit eins auf einen Punkt aus $S$. Da mit $E$
auch jeder Untervektorraum $E'$, der $E$ enth"alt, ein Testraum ist, ist leicht
einzusehen, da"s die Pr"avalenz einer Eigenschaft f"ur endlich dimensionale Vektorr"aume zu
der "ublichen Definition von \naja(f"ur fast alle) bzw.\ \naja(mit Wahrscheinlichkeit
eins) "aquivalent ist.


Aufgrund des oben beschriebenen Mankos von Takens' Theorem bewiesen \autor(Sauer),
\autor(Yorke) und \autor(Casdagli) ein erweitertes Einbettungstheorem \cite{Sauer91}, das
sogenannte \begriff(Fractal Delay Embedding Prevalence Theorem). Wir geben hier die (etwas 
verst"andlichere) Version aus ``Coping with chaos'' \cite{Ott94} wieder:

\begin{theorem}
  Ein kontinuierliches dynamisches System (gegeben durch ein Vektorfeld $\vec F$) besitze
  eine kompakte, invariante glatte Mannigfaltigkeit $\M$ der Kapazit"at $\fracdim$. 
  Sei $\embed>2\fracdim$ ganzzahlig und $\delay>0$ die Verz"ogerungszeit.  $\set M$ enthalte h"ochstens
  eine endliche Anzahl Fixpunkte, keine periodischen Orbits der Periode $\delay$ oder
  $2\delay$, und h"ochstens endlich viele periodische Orbits der Periode
  $3\delay,4\delay,\dots,\embed\delay$, und die Jakobimatrizen der Wiederkehrabbildungen dieser
  periodischen Orbits haben verschiedene Eigenwerte.  Dann gilt f"ur fast alle glatten
  Me"sfunktionen $v:\set M\to\R$, da"s die Verz"ogerungskoordinatenabbildung
  $\diffeo_{(\vec F,v,\delay)}:\set M\to\R^\embed$ (siehe~\eqnref{takmod}) eine Einbettung ist.
\end{theorem}


\begin{theorem}
  Sei $\flow^t$ ein (durch ein Vektorfeld $\vec F$ erzeugter) Flu"s auf einer offenen Teilmenge $\set U$ des $\R^k$ und sei $\set A$
  eine kompakte Teilmenge von $\set U$ der Kapazit"at $\fracdim$ \comment{(siehe Abschnitt
    \ref{chapcapacity})}. Sei $\embed>2\fracdim$ ganzzahlig und $\delay>0$. $\set A$ enthalte
  h"ochstens eine endliche Anzahl Fixpunkte, keine periodischen Orbits der Periode
  $\delay$ oder $2\delay$, und h"ochstens endlich viele periodische Orbits der Periode
  $3\delay,4\delay,\dots,\embed\delay$, und die Jakobimatrizen der Wiederkehrabbildungen der
  periodischen Orbits haben paarweise verschiedene Eigenwerte. Dann
  gilt f"ur fast alle glatten Me"sfunktionen $v:\set U\to\R$, da"s die
  Verz"ogerungskoordinatenabbildung $\diffeo_{(\vec F,v,\delay)}:\set U\to\R^\embed$
  (siehe~\eqnref{takmod}): \comment{ eine Einbettung von $\set A$ liefert.}
\begin{myitemize}
\item eineindeutig auf $\set A$ ist.
\item auf jeder kompakten Teilmenge $\set C$ einer in $\set A$ enthaltenen 
  glatten Mannigfaltigkeit eine Einbettung ist.
\end{myitemize}
\end{theorem}

Die wesentlichen Unterschiede zu Takens' Theorem sind erstens, da"s statt glatter
Mannigfaltigkeiten der Dimension $\mandim$ kompakte Mengen der Kapazit"at $\fracdim$
(welche erstere beinhalten)betrachtet werden, und zweitens, da"s generisch durch die
st"arkere Eigenschaft pr"avalent ersetzt wird.


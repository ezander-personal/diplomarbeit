\chapter{Rekonstruktion seltsamer Attraktoren}


%%%%%%%%%%%%%%%%%%%%%%%%%%%%%%%%%%%%%%%%%%%%%%%%%%%%%%%%%%%%%%%%%%%%%%%%%%%%%%%%%%%%%%%%%%%%%%%%%%%%
%% Dynamische Systeme
%%%%%%%%%%%%%%%%%%%%%%%%%%%%%%%%%%%%%%%%%%%%%%%%%%%%%%%%%%%%%%%%%%%%%%%%%%%%%%%%%%%%%%%%%%%%%%%%%%%%

\section{Dynamische Systeme}
\label{chapdynsystems}
Das Konzept \begriff(seltsamer Attraktoren) hat in den letzten Jahren Bedeutung bei der
Erkl"arung von Ph"anomenen aus den unterschiedlichsten Forschungsgebieten gefunden. Gemeinsam
ist allen, da"s die untersuchten Systeme prinzipiell durch sogenannte
\begriff(Evolutionsgleichungen)
\eqn{\dot\x = \vec F(\x)}
beschrieben werden k"onnen. Die Richtung $\dot\x$, in die sich der Zustand des Systems entwickelt, und damit
auch die ganze zuk"unftige Entwicklung\footnote{Unter hinl"anglich allgemeinen Forderungen
an das Vektorfeld $\vec F$, n"amlich der Lipschitz-Stetigkeit.}, ist damit determiniert durch
den aktuellen Zustand $\x$. Man sagt auch, das Vektorfeld $\vec F$ erzeuge einen Flu"s
$\flow$.
\eqn{\x(t) = \flow^t(\x_0) }
Der Flu"s beschreibt die zeitliche Entwicklung des Systems in Abh"angigkeit des
Anfangszustandes $\x_0$.
Die Bewegung findet in einem Raum statt, der als \begriff(Phasenraum) $\phase$
bezeichnet wird. Jedem  Freiheitsgrad des Systems entspricht genau eine Koordinate des
Phasenraums. Die Anzahl der Freiheitsgrade und damit die Dimension des Phasenraums ist in
vielen F"allen endlich, kann jedoch bei ausgedehnten Systemen auch "uberabz"ahlbar
unendlich werden\footnote{Die ist auch bei eindimensionalen Systemen der Fall, die durch
sogenannte \begriff(Delay-Gleichungen) beschrieben werden, wie beispielsweise die
Mackey-Glass-Gleichung. }.


Bei nat"urlichen Systemen ist es die Regel, da"s Energie durch Prozesse wie Reibung
verloren geht. Bei solchen \begriff(dissipativen Systemen) kann man die Beobachtung
machen, da"s sich die Dynamik nach einer kurzen transienten Phase nur noch auf eine 
niedrigdimensionale \begriff(Untermannigfaltigkeit) $\M\subset\phase$ des Phasenraums
reduziert. Die Energiedissipation eines Systems ist gleichbedeutend mit einer Kontraktion
von Phasenraumvolumina unter dem Flu"s $\flow$.
\eqn{\abl{}{t}\flow^t(V)\lt 0}

Dies kann auch "uber das erzeugende Vektorfeld $\vec F$ ausgedr"uckt werden. Die Divergenz
des Vektorfeldes mu"s in den erreichbaren Teilen des Phasenraums negativ sein.
\eqn{\mathop{\mathrm{div}}\vec F \lt 0}
Hierf"ur ist es nicht notwendig, da"s die Phasenraumvolumina in alle Richtungen gestaucht
werden. Es k"onnen auch bestimmte Richtungen expandiert werden, solange die Kontraktion in
die anderen Richtungen "uberwiegt.

Da die transiente Phase der Bewegung im allgemeinen recht schnell abklingt, interessiert
man sich meist f"ur das \begriff(Langzeitverhalten) dieser Systeme. Man beschr"ankt sich bei
der Untersuchung auf die \begriff(Grenzmengen)\footnote{Die positive Grenzmenge $\limit^+$
eines Systems ist definiert als die Menge aller Punkte $\y$, f"ur die  
Folgen $t_0\lt t_1\lt t_2\dots$ existieren, so da"s $\lim\limits_{i\to\infty}\flow^{t_i}(\x)=\y$
f"ur mindestens ein $\x\in\phase$.} $\limit^+$, d.h. den Bereich des
Phasenraums, in dem sich die Dynamik im Grenzfall unendlich langer Zeiten abspielt. 

\subsection{Seltsame Attraktoren}
Lange herrschte die Auffassung, da"s sich die einzig m"oglichen Grenzmengen aus
\begriff(Punkt\-attraktoren), \begriff(Grenzzyklen) oder \begriff($k$-Tori)
zusammensetzen. Diesen ist gemeinsam, da"s sie periodisch (in den ersten beiden F"allen)
bzw.\ quasiperiodisch (im letzten Fall) sind. Sie besitzen ein diskretes \begriff(Fourierspektrum).
Die Entstehung von Turbulenz wurde durch das hinzukommen immer neuer
Fouriermoden verstanden. Das komplexe Verhalten turbulenter Systeme also als "Uberlagerung
sehr vieler inkommensurabler Frequenzen betrachtet, und somit qualitativ nicht verschieden vom
Verhalten einfacher Systeme. Bei der Untersuchung von Konvektionszellen stie"s
\autor(Lorenz) 1963 \cite{Lorenz63} jedoch auf ein System, das keinerlei Periodizit"at
und somit ein kontinuierliches Fourierspektrum aufwies (Siehe \psref{attrlor}). Diese 
\begriff(deterministischen, nichtperiodischen Fl"usse) erhielten sp"ater von \autor(Ruelle)
und \autor(Takens) \cite{Ruelle71a} den einpr"agsameren Namen \begriff(seltsame Attraktoren). 


\epsfigdouble{attraktoren/lornew}{attraktoren/fournew}
{Links eine typische Trajektorie des Lorenzsystems zu den 
Standardparameterwerten $\sigma =10$, $r=28$, $b=8/3$. Rechts das Leistungsspektrum $P(\omega)$ der 
$x$-Komponente.}{attrlor}{-0.2cm}

Eine mathematisch exakte Definition seltsamer Attraktoren ist ein schwieriges
Problem. Eine endg"ultige Fassung, auf die sich alle geeinigt h"atten, ist bis heute nicht
gelungen. Die Schwierigkeit liegt darin, die Definition sowohl
mathematisch exakt zu halten, als auch so umfassend, da"s alle wesentlichen Eigenschaften
erfa"st werden. Ein grundlegenderes Problem ist jedoch, da"s selbst "uber die
Eigenschaften seltsamer Attraktoren keine Einigkeit zwischen allen beteiligten Wissenschaftlern
besteht. Wir wollen uns hier deshalb auf einige f"ur diese 
Arbeit wichtige Eigenschaften beschr"anken. 
\begin{itemize}

\item Seltsame Attraktoren sind \begriff(durchmischend). 
Ein Attraktor $\attr$  hei"st genau dann durchmischend, wenn es f"ur beliebige,
bez"uglich $\attr$ offene\footnote{Eine Menge $\A$ hei"st 
\begriff(offen bez"uglich $\M$), wenn es offene Mengen $\tilde\A$ gibt, so
da"s $\A=\tilde\A\cap\M$. Der Begriff \naja(offen bez"uglich) stellt eine
Verallgemeinerung der Vorstellung dar, da"s sich alle Punkte einer offenen Menge
\naja(innerhalb) dieser befinden.}, nichtleere  Mengen $\set I,\set J\subset\,\attr$
mindestens einen Punkt $\x\in\set I$ gibt, so da"s $\flow^t(\x)\in\set J$ f"ur mindestens
ein $t>0$.   

Eine wichtige Folgerung daraus ist, da"s fast alle Punkte
des Attraktors unter dem Flu"s $\flow$ jedem anderen Attraktorpunkt beliebig nahe
kommen. Diese Eigenschaft ist notwendig f"ur die Definition \begriff(ergodischer Ma"se)
auf dem Attraktor. 

\item Zwei zu einem gegebenen Zeitpunkt $t_0$ eng beieinander liegende Punkte auf dem Attraktor
werden unter dem Flu"s $\flow$ mit der Zeit exponentiell voneinander getrennt. Man spricht von einer 
\begriff(exponentiellen Separation) der Trajektorien oder auch \begriff(sensibler
Abh"angigkeit von den Anfangsbedingungen). Dies findet Ausdruck in der Existenz positiver
\begriff(Lyapunov-Exponenten).

\item \begriff(Periodische Orbits) liegen dicht auf dem Attraktor. Aufgrund der
exponentiellen Separation sind diese Orbits jedoch alle instabil. Der Attraktor ist
darstellbar als "Uberdeckung seiner instabilen periodischen Orbits (IPOs). Diese Tatsache
kann zur beschleunigten Bestimmung von Ma"szahlen wie Lyapunov-Exponenten und
Dimensionen ausgenutzt werden. Es wird nicht die ganze zur Verf"ugung stehende Datenmenge
zur Berechnung herangezogen, sondern nur "uber die IPOs gemittelt \cite{Pawelzik91,Pawelzik91a}.
\end{itemize}
Die aufgez"ahlten Eigenschaften sind nicht voneinander unabh"angig. Es konnte gezeigt
werden \cite{Banks92}, da"s die exponentielle Separation der Trajektorien aus den beiden anderen
Eigenschaften gefolgert werden kann. Da die Eigenschaften des Durchmischens als auch
das Dichtliegen periodischer Orbits topologische Invarianten sind, folgt dies auch f"ur
die exponentielle Separation der Trajektorien. Unter topologischen Abbildungen bleiben also die dynamischen 
und geometrischen Eigenschaften seltsamer Attraktoren unber"uhrt.

Oft werden seltsame Attraktoren auch "uber ihre \begriff(fraktale) Struktur
definiert \cite{Peitgen92}. Diesem Ansatz soll hier nicht gefolgt werden, da er nur
geometrische, jedoch keine dynamischen Aspekte des Attraktors beschreibt
\cite{eckmann-ruelle}. Es existieren Attraktoren, die seltsam\footnote{Nach obiger
Definition.} aber nicht fraktal sind (beispielsweise \begriff(Arnolds Katzenabbildung)
\cite{Arnold68}). Demgegen"uber ist der \begriff(Feigenbaumattraktor) fraktal aber nicht
seltsam. Die meisten Attraktoren (Lorenzattraktor, R"osslerattraktor \dots) sind jedoch
sowohl seltsam als auch fraktal.


\section{Rekonstruktionsverfahren}

Die Theorie seltsamer Attraktoren kann oftmals gute Erkl"arungsans"atze f"ur das Ver\-st"andnis
dynamischer Systeme liefern. So kann beispielsweise beim Lorenzsystem durch
Variation des Parameters $R$\korrektur(nachschlagen) der "Ubergang einer Konvektionszelle
von laminarem zu turbulentem Verhalten studiert werden. Die vielen N"aherungen, die f"ur
dieses System gemacht werden, erlauben zwar keine direkten Voraussagen "uber reale 
Konvektionszellen, die Dynamik kann jedoch prinzipiell verstanden werden.

Anders liegt der Fall bei biologischen, medizinischen oder auch komplizierteren
hydrodynamischen Systemen. Hier tauchen eine Reihe von Problemen auf.
\begin{itemize}
\item Die relevanten Phasenraumvariablen sind oft nicht bekannt. Gerade bei
medizinischen und biologischen Systemen ist dies sehr oft der Fall.
\item Die Anzahl der Freiheitsgrade ausgedehnter Systeme (z.B. meteorologischer oder
hydrodynamischer) ist in der Regel abz"ahlbar oder gar "uberabz"ahlbar unendlich. 
\item Selbst bei Kenntnis aller Phasenraumvariablen sind die f"ur die Dynamik zust"andigen
Evolutionsgleichungen unbekannt.
\end{itemize}
Daten, die in experimentellen Situationen gewonnen werden, beschr"anken sich so meistens
auf Me"sreihen einer oder weniger Gr"o"sen, von denen angenommen wird, da"s  sie die
Dynamik des Systems charakterisieren. Stellt sich nun heraus, da"s die so gewonnene
\begriff(Zeitreihe) einen nicht trivialen\footnote{Die Zeitreihe l"a"st sich nicht als
"Uberlagerung verschiedener Frequenzen darstellen, d.h.\ sie hat kein diskretes
Fourierspektrum.} zeitlichen Verlauf aufweist, ergeben sich daraus einige Fragen:
\begin{itemize}
\item Liegt der Zeitreihe ein deterministisches System zugrunde oder ist das erratische
Verhalten eine Folge additiven Rauschens?
\item L"a"st sich das System durch einen seltsamen Attraktor beschreiben? Wenn ja, wie
k"onnen wir diesen aus den vorhandenen Daten rekonstruieren?
\item Wie k"onnen f"ur den rekonstruierten Attraktor charakteristische Gr"o"sen wie
fraktale Dimension oder Lyapunov-Exponenten bestimmt werden?
\end{itemize}
Wir wollen uns zuerst mit der zweiten Frage besch"aftigen, wobei wir im folgenden annehmen
(zumindest als Arbeitshypothese), dem System l"age ein seltsamer Attraktor zugrunde.
Die erste und dritte Frage werden wir in sp"ateren Abschnitten angehen.

\subsection{Verz"ogerungskoordinaten (MOD)}

Den ersten Ansatz zur L"osung des Problems der Attraktorrekonstruktion  lieferten
\linebreak \autor(Packard \etal) 1980
mit dem Konzept der \begriff(Verz"ogerungskoordinaten), teilweise auch kurz als
MOD\footnote{Abk"urzung f"ur \begriff(Method of Delays).} bezeichnet \cite{packard80}. Um das Verfahren 
zu veranschaulichen, verwenden wir ein System dessen Dynamik uns bereits bekannt ist. An
diesem soll eine einzige Observable\footnote{D.h. eine beliebige, glatte Funktion
$v:\phase\to\R$ der Phasenraumvariablen.} gemessen werden, die uns als Zeitreihe
dient. Hieraus soll dann die Dynamik wieder rekonstruiert und mit der urspr"unglichen Dynamik
verglichen werden.

Als dynamisches System w"ahlen wir das R"osslersystem \cite{Roessler76}, das durch den folgenden Satz von
Differentialgleichungen bestimmt ist
\eqna{
\dot x &=& -z-x \nonumber \\
\dot y &=& x + a y \nonumber \\
\dot z &=& b + z(x-c) 
}
Das System ist f"ur den Parametersatz $a=0.38$, $b=0.3$, $c=4.5$ chaotisch. Der Attraktor besitzt
eine fraktale Struktur\footnote{Durch Analyse der Gleichungen \cite{Peitgen92} erkennt man
hier sehr sch"on den Chaos erzeugenden \metapher(Streck-und-Falt)-Proze"s. Die Struktur
entspricht lokal dem Produkt einer 2-dimensionalen Mannigfaltigkeit und einer Cantormenge.}.

Aus den Differentialgleichungen erzeugen wir durch numerische
Integration\footnote{Verwendet wird ein Runge-Kutta-Verfahren vierter Ordnung mit
Schrittweite $\sample=0.009$. Nach einer Transienzzeit von $1000\sample$, nach der sich das
System dem Attraktor hinreichend gen"ahert hat, wird der Orbit in
diskreten Schritten $\sample$ aufgezeichnet.  Dargestellt sind die, durch Linien
verbundenen, Orbitpunkte $\folge(\x,1,N)$ ($N=20.000$).} einen diskreten Orbit $\folge(\x,1,N)$. 
Als Observable dient die $x$-Komponente der Punkte $v(\x)=x$.


\epsfigdouble{rekonstruktion/roenew}{rekonstruktion/timenew}
{Links der R"osslerattraktor aus einer numerischen
Integration. Rechts das gemessene Signal $v_i=x(i\sample)$ zur Verdeutlichung des
diskreten Charakters der Zeitreihe durch Punkte dargestellt.}{rekroe}{-0.2cm}

Die Informationen "uber die Werte der Koordinaten $y$ und $z$ stehen uns nun nicht mehr zur
Verf"u\-gung. Nach der Idee von \autor(Packard \etal) kann jedoch der Zustand 
eines $n$-dimensio\-na\-len Systems zu einer gegebenen Zeit durch $n$ unabh"angige, sonst aber
beliebige,  Koordinaten 
spezifiziert werden \cite{packard80}. Der Zustand des R"osslersystems zu einer Zeit $t_i$ sollte
also statt durch $(x_i,y_i,z_i)$ ebenso durch $(x_i,\dot x_i,\ddot x_i)$ oder $(x_i,x_{i+1},x_{i+2})$
beschrieben werden k"onnen.

Wir wollen nun anhand einer "Uberlegung von \autor(Theiler), die wir hier auf drei
Koordinaten ausgedehnt haben, plausibel machen, da"s in den Verz"ogerungskoordinaten
tats"achlich die gleiche Information steckt, wie in den originalen Phasenraumkoordinaten.
Wenn die Zeit zwischen zwei Messungen $\sample$ klein ist, gilt 
\eqnal[recidea1]{\dot x_i &\simeq& (x_{i+1} - x_i) / \sample \nonumber \\
\ddot x_i &\simeq& (x_{i+2} - 2x_{i+1} + x_i) / \sample^2}
Nun gilt jedoch f"ur die nullte bis zweite Zeitableitung
\eqnal[recidea2]{ x_i &=& x_i \nonumber \\
\dot x_i &=& f_1(x_i, y_i, z_i) \nonumber \\
\ddot x_i &=& \pabl{f_1}{x} f_1(x_i,y_i,z_i) + \pabl{f_1}{y} f_2(x_i,y_i,z_i) +
\pabl{f_1}{z} f_3(x_i,y_i,z_i) } 
wobei wir die Koordinatenfunktionen mit $f_1,f_2,f_3$ bezeichnet haben. Unter -- im
betrachteten Zusammenhang -- recht allgemeinen Bedingungen\footnote{Die
Funktionaldeterminante der rechten Seite von \eqnref{recidea2} darf nicht verschwinden. Da
in chaotischen Systemen, die Dynamik nicht separierbar ist, kann man diese Annahme hier
(bis auf weiteres) akzeptieren.} an die $f_j$ l"a"st sich \eqnref{recidea2} wieder
nach den Phasenraumvariablen $x_i$, $y_i$ und $z_i$ aufl"osen. Das hei"st jedoch mit
\eqnref{recidea1}, da"s wir den kompletten Systemzustand $x_i,y_i,z_i$ aus den
\begriff(verz"ogerten) Werten der Zeitreihe $x_{i},x_{i+1},x_{i+2}$
\metapher(zur"uckholen) k"onnen. Man sagt, der Systemzustand sei in der Zeitreihe
\metapher(konserviert).Diese  "Uberlegungen k"onnen auch auf h"oherdimensionale
\begriff(Einbettungsr"aume) $\R^\embed$  erweitert werden. Au"serdem k"onnen ebenso
gr"o"sere Zeitabst"ande $k\sample$ zwischen den Verz"ogerungskoordinaten genommen werden,
so da"s wir als Rekonstruktionsvektoren die Folge $(x_{i},
x_{i+k},\dots,\x_{i+(\embed-1)k})$ erhalten. Diese Abbildung aus dem Original- in 
den Rekonstruktionsphasenraum bezeichnet man als Verz"ogerungskoordinatenabblidung $\diffeo_{k,d}$.

Wir wollen sie nun bei dem oben erw"ahnten R"osslersystem anwenden.
Da wir die Anzahl der Freiheitsgrade des R"osslersystems kennen, lassen wir es bei der uns
bekannten und \naja(ausreichenden) Einbettungsdimension $\embed=3$. F"ur die Verz"ogerung\footnote{Um Mi"sverst"andnisse
zu vermeiden, wird im folgenden die in Einheiten der \begriff(Samplingtime) $\sample$ gemessene 
\begriff(Verz"ogerungszeit) $\delay=k\sample$ durchg"angig als \begriff(Verz"ogerung) $k$
bezeichnet.} w"ahlen wir $k=30$. 

%\afterpage
{\epsfigsingle{rekonstruktion/recnew}
{Rekonstruktion des R"osslerattraktors aus der Zeitreihe in \psref{rekroe} (rechts) mit 
Verz"ogerung $k=30$.}{rekrek}{-0.5cm}}

Die Rekonstruktion in \psref{rekrek} "ahnelt einer verzerrten Kopie des
Originalattraktors in \psref{rekroe}. Insofern leistet die
Verz"ogerungskoordinatenabbildung schon gute Dienste. Die rein optische "Ahnlichkeit
reicht aber nicht aus. Mehrere Fragen bleiben offen
\begin{myitemize}
\item Sind die Dynamik des rekonstruierten und des Originalattraktors zueinander
konjugiert? In anderen Worten, ist die Verz"ogerungskoordinatenabbildung $\diffeo_{k,d}$
ein Diffeomorphismus?
\item Bleiben Dimensionen und Lyapunov-Exponenten unter der Rekonstruktion durch 
die Verz"ogerungskoordinatenabbildung invariant?
\item Wie sind die Einbettungsparameter $d$ und $k$ zu w"ahlen, wenn das urspr"ungliche
System nicht bekannt ist?
\end{myitemize}
Diese Fragen sollen in den n"achsten Abschnitten beantwortet werden.




%%%%%%%%%%%%%%%%%%%%%%%%%%%%%%%%%%%%%%%%%%%%%%%%%%%%%%%%%%%%%%%%%%%%%%%%%%%%%%%%%%%%%%%%%%%%%%%%%%%%
%% Einbettungen
%%%%%%%%%%%%%%%%%%%%%%%%%%%%%%%%%%%%%%%%%%%%%%%%%%%%%%%%%%%%%%%%%%%%%%%%%%%%%%%%%%%%%%%%%%%%%%%%%%%%

\subsubsection{Einbettungen}

Bevor wir uns mit den Einbettungstheoremen besch"aftigen, soll erst einmal der Begriff der
\begriff(Einbettung) gekl"art werden. Eine Einbettung ist erstmal ein
\begriff(Hom"oomorphismus) $\diffeo$, d.h. eine bijektive, stetige Abbildung mit stetiger
Umkehrfunktion. Hom"oomorphismen werden auch als \begriff(topologische) Abbildungen
bezeichnet, da sie die topologischen Eigenschaften von Punktmengen invariant lassen.
Unter topologischen Abbildungen werden keine Punkte kollabiert, d.h.\  es werden keine
verschiedenen Originalpunkte auf den gleichen Bildpunkt abgebildet. Nun sollen bei einer
Einbettung aber auch keine Tangentenrichtungen kollabiert werden, was beispielsweise f"ur die
Bestimmung von Lyapunov-Exponenten wichtig ist. Wir fordern deshalb
zus"atzlich die stetige Differenzierbarkeit von $\diffeo$ und $\diffeo^{-1}$,
d.h.\ die Einbettung $\diffeo$ soll ein $\sm^1$-Diffeomorphismus sein.
\comment{
Da bei der Phasenraumrekonstruktion nicht nur die topologischen Eigenschaften des
Attraktors erhalten bleiben sollen, sondern auch die Stetigkeit der ersten und zweiten
Ableitung an Trajektorien (d.h. Richtung und Kr"ummung), fordern wir zus"atzlich die
zweifache stetige Differenzierbarkeit, d.h. $\diffeo$ soll ein $\sm^2$-Diffeomorhismus sein.
\inkorrektur(nicht ganz richtig, auf immersions eingehen)
}

Ein weiterer Begriff der im Rahmen der Einbettungstheoreme "ofters auftritt 
ist \naja(generisch). Dieser Begriff beschreibt das Auftreten bestimmter Eigenschaften bei 
Elementen aus einer Menge $\A$. 
 \naja(Generisch) wird oft in der Weise interpretiert, da"s eine bestimmte Eigenschaft auf
der Menge $\A$ mit der Wahrscheinlichkeit eins 
auftr"ate. Der Begriff ist jedoch schw"acher und sagt nichts "uber diese
Wahrscheinlichkeit aus. 

Eine \label{generisch} Eigenschaft hei"st schon dann generisch auf $\set A$, wenn
eine \begriff(residuale Teilmenge)\footnote{Eine Menge $\set R\subset\set A$ hei"st
\begriff(residuale) Teilmenge von $\set A$, wenn $\set R$ abz"ahlbarer Durchschnitt
offener, dichter Teilmengen von $\set A$ ist. Residuale Mengen sind selbst wieder dicht, jedoch nicht notwendig offen. }
 $\set R$ von $\set A$ existiert, so
da"s alle Elemente von $\set R$ diese Eigenschaft aufweisen \cite{Liebert91}.
Zu jedem Element aus $\set A$ findet man also in
jeder endlichen, beliebig kleinen Umgebung ein Element aus $\set R$, da"s diese
Eigenschaft aufweist. Dies sagt jedoch nichts "uber die
Wahrscheinlichkeit ein Element dieser Menge zuf"allig zu treffen. Es existieren Beispiele
f"ur diese Wahrscheinlichkeit sogar Null wird. So ist beispielsweise
die Menge $\Omega_{\text{stab}}$ der Parameterwerte $\omega\in[0,2\pi]$, f"ur die die Abbildung 
\eqn{g_{\omega,k}=x+\omega+k\sin(x) \nonumber}
stabile Orbits besitzt, eine residuale Teilmenge von $[0,2\pi]$. F"ur $k\to0$ verschwindet
jedoch das Lebesgue-Ma"s von $\Omega_{\text{stab}}$, die Wahrscheinlichkeit einen Punkt
dieser Menge zu treffen geht  gegen Null \cite{sauer91}.

Um auszudr"ucken, eine Eigenschaft treffe mit Wahrscheinlichkeit eins auf die Elemente
einer Menge $\set A$ zu, sagen wir, die Eigenschaft sei \begriff(pr"avalent) auf $\set
A$. Da der Begriff der Pr"avalenz auch f"ur sehr allgemeine R"aume sinnvoll sein soll,
kann er nicht "uber das Lebesgue-Ma"s definiert werden\footnote{Die in endlich dimensionalen R"aumen
g"angige Definiton "uber das verschwindende Lebesgue-Ma"s der Komplement"armenge ist hier
nicht anwendbar, da wir es i.allg. mit unendlich dimensionalen Funktionenr"aumen zu tun
haben.}. Auf die genaue Definition soll hier jedoch verzichtet werden \cite{sauer91}.

\paragraph{Einbettungstheoreme}
Nach dieser kurzen Einleitung 
 wollen wir uns nun mit dem Problem besch"aftigen, unter welchen Vorraussetzungen die
Verz"ogerungskoordinatenabbildung $\diffeo_{k,d}$ eine Einbettung in obigem Sinne ist. Den 
ersten Ansatz zur Beantwortung dieser Frage lieferte \autor(Takens) 1980 \cite{takens80}.
Sein erstes Theorem soll hier vollst"andig zitiert werden

\begin{theorem}[Takens' Theorem 1]
Sei $\M$ eine kompakte Mannigfaltigkeit der Dimension $\mandim$. F"ur Paare $(\flow,v)$,
wobei  \linebreak[4] $\flow:\M\to\M$ ein  glatter Diffeomorhismus und $v:\M\to\R$ eine glatte Funktion
ist, ist es eine generische Eigenschaft, da"s die 
Abbildung $\diffeo_{(\flow,v)} : \M \to\R^{2\mandim+1}$, definiert durch 
\eqn{\diffeo_{(\flow,v)}(\x) = (v(\x),v(\flow^1(\x)), \dots, v(\flow^{2\mandim}(\x)))}
eine Einbettung ist; \metapher(Glatt) bedeutet hier mindestens $\sm^2$.
Hierbei werden zus"atzlich folgende Vorraussetzungen an den Flu"s $\flow$ gestellt:
\begin{myitemize}
\item Wenn $\x$ periodischer Punkt der Periode $k\le 2\mandim+1$ ist, sind alle Eigenwerte
von $\d\flow^k\at{\x}$ paarweise verschieden und verschieden von 1.
\item F"ur verschiedene Fixpunkte $\x^*$ von $\flow$, sind auch die $v(\x^*)$
verschieden\footnote{Der Satz gilt entsprechend f"ur durch $\sm^2$-Vektorfelder erzeugte Fl"usse,
wobei sich die beiden Vorraussetzungen leicht auf das Vektorfeld "ubertragen lassen.}.
\end{myitemize}
\end{theorem}

Um dieses Theorem anwenden zu k"onnen, m"ussen wir es auf unsere Situation
"ubertragen. Der glatte Diffeomorphismus $\flow$ stellt den (uns unbekannten) Flu"s des
betrachteten dynamischen Systems dar, w"ahrend $v$ unsere Observable ist. Die Abbildung
$\diffeo_{(\flow,v)}$ entspricht unseres Verz"ogerungskoordinatenabbildung
$\diffeo_{k,\embed}$ mit dem Unterschied, da"s hier aus dem Originalphasenraum und nicht aus
der von der Obervablen gebildeten Zeitreihe abgebildet wird.

Von den Vorraussetzungen des Systems ist die zweite unmittelbar einzusehen, da ansonsten
verschiedene Fixpunkte von $\flow$ auf den gleichen Punkt abgebildet w"urden. Die erste
ist dagegen komplizierter und nur im Rahmen des recht beschwerlichen Beweises dieses
Theorems zu verstehen. Es reicht hier festzuhalten, da"s beide unter generischen
Bedingungen erf"ullt sind. 

Das Theorem versichert uns also, da"s f"ur generische Fl"usse $\flow$ und
Me"sfunktionen $v$, die Verz"ogerungskoordinatenabbildung $\diffeo_{(\flow,v)}$ eine
Einbettung ist. Da im Experiment nur in diskreten Zeitschritten $\sample$ gemessen werden
kann, steht uns jedoch nicht, wie vorrausgesetzt, eine kontinuierliche Funktion $v(t)$,
sondern nur die diskrete Me"sreihe $v_i=v(i\sample)$ zur Verf"ugung. Wir m"ochten nun
wissen, ob auch die Grenzmenge der diskreten Folge konjugiert zu der des Originalsystems
ist. Dies beantwortet uns eine Korollar zu Takens' viertem Theorem.
\begin{corollar}[Korollar 5 zu Takens' Theorem 4]
Sei $\M$ eine kompakte Mannigfaltigkeit der Dimension $\mandim$. Wir betrachten Viertupel,
bestehend aus einem Vektorfeld $\vec F$, einer Funktion $v$, einem Punkt $\x$ und einer
positiven reellen Zahl $\delay$. F"ur generische $(\vec F, v, \x, \delay)$ ist die
positive Grenzmenge $\limitp(\x)$ \naja(diffeomorph) zu der Grenzmenge der folgenden
Sequenz im $\R^{2m+1}$
\eqn{ \left\{ \left( v(\flow^{i\delay}(\x)),v(\flow^{(i+1)\delay}(\x)), \dots
,v(\flow^{(i+2m)\delay}(\x))    \right) \right\}^\infty_{i=0} }
\naja(Diffeomorph) hei"st hier: es gibt eine glatte Einbettung von $\M$ nach $\R^{2m+1}$,
die $\limitp(\x)$ bijektiv auf die Grenzmenge dieser Punktfolge abbildet.
\end{corollar}

Wir k"onnen nun schlie"sen, da"s unter generischen Bedingungen an die Verz"ogerung $k$ und unsere
Me"sfunktion $v$, die Verz"ogerungskoordinatenabbildung $\diffeo_{k,\embed}$ eine Einbettung
des Attraktors liefert, solange nur $\embed\geq2m+1$ ist. 

F"ur den Experimentator ist diese Aussage jedoch nicht ausreichend, da nach den Ausf"uh\-rungen von
Seite \pageref{generisch}, Generizit"at nichts "uber die Wahrscheinlichkeit, da"s wir hier wirklich eine Einbettung
vorliegen haben, aussagt. Wir m"ochten wissen, da"s die Verz"ogerungskoordinatenabbildung  mit
Wahrscheinlichkeit eins, eine Einbettung ist. 


Entsprechende Aussagen finden sich bei \autor(Sauer), \autor(Yorke) und \autor(Casdagli)
``Embedology'' (1991). Wo bei Takens Einbettungen von glatten Mannigfaltigkeiten der
Dimension $\mandim$ betrachtet werden, beziehen sich \autor(Sauer \etal) auf kompakte Mengen der
Kapazit"at $\fracdim$ (siehe Kapitel \ref{chapcapacity}). Au"serdem wird \begriff(generisch)
durch \begriff(pr"avalent) ersetzt.

\comment{
\begin{theorem}[Sauers Theorem 2.5 -- Fractal Delay Embedding Prevalence Theorem]
Sei $\flow$ ein Flu"s auf einer offenen Teilmenge $\set U$ des $\R^k$ und sei $\set A$
eine kompakte Teilmenge von $\set U$ der Kapazit"at $d$ (siehe Kapitel
\ref{chapcapacity}). Sei $n>2d$ ganzzahlig und $t>0$. $\set A$ enthalte h"ochstens
eine endliche Anzahl Fixpunkte, keine Orbits periodischen Punkte der Periode $t$ 
oder $2t$, und h"ochstens endlich viele periodische Punkte der Periode $3t,4t,\dots,nt$,
und die Linearisierungen der periodischen Orbits haben verschiedene Eigenwerte. Dann gilt
f"ur fast alle glatten Funktionen $v:\set U\to\R$, da"s 
\end{theorem}
}



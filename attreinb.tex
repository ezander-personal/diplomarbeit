\clearpage
\chapter{Rekonstruktion seltsamer Attraktoren}


%%%%%%%%%%%%%%%%%%%%%%%%%%%%%%%%%%%%%%%%%%%%%%%%%%%%%%%%%%%%%%%%%%%%%%%%%%%%%%%%%%%%%%%%%%%%%%%%%%%%
%% Dynamische Systeme
%%%%%%%%%%%%%%%%%%%%%%%%%%%%%%%%%%%%%%%%%%%%%%%%%%%%%%%%%%%%%%%%%%%%%%%%%%%%%%%%%%%%%%%%%%%%%%%%%%%%

\section{Dynamische Systeme}
\label{chapdynsystems}
Das Konzept \begriff(seltsamer Attraktoren) hat in den letzten Jahren Bedeutung bei der
Erkl"arung von Ph"anomenen aus den unterschiedlichsten Forschungsgebieten gefunden. Gemeinsam
ist allen, da"s die untersuchten Systeme prinzipiell durch sogenannte
\begriff(Evolutionsgleichungen)
\eqn{\dot\x = \vec F(\x)}
beschrieben werden k"onnen. Die Richtung $\dot\x$, in die sich der Zustand des Systems entwickelt, und damit
auch die ganze zuk"unftige Entwicklung\footnote{Unter hinl"anglich allgemeinen Forderungen
an das Vektorfeld $\vec F$, n"amlich der Lipschitz-Stetigkeit.}, ist damit determiniert durch
den aktuellen Zustand $\x$. Man sagt auch, das Vektorfeld $\vec F$ erzeuge einen Flu"s
$\flow$.
\eqn{\x(t) = \flow^t(\x_0) }
Der Flu"s beschreibt die zeitliche Entwicklung des Systems in Abh"angigkeit vom
Anfangszustand $\x_0$.
Die Bewegung findet in einem Raum statt, der als \begriff(Phasenraum) $\phase$
bezeichnet wird. Jedem  Freiheitsgrad des Systems entspricht genau eine Koordinate des
Phasenraums. Die Anzahl der Freiheitsgrade und damit die Dimension des Phasenraums ist in
vielen F"allen endlich, kann jedoch bei ausgedehnten Systemen\footnote{Gemeint sind hier
  beispielsweise hydrodynamische Systeme deren, 
  Zustand durch Dichte- und Temperaturfelder beschrieben wird.} auch "uberabz"ahlbar
unendlich werden\footnote{Dies ist auch bei eindimensionalen Systemen der Fall, die durch
sogenannte \begriff(Delay-Gleichungen) beschrieben werden, wie beispielsweise die
Mackey-Glass-Gleichung. }.


Bei nat"urlichen Systemen ist es die Regel, da"s Energie durch Prozesse wie Reibung
verlorengeht. Bei solchen \begriff(dissipativen Systemen) kann man die Beobachtung
machen, da"s sich die Dynamik nach einer transienten Phase nur noch auf eine 
niedrigdimensionale \begriff(Untermannigfaltigkeit) $\M\subset\phase$ des Phasenraums
reduziert. Die Energiedissipation eines Systems ist gleichbedeutend mit einer Kontraktion
von Phasenraumvolumina unter dem Flu"s $\flow$.
\eqn{\abl{}{t}\flow^t(V)\lt 0}

Dies kann auch "uber das erzeugende Vektorfeld $\vec F$ ausgedr"uckt werden. Die Divergenz
des Vektorfeldes mu"s in den erreichbaren Teilen des Phasenraums negativ sein.
\eqn{\mathop{\mathrm{div}}\vec F \lt 0}
Hierf"ur ist es nicht notwendig, da"s die Phasenraumvolumina in alle Richtungen gestaucht
werden. Es k"onnen auch bestimmte Richtungen expandiert werden, solange die Kontraktion in
die anderen Richtungen "uberwiegt.

Da die transiente Phase der Bewegung im allgemeinen recht schnell abklingt, interessiert
man sich meist f"ur das \begriff(Langzeitverhalten) dieser Systeme. Man beschr"ankt sich bei
der Untersuchung auf die \begriff(Grenzmenge)\footnote{Die positive Grenzmenge $\limit^+$
eines Systems ist definiert als die Menge aller Punkte $\y$, f"ur die  
Folgen $t_0\lt t_1\lt t_2\dots$ existieren, so da"s $\lim\limits_{i\to\infty}\flow^{t_i}(\x)=\y$
f"ur mindestens ein $\x\in\phase$.} $\limit^+$, d.h.\   den Bereich des
Phasenraums, in dem sich die Dynamik im Grenzfall unendlich langer Zeiten abspielt. 

\subsection{Seltsame Attraktoren}
Lange herrschte die Auffassung, da"s sich die einzig m"oglichen Grenzmengen aus
\begriff(Punkt\-attraktoren), \begriff(Grenzzyklen) oder \begriff($k$-Tori)
zusammensetzen. Diesen ist gemeinsam, da"s sie periodisch (in den ersten beiden F"allen)
bzw.\ quasiperiodisch (im letzten Fall) sind. Sie besitzen ein diskretes \begriff(Fourierspektrum).
Die Entstehung von Turbulenz wurde durch das Hinzukommen immer neuer
Fouriermoden verstanden. Das komplexe Verhalten turbulenter Systeme w"urde also als "Uberlagerung
sehr vieler inkommensurabler Frequenzen betrachtet und w"are somit qualitativ nicht verschieden vom
Verhalten einfacher Systeme. Bei der Untersuchung von Konvektionszellen stie"s
\autor(Lorenz) 1963 \cite{Lorenz63} jedoch auf ein System, das keinerlei Periodizit"at
und somit ein kontinuierliches Fourierspektrum aufwies (s.~\psref{attrlor}). Diese 
\begriff(deterministischen nichtperiodischen Fl"usse) erhielten sp"ater von \autor(Ruelle)
und \autor(Takens) \cite{Ruelle71a} den einpr"agsameren Namen \begriff(seltsame Attraktoren). 


\epsfigdouble{attraktoren/lornew}{attraktoren/fournew}
{Links eine typische Trajektorie des Lorenzsystems. \comment{ zu den 
Standardparameterwerten $\sigma =10$, $r=28$, $b=8/3$} Rechts das Leistungsspektrum $P(\omega)$ der 
$x$-Komponente.}{attrlor}{-0.2cm}

Eine mathematisch exakte Definition seltsamer Attraktoren ist ein schwieriges
Problem. Eine endg"ultige Fassung, auf die sich alle geeinigt h"atten, ist bis heute nicht
gelungen \cite{Pawelzik91}. Die Schwierigkeit liegt darin, die Definition sowohl
mathematisch exakt zu halten als auch so umfassend, da"s alle wesentlichen Eigenschaften
erfa"st werden. Ein grundlegenderes Problem ist jedoch, da"s selbst "uber die
Eigenschaften seltsamer Attraktoren keine Einigkeit zwischen allen beteiligten Wissenschaftlern
besteht. Wir wollen uns hier deshalb auf einige f"ur diese 
Arbeit wichtige Eigenschaften beschr"anken. 
\begin{itemize}

\item Seltsame Attraktoren sind \begriff(durchmischend). 
Ein Attraktor $\attr$  hei"st genau dann durchmischend, wenn es f"ur beliebige,
bez"uglich $\attr$ offene\footnote{Eine Menge $\A$ hei"st 
\begriff(offen bez"uglich $\M$), wenn es offene Mengen $\tilde\A$ gibt, so
da"s $\A=\tilde\A\cap\M$. Der Begriff \naja(offen bez"uglich) stellt eine
Verallgemeinerung der Vorstellung dar, da"s sich alle Punkte einer offenen Menge
\naja(innerhalb) dieser befinden.}, nichtleere  Mengen $\set I,\set J\subset\,\attr$
mindestens einen Punkt $\x\in\set I$ gibt, so da"s $\flow^t(\x)\in\set J$ f"ur mindestens
ein $t>0$.   

Eine wichtige Folgerung daraus ist, da"s fast alle Punkte
des Attraktors unter dem Flu"s $\flow$ jedem anderen Attraktorpunkt beliebig nahe
kommen. Diese Eigenschaft ist notwendig f"ur die Definition \begriff(ergodischer Ma"se)
auf dem Attraktor. 

\item Zwei zu einem gegebenen Zeitpunkt $t_0$ eng beieinander liegende Punkte auf dem Attraktor
werden unter dem Flu"s $\flow$ mit der Zeit exponentiell voneinander getrennt. Man spricht von einer 
\begriff(exponentiellen Separation) der Trajektorien oder auch \begriff(sensibler
Abh"angigkeit von den Anfangsbedingungen). Dies findet Ausdruck in der Existenz positiver
\begriff(Lyapunov-Exponenten).

\item \begriff(Periodische Orbits) liegen dicht auf dem Attraktor. Aufgrund der
exponentiellen Separation sind diese Orbits jedoch alle instabil. Der Attraktor ist
darstellbar als "Uberdeckung seiner instabilen periodischen Orbits (IPOs). 

Diese Tatsache kann im Rahmen der Zeitreihenanalyse zur beschleunigten Bestimmung von
Ma"szahlen wie Lyapunov-Exponenten und Dimensionen ausgenutzt werden. Es mu"s nicht die
ganze zur Verf"ugung stehende Datenmenge f"ur die Berechnung herangezogen werden, da
oftmals Mittelungen "uber die experimentell bestimmten IPOs
ausreichend sind \cite{Pawelzik91,Pawelzik91a}.
\end{itemize}
Die aufgez"ahlten Eigenschaften sind nicht voneinander unabh"angig. \autor(J. Banks \etal)
konnten zeigen, da"s die exponentielle Separation der Trajektorien aus den beiden anderen
Eigenschaften gefolgert werden kann \cite{Banks92}. Da die Eigenschaften des Durchmischens
als auch das Dichtliegen periodischer Orbits topologische Invarianten sind, gilt
dies somit abenfalls f"ur die exponentielle Separation der Trajektorien. Unter
topologischen Abbildungen bleiben also die dynamischen und geometrischen Eigenschaften
seltsamer Attraktoren unber"uhrt.

Oft werden seltsame Attraktoren auch "uber ihre \begriff(fraktale) Struktur
definiert \cite{Peitgen92}. Diesem Ansatz soll hier nicht gefolgt werden, da er nur
geometrische, jedoch keine dynamischen Aspekte des Attraktors beschreibt
\cite{Eckmann-ruelle}. Es existieren Attraktoren, die seltsam\footnote{Nach obiger
Definition.}, aber nicht fraktal sind (beispielsweise \begriff(Arnolds Katzenabbildung)
\cite{Arnold68}). Demgegen"uber ist der \begriff(Feigenbaumattraktor) fraktal, aber nicht
seltsam. Die meisten Attraktoren (Lorenzattraktor, R"osslerattraktor \dots) sind jedoch
sowohl seltsam als auch fraktal.


\section{Rekonstruktionsverfahren}

Die Theorie seltsamer Attraktoren kann oftmals gute Erkl"arungsans"atze f"ur das Ver\-st"andnis
dynamischer Systeme liefern. So kann beispielsweise beim Lorenzsystem durch
Variation eines Parameters der "Ubergang einer Konvektionszelle
von laminarem zu turbulentem Verhalten studiert werden. Die vielen N"aherungen, die f"ur
dieses System gemacht werden, erlauben zwar keine direkten Voraussagen "uber reale 
Konvektionszellen, die Dynamik kann jedoch prinzipiell verstanden werden.

Anders liegt der Fall bei biologischen, medizinischen oder auch komplizierteren
hydrodynamischen Systemen. Hier tauchen eine Reihe von Problemen auf.
\begin{itemize}
\item Die relevanten Phasenraumvariablen sind oft nicht bekannt. Gerade bei
medizinischen und biologischen Systemen ist dies sehr oft der Fall.
\item Die Anzahl der Freiheitsgrade ausgedehnter Systeme (z.B. meteorologischer oder
hydrodynamischer) ist in der Regel abz"ahlbar oder gar "uberabz"ahlbar unendlich. 
\item Selbst bei Kenntnis aller Phasenraumvariablen sind die f"ur die Dynamik zust"andigen
Evolutionsgleichungen unbekannt.
\end{itemize}
Daten, die in experimentellen Situationen gewonnen werden, beschr"anken sich so meistens
auf Me"sreihen einer oder weniger Gr"o"sen, von denen angenommen wird, da"s  sie die
Dynamik des Systems charakterisieren. Stellt sich nun heraus, da"s die so gewonnene
\begriff(Zeitreihe) einen nicht trivialen\footnote{\begriff(Trivial) bedeutet in diesem
  Zusammenhang, da"s sich die Zeitreihe nicht als "Uberlagerung weniger Frequenzen
  darstellen l"a"st, d.h.\  sie hat kein diskretes
Fourierspektrum.} zeitlichen Verlauf aufweist, ergeben sich daraus einige Fragen:
\begin{itemize}
\item Liegt der Zeitreihe ein deterministisches System zugrunde, oder ist das erratische
Verhalten eine Folge additiven Rauschens?
\item L"a"st sich das System durch einen seltsamen Attraktor beschreiben? Wenn ja, wie
k"onnen wir diesen aus den vorhandenen Daten rekonstruieren?
\item Wie k"onnen f"ur den rekonstruierten Attraktor charakteristische Gr"o"sen wie
fraktale Dimension oder Lyapunov-Exponenten bestimmt werden?
\end{itemize}
Wir wollen uns zuerst mit der zweiten Frage besch"aftigen, wobei wir im folgenden annehmen
(zumindest als Arbeitshypothese), dem System l"age ein seltsamer Attraktor zugrunde.
Die erste und dritte Frage werden wir in sp"ateren Abschnitten angehen.

\subsection{Verz"ogerungskoordinaten (MOD)}

Den ersten Ansatz zur L"osung des Problems der Attraktorrekonstruktion  lieferten
\linebreak \autor(Packard \etal) 1980
mit dem Konzept der \begriff(Verz"ogerungskoordinaten), auch kurz als
MOD\footnote{Abk"urzung f"ur \begriff(Method of Delays).} bezeichnet \cite{Packard80}. Um das Verfahren 
zu veranschaulichen, verwenden wir ein System dessen Dynamik uns bereits bekannt ist. An
diesem soll eine einzige Observable\footnote{D.h.\   eine beliebige, glatte Funktion
$v:\phase\to\R$ der Phasenraumvariablen.} gemessen werden, die uns als Zeitreihe
dient. Hieraus soll die Dynamik wieder rekonstruiert und mit der urspr"unglichen Dynamik
verglichen werden.

%Als dynamisches System w"ahlen wir 
Das R"osslersystem \cite{Roessler76} ist durch das folgende System von
Differentialgleichungen bestimmt
\eqna{
\dot x &=& -z-x \nonumber \\
\dot y &=& x + a y \nonumber \\
\dot z &=& b + z(x-c) 
}
Das System ist in der dritten Gleichung nichtlinear und wird f"ur den Parametersatz
$a=0.38$, $b=0.3$, $c=4.5$ chaotisch. Der Attraktor besitzt eine fraktale
Struktur\footnote{Durch Analyse der Gleichungen erkennt man hier sehr sch"on den Chaos
  erzeugenden \metapher(Streck-und-Falt)-Proze"s \cite{Peitgen92} .  Die Struktur
  entspricht lokal dem kartesischen Produkt einer 2-dimensionalen Mannigfaltigkeit und einer
  Cantormenge.}.

Aus den Differentialgleichungen erzeugen wir durch numerische Integration einen diskreten
Orbit $\folge(\x,1,N)$\footnotemark.  Als Observable dient die $x$-Komponente der Punkte $v(\x)=x$.
\footnotetext{Verwendet wird ein Runge-Kutta-Verfahren vierter Ordnung mit
Schrittweite $\sample=0.009$. Nach einer Transienzzeit von $1000\sample$, nach der sich das
System dem Attraktor hinreichend gen"ahert hat, wird der Orbit in
diskreten Schritten $\sample$ aufgezeichnet.  Die Anzahl der berechneten Orbitpunkte betr"agt
$N=20000$. \comment{Dargestellt sind die, durch Linien 
verbundenen, Orbitpunkte $\folge(\x,1,N)$ ($N=20.000$)}.} 


\epsfigdouble{rekonstruktion/roenew}{rekonstruktion/timenew}
{Links der R"osslerattraktor aus einer numerischen
Integration. Rechts das gemessene Signal $v_i=x(i\sample)$ zur Verdeutlichung des
diskreten Charakters der Zeitreihe durch Punkte dargestellt.}{rekroe}{-0.2cm}

Die Informationen "uber die Werte der Koordinaten $y$ und $z$ stehen nun nicht mehr zur
Verf"u\-gung. Nach der Idee von \autor(Packard \etal) kann jedoch der Zustand eines
$n$-dimensio\-na\-len Systems zu einer gegebenen Zeit durch jeden Satz von $n$
unabh"angigen, sonst aber beliebigen, Koordinaten spezifiziert werden \cite{Packard80}.
Der Zustand des R"osslersystems zu einer Zeit $t_i$ sollte also statt durch
$(x_i,y_i,z_i)$ ebenso durch $(x_i,\dot x_i,\ddot x_i)$ oder $(x_i,x_{i+1},x_{i+2})$
beschrieben werden k"onnen.

Wir wollen nun anhand einer "Uberlegung von \autor(Theiler), die wir hier auf drei
Koordinaten ausgedehnt haben, plausibel machen, da"s in den Verz"ogerungskoordinaten
tats"achlich die gleiche Information steckt, wie in den originalen Phasenraumkoordinaten.
Wenn die Zeit $\sample$ zwischen zwei Messungen klein ist, gilt n"aherungsweise
\eqnal[recidea1]{\dot x_i &\simeq& (x_{i+1} - x_i) / \sample \nonumber \\
\ddot x_i &\simeq& (x_{i+2} - 2x_{i+1} + x_i) / \sample^2}

Weiterhin gilt f"ur die nullte bis zweite Zeitableitung
\eqnal[recidea2]{ x_i &=& x_i \nonumber \\
\dot x_i &=& f_1(x_i, y_i, z_i) \nonumber \\
\ddot x_i &=& \pabl{f_1}{x} f_1(x_i,y_i,z_i) + \pabl{f_1}{y} f_2(x_i,y_i,z_i) +
\pabl{f_1}{z} f_3(x_i,y_i,z_i) } 
wobei wir die Koordinatenfunktionen mit $f_1$, $f_2$ und $f_3$ bezeichnet haben. Unter im
betrachteten Zusammenhang recht allgemeinen Bedingungen an die $f_j$,
lassen sich $x_i$, $\dot x_i$ und $\ddot x_i$ wieder nach den Phasenraumvariablen $x_i$,
$y_i$ und $z_i$ aufl"osen.  Da andererseits nach \eqnref{recidea1} die Ableitungen durch die
Verz"ogerungskoordinaten bestimmt sind, k"onnen wir den kompletten Systemzustand
$x_i,y_i,z_i$ aus den \begriff(verz"ogerten) Werten der Zeitreihe $x_{i},x_{i+1},x_{i+2}$
\metapher(zur"uckholen). Man sagt, der Systemzustand sei in der Zeitreihe
\metapher(konserviert).
\comment{Die Funktionaldeterminante der rechten Seite von \eqnref{recidea2} darf
  nicht verschwinden. Da in chaotischen Systemen, die Dynamik nicht separierbar ist, kann
  man diese Annahme hier akzeptieren.}


Diese "Uberlegungen k"onnen auch auf beliebige Observable $v$ sowie auf h"oherdimensionale
\begriff(Einbettungsr"aume) $\R^\embed$ erweitert werden. Weiterhin k"onnen gr"o"sere
Zeitabst"ande $k\sample$ zwischen den Verz"ogerungskoordinaten gew"ahlt werden.  Man
erh"alt so als Rekonstruktionsvektoren die Folge $(v_{i},
v_{i+k},\dots,v_{i+(\embed-1)k})$. Dieses Rekonstruktionsverfahren entspricht einer
Abbildung $\diffeo_{k,\embed,v}:\phase\to\R^\embed$ aus dem Original- in den
Rekonstruktionsphasenraum, welche durch
\eqn{\diffeo_{k,\embed,v}(\x) = (v(\x),v(\flow^{k\sample}(\x)), \dots, v(\flow^{(\embed-1)k\sample}(\x)))}
gegeben ist. Man bezeichnet diese Abbildung  als \begriff(Verz"ogerungskoordinatenabbildung).

Wir wollen das Verfahren nun bei dem oben erw"ahnten R"osslersystem anwenden.
Da wir die Anzahl der Freiheitsgrade des R"osslersystems kennen, lassen wir es bei der uns
bekannten und \naja(ausreichenden) Einbettungsdimension $\embed=3$. F"ur die Verz"ogerung\footnote{Um Mi"sverst"andnisse
zu vermeiden, wird im folgenden die in Einheiten der \begriff(Samplingtime) $\sample$ gemessene 
\begriff(Verz"ogerungszeit) $\delay=k\sample$ durchg"angig als \begriff(Verz"ogerung) $k$
bezeichnet. Da Verz"ogerung und Verz"ogerungszeit {\em immer}  "uber diese Relation
eindeutig verkn"upft sind ($\sample$ ist konstant), wird im folgenden je nach Kontext der
besser geeignete der beiden benutzt.} w"ahlen wir $k=30$. 

%\afterpage
{\epsfigsingle{rekonstruktion/recnew}
{Rekonstruktion des R"osslerattraktors aus der Zeitreihe in \psref{rekroe} (rechts) zur
  Einbettungsdimension $\embed=3$ mit Verz"ogerung $k=30$. Die Rekonstruktionspunkte
  wurden zur besseren Darstellung durch Linien verbunden.}{rekrek}{-0.5cm}}

Die Rekonstruktion in \psref{rekrek} "ahnelt einer verzerrten Kopie des
Originalattraktors in \psref{rekroe}. Insofern leistet die
Verz"ogerungskoordinatenabbildung schon gute Dienste. Die rein visuelle "Ahnlichkeit
reicht aber f"ur eine genauere Analyse der Dynamik nicht aus. Es stellen sich mehrere
Fragen, die noch zu beantworten sind
\begin{myitemize}
\item Sind die Dynamik des rekonstruierten und des Originalattraktors zueinander
konjugiert? Mit anderen Worten: Ist die Verz"ogerungskoordinatenabbildung $\diffeo_{k,\embed,v}$
ein Diffeomorphismus?
\item Bleiben Dimensionen und Lyapunov-Exponenten unter der Rekonstruktion durch 
die Verz"ogerungskoordinatenabbildung invariant?
\item Wie sind die Einbettungsparameter $d$ und $k$ zu w"ahlen, wenn das urspr"ungliche
System nicht bekannt ist?
\end{myitemize}
Diese Fragen sollen in den n"achsten Abschnitten beantwortet werden.




%%%%%%%%%%%%%%%%%%%%%%%%%%%%%%%%%%%%%%%%%%%%%%%%%%%%%%%%%%%%%%%%%%%%%%%%%%%%%%%%%%%%%%%%%%%%%%%%%%%%
%% Einbettungen
%%%%%%%%%%%%%%%%%%%%%%%%%%%%%%%%%%%%%%%%%%%%%%%%%%%%%%%%%%%%%%%%%%%%%%%%%%%%%%%%%%%%%%%%%%%%%%%%%%%%

\subsubsection{Einbettungen}

Bevor wir uns mit den Einbettungstheoremen besch"aftigen, soll erst einmal der Begriff der
\begriff(Einbettung) selbst gekl"art werden. Bei einer Einbettung handelt es sich immer um
eine Abbildung (einer Menge) aus einem Phasenraum in einen anderen. Nun sollen unter
dieser Abbildung (und auch der Umkehrabbildung) keine Punkte kollabiert werden, d.h. es
sollen keine verschiedenen Originalpunkte auf den gleichen Bildpunkt abgebildet werden.
Solche Abbildungen, die die topologischen Eigenschaften von Punktmengen invariant lassen
bezeichnet man als \begriff(Hom"oomorphismen). Weiterhin sollen durch die Einbettung auch
keine Tangentenrichtungen kollabiert werden, was beispielsweise f"ur die Bestimmung von
Lyapunov-Exponenten von Bedeutung ist.  Zus"atzlich wird also die stetige
Differenzierbarkeit der Abbildung und der Umkehrabbildung gefordert. Eine
Einbettung mu"s somit ein $\sm^1$-Diffeomorphismus sein.

\comment{
Eine Einbettung ist erstmal ein
\begriff(Hom"oomorphismus) $\diffeo$, d.h.\   eine bijektive, stetige Abbildung mit stetiger
Umkehrfunktion. Hom"oomorphismen werden auch als \begriff(topologische Abbildungen)
bezeichnet, da sie die topologischen Eigenschaften von Punktmengen invariant lassen.
Unter topologischen Abbildungen werden keine Punkte kollabiert, d.h.\  es werden keine
verschiedenen Originalpunkte auf den gleichen Bildpunkt abgebildet. Nun sollen bei einer
Einbettung aber auch keine Tangentenrichtungen kollabiert werden, was unter anderem f"ur die
Bestimmung von Lyapunov-Exponenten wichtig ist. Wir fordern deshalb
zus"atzlich die stetige Differenzierbarkeit von $\diffeo$ und $\diffeo^{-1}$,
d.h.\  die Einbettung $\diffeo$ soll mindestens ein 
Da bei der Phasenraumrekonstruktion nicht nur die topologischen Eigenschaften des
Attraktors erhalten bleiben sollen, sondern auch die Stetigkeit der ersten und zweiten
Ableitung an Trajektorien (d.h.\   Richtung und Kr"ummung), fordern wir zus"atzlich die
zweifache stetige Differenzierbarkeit, d.h.\   $\diffeo$ soll ein $\sm^2$-Diffeomorphismus sein.
\inkorrektur(nicht ganz richtig, auf immersions eingehen)
}


\paragraph{Einbettungstheoreme}
\comment{Nach dieser kurzen Einleitung}
Wir wollen uns nun mit dem Problem besch"aftigen, unter
welchen Voraussetzungen die Verz"ogerungskoordinatenabbildung $\diffeo_{k,\embed,v}$ eine
Einbettung in obigem Sinne ist. Den ersten Ansatz zur Beantwortung dieser Frage lieferte
\autor(Takens) 1980 \cite{Takens80}.  \comment{Sein erstes Theorem soll hier vollst"andig zitiert
werden}

\begin{theorem}
Sei $\M$ eine kompakte Mannigfaltigkeit der Dimension $\mandim$. F"ur Paare $(\flow,v)$,
wobei  \linebreak[4] $\flow:\M\to\M$ ein  glatter Diffeomorphismus und $v:\M\to\R$ eine
glatte Funktion ist, ist es eine generische Eigenschaft, da"s die 
Abbildung $\diffeo_{(\flow,v)} : \M \to\R^{2\mandim+1}$, definiert durch 
\eqnl[takmod]{\diffeo_{(\flow,v)}(\x) = (v(\x),v(\flow^1(\x)), \dots, v(\flow^{2\mandim}(\x))),}
eine Einbettung ist; \metapher(Glatt) bedeutet hier mindestens $\sm^2$.
Hierbei werden zus"atzlich folgende Voraussetzungen an den Flu"s $\flow$ gestellt:
\begin{myitemize}
\item Wenn $\x$ periodischer Punkt der Periode $k\le 2\mandim+1$ ist, sind alle Eigenwerte
von $\mathrm{D}\flow^k(\x)$ paarweise verschieden und verschieden von 1.
\item F"ur verschiedene Fixpunkte $\x^*$ von $\flow$, sind auch die $v(\x^*)$
verschieden\footnote{Der Satz gilt entsprechend f"ur durch $\sm^2$-Vektorfelder erzeugte Fl"usse,
wobei sich die beiden Voraussetzungen leicht auf das Vektorfeld "ubertragen lassen.}.
\end{myitemize}
\end{theorem}

Um dieses Theorem anwenden zu k"onnen, mu"s es auf unsere Situation "ubertragen werden.
$v$ ist unsere Observable, und die Abbildung $\diffeo_{(\flow,v)}$ entspricht der
Verz"ogerungskoordinatenabbildung $\diffeo_{k,\embed,v}$.  Der glatte Diffeomorphismus
$\flow$ stellt den (uns unbekannten) Flu"s des betrachteten dynamischen Systems dar.
Takens' normiert hierbei die Verz"ogerungszeit auf $\delay=1$.  Dies ist jedoch keine
wesentliche Einschr"ankung, da die Flu"sabbildung zeitlich immer so umskaliert werden
kann, da"s die Verz"ogerungszeit $\delay$ gleich $1$ wird\footnote{D.h. wir betrachten
  statt des Flusses $\flow^t(\x)$ den Flu"s $\tilde\flow^{t}(\x) = \flow^{t/\delay}(\x)$.}.

Von den Voraussetzungen des Systems ist die zweite unmittelbar einzusehen, da bei einer
Verletzung selbiger verschiedene Fixpunkte $\x^*$ von $\flow$ auf den gleichen Punkt
$\diffeo_{(\flow,v)}(\x^*)$ abgebildet w"urden. Die erste ist dagegen komplizierter und
nur im Rahmen des recht beschwerlichen Beweises dieses Theorems zu verstehen. Es reicht
hier festzuhalten, da"s beide unter generischen Bedingungen erf"ullt sind.

Das Theorem versichert uns also, da"s f"ur generische Fl"usse $\flow$ und
Me"sfunktionen $v$, die Verz"ogerungskoordinatenabbildung $\diffeo_{(\flow,v)}$ eine
Einbettung ist. Da im Experiment nur in diskreten Zeitschritten $\sample$ gemessen werden
kann, steht uns jedoch nicht, wie vorausgesetzt, eine kontinuierliche Funktion $v(t)$,
sondern nur die diskrete Me"sreihe $v_i=v(i\sample)$ zur Verf"ugung. Wir m"ochten nun
wissen, ob auch die Grenzmenge der diskreten Folge konjugiert zu der des Originalsystems
ist. Dies wird durch ein Korollar zu Takens' viertem Theorem beantwortet.

\begin{corollar}
Sei $\M$ eine kompakte Mannigfaltigkeit der Dimension $\mandim$. Wir betrachten Viertupel,
bestehend aus einem Vektorfeld $\vec F$, einer Funktion $v$, einem Punkt $\x$ und einer
positiven reellen Zahl $\delay$. F"ur generische $(\vec F, v, \x, \delay)$ ist die
positive Grenzmenge $\limitp(\x)$ \naja(diffeomorph) zu der Grenzmenge der folgenden
Sequenz im $\R^{2m+1}$
\eqn{ \left\{ \left( v(\flow^{i\delay}(\x)),v(\flow^{(i+1)\delay}(\x)), \dots
,v(\flow^{(i+2m)\delay}(\x))    \right) \right\}^\infty_{i=0} }
\naja(Diffeomorph) hei"st hier: es gibt eine glatte Einbettung von $\M$ nach $\R^{2m+1}$,
die $\limitp(\x)$ bijektiv auf die Grenzmenge dieser Punktfolge abbildet.
\end{corollar}

Wir k"onnen nun schlie"sen, da"s f"ur generische Verz"ogerungen $k$ und Me"sfunktionen
$v$, die Verz"ogerungskoordinatenabbildung $\diffeo_{k,\embed,v}$ eine Einbettung des
Attraktors liefert, solange nur $\embed\geq2m+1$ ist.
Der Begriff \naja(generisch) ist jedoch relativ schwach und sagt nichts "uber die
Wahrscheinlichkeit aus, da"s dies tats"achlich der Fall ist, wenngleich man dies gerne
meinen m"ochte. 

Dies soll genauer erl"autert werden.  Der Audruck \naja(Generisch) beschreibt das
Auftreten bestimmter Eigenschaften bei Elementen aus einer Menge $\A$.
Eine \label{generisch} Eigenschaft hei"st bereits dann generisch auf $\set A$, wenn
eine \begriff(residuale Teilmenge)\footnote{Eine Menge $\set R\subset\set A$ hei"st
\begriff(residuale) Teilmenge von $\set A$, wenn $\set R$ abz"ahlbarer Durchschnitt
offener, dichter Teilmengen von $\set A$ ist. Residuale Mengen sind selbst wieder dicht, jedoch nicht notwendig offen. }
 $\set R$ von $\set A$ existiert, so
da"s alle Elemente von $\set R$ diese Eigenschaft aufweisen \cite{Liebert91}.
Zu jedem Element aus $\set A$ findet man also in
jeder endlichen, beliebig kleinen Umgebung ein Element aus $\set R$, da"s diese
Eigenschaft aufweist. Dies sagt jedoch nichts "uber die
Wahrscheinlichkeit ein Element dieser Menge zuf"allig zu treffen. Es existieren Beispiele,
f"ur diese Wahrscheinlichkeit sogar null wird. So ist beispielsweise
die Menge $\Omega_{\text{stab}}$ der Parameterwerte $\omega\in[0,2\pi]$, f"ur die die
eindimensionale Kreisabbildung
\eqn{g_{\omega,k}(x)=x+\omega+k\sin(x) \nonumber}
stabile Orbits besitzt, eine residuale Teilmenge von $[0,2\pi]$. F"ur $k\to0$ verschwindet
jedoch das Lebesgue-Ma"s von $\Omega_{\text{stab}}$, die Wahrscheinlichkeit einen Punkt
dieser Menge zu treffen geht  gegen null \cite{Sauer91}.


F"ur den Experimentator ist die Zusicherung aus Takens' Korrolar somit nicht ausreichend,
da nach den obigen Ausf"uh\-rungen, Generizit"at nichts "uber die
Wahrscheinlichkeit, da"s hier wirklich eine Einbettung vorliegt, aussagt. Wir m"ochten
wissen, da"s die Verz"ogerungskoordinatenabbildung mit Wahrscheinlichkeit eins eine
Einbettung ist.

Um auszudr"ucken, eine Eigenschaft treffe mit Wahrscheinlichkeit eins auf die Elemente
einer Menge $\set A$ zu, sagen wir, die Eigenschaft sei \begriff(pr"avalent) auf $\set A$. 
Da die Definition diese Begriffs auch auf "uberabz"ahlbar dimensionale Mengen sinnvoll
sein soll, kann er nicht "uber das verschwindende Lebesgue-Ma"s der Komplement"armenge
definiert werden, da ein solches hier nicht existiert. Die folgende Definition ist
entnommen aus \autor(Sauer \etal) \cite{Sauer91}.

\begin{definition}[Pr"avalent]
Eine Borel-Teilmenge $S$ eines normierten Vektorraums $V$ ist \begriff(pr"avalent), wenn
es einen endlich dimensionalen Untervektorraum $E$ aus $V$ gibt, so da"s f"ur alle $v$ aus
$V$ gilt, $v+e\in S$ f"ur fast alle $e$ aus $E$.
\end{definition}
Den Unterraum $E$ bezeichnet man als \begriff(Testraum) (engl. probe space). Die
Eigenschaft ``pr"avalent'' kann man sich nun folgenderma"sen vorstellen. Sei irgendein
Punkt $v$ aus $V$ vorgegeben, dann kann man von da aus in jede beliebige Richtung aus $E$
\naja(wandern) und trifft mit Wahrscheinlichkeit eins auf einen Punkt aus $S$. Da mit $E$
auch jeder Untervektorraum $E'$, der $E$ enth"alt, ein Testraum ist, ist leicht
einzusehen, da"s die Eigenschaft ``pr"avalent'' f"ur endlich dimensionale Vektorr"aume zu
der "ublichen Definition von \naja(f"ur fast alle) bzw.\ \naja(mit Wahrscheinlichkeit
eins) "aquivalent ist.


Aufgrund des oben beschriebenen Mankos von Takens' Theorem bewiesen \autor(Sauer),
\autor(Yorke) und \autor(Casdagli) ein erweitertes Einbettungstheorem \cite{Sauer91}, das
sogenannte \begriff(Fractal Delay Embedding Prevalence Theorem).

\begin{theorem}
Sei $\flow$ ein Flu"s auf einer offenen Teilmenge $\set U$ des $\R^k$ und sei $\set A$
eine kompakte Teilmenge von $\set U$ der Kapazit"at $\fracdim$ \comment{(s. Abschnitt
\ref{chapcapacity})}. Sei $n>2\fracdim$ ganzzahlig und $\delay>0$. $\set A$ enthalte h"ochstens
eine endliche Anzahl Fixpunkte, keine Orbits periodischen Punkte der Periode $\delay$ 
oder $2\delay$, und h"ochstens endlich viele periodische Punkte der Periode $3\delay,4\delay,\dots,n\delay$,
und die Linearisierungen der periodischen Orbits haben verschiedene Eigenwerte. Dann gilt
f"ur fast alle glatten Funktionen $v:\set U\to\R$, da"s die
Verz"ogerungskoordinatenabbildung $\diffeo(v,\flow,\delay):\set U\to\R^n$ (s.~\eqnref{takmod})
\begin{myitemize}
\item eineindeutig auf $\set A$ ist.
\item auf jeder kompakten Teilmenge $\set C$ einer in $\set A$ enthaltenen 
  glatten Mannigfaltigkeit eine Einbettung ist.
\end{myitemize}
\end{theorem}

Die wesentlichen Unterschiede zu Takens' Theorem sind erstens, da"s statt glatter
Mannigfaltigkeiten der Dimension $\mandim$ kompakte Mengen der Kapazit"at $\fracdim$
(welche erstere beinhalten)betrachtet werden, und zweitens, da"s generisch durch die
st"arkere Eigenschaft pr"avalent ersetzt wird.


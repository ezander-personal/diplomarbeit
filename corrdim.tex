
\subsection{Die Korrelationsdimension}
\label{chapcorrdim}

Die momentan g"angigste Methode der Dimensionsbestimmung ist die Berechnung der
\begriff(Korrelationsdimension) nach \autor(Grassberger) und \autor(Procaccia)
\cite{Grassberger-procaccia}. Die Korrelationsdimension ist definiert durch:
\eqnl[cdimdef]{D_C = \lim_{r\to 0} \frac{\log C(r)}{\log r},}
wobei $C(r)$ das sogenannte \begriff(Korrelationsintegral) darstellt: 
\eqnl[cintdef]{C(r) = \frac{1}{N(N-1)}\sum_{i\neq j}\Theta(r-\norm{\x_i-\x_j}).}
Hierbei ist $\Theta$ die Heavysidefunktion. 
$C(r)$ \naja(z"ahlt) wieviel Paare von Punkten existieren, die einen Abstand
$\norm{\x_i-\x_j}$ kleiner als $r$ haben. Wir wollen nun zeigen, da"s die "uber
\eqnref{cdimdef} definierte Korrelationsdimension mit der generalisierten Dimension $\corrdim$ 
"ubereinstimmt, wobei hier kein exakter Beweis sondern nur eine Beweisskizze erfolgen
soll.


Die verallgemeinerte Dimension $\corrdim$ ist gegeben durch
\eqnl[d2def]{\corrdim =  \lim_{r\to 0} \frac{\log \sum_i \Prob_i^2}{\log r}.} 
$\Prob_i$ ist die Wahrscheinlichkeit einen Attraktorpunkt in der $i$-ten Box der
gew"ahlten Zerlegung zu finden. $\Prob_i^2$ ist somit die Wahrscheinlichkeit zwei
beliebige, aber verschiedene Punkte gleichzeitig in der Box~$i$ anzutreffen:
\eqn{\Prob_i^2 =  \frac{1}{N(N-1)}\sum_{j,k} I_i(\x_j) I_i(\x_k).}
Hierbei ist $I_i$ die Indikatorfunktion der Box~$i$.  Die Summation von $\Prob_i^2$ "uber
alle Boxen ergibt nun die Wahrscheinlichkeit, irgend zwei Punkte gleichzeitig in einer
beliebigen Box anzutreffen.  Diese kann angen"ahert werden durch die Wahrscheinlichkeit,
zwei Punkte in einer Entfernung kleiner als der Boxdurchmesser anzutreffen;
letztere ist gegeben durch das Verh"altnis aller Punktepaare mit einem
Abstand kleiner als $r$ zur Gesamtanzahl aller Punktepaare, d.h.
\eqn{\sum_i \Prob_i^2 \simeq \frac{1}{N(N-1)} \, \# \left\{ (i,j) \, \vert \, i \neq j \land  \norm{\x_i-\x_j}\leq r \right\} .} 
Die rechte Seite dieser Gleichung kann aber durch das Korrelationsintegral
\eqnref{cintdef} ausgedr"uckt werden. Es kann nun weiterhin gezeigt werden, da"s die
gemachten N"aherungen f"ur $r\to0$ verschwinden, und wir erhalten somit:
\eqn{\corrdim=D_C.} 
Die Korrelationsdimension ist also gleich der verallgemeinerten Dimension $\corrdim$ und dient
uns so als untere Absch"atzung f"ur die  Kapazit"atsdimension. 

Aufgrund der Unabh"angigkeit von $D_q$ von der speziellen Form der Zerlegung
(siehe \eqnref{gendim}) ist das Korrelationsintegral (im Grenzfall $r\to0$) unabh"angig von
der Wahl der Norm. F"ur die numerische Berechnung des Korrelationsintegrals ist es daher
aus Gr"unden der Geschwindigkeit sinnvoll, statt der euklidischen Norm $\norm{\cdot}_2$ die
Maximumsnorm $\norm{\cdot}_\infty$ zu verwenden. Da die Definition des
Korrelationsintegrals in \eqnref{cintdef} symmetrisch bez"uglich $i$ und $j$ ist, gen"ugt
die Berechnung desselben f"ur $i<j$:
\eqnl[cintdef2]{C(r) = \frac{2}{N(N-1)}\sum_{i<j}\Theta(r-\norm{\x_i-\x_j}_\infty).}



%%%%%%%%%%%%%%%%%%%%%%%%%%%%%%%%%%%%%%%%%%%%%%%%%%%%%%%%%%
\subsubsection{Numerische Berechnung des Korrelationsintegrals}
Bedeutung erlangt hat die Korrelationsdimension vor allem wegen der schnellen und
relativ genauen Berechenbarkeit des Korrelationsintegrals. 
Die Berechnung des Korrelationsintegrals geschieht nun in den folgenden Schritten.
\begin{enumerate}
\item Der minimale und der maximale Abstand $\rmin$ und $\rmax$ zweier Punkte der
Zeitreihe werden bestimmt. Unter Verwendung der Maximumsnorm kann f"ur den maximalen
Abstand $\rmax=\max\limits_{i,j}{\lvert x_i-x_j \rvert}$ gew"ahlt werden bzw.\  f"ur den
minimalen $\rmin=\min\limits_{i,j}{\lvert x_i-x_j \rvert}$. 
\item Der Bereich $[\rmin,\rmax[$ wird in $m$ Intervalle $[r_l,r_{l+1}[$ mit
$r_0=\rmin$ und $r_m=\rmax$ aufgeteilt. Die Aufteilung sollte logarithmisch (d.h.\
$r_{l+1}/r_l=\rho=\const$) erfolgen, um dem Skalierungsverhalten des Korrelationsintegrals
Rechnung zu tragen.
\item Den Intervallen wird ein Array $K$ zugeordnet, so da"s dem Intervall $[r_l,r_{l+1}[$
der Wert $K_l$ entspricht. Das Array $K$ dient dazu, bei der sp"ateren Kalkulation die
Anzahl der Punktepaare, deren Abstand in dem entsprechenden Intervall liegt, aufzunehmen.
\item F"ur alle $i<j$ wird der Abstand $r_{ij}=\norm{\x_i-\x_j}_\infty$
berechnet. Dasjenige $K_l$ mit $r_l\leq r_{ij}\lt r_{l+1}$ wird um eins erh"oht.

Die Berechnung des Index $l$ aus dem Abstand $r_{ij}$ ist einer der zeitkritischen Teile 
des Algorithmus. Der Index $l$ kann "uber $l=\lfloor log(r_{ij} / \rmax ) / log(\rho)
\rfloor + m$ berechnet werden. Da die Berechnung des Logarithmus sehr langsam ist, wird
hier folgenderma"sen verfahren: Statt des nat"urlichen Logarithmus wird der
Zweierlogarithmus benutzt. W"ahlt man nun das Abstandsverh"altnis $\rho$ so, da"s
$\rho=2^{1/k}$ mit $k\in\N$ gilt, dann kann die obige Formel umgeschrieben werden zu $l=\lfloor
log_2\left( (r_{ij} / \rmax )^k \right) \rfloor + m$. Die Funktion $\lfloor
log_2(\cdot) \rfloor$ kann aufgrund der internen Zahlendarstellung von Computern sehr
schnell berechnet werden.
\item Da $K_l$ gleich der H"alfte der Anzahl aller Punktepaare mit Abstand zwischen $r_l$
  und $r_{l+1}$ ist, kann aus diesen das Korrelationsintegral "uber
  $C(r_k)=\frac{2}{N(N-1)} \sum\limits_{l=0}^{k-1} K_l$ bestimmt werden.
\end{enumerate}
Am effektivsten ist das Verfahren, wenn das Korrelationsintegral in einem Durchlauf f"ur verschiedene 
Einbettungsdimensionen $d=1\dots d_\tmax$ bestimmt wird. In diesem Fall mu"s das
eindimensionale Array $K$ durch ein zweidimensionales ersetzt werden, dessen zweite
Komponente die Einbettungsdimension spezifiziert. Schritt 4 ist so
abzu"andern, da"s zuerst der Abstand $r_{ij,1}=\abs{\x_{i,1}-\x_{j,1}}$ f"ur die
Einbettungsdimension $d=1$ berechnet wird. Die folgenden Abst"ande folgen dann bei Verwendung
der Maximumsnorm aus $r_{ij,d}=\max(r_{ij,d-1},\abs{\x_{i,d}-\x_{j,d}})$. Der Index $l$
mu"s nur dann neu berechnet werden, wenn $r_{ij,d}\gt r_{ij,d-1}$ ist.

\epsfigsingle{corrint/perfect/corrint700b}
{Korrelationsintegral f"ur eine Zeitreihe ($N=7\times 10^5$)des R"ossler-Attraktor mit $\rho=2^{1/10}$,
$m=102$ und $d=1\dots 8$ (von oben nach unten). Die Abst"ande sind einheitlich auf
$\rmax=1$ skaliert worden. Aufgrund des Skalierungsverhaltens des
Korrelationsintegrals erfolgt die Darstellung doppelt logarithmisch.} 
{corrintperf}{-0.2cm}

\psref{corrintperf} zeigt eine Berechnung des Korrelationsintegrals f"ur eine aus dem
R"ossler-System \cite{Roessler76} gewonnene Zeitreihe mit $7\times 10^5$ Punkten 
f"ur Einbettungsdimensionen $d=1\dots 8$. In der Abbildung sind deutlich zwei Bereiche zu
unterscheiden. F"ur $\ln r>-1$ geht $\ln C(r)$ gegen null. Der Grund liegt einfach
darin, da"s der Attraktor nur einen begrenzten Raumbereich aufspannt und f"ur hinreichend
gro"ses $r$ alle Paare von Rekonstruktionspunkten  in der Summe von \eqnref{cintdef2}
erfa"st sind. Dieser Bereich nennt sich auch \begriff(S"attigungsbereich) des
Korrelationsintegrals. Den Abstand, ab dem S"attigung auftritt, bezeichnen wir mit
$\rsaett$. F"ur $\log r<-1$ zeigt das Korrelationsintegral das 
 erwartete Skalierungsverhalten $C(r)\propto r^\nu$, wobei die Konstante
$\nu$ f"ur unterschiedliche Einbettungsdimensionen verschiedene Werte annimmt. Da der
R"ossler-Attraktor die Korrelationsdimension $\corrdim\simeq 2.07>2$ besitzt, skaliert $C(r)$
f"ur Einbettungsdimensionen $d\leq 2<\corrdim$ mit $r^d$, da in diesem Fall der gesamte Phasenraum
aufgespannt wird. F"ur Einbettungsdimensionen $d>\corrdim$ skaliert das Korrelationsintegral
mit $r^{\corrdim}$. Dieser Bereich, auf dessen Bestimmung bei der Korrelationsanalyse das
Hauptaugenmerk liegt, hei"st \begriff(Skalierungsbereich)
\footnote{F"ur die Dimensionsberechnung ben"otigen wir also nur $d>\corrdim$. Die Bedingung
$d>2D_0$ (nach Einbettungstheorem 2) ist hier nicht notwendig zu erf"ullen, da es bei
Dimensionsberechnungen nicht auf die Eineindeutigkeit der
Verz"ogerungskoordinatenabbildung  $\diffeo_{(k,d,v)}$ ankommt. Dieses Resultat geht
auch aus dem Projektionssatz f"ur fraktale Mengen hervor \cite{Falconer90}.}.

\subsubsection{Berechnung der Korrelationsdimension aus dem Korrelationsintegral}
\paragraph{Ableitung des Korrelationsintegrals}
F"ur die numerische Berechnung der Korrelationsdimension ist die Definition
\eqnref{cdimdef} schlecht geeignet, da die Gleichheit nur im Grenzfall $r\to 0$ gilt. F"ur 
endliche $r$ gilt wegen $C(r)=k r^{\corrdim}:$
\eqn{\tilde\corrdim(r)=\corrdim + \frac{\log k}{\log r}.}
Aufgrund des zweiten Summanden ist die Konvergenz gegen $\corrdim$ logarithmisch langsam.
Besser geeignet ist die Berechnung von $\corrdim$ "uber die Ableitung, da hier der Einflu"s
des Proportionalit"atsfaktors wegf"allt:
\eqn{\mcorrdim=\abl{\log C(r)}{\log r}=\frac{\abls{C(r)}{r}}{C(r)/r} .}
Numerisch bestimmt wird die Ableitung an einer Stelle $r_i$, indem durch $2k+1$
Nachbarpunkte\footnote{D.h. die Punkte $(\log r_{i-k}, \log C(r_{i-k})),\dots,(\log r_i,
\log C(r_i)),\dots,(\log r_{i+k}, \log C(r_{i+k}))$. } von $r_i$ eine Regressionsgerade
gelegt und deren Steigung ermittelt wird. Einen Graphen von $\corrdim$ "uber $\ln r$
(h"aufig auch als \begriff(Slopekurve) bezeichnet) zeigt \psref{corrslpperf}.

\epsfigsingle{corrint/perfect/corrslp700b}
{Ableitung des Logarithmus des Korrelationsintegrals aus \psref{corrintperf}. In die
Berechnung der Steigung wurden f"unf Nachbarpunkte (d.h. $k=2$) mit einbezogen. Der
Skalierungsbereich liegt ungef"ahr zwischen $\log(\rmin)=-5.7$ und $\log(\rmax)=-3.3$.}
{corrslpperf}{-0.2cm}
Die Berechnung der Korrelationsdimension "uber die Ableitung hat jedoch einige
Nachteile. Zum einen schwankt der Wert von $\mcorrdim(r)$ relativ stark f"ur
unterschiedliche $r$. Zum anderen wird ein gro"ser Teil Informationen, die das
Korrelationsintegral liefert \naja(verschwendet), da nur ein sehr begrenzter Teil f"ur
die Berechnung verwendet wird. \psref{corrslpperf} liefert trotzdem wichtige Informationen zur 
Berechnung der Korrelationsdimension. Aus der Abbildung 
kann gut abgelesen werden in welchem Bereich die Steigung des
Korrelationsintegrals konstant bleibt. Wir k"onnen hier"uber die Grenzen des
Skalierungsbereiches $\rsmin$ und $\rsmax$ bestimmen. 

\paragraph{Steigung und lineare Regression}
Eine verl"a"slichere Sch"atzung der Korrelationsdimension erhalten wir, indem wir die
Steigung "uber den gesamten Skalierungsbereich ermitteln. Als Grenzen des
Skalierungsbereiches k"onnen beispielsweise die im vorigen Abschnitt gewonnenen Werte
$\rsmin$ und $\rsmax$ genommen werden. F"ur die Korrelationsdimension ergibt sich dann:
\eqn{\mcorrdim=\frac{\log C(\rsmax)-\log C(\rsmin)}{\log \rsmax-\log(\rsmin)} .}
Die Ermittelung von $\rsmin$ und $\rsmax$ aus der Slopekurve geschieht im allgemeinen
manuell. H"aufig ist dies auch sinnvoll, da die Daten zuerst mit dem Auge begutachtet
werden sollten, bevor eine Aussage "uber die Korrelationsdimension einer Zeitreihe gemacht
wird. 

Die Steigung des Korrelationsintegrals im Bereich $I=[\rmin,\rmax[$ kann nat"urlich auch
durch Bestimmung einer Regressionsgeraden berechnet werden. Die Steigung dieser Geraden
ist durch
\eqn{\mcorrdim = \frac{\sum\limits_{i\in I} (\log C(r_i)-\overline{\log C(r)}(\log r_i
-\overline{\log r_i}))}{\sum\limits_{i\in I}(\log r_i-\overline{\log r})^2} }
gegeben, wobei $\overline{\log C(r)}$ und $\overline{\log r_i}$ die Mittelwerte von $\log C(r)$
bzw.\ $\log r$ im betrachteten Intervall $I$ sind. Es mu"s jedoch angemerkt werden, da"s
eine der Voraussetzungen f"ur die Durchf"uhrung einer linearen Regression hier nicht
erf"ullt ist. Die Werte von $\log C(r)$ sind f"ur verschiedene Werte von $r$ nicht
voneinander unabh"angig. Zudem hat die Absch"atzung des Fehlers bei einer solchen Methode
kleinster Quadrate meist wenig mit dem wirklichen Fehler bei der Dimensionsberechnung zu
tun. Wir werden sp"ater eine Methode entwickeln, die diese Schwachstellen nicht teilt (siehe
Abschnitt \ref{chaptakensest}).

\paragraph{Der Korrelationskoeffizient}
F"ur die Bestimmung der Korrelationsdimension vieler Zeitreihen ist ein Verfahren
zur automatischen Bestimmung des Skalierungsbereiches angebracht. Verfahren dieser Art
sind vielf"altig entwickelt worden, wir wollen jedoch nur eines, welches hier auch
Anwendung gefunden hat, besprechen. Der Skalierungsbereich ist derjenige, in der die
Auftragung $\log C(r)$ "uber $\log r$ mit konstanter Steigung $\mcorrdim$
verl"auft. Es ist also Aufgabe des Verfahrens, aus dem $\log C(r)$-$\log r$-Graphen den
\naja(geradesten) Teil herauszufinden. 

Daf"ur ist es notwendig, ein Ma"s daf"ur zu finden, wie \naja(gerade) der Graph in einem
bestimmten Bereich verl"auft \cite{Raidl}. Ein solches Ma"s ist der 
\begriff(Korrelationskoeffizient)\footnote{Der (\begriff(Pearsonsche))
\begriff(Korrelationskoeffizient) $K(X,Y)=\mathrm{Cov}(X,Y)/\sigma_X \sigma_Y$ beschreibt die G"ute 
der linearen Vorhersagbarkeit der Zufallsvariable $Y$ durch die Zufallsvariable $X$, wobei 
$\mathrm{Cov}(X,Y)$ die Kovarianz der beiden Zufallsvariablen ist. Er ist 
eng gekoppelt mit dem Optimierungsproblem $Y$ m"oglichst gut (linear) aus $X$
vorherzusagen, d.h.\  den Vorhersagefehler $E(Y-(a+bX))^2$ zu minimieren. Es gilt
$\min(E(Y-(a+bX))^2) = \sigma_Y(1-K^2(X,Y))$.} 
$K(X,Y)$, wobei $X$ und $Y$ zwei beliebige Zufallsvariablen sein m"ogen. Er wird $1$ 
bzw.\  $-1$, wenn zwischen $X$ und $Y$ ein exakt linearer Zusammenhang
besteht. Bei schw"acheren Zusammenh"angen wird er betragsm"a"sig kleiner als eins bzw.\  null, falls 
gar keiner besteht. Betrachten wir nun $\log C(r)$ und $\log r$ in einem Intervall
$I=[r_1,r_2[$ als Zufallsvariablen, so erhalten wir mit $L(I)=\abs{ r(\log r, \log
C)\rvert_{I}}$ ein Ma"s f"ur die Linearit"at des Korrelationsintegrals in diesem
Bereich:
\eqn{L(I) = \left| \frac{\frac{1}{n-1} \sum\limits_{r\in I}\left( \log r - \overline{\log
r}\right)\left( \log C(r) - \overline{\log C(r)} \right) } {\sqrt{\left[ \frac{1}{n-1}
\sum\limits_{r\in I}\left( \log r - \overline{\log r}\right) \right] \left[ \frac{1}{n-1}
\sum\limits_{r\in I}\left( \log C(r) - \overline{\log C(r)} \right) \right ]}} \right| .} 
Hierbei bezeichnen $\overline{\log r}$ und $\overline{\log C(r)}$ jeweils die
im Intervall $I$ gemittelten Gr"o"sen; $n$ ist die Anzahl der im Intervall $I$ liegenden
$r$-Werte d.h. $n=\#\{r_i\vert r_i\in I\}$. Um den Skalierungsbereich zu finden, wird nun $L(I)$ 
f"ur alle m"oglichen Intervalle berechnet und dasjenige, welches das maximale $L$
liefert, als Skalierungsbereich benutzt. Hier sind jedoch zwei Einschr"ankungen zu
beachten. Zum einen sollte eine Mindestl"ange f"ur die Intervalle vorgegeben
sein. Ansonsten tendiert der Algorithmus dazu, sehr kleine Intervalle auszuw"ahlen. Zum
anderen m"ussen Unter- und Obergrenzen f"ur $r$ vorgegeben werden, da das
Korrelationsintegral im Bereich des Digitalisierungsrauschens und der S"attigung  auch
einen linearen Bereich (mit Steigung null) besitzt. F"ur das Korrelationsintegral aus \psref{corrintperf} 
liefert das Verfahren die Werte $\log \rmin = -5,7$ und $\log \rmax = -3,6$. Diese stimmen gut 
mit denen "uberein, die man auch aus \psref{corrslpperf} absch"atzen w"urde.

\paragraph{Takens' Sch"atzverfahren}
\label{chaptakensest}
Die Berechnung der Korrelationsdimension "uber eine Regressionsgerade durch ein auf
irgendeine Weise festgelegtes Intervall hat zwei Nachteile. Zum einen bleiben
Informationen "uber den Verlauf von $C(r)$ unterhalb von $\rmin$ ungenutzt. Gerade kleine
Werte von $r$ sollten jedoch nach \eqnref{cdimdef} Informationen "uber die
Korrelationsdimension liefern. Zum anderen wird nicht ber"ucksichtigt, da"s
aufeinanderfolgende Werte von $C(r)$ nicht voneinander unabh"angig sind. Einen Ausweg
hieraus bietet das Takenssche Sch"atzverfahren \cite{Takens85a}. 

Wir m"ussen hier einen wahrscheinlichkeitstheoretischen Ansatz f"ur das
Korrelationsintegral machen. Betrachten wir die Wahrscheinlichkeit $\Prob$, da"s zwei
Attraktorpunkte $\x_i$ und $\x_j$ einen Abstand $\rij$ kleiner als $r$ haben, so erhalten wir
\eqn{\Prob(\norm{\x_i-\x_j}<r)=\frac{\sum\limits_{i\neq
      j}\Theta(r-\norm{\x_i-\x_j})}{N(N-1)}=C(r) .}

Die betrachtete Wahrscheinlichkeit ist also genau gleich dem Korrelationsintegral $C(r)$.
Da uns nur das Skalierungsverhalten unterhalb des S"attigungsbereiches interessiert,
legen wir eine Obergrenze $\rmax<\rsaett$ f"ur $r$ fest. Alle Punktepaare mit Abstand
$r\geq\rmax$ werden ignoriert und wir betrachten die bedingte Wahrscheinlichkeit $\tilde
\Prob$, da"s zwei dieser Punktepaare einen Abstand kleiner $r$ haben m"ogen:
\eqn{\tilde \Prob(\rij<r)=\Prob(\rij<r \vert \rij < \rmax) = \frac{C(r)}{C(\rmax)}   .}

Da $C(r)$ im Skalierungsbereich idealerweise wie $kr^\corrdim$ skaliert, verh"alt sich
$\tilde \Prob(\rij<r)$ idealerweise wie $(r/\rmax)^\corrdim$. 

Unsere Aufgabe ist es nun, den freien Parameter $\corrdim$ der Wahrscheinlichkeitsverteilung
$\tilde\Prob$ abzusch"atzen.  Dies geschieht in der Statistik "ublicherweise durch
Anwendung der \begriff(Maximum-Likelihood-Regel).  Sie besagt, da"s als Sch"atzwert f"ur
einen unbekannten Parameter einer Wahrscheinlichkeitsverteilung ein solcher Wert des
Parameters verwendet wird, bei dessen Vorliegen der konkreten Stichprobe eine m"oglichst
gro"se Wahrscheinlichkeit zukommt.


Die vorliegende konkrete Stichprobe
besteht aus den gemessenen Abst"anden $r_{ij}<\rmax$. Aus Gr"unden der Vereinfachung
"andern wir hier die Indizierung und betrachten die Folge $\folge(r,1,m)$, welche alle
$r_{ij}$ beinhalten soll.  Die Wahrscheinlichkeit $L$, eine solche Stichprobe zu erhalten,
ist das Produkt der Wahrscheinlichkeiten f"ur die Messung jedes einzelnen Abstands:
\eqn{L_\mathrm{disk}(\folge(r,1,m);\corrdim) = \prod_{i=1}^m\tilde\Prob(r=r_i) .}
Die so definierte \begriff(Likelihood-Funktion) ist jedoch nur f"ur diskrete
Wahrscheinlichkeitsverteilungen verwendbar.  Da die vorliegende Verteilung kontinuierlich
ist, ist die Wahrscheinlichkeit f"ur die exakte Gleichheit $\tilde\Prob(r=r_i)$ null, und
somit ist auch die Likelihood-Funktion identisch null. Es kann nun gezeigt werden, da"s f"ur
kontinuierliche Verteilungen die Wahrscheinlichkeitsdichte ma"sgeblich ist\footnote{Man
  betrachtet statt den Wahrscheinlichkeiten $\Prob(r=r_i)$ die Wahrscheinlichkeiten $\Prob(r_i\leq r <r_i+
  dr_i) = f(r_i)dr_i$. Das Maximum der Likelihood-Funktion ist unabh"angig von
  der Wahl der $dr_i$, so da"s direkt die Wahrscheinlichkeitsdichte $f$ benutzt werden kann.  }.
Somit erhalten wir als Likelihood-Funktion:
\eqnl[maxlike1]{L(\folge(r,1,m);\corrdim) = \prod_{i=1}^m f(r_{i}) ,}
wobei die Wahrscheinlichkeitsdichte durch
\eqn{f(r) = \abl{\tilde \Prob}{r} =   \corrdim  (r/\rmax)^{\corrdim-1}  }
gegeben ist.  Der Sch"atzwert f"ur $\corrdim$ ist nun derjenige, f"ur den $L$ ein Maximum
annimmt. 

Da die Ableitung des Produkts in \eqnref{maxlike1} sowie die Berechnung ihrer
Nullstellen recht kompliziert ist, verwendet man einen \naja(Standardtrick). Da der
Logarithmus eine streng monoton wachsende Funktion ist, besitzen $L$ und $\ln
L$ die gleichen Extremstellen. Das Produkt geht hierbei in eine Summation "uber, und die
Berechnung des Maximums vereinfacht sich erheblich:
\eqna{\abl{\ln L(\corrdim)}{\corrdim}&=& \abl{}{\corrdim}\left( m\ln\corrdim + (\corrdim-1)\sum_{i=1}^m\ln (r_i/\rmax) \right)\\
&=& \frac{m}{\corrdim}+\sum_{i=1}^m\log (r_i/\rmax).}

Nullsetzen der Gleichung ergibt den Sch"atzwert f"ur $\corrdim$:
\eqn{\corrdim=-\frac{m}{\sum_{i=1}^m \ln(r_i/\rmax)}=-\frac{1}{<\ln (r_i/\rmax)>} .}

 \autor(Takens) zeigte, da"s
diese Sch"atzung erwartungstreu ist, was bei Maximum-Likelihood-Methoden
nicht immer der Fall ist\footnote{F"ur eine \begriff(erwartungstreue) Sch"atzung ist
  der Erwartungswert des Sch"atzwerts gleich dem zu sch"atzenden Wert. Beispielsweise ist
  die Sch"atzung $s^2=\frac1{N-1}\sum_{i=1}^N(x_i-\bar x)^2$ eine erwartungstreue
  Sch"atzung f"ur die Varianz $\sigma^2$ der Stichprobe $(\folge(x,1,N))$, da
  $<s^2>=\sigma^2$ gilt. Die "uber die Maximum-Likelihood-Methode gewonnene Sch"atzung
  ${s^*}^2=\frac1{N}\sum_{i=1}^N(x_i-\bar x)^2$ ist nur \begriff(asymptotisch
  erwartungstreu), da nur f"ur den Grenzwert $\lim_{N\to\infty}<{s^*}^2> = \sigma^2$ gilt.
  Maximum-Likelihood-Sch"atzungen sind immer mindestens asymptotisch erwartungstreu. }.
Weiterhin ist diese Sch"atzung die \begriff(wirksamste), da die Varianz des
Sch"atzwertes f"ur diese Sch"atzung ein Minimum annimmt. 

Da das Korrelationsintegral, wie oben gesehen, eine Wahrscheinlichkeitsverteilung f"ur die
$r_i$, darstellt, kann der Erwartungswert $<\ln (r_i/\rmax)>$ jetzt "uber $C(r)$ ausgedr"uckt
werden. Die Wahrscheinlichkeit, bei der gemessenen Verteilung einen Wert zwischen $r$ und
$r+\d r$ zu messen, betr"agt $\frac{\d  C(r)}{\d r}\frac{\d r}{C(\rmax)}$. Wir erhalten
also f"ur ein gegebenes $\rmax$:
\eqn{\corrdim(\rmax) = \int\limits_0^\rmax \abl{C(r)}{r}\frac{\ln (r/\rmax)}{C(\rmax)}\d r .}
Da wir das Korrelationsintegral jedoch nur f"ur diskrete Werte $r_i$ bestimmen k"onnen,
geht die obige Gleichung "uber in:
\eqn{\corrdim(r_n) = \sum\limits_{i=1}^{n-1} \frac{C(r_i+1)-C(r_i)}{C(r_n)}\ln(r_i/r_n), }
wobei $r_n=\rmax$ gesetzt wurde.  Ein Vergleich dieser Methode mit der im vorigen
Abschnitt beschriebenen zeigt, da"s Takens' Sch"atzverfahren bei bekannten Systemen im
allgemeinen bessere Werte liefert (siehe \psref{corrdimcomp}); die berechneten Werte der
Korrelationsdimension $\corrdim=2,07\pm0.01$ stimmen deutlich besser mit den theoretischen
Werten "uberein.

\epsfigdouble{corrint/perfect/corrdim700b}{corrint/perfect/takdim700b} { Vergleich der
  Bestimmung der Korrelationsdimension $\corrdim$ in Abh"angigkeit von der
  Einbettungsdimension $\embed$ durch Regressionsgeraden (oben) bzw.\ Takens' Sch"atzverfahren
  (unten) f"ur das Korrelationsintegral aus \psref{corrintperf}. 
}  {corrdimcomp}{-0.2cm}

\subsubsection{Fehler bei der Korrelationsanalyse}
Das Korrelationsintegral in \psref{corrintperf} ist nahezu optimal im Sinne des erwarteten 
theoretischen Verlaufs. Der Skalierungsbereich 
erstreckt sich "uber einige Gr"o"senordnungen ($\rmax/\rmin\simeq 54)$ und die Steigung innerhalb
dieses Bereichs ist nahezu konstant (siehe \psref{corrslpperf}). Der Grund liegt vor allem
darin, da"s f"ur die Berechnung eine gro"se Anzahl Datenpunkte ($7\times10^5$) vorlag und
mit nahezu rauschfreien Daten gearbeitet wurde. In experimentellen Situationen liegt
beides meist nicht vor. Die Fehler, die hieraus (und auch aus anderen Quellen) resultieren,
sollen im folgenden diskutiert werden.




\paragraph{Endliche Datenmengen}
Ein Faktor, der die verl"a"sliche Bestimmung der Korrelationdimension wesentlich
beeinflu"st, ist die Menge der verf"ugbaren Daten. Das Korrelationsintegral hat einen
Wertebereich von $2/N(N-1)$ bis $1$. Da $C(r)$ f"ur $r<\rmax$ mit $(r/\rmax)^\corrdim$ skaliert, tragen
bei Abst"anden der Gr"o"senordnung $\rmax(2/N^2)^{1/\corrdim}$ nur noch wenige Punktepaare 
zum Korrelationsintegral bei. Daraus resultieren in diesem Bereich starke statistische
Schwankungen von $\mcorrdim(r)$. Aus diesem Verhalten leiteten \autor(Eckmann) und \autor(Ruelle)
\cite{Eckmann-ruelle2} eine Gleichung f"ur die minimal erforderliche Datenmenge f"ur
Dimensionsberechnungen ab. Bezeichnet man das Verh"altnis von gr"o"sten zu kleinsten
Skalen mit $\rho$, so ergibt sich f"ur die minimale Datenmenge:
\eqnl[corrdimnminer]{N_\tmin=\sqrt{2\rho^\corrdim}}
oder andererseits bei fester Datenmenge die maximal berechenbare Dimension:
\eqnl[corrmaxdim]{D_{2,\tmax}=\frac{2\log N}{\log \rho}.}
Nun mu"s f"ur eine vern"unftige Absch"atzung der Dimension das Skalenverh"altnis $\rho$
hinreichend gro"s sein. Eckmann und Ruelle geben hier ein minimales Verh"altnis von
$\rho_\tmin=10$ an, so da"s man etwa bei $N=1000$ Datenpunkten maximal eine
Korrelationsdimension $\corrdim\leq 6$ sinnvoll bestimmen kann. Diese Grenze scheint mir
allerdings sehr hoch, da sich bei $N=1000$ schon die Dimensionen von Attraktoren mit
$\corrdim\simeq 2$ nur sehr schlecht bestimmen lassen. Andererseits kann das Ergebnis gut
als wirkliche obere Grenze f"ur die Dimensionsberechnung angesehen werden. 




\paragraph{Kanten und endliche Ausdehnung des Attraktors}
Wie wir bereits in \psref{corrintperf} gesehen haben, geht das Korrelationsintegral ab einem
bestimmten Wert $\rsaett$ in einen S"attigungsbereich "uber. Dies ist eine Konsequenz der
endlichen Ausdehnung des Attraktors. Effekte, die aus dem S"attigungsverhalten resultieren, 
sind schon auf L"angen\-skalen weit unterhalb der linearen Ausdehnung des Attraktors
erkennbar. Sobald eine wesentliche Anzahl Punkte einen Abstand von weniger als $r$ vom
\naja(Rand) des Attraktors hat, weicht das Skalierungsverhalten des Korrelationsintegrals
stark vom theoretischen Verlauf ab.

Der Wert $r_s$, f"ur den das Skalierungsverhalten durch den \begriff(Kanteneffekt)
(engl.: edge effect) abbricht, h"angt stark von
der Dimension und der Geometrie des Attraktors ab. 
F"ur einen $m$-dimensionalen Hyperkubus der Kantenl"ange 1 kann das 
Korrelationsintegral jedoch exakt berechnet werden\footnote{Wie bereits gezeigt wurde gilt f"ur das Korrelationsintegral
$C(r)=\Prob(\rij<r)$. Bei einem homogenen Hyperkubus sind die einzelnen Komponenten des
Abstandsvektors voneinander statistisch unabh"angig, und es gilt bei Verwendung der
Maximumsnorm $\Prob(\rij)=\Prob((\vec r_{ij,1} < r) \land\dots\land (\vec r_{ij,m} < r)) = \Prob_1(\vec
r_{ij,1} < r)\cdot\dots\cdot \Prob_1(\vec r_{ij,m} < r)$. Da au"serdem die Verteilungen der
einzelnen Komponenten gleich sind folgt $\Prob(\rij)=\Prob_1(\vec r_{ij,1} < r)^m$. F"ur die
Verteilung $\Prob_1$ gilt nun $\Prob_1(\rij<r)=\int_0^1 \{\min(1,r'+r)-\max(0,r'-r)\}\d r'=
2r-r^2$. Somit folgt die Behauptung.
}:
\eqnl[corrinthyper]{C(r)=(2r-r^2)^m.}
Unter der vereinfachenden Annahme, da"s sich der Attraktor "ahnlich einem Hyperkubus
der Dimension $\corrdim$ und der Kantenl"ange $\rmax$ verh"alt, kann die Korrelationsdimension entsprechend
\eqnref{corrinthyper} durch Auftragung von $\log C(r)$ "uber $\log(2r/\rmax-(r/\rmax)^2)$
abgesch"atzt werden. Die sich hieraus ergebenden Korrekturen  
sind jedoch "au"serst gering (wenige Promille).

Die Eigenschaft des Korrelationsintegrals ab einer bestimmten Obergrenze in S"attigung zu
gehen wurde von \autor(Nerenberg) und \autor(Essex) benutzt, um eine Untergrenze der
ben"otigten Datenmenge zur Berechnung der Korrelationsdimension seltsamer Attraktoren
 zu bestimmen \cite{Nerenberg-essex}. Ihre Berechnung beruht darauf, da"s der Skalierungsbereich
von unten durch die Menge der verf"ugbaren Daten beschr"ankt wird. Kennzeichnend hierf"ur
ist der charakteristische Abstand n"achster Nachbarn $r_n$. Von oben ist der Skalierungsbereich
beschr"ankt durch den durchschnittlichen Abstand der Punkte zum Rand des Attraktors
$r_s$. Fallen beide zusammen, verschwindet der Skalierungsbereich und eine
Dimensionsbestimmung ist nicht mehr m"oglich. Beide Werte werden nun wiederum f"ur einen
homogenen Hyperkubus der Dimension $m$ und der Kantenl"ange 1 berechnet:
\eqna{r_n &=& \frac{1}{2(m+1)}\nonumber\\
r_s &=& N^{-1/m} .}
Gleichsetzen der beiden Grenzen $r_n$ und $r_s$ ergibt:
\eqnl[corrnminnea]{N_\tmin(m)=\{2(m+1)\}^m .}
Zur Berechnung der Dimension eines Attraktors der Korrelationsdimension $\corrdim$ sind nach
dieser Absch"atzung $N_\tmin(\lceil \corrdim \rceil)$ Datenpunkte n"otig\footnote{$\lceil x \rceil$
bedeutet hier die n"achste ganze Zahl gr"o"ser oder gleich $x$.}. Zum Vergleich mit der
weiter oben angegebenen Formel von Eckmann  und Ruelle setzen wir das Verh"altnis
von $r_s$ zu $r_n$ gleich $\rho$. Damit erhalten wir:
\eqnl[corrnminneb]{N_\tmin(m,\rho)=\{2(m+1)\rho\}^m.}
Dies ergibt beispielsweise f"ur Attraktoren der Dimension 2 eine minimale Datenmenge von
ca.\  4000 Punkten bzw.\  500000 Punkte f"ur die Dimension 3. Diese Absch"atzung deckt
sich eher mit meinen Erfahrungen bei Dimensionsberechnungen. 





\paragraph{Wei"ses und Digitalisierungsrauschen}
Der Skalierungsbereich des Korrelationsintegrals ist nach unten, au"ser durch die Menge der 
verf"ugbaren Daten, durch Rauschen beschr"ankt. Die Auswirkungen beider Arten von
Rauschen -- wei"sem und Digitalisierungsrauschen -- m"ussen zun"achst getrennt behandelt
werden. 

Wir betrachten dem Signal "uberlagertes wei"ses Rauschen der St"arke $\xi$. F"ur Abst"ande 
$r>\xi$ spielt das Rauschen nur eine untergeordnete Rolle. Die Abst"ande zwischen den
Attraktorpunkten werden nicht wesentlich beeinflu"st, und das Korrelationsintegral skaliert 
wie bei rauschfreien Daten. 
Im Bereich $r<\xi$ wird das Signal jedoch stark vom Rauschen dominiert. Die
Wahrscheinlichkeit, in der Umgebung eines Attraktorpunktes einen weiteren Punkt zu finden,
verh"alt sich also wie bei einem reinen Rauschsignal. Da Rauschen jedoch immer den ganzen
Phasenraum aufspannt, skaliert das Korrelationsintegral hier mit der Einbettungsdimension
$d$, d.h.\  $C(r)\propto r^d$. Dieser Effekt ist dargestellt in \psref{corrnoise} oben.
Man erkennt deutlich das \naja(Abknicken) des Korrelationsintegrals bei ca.\  $\ln r=-4$. 

Bei der Umwandlung analoger in digitale Signale wird das Signal in diskreten Stufen
(sogenannten \begriff(Quantisierungsstufen)) abgetastet.  Die hierbei erzeugten
(zuf"alligen) Fehler werden als \begriff(Digitalisierungs-) oder
\begriff(Quantisierungsrauschen) bezeichnet.  Die H"ohe der Quantisierungsstufen sei
$\eps$.  Dann ist jeder Me"swert und somit auch jeder Abstand zwischen Attraktorpunkten
ein ganzzahliges Vielfaches von $\eps$ \footnote{Vorausgesetzt der Abstand wird "uber
  die Maximums- oder die Absolutnorm bestimmt.}. Das Korrelationsintegral verl"auft
daher nicht kontinuierlich sondern in Spr"ungen bei jedem $r=n\eps$ ($n\in\N$) (siehe
\psref{corrnoise} unten).  Zus"atzlich werden verschiedene Attraktorpunkte auf ein und
denselben Punkt abgebildet, so da"s Punkte mit $\rij=0$ mit endlicher Wahrscheinlichkeit
auftreten. Ein Algorithmus zur Berechnung der Korrelationsdimension mu"s mit diesen
Punkten umgehen k"onnen.

Ein von \autor(Theiler) hergeleitetes Modell f"ur das Skalierungsverhalten des
Korrelationsintegrals eines $m$-dimensionalen Gitters ergibt $C(r)\propto(r+\eps/2)^m$ \cite{Theiler}. Er 
schl"agt daher vor, bei einer diskretisierten Zeitreihe $\log C(r)$ gegen $\log(r+\eps/2)$
aufzutragen. Diese Korrektur hat nach meiner Erfahrung jedoch wenig
Einflu"s auf tats"achliche Dimensionsbestimmungen, sondern hilft eher, Singularit"aten 
bei der Berechnung von $\log r$ zu vermeiden.
\comment{Sei $r_\eps$ das n"achste Vielfache von $\eps$ kleiner als $r$. Dann wird
$r$ mit 50 prozentiger Wahrscheinlichkeit auf $r_\eps$ bzw auf $r_\eps+\eps$ abgebildet. F"ur 
das Korrelationsintegral der diskretisierten Daten $C'(r)$ folgt daraus
$C'(r)=\{C(r_\eps)+C(r_\eps+\eps)\}/2$. N"aherung und ausnutzung des urspr"unglichen
Skalierungsverhaltens f"uhrt auf den oben angegebenen Term.}

Um Rauschen zu vermindern k"onnen verschiedene Filtertechniken angewandt werden. Die
einfachsten hiervon sind Tief- oder Bandpa"sfilter. Diese haben jedoch den Nachteil, da"s
ihre Anwendung dem System einen Freiheitsgrad hinzuf"ugt und somit auch die gemessene
Dimension erh"ohen kann (siehe Abschnitt \ref{chapcorrdimfiltered}). Zudem ist im
allgemeinen nicht direkt ersichtlich, ab welcher Oberfrequenz gefiltert werden sollte. Das
Abschneiden aller Frequenzen oberhalb einer bestimmten Grenzfrequenz birgt auch die
Gefahr, da"s eventuell f"ur die Dynamik wesentliche Informationen verloren gehen.

Das Verfahren, welches hier zum Einsatz kommt, beruht auf der in Abschnitt \ref{chapsvd}
beschriebenen Methode der Singular Value Decomposition. Hierbei verwenden wir allerding
nur die erste Komponente der Rekonstruktion, welche "uber die
Verz"ogerungskoordinatenabbildung dann wieder eingebettet wird (\begriff(Reembedding)).
Die oben erw"ahnten Nachteile 
anderer Verfahren teilt diese Methode nicht, da sie sich automatisch dem System anpa"st
und das Problem der Dimensionserh"ohung hier prinzipiell nicht auftreten kann.

%Das Ergebnis des Verfahrens sieht man in \psref{corrfilter}. 
Das Verfahren der SVD-Filterung wurde auf die in \psref{corrnoise} benutzten verrauschten
bzw.\  diskretisierten Zeitreihen angewendet.
Der Skalierungsbereich der
Zeitreihe mit "uberlagertem wei"sen Rauschen hat sich deutlich vergr"o"sert (siehe \psref{corrfilter}
oben); dies erkennt man auch sehr gut am direkten Vergleich der Slopekurven in
\psref{corrslpfilter}. Im Korrelationsintegral der diskretisierten Zeitreihe sind die
Stufen nach der SVD-Filterung nicht mehr zu erkennen (siehe \psref{corrfilter} unten).


\epsfigdouble{corrint/errors/noise/svd/corrint10}{corrint/errors/discrete/corrint05b}
{Oben das Korrelationsintegral zu einer Zeitreihe des Lorenz-Systems dem wei"ses
Rauschen "uberlagert ist. Die Varianz des Rauschsignals betr"agt 1,0 Prozent der Varianz des
Originalsignals. Unten wurde dasselbe Signal diskretisiert mit Quantisierungsstufen von 0,5 Prozent
der Varianz des Signals.
}{corrnoise}{-0.2cm}


\epsfigdouble{corrint/errors/noise/svd/corrint10sf}{corrint/errors/discrete/corrint5bsf}
{Korrelationsintegrale zu den gleichen Signalen wie in \psref{corrnoise}, wobei die
  Signale vor der Berechnung SVD-gefiltert wurden.
}{corrfilter}{-0.2cm}

\epsfigdouble{corrint/errors/noise/svd/corrslp10}{corrint/errors/noise/svd/corrslp10sf}
{Oben die Slopekurve zum Korrelationsintegral der verrauschten Zeitreihe aus
  \psref{corrnoise} (oben). Unten die Slopekurve zum Korrelationsintegral der entsprechenden
  SVD-gefilterten Zeitreihe aus \psref{corrfilter} (oben).
}{corrslpfilter}{-0.2cm}



\paragraph{Gefilterte Zeitreihen}
\label{chapcorrdimfiltered}
Wie im vorigen Abschnitt bereits erw"ahnt, kann die Filterung von Zeitreihen "uber einfache 
\begriff(Tief-) oder \begriff(Bandpa"sfilter) die Dimension des Systems erh"ohen \cite{Badii-politi}. Wir wollen dazu einen
einfachen Tiefpa"s erster Ordnung betrachten. Sei $x$ das Originalsignal und $z$ das
gefilterte Signal. Dann kann die Zeitabh"angigkeit des gefilterten Signals durch die
folgende Differentialgleichung beschrieben werden:
\eqnl[corrlowpass]{\dot z(t) = -\eta z(t) + x(t).}
Hierbei ist $\eta$ die Grenzfrequenz des Filters. Offensichtlich wird die Zahl der
Freiheitsgrade des Systems durch Anwendung dieses Filters um eins erh"oht. Da"s durch den
Filter auch die Dimension des Systems erh"oht werden kann, zeigten \autor(Badii \etal) 
\cite{Badii-politi}. Die Gleichung f"ugt dem System einen neuen Lyapunov-Exponenten
$\lambda_f=-\eta$ hinzu. Dies f"uhrt je nach Gr"o"se von $\lambda_f$ in Bezug auf die
anderen Lyapunov-Exponenten des Systems zu einer Erh"ohung der Lyapunov-Dimension
$D_L$. Unter der Annahme der G"ultigkeit der  Kaplan-Yorke-Vermutung $D_L=D_1$ f"uhrt dies
auch zu einer Zunahme der Informationsdimension $D_1$. Da"s auch die weiteren
verallgemeinerten Dimensionen erh"oht werden, kann hier nur vermutet werden, ist jedoch
wahrscheinlich.

Um dem entgegen zu wirken, ist von \autor(Chennaoui \etal) ein Verfahren entwickelt worden 
um den unbekannten Filterparameter $\eta$ zu bestimmen \cite{Chennaoui}. Durch Inversion von
\eqnref{corrlowpass} kann dann, mit bekanntem $\eta$, die originale Zeitreihe wieder
extrahiert werden. Das Verfahren beruht jedoch auf der Berechnung der
Informationsdimension und funktioniert auch nur f"ur Filter der oben beschriebenen
Art. F"ur \begriff(akausale Filter)\footnote{Akausale Filter sind Filter, deren
Sprungantwort $h(t)$ bereits f"ur $t<0$ ungleich 0 ist. Der \begriff(ideale) Tiefpa"s
ist ein Beispiel f"ur solch einen Filter. Akausale Filter sind zwar in Echtzeit nicht zu
realisieren, ihrer Anwendung auf komplett vorliegende Zeitreihen steht jedoch nichts
entgegen.} konnte \autor(Mitschke)  allerdings zeigen, da"s dieses Problem nicht 
auftaucht \cite{Mitschke}.

\paragraph{Autokorrelation}
\label{corrdimtheiler}
Aus deterministischen Systemen gewonnene Signale sind grunds"atzlich autokorreliert. Es
existiert eine Autokorrelationszeit $\tau_\ac$, so da"s 
$x(t)$ und $x(t+\tau)$ f"ur Zeiten $\tau<\tau_\ac$ stark miteinander korreliert sind. Falls die Autokorrelationszeit gro"s 
gegen sie Sampling Time $\sample$ ist, kann im Korrelationsintegral eine anormale Stufe
auftreten. Nach einem Vorschlag von \autor(Theiler) l"a"st sich dies vermeiden, indem das
Korrelationsintegral nur "uber Punktepaare  gebildet wird, die zeitlich mindestens
$W\sample>\tau_\ac$ auseinanderliegen \cite{Theiler}. Das so korrigierte Korrelationsintegral lautet dann:
\eqnl[cintdefac]{C(r) = \frac{2}{(N-W+1)(N-W)}\sum_{i<=j-W}\Theta(r-\norm{\x_i-\x_j}_\infty).}
F"ur $W=1$ geht dies wieder in die urspr"ungliche Form von \eqnref{cintdef2} "uber. Bei
den hier untersuchten Zeitreihen trat dieser Effekt niemals deutlich auf. Da diese
Korrektur jedoch die Anzahl der in die Berechnung eingehenden Punktepaare nicht
wesentlich herabsetzt, wurde sie in allen Berechnungen des Korrelationsintegrals zur
Sicherheit vorgenommen.

\paragraph{Lakunarit"at}
Die fraktale Struktur einer Menge kann au"ser durch ihre Dimension auch
durch die sogenannte \begriff(Lakunarit"at) (lat.\ lacuna = Loch, H"ohle), einem von
\autor(Mandelbrot) \cite{Mandelbrot82} gepr"agten Begriff, charakterisiert werden.  Bei
Fraktalen gleicher Dimension scheint dasjenige mit der h"oheren Lakunarit"at eine feinere
ver"asteltere Struktur zu besitzen.  Ein Beispiel einer Cantor-Menge
mit relativ hoher Lakunarit"at zeigt \psref{lacunarity}.

%\vspace{0.7cm}
\epsfigsingle{corrint/errors/lacunarity/sevenb}
{Die ersten f"unf Konstruktionschritte einer Cantor-Menge. In jedem Schritt werden die
Intervalle in sieben gleich gro"se Teilintervalle unterteilt von denen jeweils
das zweite, vierte und sechste Teilintervall gestrichen wird. 
}{lacunarity}{-0.2cm}

Die Lakunarit"at eines Fraktals hat Auswirkungen auf die Dimensionsbestimmung. Die
Relation $C(r)\propto r^\corrdim$ ist hier falsch, da $C(r)$ eine stufenf"ormige Funktion
ist \cite{Broggi88}.  Das Resultat sind Oszillationen im Korrelationsintegral, d.h.\ 
$C(r)$ ist nun proportional zu $k(r)r^\corrdim$ (siehe \psref{corrintlac} oben), wobei
$k(r)$ im allgemeinen periodisch in $\log r$ ist; diese Periodizit"at l"a"st sich
besonders gut in den berechneten Slopekurven erkennen (siehe \psref{corrintlac} unten).
Der Einflu"s der Lakunarit"at kann minimiert werden, wenn bei der Dimensionbestimmung
"uber volle Perioden von $k(r)$ gemittelt wird. Der so berechnete Wert $\corrdim=0,718$
der hier betrachteten Cantor-Menge 
stimmt sehr gut mit dem theoretischen Wert $\corrdim=\ln4/\ln7\simeq 0,712$
"uberein\footnote{Da es sich hier um ein homogenes Fraktal handelt, gilt $D_q=\const$,
  also auch $\corrdim=D_0$. Die Korrelationsdimension kann somit "uber die g"angige Formel
  zur Berechnung der Kapazit"at von Cantor-Mengen erhalten werden.}.  \autor(Grassberger)
zeigte andererseits, da"s die Oszillationen im Grenzfall $r\to 0$ ausged"ampft
werden\cite{Grassberger88}.

\epsfigdouble{corrint/errors/lacunarity/canbcint}{corrint/errors/lacunarity/canbcslp}
{Korrelationsintegral (oben) und Slopekurve (unten) f"ur die Cantor-Mengen aus \psref{lacunarity}.
}{corrintlac}{-0.2cm}







\chapter{Anwendung}

Zitat siehe \autor(Theiler)
\begin{quote} \em
Es ist nicht schwierig einen Algorithmus zu entwickeln, der Zahlen liefert die als Dimension
bezeichnet werden k"onnen, aber es ist weit schwieriger, sicher zu sein, da"s diese Zahlen
wirklich die Dynamik des System repr"asentieren.
\end{quote}



\section{Physiologie Fr"uhgeborener}

\subsection{Atmungstypen}

\subsubsection{Regelm"a"sige Atmung}

\subsubsection{Periodische Atmung}

\section{Me"sverfahren}

\subsection{Torakale Impedanz (TI)}

\subsection{Volumenstrommessungen (Flow)}

\section{Analyse der Atmungsdaten}

\subsection{Spektren}

\subsubsection{Filterung}

\subsection{Bestimmung der Einbettungsparameter}

\subsection{Rekonstruktion}

\subsubsection{Singular Value Decomposition}

\subsection{Korrelationsdimension}

\subsubsection{Regelm"a"sige Atmung (TI)}

\subsubsection{Periodische Atmung (TI)}

\subsubsection{Regelm"a"sige Atmung (Flow)}

\subsection{Determinismustests}

\epsfigdouble{surrogate/aaft/regel/timeseries}{surrogate/aaft/regel/surronew}
{Links die Originalzeitreihe (Regelm"a"siges Atmen), rechts eine typische, durch AAFT
erzeugte Surrogat-Zeitreihe. Zur besseren Vergleichbarkeit  werden nur die ersten 30
Sekunden adrgestellt.}{surreg1}{-0.2cm}

\epsfigdouble{surrogate/aaft/regel/corrslp}{surrogate/aaft/regel/surcslp}
{Vergleich der Steigungen des Korrelationsintegrals $C(r)$ f"ur Original- (links) und eine
typische Surrogat-Zeitreihe (rechts).}{surreg2}{-0.2cm}

\epsfigdouble{surrogate/aaft/regel/cmpsurcdim}{surrogate/aaft/regel/signi}
{Links: Gemessene Korrelationsdimension f"ur Original- \captimes und Surrogatdaten
\capplus. Der Skalierungsbereich wurde anhand von \psref{surreg2} f"ur beide 
zu $-1.5\leq\ln r\leq\-0.9$ gew"ahlt. Rechts: Die Signifikanz $\Delta\nu/\sigma$ liegt
f"ur Einbettungsdimensionen $d\geq 2$ deutlich "uber $3$.}{surreg3}{-0.2cm} 


\newpage

\comment{
\epsfigdouble{surrogate/ft/regel/timeseries}{surrogate/ft/regel/surronew}
{Links die Originalzeitreihe (Regelm"a"siges Atmen), rechts eine typische, durch FT
erzeugte Surrogat-Zeitreihe. Zur besseren Vergleichbarkeit  werden nur die ersten 30
Sekunden adrgestellt.}{surareg1}{-0.2cm}
}

\epsfigdouble{surrogate/ft/regel/corrslp}{surrogate/ft/regel/surcslp}
{Vergleich der Steigungen des Korrelationsintegrals $C(r)$ f"ur Original- (links) und eine
typische Surrogat-Zeitreihe (rechts).}{surareg2}{-0.2cm}

\epsfigdouble{surrogate/ft/regel/cmpsurcdim}{surrogate/ft/regel/signi}
{Links: Gemessene Korrelationsdimension f"ur Original- \captimes und Surrogatdaten
\capplus. Der Skalierungsbereich wurde anhand von \psref{surareg2} f"ur beide 
zu $-1.5\leq\ln r\leq\-0.9$ gew"ahlt. Rechts: Die Signifikanz $\Delta\nu/\sigma$ liegt
f"ur Einbettungsdimensionen $d\geq 2$ deutlich "uber $3$.}{surareg3}{-0.2cm} 

\epsfigdouble{surrogate/ft/regel/B/cmpsurcdim}{surrogate/ft/regel/B/signi}
{Links: Gemessene Korrelationsdimension f"ur Original- \captimes und Surrogatdaten
\capplus. Der Skalierungsbereich wurde anhand von \psref{surareg2} f"ur beide 
zu $-1.5\leq\ln r\leq\-0.9$ gew"ahlt. Rechts: Die Signifikanz $\Delta\nu/\sigma$ liegt
f"ur Einbettungsdimensionen $d\geq 2$ deutlich "uber $3$.}{suraregb3}{-0.2cm} 












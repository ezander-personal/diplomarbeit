\chapter{Anwendung}

% \section{Physiologie Fr"uhgeborener}

\section{Zur Atmung von Fr"uhgeborenen}


\subsection{Problemstellung}

Die Fortschritte auf dem Gebiet der intensivmedizinischen Betreuung w"ahrend der letzten
20 Jahre erm"oglichen zunehmend das "Uberleben kleiner und sehr kleiner Fr"uhgeborener\footnote{Als
  \begriff(Fr"uhgeborene) bezeichnet man S"auglinge mit einem Gestationsalter unter 37
  Wochen. Die Angabe \naja(klein) bzw.  \naja(sehr klein) bezieht sich auf das
  Geburtsgewicht von unter 1500 g bzw.\ unter 1000 g.}.
In der auf die Zeit der Intensivbehandlung folgenden Zeit der
Stabilisierung und ``Aufzucht'' sind Schwestern und "Arzte mit verschiedenen durch die
Unreife bzw.  Instabilit"at dieser Patienten bedingten Problemen konfrontiert. Hierunter
erfordert die Instabilit"at der Atmung unter allen Reifealtern vordringliches Augenmerk:
Beim eigenst"andig atmenden Fr"uhgeborenen kann es ohne vorausgehende Anzeichen
zum Atemstillstand (\begriff(Apnoe)) kommen.  Dieser f"uhrt zum Sauerstoffmangel (\begriff(Hypox"amie))und
konsekutiv zu einem Abfall der Herzfrequenz (\begriff(Bradykardie)) auf kritische, die
Kreislaufversorgung gef"ahrdende Verh"altnisse.  Neben der Gefahr eines akuten
Herz-Kreislaufversagens bringen h"aufige Atemstillst"ande die Gef"ahrdung des sich
entwickelnden Gehirns durch latenten Sauerstoffmangel mit sich.  Die Kinder werden durch
Ableitung des EKG\footnote{Man spricht hier von einer \begriff(Ableitung) des EKG, da die 
  interessierenden Aktionspotentiale des Herzens nicht direkt gemessen sondern nur
  indirekt durch Anbringung von Elektroden auf der K"orperoberfl"ache \naja(abgeleitet)
  werden k"onnen.} und des atemsynchronen thorakalen Impedanzsignals (TI) fortlaufend
"uberwacht. Besonders gef"ahrdete Patienten werden zus"atzlich durch Messung der
Sauerstoffs"attigung des Blutes "uberwacht (\begriff(Pulsoxymetrie)).

\subsection{Atemrhythmen Fr"uhgeborener}

Heute wird vom Konzept eines lokalisierten Rhythmusgenerators der Atmung abgesehen.
Stattdessen ist experimentell ein verteiltes Netzwerk atemkompetenter Nervenzellen belegt.
Zur Zeit wird von sechs jeweils verschiedenen Atemphasen zugeordneten Nervenzellgruppen im
Hirnstamm ausgegangen. Ihre gemeinsame Endstrecke ist der das Zwerchfell stimulierende
\begriff(Nervus Phrenicus). Die f"ur die Atmung zust"andigen \comment{respiratorisch
  kompetenten} Gruppen unterliegen ihrerseits einer Anzahl peripherer (Lungendehnung,
$\mathrm{O}_2$-Gehalt des Blutes) und zentraler  (Wachheit) Einfl"usse, welche sie
integrativ verarbeiten. \comment{(\autor(Richter))}

F"ur die hier untersuchten Fr"uhgeborenen, werden im wesentlichen die
folgenden Atemrhythmen unterschieden, wobei diese Einteilung ph"anomenologisch erfolgt:
\begriff(regelm"a"siges) Atmen einerseits und \begriff(periodisches) Atmen andererseits.
Die Atemrhythmen k"onnen grob den verschiedenen Schlafphasen zugeordnet werden: W"ahrend
des REM-Schlafs tritt ausschlie"slich periodische, w"ahrend des Non-REM-Schlafs
haupts"achlich regelm"a"sige, aber auch periodische Atmung auf.

Der Begriff \begriff(regelm"a"sige Atmung) beschreibt das Auftreten regelm"a"siger
Atemz"uge gleicher Atemtiefe (siehe \psref{medrhythm} oben). F"ur den Terminus
\begriff(periodische Atmung) folgen wir der Definition von \autor(Kryger); siehe \cite{Hoch96}. F"ur ihn sind
die Kriterien dieses Atmungsmusters erf"ullt, wenn eine Folge von Atmungspausen von
jeweils mehr als 3 Sekunden L"ange, unterbrochen von regul"aren Atmungsperioden mit einer
Dauer von bis zu 20 Sekunden, vorliegt (siehe \psref{medrhythm} unten). Ein Zusammenhang
zwischen periodischer Atmung und dem Auftreten von Fr"uhgeborenenapnoen scheint
nicht gegeben \cite{Hoch96}.

\epsfigdouble
{anwendung/zeitreihen/regeln}
{anwendung/zeitreihen/periodn}
{Beispiele f"ur die verschiedenen Atemrhythmen: regelm"a"siges Atmen (oben) bzw.\  
  periodisches Atmen (unten). Dargestellt ist das atemkorrelierte thorakale
  Impedanzsignal.  }{medrhythm}{-0.5cm}

"Uber die Definition einer Fr"uhgeborenenapnoe bestehen einige unterschiedliche
Auffassungen. In vielen Ver"offentlichungen wird als Indikation f"ur eine Apnoe neben dem Stillstand der
Atmung  ein Abfall der Herzfrequenz und des Sauerstoffpartialdrucks des Blutes als
Kriterium angegeben \cite{Poets93}. Weitere Kennzeichen sind eine Zyanose oder eine Bl"asse
des Patienten. Da das Vorliegen einer Hypox"amie oder Bradykardie aus den vorliegenden
Daten (thorakale Impedanz) nicht hervorgeht, verwenden wir die Definition von
\autor(Glotzbach) (siehe \cite{Hoch96}), nach der bei einem Atemstillstand von mindestens 15 Sekunden eine 
Apnoe vorliegt.  Ein Beispiel einer ca.\ 20 Sekunden dauernden Apnoe w"ahrend einer Phase
regelm"a"siger Atmung zeigt \psref{medapnoe}.

\epsfigdouble
{anwendung/zeitreihen/reg_apnoe}
{anwendung/zeitreihen/reg_apnoe2}
{ Apnoe w"ahrend einer Phase regelm"a"siger Atmung. Unten
  ein vergr"o"serter Ausschnitt der oben dargestellten Zeitreihe. Aufgrund des eingeschr"ankten Me"sbereichs ist
  das Signal des der Apnoe vorausgehenden tiefen Atemzugs nach oben und unten abgeschnitten.
}{medapnoe}{-0.5cm}


\subsection{Stand der Forschung}

Mit erheblichem Aufwand wird versucht, die Bedingungen, welche zu Atemstillst"anden
f"uhren, zu identifizieren.  Nur wenige Arbeiten gehen bisher der Frage nach dem
dynamischen Charakter der Fr"uhgeborenenatmung nach. \autor(Pilgrim \etal) konnten
ausgehend von Messungen der Korrelationsdimension ($\corrdim<3$) f"ur die periodische
Atmung einen deterministischen Charakter dieses Atemrhythmus belegen \cite{Pilgram94}. Weitere Aussagen zum
dynamischen Charakter der Atmung im allgemeinen finden sich vor allem in
tierexperimentellen Arbeiten. \autor(Sammon) konnte in Versuchen mit narkotisierten Ratten
nachweisen, da"s durch Stimulation des Nervus Vagus eine Erh"ohung der Dimension der
Atmungsdynamik stattfindet \cite{Sammon91}.  \autor(Eldridge) konnte -- ebenfalls an einem
Experiment mit Ratten -- zeigen, da"s die experimentell gefundene Abh"angigkeit der durch
eine St"orung des Atemzyklus hervorgerufenen Reaktion von der Phasenlage durch
Modellierung mittels eines periodisch erregten \begriff(Van-der-Pol-Oszillators)
nachgebildet werden konnte \cite{Eldridge89}.  Insgesamt fehlen bisher f"ur das
menschliche Neugeborene deutliche Hinweise auf einen deterministischen Charakter des
Atemrhythmus. Von Interesse ist dies vor allem im Hinblick auf die m"ogliche
Vorhersagbarkeit von St"orungen der Atmungsaktivit"at.  Dar"uber hinaus spielt diese Frage auch f"ur das
Verst"andnis des Zusammenhangs zwischen der lebensnotwendigen Stabilit"at der Atmung
einerseits und den vielen Unterbrechungen des Atemrhythmus (Schlucken, Innehalten,
Sprechen etc.) andererseits ein Rolle.

\subsection{Signale, Daten und Patienten}

F"ur die vorliegende Arbeit standen Aufzeichnungen des thorakalen Impedanzsignals von acht
Fr"uhgeborenen zur Verf"ugung. Diese entstammen der Routine"uberwachung. Die Messung der
thorakalen Impedanz registriert den atemabh"angig schwankenden Wechselstromwiderstand des
Brustkorbes anhand eines Me"sstroms (30 kHz, 3 $\mu$A), welcher "uber zwei Elektroden
eingebracht wird. Sie bestimmt somit den Anteil des leitenden zum nichtleitenden Material
im Volumenleiter Thorax. Da nichtleitendes Atemgas periodisch ein und ausstr"omt, variiert
der gemessene Wechselstromwiderstand im wesentlichen mit der Atmung. Das Signal ist allerdings
wegen der Herzaktion und der sich aus dieser ergebenden "Anderung des leitenden Volumens
im Brustkorb auch an das Herzkreislaufsystem gekoppelt. 

Die Sampling Rate betrug 100 Hz bei einer Aufl"osung von 12 Bit. Die Me"swerte wurden
auf einen Bereich ganzer Zahlen von -2048 bis +2047 abgebildet, wobei das
Abbildungsverh"altnis und die Nullage jeweils dem Patienten angepa"st wurden. Die
Aufzeichnung erfolgte zusammen mit dem EKG "uber 24 Stunden. Nach Abschlu"s der Messungen
wurden 13 Zeitreihen mit regelm"a"siger und 12 mit periodischer Atmung ausgew"ahlt. Die
L"ange der ausgew"ahlten Zeitreihen betrug 4 min, was einem Datenumfang von je 24.000 Punkten
entspricht.  In 4 dieser Zeitreihen traten Apnoen auf.

Die bei den folgenden Auswertungen abgebildeten Darstellungen beruhen, soweit nicht anders 
angegeben, auf den Zeitreihen aus \psref{medrhythm}. Die Auswertungen selber wurden f"ur
alle vorliegenden Zeitreihen durchgef"uhrt, zeigten jedoch i.a.\    keine signifikanten
Unterschiede zu den beiden Beispielzeitreihen, so da"s auf deren separate Darstellung
verzichtet wurde.



\section{Auswertung} 

\subsection{Fourieranalyse}
Die gemessen Zeitreihen wurden zuerst einer Fourieranalyse
unterzogen. Die Fourierspektren von zwei ausgew"ahlten Zeitreihen zeigt \psref{medfourier}
(oben). Im Spektrum der regelm"a"sigen Atmung sind drei Peaks gut zu unterscheiden. Der
erste liegt bei ca.\ 0,8 Hz und, entspricht der mittleren Atemfrequenz des Kindes. Die
Atemfrequenzen Neugeborener liegen deutlich "uber denen Erwachsener bei ca.\ 
0,6-0,8 Hz, und k"onnen in Ausnahmef"allen auch bis 1,2 Hz reichen, ohne da"s dies als
pathologische Indikation zu werten w"are. Die beiden weiteren Peaks (bei 2,2 bzw. 4,3 Hz)
r"uhren von der Herzt"atigkeit her, was durch Vergleich mit den Spektren des gleichzeitig
aufgenommenen EKGs best"atigt wird (s. \psref{medfourier} unten). Auch die Herzfrequenz
Neugeborener liegt h"oher als die Erwachsener, wobei eine Fequenz von ca.\ 2 Hz als normal
anzusehen ist. Die deutlich zu erkennenden Oberfrequenzen werden vornehmlich hervorgerufen
durch die steilen Flanken des QRS-Komplexes.

Da die Peaks bei Frequenzen "uber 2 Hz i.allg. von der Herzt"atigkeit herr"uhren, ist es
h"aufige Praxis, auf die Zeitreihen Tiefpa"sfilter mit Grenzfrequenzen von 2 -- 2,5 Hz
anzuwenden. Da hierbei jedoch nicht klar ist, ob durch die Filterung eventuell f"ur die
Atmungsdynamik wesentliche Informationen verlorengehen, findet diese Vorgehensweise hier
keine Anwendung. 

Bei der periodischen Atmung liegen die Peaks an etwa der gleichen Stelle. Im dargestellten
Spektrum (\psref{medfourier} links) erkennt man Peaks bei 0,7 und 2,1 Hz sowie
andeutungsweise bei 4,3 Hz.  Die Auspr"agung ist jedoch weniger deutlich als bei der
regelm"a"sigen Atmung.


\epsfigfour
{anwendung/fourier/fr_r_3}
{anwendung/fourier/fr_p_2}
{anwendung/herzfreq/hfr_r}
{anwendung/herzfreq/hfr_p}
{Oben die Fourierspektren einer Zeitreihe des thorakalen Impedanzsignals bei regelm"a"siger (links)
  bzw. periodischer Atmung (rechts).  Darunter zum Vergleich die entsprechenden Spektren
  des gleichzeitig aufgenommen EKGs.
}
{medfourier}{-0.5cm}


\subsection{Bestimmung der Verz"ogerungszeit durch Redundanzanalyse}
Die Verz"ogerungszeiten f"ur die Phasenraumrekonstruktionen wurden "uber das Verfahren der 
Redundanzanalyse bestimmt (s. \psref{medredund}).  Die bestimmten Verz"ogerungszeiten
lagen bei 0,27--0,42 s f"ur regelm"a"sige und bei 0,24 -- 0,36 s f"ur periodische Atmung,
wobei der Unterschied nicht signifikant ist. Eine "Ubersicht "uber die gemessenen Werte
zeigt die untenstehende Tabelle. Da die Verz"ogerunsgzeiten eine recht geringe Streuung
aufweisen, wurde f"ur die meisten Auswertungen ein Standardwert von $\delay_r=0,32$ s
benutzt. 

\begin{center}
\begin{tabular}{|l||c|c|c|}
  \hline
  Atmungstyp & $\delay_{R,\tmin}/$s & $\delay_{R,\tmax}/$s & $\bar \delay_{R}/$s \\
  \hline
  regelm"a"sig   &  0,27 &  0,42   &  0,33$\pm$0,04 \\
  periodisch        &  0,24 &  0,36   &  0,31$\pm$0,03 \\
  \hline
\end{tabular}
\end{center}

\epsfigdouble
{anwendung/redund/mut_r_3}
{anwendung/redund/mut_p_1}
{Redundanzanalyse f"ur die Zeitreihen aus \psref{medfourier}. Die Verz"ogerungszeiten
  liegen bei $\delay_R=0,33 s$ f"ur regelm"a"siges bzw.\ bei $\delay_R=0,27 s$ f"ur
  periodisches Atmen. 
}
{medredund}{-0.5cm}

\subsection{Versuch einer Phasenraumrekonstruktion}
Mit den so gewonnenen Verz"ogerungszeiten wurden Bilder der Phasenraumrekonstruktionen
erzeugt. Es sei darauf hingewiesen, da"s diese Bilder {\em keine} g"ultigen
Rekonstruktionen eventuell zugrundliegender seltsamer Attraktoren darstellen. Sie sollen
nur die Struktur der Dynamik veranschaulichen, und Anhaltspunkte f"ur die weitere
Verfahrensweise geben.

In der Rekonstruktion des regelm"a"sigen Atemsignals erkennt man deutlich eine toroidale
Struktur (s. \psref{medrekonst} links oben). Diese ist allerdings gest"ort durch Rauschen
sowie das \naja(Hin-und-her-Driften) der Trajektorien entlang der Hauptdiagonalen, was auf
eine Instationarit"at der Zeitreihe hinweist. Die Rekonstruktion des periodischen Atmens
zeigt keine klare Struktur sondern eher ein \naja(Kn"auel) durcheinander laufernder
Trajektorien (s. \psref{medrekonst} rechts oben).  Um das sichtbar starke Rauschen zu
unterdr"ucken wurde, f"ur beide Signale eine Singular Value Decomposition durchgef"uhrt und
die erste Komponente der SVD-Rekonstruktion wiedereingebettet (s. \psref{medrekonst}
unten).

Die Instationarit"at der Zeitreihen l"a"st sich auch an "Anderungen Dichteverteilungen der
Me"swerte "uber verschiedene Zeitr"aume beobachten. \psref{medinstat} zeigt jeweils die
Verteilungen sowohl f"ur die gesamte als auch getrennt f"ur die erste und zweite H"alfte
der Zeitreihe. Zwischen den Verteilungen f"ur beide H"alften bestehen siginifikante
Unterschiede. Es sei an dieser Stelle nur darauf hingewiesen, da"s die bei diesen
Zeitreihen beobachteten Instationarit"aten noch relativ gering sind, und bei anderen weit
gravierender ausfallen.

\epsfigfour
{anwendung/rects/rct_r_3}
{anwendung/rects/rct_p_1}
{anwendung/rects/rctsf_r_3}
{anwendung/rects/rctsf_p_1}
{Rekonstruktionen eines regelm"a"sigen (links) bzw.\  periodischen Atemsignals (rechts)
  zur Einbettungsdimension $\embed=2$ und Verz"ogerungszeit $\delay_R=3,2$ s. Unten
  dargestellt sind die entsprechenden Rekonstruktionen der "uber SVD gefilterten Signale.
}
{medrekonst}{-0.5cm}

\epsfigdouble
{anwendung/instat/cmphsregel}
{anwendung/instat/cmphsperiod}
{ Dichteverteilungen der Me"swerte f"ur das regelm"a"sige (links) und das periodische
  Atemsignal (rechts). Die Kurve f"ur das gesamte Signal ist durchgezogen, f"ur die beiden 
  H"alften des Signals gestrichelt dargestellt.
}
{medinstat}{-0.5cm}


\subsection{Bestimmung der Korrelationsintegrale und -dimensionen}
F"ur die Atemsignale wurden Korrelationsintegrale bis zu einer Einbettungsdimension
$\embed=8$ berechnet. Zwar sind bei dem gegebenen Datenumfang maximal
Korrelationsdimensionen bis $D_2\simeq 4$ berechenbar, was auch nur eine
Einbettungsdimension von $d=4$ erforden w"urde, jedoch l"a"st die Berechnung  bis $d=8$
erkennen, ob die berechneten Integrale bzw.\  Dimensionen konvergieren oder nicht.

Die zu den Parameter $d_\tmax=8$, $\delay=0,32$ s und $W=60$ berechneten
Korrelationsintegrale zeigt \psref{medcorrint} (oben). Ob vern"unftige Skalierungsbereiche
vorliegen l"a"st sich jedoch besser an den Steigungen des Korrelationsintegrals erkennen
(s. \psref{medcorrint} mitte). F"ur die regelm"a"sige Atmung ist weder ein Bereich
konstanter Steigung noch eine Konvergenz der Steigungen festzustellen. F"ur die
periodische Atmung scheint bei etwa $\ln r=-4$ Konvergenz aufzutreten. Dies ist jedoch eher
als Artefakt des bei h"ohreren Einbettungsdimensionen auftretenden \begriff(Randeffekts)
zu deuten. Auch die "uber Takens' Sch"atzer zum Maximalabstand $\ln\rmax=-2,5$ bestimmten
Dimensionen zeigen keine Konvergenz (s. \psref{medcorrint} unten).

\epsfigsix
{anwendung/corrint/ci_r_3}
{anwendung/corrint/ci_p_1}
{anwendung/corrint/cs_r_3}
{anwendung/corrint/cs_p_1}
{anwendung/corrint/td_r_3}
{anwendung/corrint/td_p_1}
{
Korrelationsintegrale, logarithmische Ableitung und Korrelationsdimension ("uber Takens'
Sch"atzer) f"ur regelm"a"sige (links) und periodische Atmung (rechts).
}
{medcorrint}{-0.5cm}

Die Berechnungen wurden auch mit Variationen der Parameter sowie der Filtertechniken
ausgef"uhrt. Die nachfolgend aufgez"ahlten Parametereinstellungen und Transformationen der
Zeitreihen wurden auf verschiedene Weisen kombiniert, f"uhrten jedoch zu keiner
Verbesserung der Ergebnisse.
\begin{description}
\item[Filterung:] Es wurden Zeitreihen mit oder ohne SVD-Filterung verwendet, wobei f"ur
  die Singular Value Decomposition Einbettungsdimensionen zwischen $d=5$ und $d=30$
  gew"ahlt wurden. Weiterhin wurden Tiefpa"sfilterungen mit Grenzfrequenz bei 2,5 Hz
  durchgef"uhrt, um Beeinflussungen durch die Herzdynamik zu minimieren.
\item[Stationarit"at:] Biologische Zeitreihen sind oftmals instation"ar, und weisen von
  der eigentlichen Dynamik unabh"angige langreichweitige Schwankungen (Bias) auf. 
  Um diese herauszufiltern wurden Fenster "uber die Zeitreihe gelegt, in denen dieser
  Grundwert berechnet und nachfolgend von den Signalwerten subtrahiert wurde.
\item[Verzo"ogerungszeiten:] Kleine Verz"ogerungszeiten bringen bei der Berechnung der
  Korrelationsintegrale oftmals besseres Ergebnisse als die berechneten
  Verz"ogerungszeiten. Es wurden Verz"ogerungszeiten bis hinunter zu $\delay=0,1$ s
  getestet.
\item[Autokorrelationszeit:] Um Effekte durch Autokorrelationen auszuschlie"sen, wurde
  der Parameter f"ur die Autokorrelationsl"ange $W$ von 1 bis zu Vielfachen der
  Verz"ogerungszeit ausgetestet.
\end{description}
Da eine endliche fraktale Dimension Voraussetzung f"ur die Existenz eines seltsamen
Atraktors ist und die Korrelationsanalyse (die Korrelationsdimension ist eine untere
Absch"atzung f"ur die fraktale Dimension) keine konvergenten Ergebnisse erbrachte, k"onnen wir somit
nicht auf die Existenz eines solchen schlie"sen. 

F"ur den Atemtypus der regelm"a"sigen Atmung ist noch eine andere (sehr vorsichtige)
Deutung m"oglich. Es k"onnte sich hierbei um eine Grenzzyklus handeln, der jedoch stark
verrauscht ist. Dies st"ande sowohl in Einklang mit dem gemessenen Fourierspektrum, das
relativ klar hervortretende Peaks aufweist, als auch mit 2 bzw.\ 3-dimensionalen
Rekonstruktionen, die eine toroidale Struktur zeigen. Zudem weisen nicht-chaotische Systeme
einen verk"urzten Skalierungsbereich im Korrelationsintegral auf: Das Korrelationsintegral
skaliert nur "uber einen Bereich der Ordnung \order{N} statt der Ordnung \order{N^2}, wie
im chaotischen Fall \cite{Theiler}.

\subsection{Determinismustest "uber AAFT-Surrogate}

Trotz der negativen Ergebnisse bei der Dimensionsanalyse k"onnen die f"ur bestimmte
Einbettungsdimensionen und Maximalabstand $\rmax$ berechneten Werte der
Korrelationsdimension zum Vergleich mit entsprechenden Werten von Surrogatdaten im Rahmen
eines Determinismustest herangezogen werden. Da die Dichteverteilung der Me"swerte eine
deutlich nicht gau"ssche Verteilung zeigen (s. \psref{medinstat}) wurden als Surrogatdaten
AAFT-Surrogate verwendet. Die Anzahl der erstellten Datenreihen betrug in beiden F"allen
(d.h. f"ur regelm"a"sige bzw.\ periodische Atmung) 39. F"ur diese wurden mit den gleichen
Parametern ($\embed_\tmax=8, \delay=0,32, W=60)$, wie f"ur die Originalzeitreihen
Korrelationsintegrale berechnet. Aus diesen wurde "uber Takens' Sch"atzer die
Korrelationsdimension $D_2$ in Abh"angigkeit von der Einbettungsdimension bestimmt. Die
Maximalabst"ande lagen bei $\rmax=-2,5$. Den Vergleich der Dimensionsberechnungen zwischen
Original- und Surrogatdaten zeigt \psref{medsurro} (oben).  Die Signifikanz $\mathcal S$
der Abweichung ist in den meisten F"allen sehr hoch (s.  \psref{medsurro} mitte). Der mit
der Signifikanz verbundene $p$-Wert liegt, au"ser bei der periodischen Atmung f"ur
Einbettungsdimensionen $d\leq3$, bei Werten deutlich unter 0,05. Wir k"onnen daher die
Nullhypothese, da"s es sich bei der Atmung um ein durch ein ARMA-Modell beschreibbares
stochastisches System handelt, auf einem 0,05-Signifikanzniveau ablehnen. Wie man schon
intuitiv vermuten w"urde, handelt es sich der Atmung also um ein determinsitisches System.

\epsfigsix
{anwendung/surrogat/regel/cmpsurcdim}
{anwendung/surrogat/period/cmpsurcdim}
{anwendung/surrogat/regel/signi}
{anwendung/surrogat/period/signi}
{anwendung/surrogat/regel/pvalue}
{anwendung/surrogat/period/pvalue}
{
Surrogatdatenanalyse f"ur regelm"a"sige (links) und periodische Atmung (rechts). 
Vergleich der berechneten Korrelationdimensionen zwischen Original- und Surrogatdaten
(oben), Signifikanzniveau der Abweichung (mitte) und $p$-Wert (unten).
}
{medsurro}{-0.5cm}







%\section{Auswertung der vorliegenden Me"sreihen}

\subsection{Sch"one Bildchen}

\epsfigsix
{anwendung/zeitreihen/regel}
{anwendung/zeitreihen/misch}
{anwendung/zeitreihen/misch2}
{anwendung/zeitreihen/period}
{anwendung/zeitreihen/reg_apnoe2}
{anwendung/zeitreihen/reg_apnoe}
{ts: r1, m1, m2, pa, ra2, ra1 }{pics1}{-0.2cm}

\epsfigsix
{anwendung/fourier/fr_r_3}
{anwendung/fourier/fr_r_6}
{anwendung/fourier/fr_m_1}
{anwendung/fourier/fr_ma_1}
{anwendung/fourier/fr_ra_2}
{anwendung/fourier/fr_p_2}
{fr: r3,r6,m1,ra2,p2 }{pics2}{-0.2cm}

\epsfigsix
{anwendung/histo/hst_r_3}
{anwendung/histo/hst_r_6}
{anwendung/histo/hst_m_1}
{anwendung/histo/hst_ma_1}
{anwendung/histo/hst_ra_2}
{anwendung/histo/hst_p_2}
{hst: r3,r6,m1,ma1,ra2 }{pics3}{-0.2cm}

\epsfigsix
{anwendung/rects/rct_r_3}
{anwendung/rects/rct_r_6}
{anwendung/rects/rct_m_1}
{anwendung/rects/rct_ma_1}
{anwendung/rects/rct_ra_2}
{anwendung/rects/rct_p_2}
{rct: r3,r6,m1,ma1,ra2,p2 }{pics4}{-0.2cm}

\epsfigsix
{anwendung/rectsma/rctma_r_3}
{anwendung/rectsma/rctma_r_6}
{anwendung/rectsma/rctma_m_1}
{anwendung/rectsma/rctma_ma_1}
{anwendung/rectsma/rctma_ra_2}
{anwendung/rectsma/rctma_p_2}
{rctma: r3,r6,m1,ma1,ra2,p2 }{pics5}{-0.2cm}

\epsfigsix
{anwendung/redund/mut_r_3}
{anwendung/redund/mut_r_6}
{anwendung/redund/mut_m_1}
{anwendung/redund/mut_ma_1}
{anwendung/redund/mut_ra_2}
{anwendung/redund/mut_p_2}
{mut: r3,r6,m1,ma1,ra2,p2 }{pics6}{-0.2cm}






\comment{
\section{Physiologie Fr"uhgeborener}


Zitat siehe \autor(Theiler)
\begin{quote} \em
Es ist nicht schwierig einen Algorithmus zu entwickeln, der Zahlen liefert die als Dimension
bezeichnet werden k"onnen, aber es ist weit schwieriger, sicher zu sein, da"s diese Zahlen
wirklich die Dynamik des System repr"asentieren.
\end{quote}




\subsection{Atmungstypen}

\subsubsection{Regelm"a"sige Atmung}

\subsubsection{Periodische Atmung}

\section{Me"sverfahren}

\subsection{Torakale Impedanz (TI)}

\subsection{Volumenstrommessungen (Flow)}

\section{Analyse der Atmungsdaten}

\subsection{Spektren}

\subsubsection{Filterung}

\subsection{Bestimmung der Einbettungsparameter}

\subsection{Rekonstruktion}

\subsubsection{Singular Value Decomposition}

\subsection{Korrelationsdimension}

\subsubsection{Regelm"a"sige Atmung (TI)}

\subsubsection{Periodische Atmung (TI)}

\subsubsection{Regelm"a"sige Atmung (Flow)}

\subsection{Determinismustests}

\epsfigdouble{surrogate/aaft/regel/timeseries}{surrogate/aaft/regel/surronew}
{Links die Originalzeitreihe (Regelm"a"siges Atmen), rechts eine typische, durch AAFT
erzeugte Surrogat-Zeitreihe. Zur besseren Vergleichbarkeit  werden nur die ersten 30
Sekunden adrgestellt.}{surreg1}{-0.2cm}

\epsfigdouble{surrogate/aaft/regel/corrslp}{surrogate/aaft/regel/surcslp}
{Vergleich der Steigungen des Korrelationsintegrals $C(r)$ f"ur Original- (links) und eine
typische Surrogat-Zeitreihe (rechts).}{surreg2}{-0.2cm}

\epsfigdouble{surrogate/aaft/regel/cmpsurcdim}{surrogate/aaft/regel/signi}
{Links: Gemessene Korrelationsdimension f"ur Original- \captimes und Surrogatdaten
\capplus. Der Skalierungsbereich wurde anhand von \psref{surreg2} f"ur beide 
zu $-1.5\leq\ln r\leq\-0.9$ gew"ahlt. Rechts: Die Signifikanz $\Delta\nu/\sigma$ liegt
f"ur Einbettungsdimensionen $d\geq 2$ deutlich "uber $3$.}{surreg3}{-0.2cm} 


\newpage

\comment{
\epsfigdouble{surrogate/ft/regel/timeseries}{surrogate/ft/regel/surronew}
{Links die Originalzeitreihe (Regelm"a"siges Atmen), rechts eine typische, durch FT
erzeugte Surrogat-Zeitreihe. Zur besseren Vergleichbarkeit  werden nur die ersten 30
Sekunden adrgestellt.}{surareg1}{-0.2cm}
}

\epsfigdouble{surrogate/ft/regel/corrslp}{surrogate/ft/regel/surcslp}
{Vergleich der Steigungen des Korrelationsintegrals $C(r)$ f"ur Original- (links) und eine
typische Surrogat-Zeitreihe (rechts).}{surareg2}{-0.2cm}

\epsfigdouble{surrogate/ft/regel/cmpsurcdim}{surrogate/ft/regel/signi}
{Links: Gemessene Korrelationsdimension f"ur Original- \captimes und Surrogatdaten
\capplus. Der Skalierungsbereich wurde anhand von \psref{surareg2} f"ur beide 
zu $-1.5\leq\ln r\leq\-0.9$ gew"ahlt. Rechts: Die Signifikanz $\Delta\nu/\sigma$ liegt
f"ur Einbettungsdimensionen $d\geq 2$ deutlich "uber $3$.}{surareg3}{-0.2cm} 

\epsfigdouble{surrogate/ft/regel/B/cmpsurcdim}{surrogate/ft/regel/B/signi}
{Links: Gemessene Korrelationsdimension f"ur Original- \captimes und Surrogatdaten
\capplus. Der Skalierungsbereich wurde anhand von \psref{surareg2} f"ur beide 
zu $-1.5\leq\ln r\leq\-0.9$ gew"ahlt. Rechts: Die Signifikanz $\Delta\nu/\sigma$ liegt
f"ur Einbettungsdimensionen $d\geq 2$ deutlich "uber $3$.}{suraregb3}{-0.2cm} 

}










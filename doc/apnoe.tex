\section{Zur Atmung von Frühgeborenen}


\subsection{Problemstellung}

Die Fortschritte auf dem Gebiet der intensivmedizinischen Betreuung während der letzten
20 Jahre ermöglichen zunehmend das Überleben kleiner und sehr kleiner Frühgeborener\footnote{Als
  \begriff(Frühgeborene) bezeichnet man Säuglinge mit einem Gestationsalter unter 37
  Wochen. Die Angabe \naja(klein) bzw.  \naja(sehr klein) bezieht sich auf das
  Geburtsgewicht von unter 1500 g bzw.\ unter 1000 g.}.
In der auf die Zeit der Intensivbehandlung folgenden Zeit der
Stabilisierung und ``Aufzucht'' sind Schwestern und Ärzte mit verschiedenen durch die
Unreife bzw.  Instabilität dieser Patienten bedingten Problemen konfrontiert. Hierunter
erfordert die Instabilität der Atmung unter allen Reifealtern vordringliches Augenmerk:
Beim eigenständig atmenden Frühgeborenen kann es ohne vorausgehende Anzeichen
zum Atemstillstand (\begriff(Apnoe)) kommen.  Dieser führt zum Sauerstoffmangel (\begriff(Hypoxämie))und
konsekutiv zu einem Abfall der Herzfrequenz (\begriff(Bradykardie)) auf kritische, die
Kreislaufversorgung gefährdende Verhältnisse.  Neben der Gefahr eines akuten
Herz-Kreislaufversagens bringen häufige Atemstillstände die Gefährdung des sich
entwickelnden Gehirns durch latenten Sauerstoffmangel mit sich.  Die Kinder werden durch
Ableitung des EKG\footnote{Man spricht hier von einer \begriff(Ableitung) des EKG, da die 
  interessierenden Aktionspotentiale des Herzens nicht direkt gemessen sondern nur
  indirekt durch Anbringung von Elektroden auf der Körperoberfläche \naja(abgeleitet)
  werden können.} und des atemsynchronen thorakalen Impedanzsignals (TI) fortlaufend
überwacht. Besonders gefährdete Patienten werden zusätzlich durch Messung der
Sauerstoffsättigung des Blutes überwacht (\begriff(Pulsoxymetrie)).

\subsection{Atemrhythmen Frühgeborener}

Heute wird vom Konzept eines lokalisierten Rhythmusgenerators der Atmung abgesehen.
Stattdessen ist experimentell ein verteiltes Netzwerk atemkompetenter Nervenzellen belegt.
Zur Zeit wird von sechs jeweils verschiedenen Atemphasen zugeordneten Nervenzellgruppen im
Hirnstamm ausgegangen. Ihre gemeinsame Endstrecke ist der das Zwerchfell stimulierende
\begriff(Nervus Phrenicus). Die für die Atmung zuständigen \comment{respiratorisch
  kompetenten} Gruppen unterliegen ihrerseits einer Anzahl peripherer (Lungendehnung,
$\mathrm{O}_2$-Gehalt des Blutes) und zentraler  (Wachheit) Einflüsse, welche sie
integrativ verarbeiten. \comment{(\autor(Richter))}

Für die hier untersuchten Frühgeborenen, werden im wesentlichen die
folgenden Atemrhythmen unterschieden, wobei diese Einteilung phänomenologisch erfolgt:
\begriff(regelmäßiges) Atmen einerseits und \begriff(periodisches) Atmen andererseits.
Die Atemrhythmen können grob den verschiedenen Schlafphasen zugeordnet werden: Während
des REM-Schlafs tritt ausschließlich periodische, während des Non-REM-Schlafs
hauptsächlich regelmäßige, aber auch periodische Atmung auf.

Der Begriff \begriff(regelmäßige Atmung) beschreibt das Auftreten regelmäßiger
Atemzüge gleicher Atemtiefe (siehe \psref{medrhythm} oben). Für den Terminus
\begriff(periodische Atmung) folgen wir der Definition von \autor(Kryger); siehe \cite{Hoch96}. Für ihn sind
die Kriterien dieses Atmungsmusters erfüllt, wenn eine Folge von Atmungspausen von
jeweils mehr als 3 Sekunden Länge, unterbrochen von regulären Atmungsperioden mit einer
Dauer von bis zu 20 Sekunden, vorliegt (siehe \psref{medrhythm} unten). Ein Zusammenhang
zwischen periodischer Atmung und dem Auftreten von Frühgeborenenapnoen scheint
nicht gegeben \cite{Hoch96}.

\epsfigdouble
{anwendung/zeitreihen/regeln}
{anwendung/zeitreihen/periodn}
{Beispiele für die verschiedenen Atemrhythmen: regelmäßiges Atmen (oben) bzw.\  
  periodisches Atmen (unten). Dargestellt ist das atemkorrelierte thorakale
  Impedanzsignal.  }{medrhythm}{-0.5cm}

Über die Definition einer Frühgeborenenapnoe bestehen einige unterschiedliche
Auffassungen. In vielen Veröffentlichungen wird als Indikation für eine Apnoe neben dem Stillstand der
Atmung  ein Abfall der Herzfrequenz und des Sauerstoffpartialdrucks des Blutes als
Kriterium angegeben \cite{Poets93}. Weitere Kennzeichen sind eine Zyanose oder eine Blässe
des Patienten. Da das Vorliegen einer Hypoxämie oder Bradykardie aus den vorliegenden
Daten (thorakale Impedanz) nicht hervorgeht, verwenden wir die Definition von
\autor(Glotzbach) (siehe \cite{Hoch96}), nach der bei einem Atemstillstand von mindestens 15 Sekunden eine 
Apnoe vorliegt.  Ein Beispiel einer ca.\ 20 Sekunden dauernden Apnoe während einer Phase
regelmäßiger Atmung zeigt \psref{medapnoe}.

\epsfigdouble
{anwendung/zeitreihen/reg_apnoe}
{anwendung/zeitreihen/reg_apnoe2}
{ Apnoe während einer Phase regelmäßiger Atmung. Unten
  ein vergrößerter Ausschnitt der oben dargestellten Zeitreihe. Aufgrund des eingeschränkten Meßbereichs ist
  das Signal des der Apnoe vorausgehenden tiefen Atemzugs nach oben und unten abgeschnitten.
}{medapnoe}{-0.5cm}


\subsection{Stand der Forschung}

Mit erheblichem Aufwand wird versucht, die Bedingungen, welche zu Atemstillständen
führen, zu identifizieren.  Nur wenige Arbeiten gehen bisher der Frage nach dem
dynamischen Charakter der Frühgeborenenatmung nach. \autor(Pilgrim \etal) konnten
ausgehend von Messungen der Korrelationsdimension ($\corrdim<3$) für die periodische
Atmung einen deterministischen Charakter dieses Atemrhythmus belegen \cite{Pilgram94}. Weitere Aussagen zum
dynamischen Charakter der Atmung im allgemeinen finden sich vor allem in
tierexperimentellen Arbeiten. \autor(Sammon) konnte in Versuchen mit narkotisierten Ratten
nachweisen, daß durch Stimulation des Nervus Vagus eine Erhöhung der Dimension der
Atmungsdynamik stattfindet \cite{Sammon91}.  \autor(Eldridge) konnte -- ebenfalls an einem
Experiment mit Ratten -- zeigen, daß die experimentell gefundene Abhängigkeit der durch
eine Störung des Atemzyklus hervorgerufenen Reaktion von der Phasenlage durch
Modellierung mittels eines periodisch erregten \begriff(Van-der-Pol-Oszillators)
nachgebildet werden konnte \cite{Eldridge89}.  Insgesamt fehlen bisher für das
menschliche Neugeborene deutliche Hinweise auf einen deterministischen Charakter des
Atemrhythmus. Von Interesse ist dies vor allem im Hinblick auf die mögliche
Vorhersagbarkeit von Störungen der Atmungsaktivität.  Darüber hinaus spielt diese Frage auch für das
Verständnis des Zusammenhangs zwischen der lebensnotwendigen Stabilität der Atmung
einerseits und den vielen Unterbrechungen des Atemrhythmus (Schlucken, Innehalten,
Sprechen etc.) andererseits ein Rolle.

\subsection{Signale, Daten und Patienten}

Für die vorliegende Arbeit standen Aufzeichnungen des thorakalen Impedanzsignals von acht
Frühgeborenen zur Verfügung. Diese entstammen der Routineüberwachung. Die Messung der
thorakalen Impedanz registriert den atemabhängig schwankenden Wechselstromwiderstand des
Brustkorbes anhand eines Meßstroms (30 kHz, 3 $\mu$A), welcher über zwei Elektroden
eingebracht wird. Sie bestimmt somit den Anteil des leitenden zum nichtleitenden Material
im Volumenleiter Thorax. Da nichtleitendes Atemgas periodisch ein und ausströmt, variiert
der gemessene Wechselstromwiderstand im wesentlichen mit der Atmung. Das Signal ist allerdings
wegen der Herzaktion und der sich aus dieser ergebenden Änderung des leitenden Volumens
im Brustkorb auch an das Herzkreislaufsystem gekoppelt. 

Die Sampling Rate betrug 100 Hz bei einer Auflösung von 12 Bit. Die Meßwerte wurden
auf einen Bereich ganzer Zahlen von -2048 bis +2047 abgebildet, wobei das
Abbildungsverhältnis und die Nullage jeweils dem Patienten angepaßt wurden. Die
Aufzeichnung erfolgte zusammen mit dem EKG über 24 Stunden. Nach Abschluß der Messungen
wurden 13 Zeitreihen mit regelmäßiger und 12 mit periodischer Atmung ausgewählt. Die
Länge der ausgewählten Zeitreihen betrug 4 min, was einem Datenumfang von je 24.000 Punkten
entspricht.  In 4 dieser Zeitreihen traten Apnoen auf.

Die bei den folgenden Auswertungen abgebildeten Darstellungen beruhen, soweit nicht anders 
angegeben, auf den Zeitreihen aus \psref{medrhythm}. Die Auswertungen selber wurden für
alle vorliegenden Zeitreihen durchgeführt, zeigten jedoch i.a.\    keine signifikanten
Unterschiede zu den beiden Beispielzeitreihen, so daß auf deren separate Darstellung
verzichtet wurde.


\addchap{Schluß} 

In der vorliegenden Arbeit konnte gezeigt werden, daß die bei Frühgeborenen
haupt\-säch\-lich auftretenden Atemrhythmen (regelmäßige und periodische Atmung)  nicht durch stochastische Prozesse modellierbar
sind: Die Frühgeborenenatmung besitzt mit sehr hoher Wahrscheinlichkeit einen
deterministischen Charakter. Trotz dieses zugrunde liegenden Determinismus konnte die
genaue Struktur der Dynamik nicht identifiziert werden. Eine verläßliche Bestimmung der
Korrelationsdimension als Grundlage für eine mögliche Beschreibung des Atmungssystems
durch einen seltsamen Attraktor erwies sich als  unmöglich. Ferner ist die Korrelationdimension
als möglicher Indikator für eine Vorhersage von Atemstillständen nach den Resultaten dieser Arbeit 
nicht geeignet. Als Gründe dafür, daß  die verwendeten Methoden keinen Erfolg hatten, lassen sich mehrere
Punkte anführen:

\begin{itemize}
\item Die Atmung, auf die reine Lungenaktivität reduziert, stellt kein autonomes System
  dar. In der Dynamik spielen noch weitere Kontrollparameter eine Rolle, die aus dem TI-Signal
  nicht rekonstruierbar sind. Insofern ist eine der Grundlagen für die Verwendung der
  Verzögerungskoordinatenabbildung verletzt, da in diesem Fall selbst beliebig viele verzögerte
  Werte nicht den kompletten Systemzustand wiedergeben können.  Eventuell müßte hier
  der Zustand der atemkompetenten Neuronengruppen in eine Analyse miteinbezogen werden.
  
\item In die gleiche Richtung wie der vorige Punkt geht die Vermutung, daß
  weitere psychische und physische Einflüsse existieren, die während der Aufnahme der Zeitreihe variieren können.
  Dies äußert sich in zeitlichen Änderungen der natürlichen Dichteverteilungen in den
  Phasenraumrekonstruktionen. Hierdurch sind Verfahren wie Dimensionsalgorithmen nicht
  mehr anwendbar, da sie auf stationären Dichten beruhen.
  
\item Das gemessene Signal unterliegt einer Reihe weiterer nicht atemkorrelierter
  Einflüsse, wie beispielweise der Herztätigkeit. Ein Signal, in dem solche Einflüsse
  ausgeschlossen werden können (z.B. Volumenstrommessungen), mag bessere Ergebnisse
  liefern als die thorakale Impedanz.
  
\item Im Gegensatz zu anderen physiologischen Vorgängen, wie der Herztätigkeit, kann die
  Atmung leicht bewußt beeinflußt werden. So können Vorgänge, wie Schlucken,
  Lautäußerungen oder Innehalten, die Aussagekraft des Signals bezüglich der
  Atmungsdynamik stark beeinträchtigen.
\end{itemize}

Trotz der negativen Ergebnisse bei dem Versuch einer Systemidentifikation der
Atmungsdynamik ist durch den
Nachweis ihres deterministischen Charakters ein wichtiger Grundstein gelegt. Für
weitergehende Analysen sind jedoch besser geeignete Signale und bessere Filtertechniken
vonnöten (siehe beispielsweise \cite{Rao92}).  Eine weitere interessante Möglichkeit ist
die Modellierung chaotischer Systeme über selbstlernende neuronale Netze
(Back\-pro\-pa\-ga\-tion-Netze), die -- zumindest für einfache Systeme -- sehr gut in der 
Lage sind, die Dynamik aus kleinen, stark verrauschten Datenreihen zu extrahieren
\cite{Albano92}. Dieser gerade im Hinblick auf seine Robustheit 
gegenüber Rauschen vielversprechende Ansatz könnte bei der Analyse der Atmungsdynamik
eine große Hilfe sein.




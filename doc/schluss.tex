\addchap{Schlu"s} 

In der vorliegenden Arbeit konnte gezeigt werden, da"s die bei Fr"uhgeborenen
haupt\-s"ach\-lich auftretenden Atemrhythmen (regelm"a"sige und periodische Atmung)  nicht durch stochastische Prozesse modellierbar
sind: Die Fr"uhgeborenenatmung besitzt mit sehr hoher Wahrscheinlichkeit einen
deterministischen Charakter. Trotz dieses zugrunde liegenden Determinismus konnte die
genaue Struktur der Dynamik nicht identifiziert werden. Eine verl"a"sliche Bestimmung der
Korrelationsdimension als Grundlage f"ur eine m"ogliche Beschreibung des Atmungssystems
durch einen seltsamen Attraktor erwies sich als  unm"oglich. Ferner ist die Korrelationdimension
als m"oglicher Indikator f"ur eine Vorhersage von Atemstillst"anden nach den Resultaten dieser Arbeit 
nicht geeignet. Als Gr"unde daf"ur, da"s  die verwendeten Methoden keinen Erfolg hatten, lassen sich mehrere
Punkte anf"uhren:

\begin{itemize}
\item Die Atmung, auf die reine Lungenaktivit"at reduziert, stellt kein autonomes System
  dar. In der Dynamik spielen noch weitere Kontrollparameter eine Rolle, die aus dem TI-Signal
  nicht rekonstruierbar sind. Insofern ist eine der Grundlagen f"ur die Verwendung der
  Verz"ogerungskoordinatenabbildung verletzt, da in diesem Fall selbst beliebig viele verz"ogerte
  Werte nicht den kompletten Systemzustand wiedergeben k"onnen.  Eventuell m"u"ste hier
  der Zustand der atemkompetenten Neuronengruppen in eine Analyse miteinbezogen werden.
  
\item In die gleiche Richtung wie der vorige Punkt geht die Vermutung, da"s
  weitere psychische und physische Einfl"usse existieren, die w"ahrend der Aufnahme der Zeitreihe variieren k"onnen.
  Dies "au"sert sich in zeitlichen "Anderungen der nat"urlichen Dichteverteilungen in den
  Phasenraumrekonstruktionen. Hierdurch sind Verfahren wie Dimensionsalgorithmen nicht
  mehr anwendbar, da sie auf station"aren Dichten beruhen.
  
\item Das gemessene Signal unterliegt einer Reihe weiterer nicht atemkorrelierter
  Einfl"usse, wie beispielweise der Herzt"atigkeit. Ein Signal, in dem solche Einfl"usse
  ausgeschlossen werden k"onnen (z.B. Volumenstrommessungen), mag bessere Ergebnisse
  liefern als die thorakale Impedanz.
  
\item Im Gegensatz zu anderen physiologischen Vorg"angen, wie der Herzt"atigkeit, kann die
  Atmung leicht bewu"st beeinflu"st werden. So k"onnen Vorg"ange, wie Schlucken,
  Laut"au"serungen oder Innehalten, die Aussagekraft des Signals bez"uglich der
  Atmungsdynamik stark beeintr"achtigen.
\end{itemize}

Trotz der negativen Ergebnisse bei dem Versuch einer Systemidentifikation der
Atmungsdynamik ist durch den
Nachweis ihres deterministischen Charakters ein wichtiger Grundstein gelegt. F"ur
weitergehende Analysen sind jedoch besser geeignete Signale und bessere Filtertechniken
vonn"oten (siehe beispielsweise \cite{Rao92}).  Eine weitere interessante M"oglichkeit ist
die Modellierung chaotischer Systeme "uber selbstlernende neuronale Netze
(Back\-pro\-pa\-ga\-tion-Netze), die -- zumindest f"ur einfache Systeme -- sehr gut in der 
Lage sind, die Dynamik aus kleinen, stark verrauschten Datenreihen zu extrahieren
\cite{Albano92}. Dieser gerade im Hinblick auf seine Robustheit 
gegen"uber Rauschen vielversprechende Ansatz k"onnte bei der Analyse der Atmungsdynamik
eine gro"se Hilfe sein.




\addchap{Einleitung}

Komplexes Verhalten tritt in nahezu allen Bereichen der Natur in Erscheinung. Während die
Physik diesen Phänomenen früher relativ ratlos gegenüberstand, konnten in den letzten
Jahren durch die Erkenntnisse der Chaostheorie auf diesem Gebiet große Fortschritte
erzielt werden. Die Theorie nichtlinearer dynamischer Systeme leistete hier Wesentliches,
indem sie die oftmals auftretenden komplexen Dynamiken mathematisch-physikalisch erfaßbar
und charakterisierbar machte.  Hierbei ist besonders das Konzept \begriff(seltsamer Attraktoren)
hervorzuheben.  Seltsame Attraktoren bilden die Menge der im Langzeitverhalten
erreichbaren Phasenraumpunkte dissipativer chaotischer Systeme und besitzen im allgemeinen
eine sehr verwickelte fraktale Struktur. Damit einher geht ein Verlust der Langzeitvorhersage
durch die Eigenschaft dieser Systeme, kleine Störungen exponentiell zu verstärken.  Die
Erkenntnis, daß solche Strukturen auch bei deterministischen Systemen mit wenigen
Freiheitsgraden auftreten können, hat in vielen Gebieten der Naturwissenschaften, auch
außerhalb der Physik, die Frage aufgeworfen, ob die beobachteten Verhaltensweisen und
Strukturen durch diese Konzepte zu erklären seien. (Kapitel~1.1.)


In den empirischen Naturwissenschaften liegen häufig nur unzureichende Daten über den
genauen Zustand des beobachteten Systems vor, oftmals in Form \begriff(skalarer
Zeitreihen) einer gemessenen Zustandsvariable.  Eine entscheidende Voraussetzung für die
Charakterisierung dieser Zeitreihen mittels der Methoden der nichtlinearen Dynamik ist
die Möglichkeit, aus diesen wieder Attraktoren rekonstruieren zu können.  Dies geschieht
über das Konzept der sogenannten \begriff(Verzögerungskoordinaten), welches durch eine
mathematische Theorie abgesichert werden konnte. Die Voraussetzungen der mathematischen
Theorie sind in der experimentellen Praxis jedoch nicht erfüllbar. Neben der ständigen
Anwesenheit von Rauschen ist vor allem die zeitliche Begrenztheit der zur Verfügung
stehenden Zeitreihen ein Problem. Aufgrund dieser Einschränkungen ist eine gute Wahl der
Einbettungsparameter vonnöten.  Desweiteren müssen Möglichkeiten zur Verminderung des
Rauschens gefunden werden. (Kapitel~1.2.)

Für die Charakterisierung seltsamer Attraktoren steht eine Vielzahl verschiedener Maßzahlen zur
Verfügung. Die drei am häufigsten herangezogenen sind die fraktale Dimension, die
Kolmogorov-Entropie und der Lyapunov-Exponent. Diese beschreiben respektive die Anzahl
der effektiven Freiheitsgrade, die Rate der Informationsproduktion sowie die exponentielle
Separation der Trajektorien des Systems. Da die Komplexität eines Systems meistens mit
der Anzahl seiner Freiheitsgrade korreliert ist, wendet sich diese Arbeit besonders den
verschiedenen Begriffen fraktaler Dimension zu.  Im Bereich der nichtlinearen
Zeitreihenanalyse wird zumeist die Korrelationsdimension verwendet, die durch ihre schnelle
und stabile Berechenbarkeit durch den Grassberger-Procaccia-Algorithmus wesentliche
Bedeutung erlangt hat. Probleme stellen sich hierbei hauptsächlich durch die endliche
Datenmenge, weißes und Digitalisierungsrauschen und sogenannte Randeffekte.  (Kapitel~1.3.)

Damit die Verwendung der beschriebenen Methoden überhaupt einen Sinn macht, ist es
notwendig zu wissen, daß der beobachteten Zeitreihe ein deterministisches System zugrunde
liegt.  Zur Lösung des Problems existieren eine Vielzahl von Vorschlägen, von denen hier
der Methode der Surrogatdaten auf der Basis statistischer Hypothesentests der Vorzug gegeben
wird.  Diese bieten ein breites Fundament, um Fragen nach der Struktur und
zugrunde liegendem Determinismus experimenteller Zeitreihen beantworten zu können. Durch
die Verfügbarkeit hoher Rechenleistungen in modernen Computern lassen sich eine Vielzahl
verschiedener Hypothesen durch Vergleich der Originalzeitreihen mit sogenannten
Surrogatdaten testen. (Kapitel~1.4.)

Die Ergebnisse dieser Arbeit sollen auf das konkrete Beispiel der Atmungsdynamik
Frühgeborener angewandt werden. Das Ziel
hierbei ist es zu klären, ob die vorgestellten Methoden aus der Zeitreihenanalyse bei der
Untersuchung dieser Problematik von Nutzen sein können.
Konkret geht es hierbei um folgende Fragen:
\begin{itemize}
\item Ist die den verschiedenen Atmungstypen zugrunde liegende Dynamik durch ein
  deterministisches System beschreibbar ?
\item Können für die verschiedenen Atmungstypen, regelmäßiges und periodisches Atmen, die
  fraktale Dimension berechnet und  die Existenz eines seltsamen Attraktors belegt werden ?
\item Lassen sich signifikante Änderungen der Dimension vor dem Auftreten von
  Atemstillstän\-den (Apnoen) feststellen ?
\end{itemize}
Eine Bejahung dieser Fragen könnte zu einer möglichen Früherkennung von
Atemstillständen führen, was für die Frühgeborenenmedizin von größtem Nutzen wäre.
(Kapitel~2)






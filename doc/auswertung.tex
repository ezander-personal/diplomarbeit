\section{Auswertung der Zeitreihen} 

\subsection{Fourier-Analyse}
Die gemessen Zeitreihen wurden zuerst einer Fourier-Analyse
unterzogen, wobei hier nur die Leistungsspektren $P(f)$ (d.h. der Betrag der Amplituden der
Fourier-Transfor\-mier\-ten) von Interesse sind. Die Leistungsspektren der beiden
ausgewählten Zeitreihen zeigen \psref{medfourierr} (oben) und \psref{medfourierp}
(oben). Im Spektrum der regelmäßigen Atmung sind vier Peaks gut zu unterscheiden, von
denen allerdings nur die ersten drei durchgehend in den Leistungsspektren der
regelmäßigen Atmung auftauchen. Der 
erste liegt bei ca.\ 0,8 Hz und entspricht der mittleren Atemfrequenz des Kindes. Die
Atemfrequenzen Frühgeborener liegen in der Regel bei ca.\ 
0,6-0,8 Hz, und können in Ausnahmefällen auch bis 1,2 Hz reichen, ohne daß dies als
pathologische Indikation zu werten ist. Die Atemfrequenz dieser Patienten liegt damit
deutlich höher als die von erwachsenen Individuen. Die beiden weiteren Peaks (bei 2,2 bzw. 4,3 Hz)
rühren von der Herztätigkeit her, was durch Vergleich mit den Spektren des gleichzeitig
aufgenommenen EKGs bestätigt wird (siehe \psref{medfourierr} unten). Auch die Herzfrequenz
Frühgeborener liegt höher als die Erwachsener, wobei eine Frequenz von ca.\ 2 Hz als normal
anzusehen ist. Die deutlich zu erkennenden Oberfrequenzen werden vornehmlich hervorgerufen
durch die steilen Flanken des sogenannten QRS-Komplexes, eines im EKG-Signal auftretenden sehr
starken Ausschlags.

Bei der periodischen Atmung liegen die Peaks an etwa der gleichen Stelle. Im dargestellten
Spektrum (\psref{medfourierp} oben) erkennt man Peaks bei 0,7 und 2,1 Hz sowie
andeutungsweise bei 4,3 Hz.  Die Ausprägung ist jedoch weniger deutlich als bei der
regelmäßigen Atmung. Auch hier können die Peaks mit Frequenzen über 2 Hz mit
entsprechenden Peaks des Leistungsspektrums des EKGs identifiziert werden.

Da die Peaks bei Frequenzen über 2 Hz im allgemeinen von der Herztätigkeit herrühren, ist es
häufige Praxis, auf die Zeitreihen Tiefpaßfilter mit Grenzfrequenzen von 2 bis 2,5 Hz
anzuwenden. Da hierbei jedoch nicht klar ist, ob durch die Filterung eventuell für die
Atmungsdynamik wesentliche Informationen verlorengehen, findet diese Vorgehensweise hier
keine Anwendung. 


\epsfigdouble
{anwendung/fourier/fr_r_3}
{anwendung/herzfreq/hfr_r}
{Oben das Leistungsspektrum der ausgewählten Zeitreihe des thorakalen Impedanzsignals bei regelmäßiger Atmung.
  Darunter zum Vergleich das entsprechende Spektrum  des gleichzeitig aufgenommen EKGs.
}
{medfourierr}{-0.5cm}

\epsfigdouble
{anwendung/fourier/fr_p_2}
{anwendung/herzfreq/hfr_p}
{Oben das Leistungsspektrum der ausgewählten Zeitreihe des thorakalen Impedanzsignals bei
  periodischer Atmung.
  Darunter zum Vergleich das entsprechende Spektrum  des gleichzeitig aufgenommen EKGs.
}
{medfourierp}{-0.5cm}


\subsection{Bestimmung der Verzögerungszeit}
Die Verzögerungszeiten für die Phasenraumrekonstruktionen wurden über das Verfahren der
Redundanzanalyse (siehe Seite \pageref{chapredundancy}ff) bestimmt. Die Redundanz $R(t)$
für die beiden ausgewählten Zeitreihen zeigt \psref{medredund}.  Die über das erste
lokale Minimum der Redundanz bestimmten Verzögerungszeiten $\delay_R$
lagen bei 0,27 -- 0,42 s für regelmäßige und bei 0,24 -- 0,36 s für periodische Atmung,
wobei der Unterschied nicht signifikant ist. Eine Übersicht über die gemessenen Werte
zeigt die untenstehende Tabelle. Da die Verzögerungszeiten eine recht geringe Streuung
aufweisen, wurde für die meisten Auswertungen ein Standardwert von $\delay_R=0,32$ s
benutzt. 

\begin{center}
\begin{tabular}{|l||c|c|c|}
  \hline
  Atmungstyp & $\delay_{R,\tmin}/$s & $\delay_{R,\tmax}/$s & $\bar \delay_{R}/$s \\
  \hline
  regelmäßig   &  0,27 &  0,42   &  0,33$\pm$0,04 \\
  periodisch        &  0,24 &  0,36   &  0,31$\pm$0,03 \\
  \hline
\end{tabular}
\end{center}

\epsfigdouble
{anwendung/redund/mut_r_3}
{anwendung/redund/mut_p_1}
{Redundanzanalyse für die Zeitreihen mit regelmäßiger (oben) bzw.\  periodischer
  (unten) Atmung. Die Verzögerungszeiten
  liegen bei $\delay_R=0,33 s$ für regelmäßiges bzw.\ bei $\delay_R=0,27 s$ für
  periodisches Atmen. 
}
{medredund}{-0.5cm}

\clearpage
\subsection{Versuch einer Phasenraumrekonstruktion}
Mit den so gewonnenen Verzögerungszeiten wurden Darstellungen der Phasenraumrekonstruktionen
erzeugt. Es sei darauf hingewiesen, daß diese Abbildungen {\em keine} gültigen
Rekonstruktionen eventuell zugrunde liegender seltsamer Attraktoren darstellen, da die
verwendete Einbettungsdimension $\embed=2$ mit Sicherheit nicht ausreichend ist. Sie sollen
nur Hinweise auf die  Struktur der Dynamik und Anhaltspunkte für die weitere
Verfahrensweise geben.

In der Rekonstruktion des regelmäßigen Atemsignals erkennt man deutlich eine toroidale
Struktur (siehe \psref{medrekonst} links oben). Diese ist allerdings gestört durch Rauschen
sowie ein gewisses \naja(Hin-und-her-Driften) der Trajektorien entlang der
Hauptdiagonalen; die Rekonstruktion erweckt den Eindruck eines Grenzzyklus, der entlang dieser Diagonalen hin
und her geschoben wird. Dies kann als Hinweis auf
eine Instationarität der Zeitreihe aufgefaßt werden. Die Rekonstruktion des periodischen Atmens
zeigt keine klare Struktur sondern eher ein \naja(Knäuel) durcheinander laufernder
Trajektorien (siehe \psref{medrekonst} rechts oben).  Um das sichtbar starke Rauschen zu
unterdrücken wurde für beide Signale eine Singular Value Decomposition durchgeführt und
die erste Komponente der SVD-Rekonstruktion wieder eingebettet (siehe \psref{medrekonst}
unten). Die Trajektorien scheinen ein wenig geglättet zu sein, lassen jedoch auch nicht
mehr von der Struktur der Dynamik erahnen. 

\afterpage{
\epsfigfour
{anwendung/rects/rct_r_3}
{anwendung/rects/rct_p_1}
{anwendung/rects/rctsf_r_3}
{anwendung/rects/rctsf_p_1}
{Oben dargestellt sind die Rekonstruktionen eines regelmäßigen (links) bzw.\  periodischen Atemsignals (rechts)
  zur Einbettungsdimension $\embed=2$ und zur Verzögerungszeit $\delay_R=3,2$ s; unten
 die entsprechenden Rekonstruktionen der über SVD gefilterten Signale.
}
{medrekonst}{-0.5cm}
}

Die Instationarität der Zeitreihen läßt sich auch an Änderungen der Verteilungen
$n(x)$ der Meßwerte über verschiedene Zeiträume beobachten, wobei die Größe $n(x)$
besagt wieviele Meßwerte in ein enges Intervall um den Meßwert $x$ fallen.
\psref{medinstat} zeigt jeweils die Verteilungen sowohl für die gesamte als auch getrennt
für die erste und zweite Hälfte der Zeitreihe. Zwischen den Verteilungen für beide
Hälften bestehen signifikante Unterschiede. Sowohl die Lage des Maximums
als auch die Form der Verteilung variieren zwischen den beiden Hälften der Zeitreihen 
deutlich. Dies läßt sich auch quantitativ über die Mittelwerte und die
Streuungen der Verteilungen ausdrücken. Bei der Zeitreihe des regelmäßigen Atemsignals
beträgt der Unterschied der Mittelwerte beider Hälften $1,2$ Prozent und der Unterschied
der Streuungen $9,5$ Prozent, wobei die Prozentangaben jeweils auf die Streuung
der gesamten Zeitreihe bezogen sind. Bei
der Zeitreihe aus periodischer Atmung betragen die entsprechenden Unterschiede $5,5$
Prozent bei den Mittelwerten und $55,8$ Prozent bei den Streuungen. Zum Vergleich: Bei
einer Zeitreihe des Lorenz-System mit vergleichbarer Länge betrugen die gemessenen
Unterschiede nur $0,5$ bzw. $0,4$ Prozent.

Es sei an dieser Stelle noch darauf hingewiesen, daß die bei diesen beiden
Zeitreihen beobachteten Instationaritäten noch relativ gering sind und bei anderen
Zeitreihen weit gravierender ausfallen.

\afterpage{
\epsfigdouble
{anwendung/instat/cmphsregel}
{anwendung/instat/cmphsperiod}
{ Verteilungen $n$ der Meßwerte $x$ für das regelmäßige Atmen (oben) und das periodische 
  Atmen (unten). Die Kurve für die gesamte Zeitreihe (4 min) ist jeweils durchgezogen, für die beiden 
  Hälften der Zeitreihe (je 2 min) jeweils gestrichelt dargestellt.
}
{medinstat}{-0.5cm}
}

\subsection{Bestimmung der Korrelationsintegrale und -dimensionen}
Für die Atemsignale wurden Korrelationsintegrale bis zu einer Einbettungsdimension
$\embed=8$ berechnet. Zwar sind bei dem gegebenen Datenumfang maximal
Korrelationsdimensionen bis $\corrdim\simeq 4$ berechenbar, was auch nur eine
Einbettungsdimension von $d=4$ erfordern würde, jedoch läßt die Berechnung  bis $d=8$
erkennen, ob die berechneten Integrale bzw.\  Dimensionen konvergieren oder nicht.

Die zu den Parameterwerten $d_\tmax=8$, $\delay=0,32$ s und $W=60$ berechneten
Korrelationsintegrale zeigt \psref{medcorrint}. In dem Korrelationsintegral der
regelmäßigen Atmung scheint ein Skalierungsbereich zwischen $\ln r$-Werten von  $-4$ und 
$-3$ zu liegen (siehe \psref{medcorrint} oben), bei der periodischen Atmung könnte man einen Skalierungsbereich zwischen
$-4,5$ und $-3$ vermuten (siehe \psref{medcorrint} unten).
Ob die Steigung des Korrelationsintegrals in diesen Bereichen wirklich konstant ist,
läßt sich jedoch besser an den entsprechenden Slopeplots erkennen erkennen. Für die
regelmäßige Atmung existiert (entgegen dem bei den Korrelationsintegralen gewonnenen Eindruck)
kein größerer Bereich von $r$-Werten
mit konstantem $\corrdim(r)$; weiterhin ist keine Konvergenz der Steigungen festzustellen
(siehe \psref{medcorrslp} oben). Für die
periodische Atmung scheint bei etwa $\ln r=-4$ Konvergenz aufzutreten (siehe
\psref{medcorrslp} unten). Dies ist jedoch eher
als Artefakt des bei höheren Einbettungsdimensionen auftretenden \begriff(Randeffekts)
zu deuten. Auch die über Takens' Schätzverfahren zum Maximalabstand $\ln\rmax=-2,5$ bestimmten
Dimensionen zeigen keine Konvergenz (siehe \psref{medcorrdim}). 


\epsfigdouble{anwendung/corrint/ci_r_3}{anwendung/corrint/ci_p_1}
{
Korrelationsintegrale der beiden ausgewählten Zeitreihen für regelmäßige (oben) und
periodische (unten) Atmung mit den Parameterwerten $\delay=0,32$ s und $W=60$ für die
Einbettungsdimensionen $d=1,\dots,8$ (jeweils von oben nach unten).
}
{medcorrint}{-0.5cm}

\epsfigdouble{anwendung/corrint/cs_r_3}{anwendung/corrint/cs_p_1}
{
Slopekurven zu den Korrelationsintegralen aus \psref{medcorrint} für die 
Einbettungsdimensionen $d=1,\dots,8$ (jeweils von unten nach oben).
}
{medcorrslp}{-0.5cm}

\epsfigdouble{anwendung/corrint/td_r_3}{anwendung/corrint/td_p_1}
{
Über Takens' Schätzverfahren berechnete Korrelationsdimension  zu den
Korrelationsintegralen aus \psref{medcorrint} mit dem Parameterwert $\ln\rmax=-2,5 $.
}
{medcorrdim}{-0.5cm}



Die Berechnungen wurden auch mit Variationen der Parameterwerte sowie mit verschiedenen
auf die Zeitreihen angewandten Filtertechniken
ausgeführt. Die nachfolgend aufgezählten Parametereinstellungen und Filtermethoden
wurden auf verschiedene Weisen kombiniert, führten jedoch zu keiner
Verbesserung der Ergebnisse.
\begin{description}
\item[Filterung:] Es wurden Zeitreihen mit und ohne SVD-Filterung verwendet, wobei für
  die Singular Value Decomposition Einbettungsdimensionen zwischen $d=5$ und $d=30$
  gewählt wurden. Weiterhin wurden Tiefpaßfilterungen mit Grenzfrequenz bei 2,5 Hz
  durchgeführt, um Beeinflussungen durch die Herzdynamik zu minimieren.
\item[Stationarität:] Biologische Zeitreihen sind, wie gesehen, oftmals instationär und weisen von
  der eigentlichen Dynamik unabhängige langreichweitige Schwankungen (Bias) auf. 
  Um diese herauszufiltern wurden Fenster über die Zeitreihe gelegt, in denen dieser
  Grundwert berechnet und nachfolgend von den Signalwerten subtrahiert wurde.
\item[Verzögerungszeiten:] Kleine Verzögerungszeiten bringen bei der Berechnung der
  Korrelationsintegrale oftmals besseres Ergebnisse als die berechneten
  Verzögerungszeiten. Es wurden Verzögerungszeiten bis hinunter zu $\delay=0,1$ s
  getestet.
\item[Autokorrelationszeit:] Um Effekte durch Autokorrelationen auszuschließen, wurde
  der Parameter für die Autokorrelationslänge $W$ von 1 bis zu Vielfachen der
  Verzögerungszeit ausgetestet.
\end{description}
Da eine endliche fraktale Dimension Voraussetzung für die Existenz eines seltsamen
Attraktors ist und die Korrelationsanalyse (die Korrelationsdimension ist eine untere
Abschätzung für die fraktale Dimension) keine konvergenten Ergebnisse erbrachte, können wir somit
nicht auf die Existenz eines solchen schließen. 

Für den Atemtypus der regelmäßigen Atmung ist noch eine andere (sehr vorsichtige)
Deutung möglich. Es könnte sich hierbei um eine Grenzzyklus handeln, der jedoch stark
verrauscht ist. Dies stünde sowohl im Einklang mit dem gemessenen Fourier-Spektrum, das
relativ klar hervortretende Peaks aufweist, als auch mit 2 bzw.\ 3-dimensionalen
Rekonstruktionen, die eine toroidale Struktur zeigen. Zudem weisen nicht-chaotische Systeme
einen verkürzten Skalierungsbereich im Korrelationsintegral auf: Das Korrelationsintegral
skaliert nur über einen Bereich der Ordnung \order{N} statt der Ordnung \order{N^2} wie
im chaotischen Fall \cite{Theiler}.

\subsection{Vorhersagbarkeit von Apnoen}
Da die im vorigen Abschnitt berechneten Korrelationsintegrale keinen Skalierungsbereich
aufwiesen und auch keine Konvergenz der Slopekurven vorlag, konnte die Existenz eines
niedrigdimensionalen Attraktors mit bestimmter Korrelationsdimension $\corrdim$ für die
Atmungsdynamik nicht nachgewiesen werden. Betrachtet man den Wert der für eine bestimmte
Einbettungsdimension $d$ und eine bestimmte Verzögerungszeit $\delay$ berechneten
Korrelationsdimension $\corrdim$ jedoch nicht als wirkliche Dimension des zugrunde
liegenden dynamischen Systems sondern als ein Maß, daß die relative Komplexität der
Dynamik in dem durch die Zeitreihe gegebenen Zeitabschnitt beschreibt, können die für
verschiedene zeitlich versetzte Zeitreihen berechneten Werte Änderungen dieses
Komplexitätsmasses anzeigen und so eventuell eine Vorhersage von Atemstillständen
ermöglichen.

Zum Test dieser Möglichkeit wurde eine Zeitreihe, in der zwei Apnoen kurz
hintereinander auftraten,
verwendet (siehe \psref{medtestapnoe} oben). Die Länge dieser Zeitreihe betrug 9 Minuten
(statt der üblicherweise benutzen 4 minütigen Zeitreihen), damit auch in  größeren
Zeitabständen vor Auftreten der Apnoen die Korrelationsdimensionen berechnet werden
konnten. Aus dieser Zeitreihe wurden Stücke von jeweils 2 Minuten Länge
extrahiert, wobei die Anfangszeiten dieser Zeitreihen jeweils um 15 Sekunden verschoben
wurden. Jede dieser Zeitreihen erstreckt sich somit über einen Zeitraum von $t_{i,a}=i
15\sek$ bis $t_{i,e}=i 15\sek + 120\sek$. Es wurden nun die
entsprechenden Korrelationsintegrale zur Einbettungsdimension $d=5$ berechnet und die Korrelationsdimensionen $D_{2,i}$ über Takens' 
Schätzverfahren mit dem Maximalabstand $\ln\rmax=-2,5$
bestimmt. Die Berechnung der Korrelationsdimension geschieht immer über einen bestimmten
Zeitraum, so daß eine Auftragung der Korrelationsdimension über der Zeit nicht ganz
trivial ist.
Da zur Berechnung der Korrelationsdimension die Zeitreihe jedoch komplett vorliegen
muß, ist es sinnvoll bei dieser Auftragung jeweils das zeitliche Ende der Reihe als Wert auf
der Zeitachse zu wählen, d.h. man betrachtet die Auftragung von $D_{2,i}$ über
$t_{i,e}$. Das Ergebnis einer solchen Berechnung zeigt \psref{medtestapnoe} (unten).

\epsfigdouble
{anwendung/apnoe/apn2}
{anwendung/apnoe/dimensionen}
{
Oben die verwendete Zeitreihe mit Apnoen bei $t=390\sek$ und $t=460\sek$. Unten die
zeitabhängige Korrelationsdimension $D_2(t)$ für $t\geq 120\sek$. Für $t<120\sek$ ließ 
sich die Berechnung aufgrund der Länge der Zeitreihen von 2 min nicht durchführen.
}
{medtestapnoe}{-0.5cm}

Die Werte der Korrelationsdimension für $t\leq390\sek$ schwanken in einem Bereich
zwischen ca.\ $4,0$ und ca.\ $4,3$, wobei in diesen Schwankungen kein bestimmter Trend auszumachen ist.
Ferner liegen die Werte in einem Bereich, der auch bei anderen Zeitreihen, in denen
keine Apnoen auftreten, typisch ist. Erst bei $t=405\sek$ zeigt sich ein deutlicher Abfall
der Korrelationsdimension auf ca.\ $2,1$, die sich nachfolgend langsam wieder zu höheren
Werten hin entwickelt.  Da der Atemstillstand jedoch schon bei $t=400\sek$ eintritt, ist
die Korrelationsdimension folglich für die Vorhersage von Apnoen nicht zu gebrauchen.


\subsection{Determinismustest über AAFT-Surrogate}

Trotz der negativen Ergebnisse bei der Dimensionsanalyse können die für bestimmte
Einbettungsdimensionen $d$ und für einen bestimmten Maximalabstand $\rmax$ berechneten Werte der
Korrelationsdimension zum Vergleich mit entsprechenden Werten von Surrogatdaten im Rahmen
eines Determinismustests herangezogen werden. Da die Dichteverteilung der Meßwerte eine
deutlich nicht-gaußsche Verteilung zeigen (siehe \psref{medinstat}) wurden als Surrogatdaten
AAFT-Surrogate verwendet. Die Anzahl der erstellten Datenreihen betrug in beiden Fällen
(d.h. für regelmäßige bzw.\ periodische Atmung) 39. Für diese wurden mit den gleichen
Parametern ($\embed_\tmax=8, \delay=0,32, W=60)$ wie für die Originalzeitreihen
Korrelationsintegrale berechnet. Aus diesen wurde über Takens' Schätzverfahren die
Korrelationsdimension $\corrdim$ in Abhängigkeit von der Einbettungsdimension bestimmt. Die
Maximalabstände lagen bei $\rmax=-2,5$. Den Vergleich der Dimensionsberechnungen zwischen
Original- und Surrogatdaten zeigt \psref{medsurrodim}. Deutlich zu erkennen ist, daß sich 
für beide Atemtypen bei Einbettungsdimensionen $d\geq4$ die Korrelationsdimension der
Originaldaten stark von den Korrelationsdimensionen der Surrogatdaten unterscheidet.
Aus der Verteilung der Korrelationdimensionen wurde gemäß \eqnref{eqnsigni} die
Signifikanz $\mathcal S$ der Abweichungen berechnet (siehe \psref{medsurrosigni}).
Außer für $d=7$ und $d=8$ bei der regelmäßigen Atmung und $d=1,\dots,3$ bei der
periodischen Atmung liegt die Signifikanz sehr hoch ($\mathcal{S}>5$). 
Die entsprechenden $p$-Werte (siehe \eqnref{eqnpvalue}) liegen in diesen Fällen unter
$10^{-7}$. 

Da die $p$-Werte angeben, mit welcher Wahrscheinlichkeit die Ablehnung der Nullhypothese
falsch ist, können wir mit über $99,9$ prozentiger Sicherheit davon ausgehen, daß es
sich bei der Atmungsdynamik um kein durch ein ARMA-Modell beschreibbares stochastisches
System handelt. Selbst bei Zugrundelegung des vorher festlegegten Signifikanzniveaus
$\alpha=0,05$, das ja für die minimale Anzahl der berechneten Surrogatzeitreihen
ausschlaggebend ist, liegt die Wahrscheinlichkeit, daß der Atmung ein deterministischer
Charakter zugesprochen werden kann, bei mindestens $95$ Prozent.

\epsfigdouble{anwendung/surrogat/regel/cmpsurcdim}{anwendung/surrogat/period/cmpsurcdim}
{
Vergleich der berechneten Korrelationdimensionen zwischen Original- \gpmarkb\  und
Surrogatdaten \gpmarka\  für regelmäßige Atmung (oben) und periodische Atmung (unten). 
}
{medsurrodim}{-0.5cm}

\epsfigdouble{anwendung/surrogat/regel/signi}{anwendung/surrogat/period/signi}
{
Signifikanz $\mathcal{S}$ der Abweichung der Korrelationsdimension der Originaldaten vom
Mittelwert der Korrelationsdimensionen der Surrogatdaten für regelmäßige Atmung (oben)
und periodische Atmung (unten).
}
{medsurrosigni}{-0.5cm}

\comment{
{anwendung/surrogat/regel/pvalue}
{anwendung/surrogat/period/pvalue}
}


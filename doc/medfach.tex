\chapter{Medizinische Fachbegriffe} 

Die meisten der hier erl"auterten medizinischen Fachbegriffe stammen aus ``Pschyrembel --
Klinisches W"orterbuch'' \cite{Pschyrembel}.  F"ur Fr"uhgeborene spezifische "Anderungen
oder Erweiterungen der Definitionen sind aus \autor(Poets)  \cite{Poets93} und
\autor(Hoch) und \autor(Bergmann) \cite{Hoch96} erg"anzt worden.

\begin{description}
\item[Bradykardie:] langsame Schlagfolge des Herzens mit einer Pulsfrequenz unter 60/min,
  bei Fr"uhgeborenen schon ab unter 90-100/min.
\item[Gestationsalter:] Schwangerschaftsdauer, Reifezeichen des Neugeborenen.
\item[Hypox"amie:] niedriger Sauerstoffpartialdruck im arterielle Blut ($\mathrm{pO_2}<70$
  mmHg). Bei Neugeborenen ein Abfall auf unter 40 -- 45 mmHg bzw.\  unter 20\% des
  Basalwertes. 
\item[Nervus Vagus:] Wurzel des Parasympathikus (Teil des vegetativen Nervensystems), steuert
  Erholungsprozesse von Herz, Lungen, Magen-Darm-Trakt und einigen weiteren Systemen.
\item[Pulsoxymetrie:] transkutane (unblutige) Messung der arteriellen
  Sauerstoffs"attigung.
\item[QRS-Komplex:] Phase der Erregungsausbreitung in den Herzkammern. "Au"sert sich im
  EKG durch kurzen, sehr starken Ausschlag. 
\item[REM-Schlaf:] Abk"urzung f"ur engl.:\  rapid eye movement. Schlafphase mit raschen
  Augenbewegungen und erh"ohter Herz- und Atemfrequenz.
\item[thorakal:] zum Brustkorb (Thorax) geh"orig.
\item[Zyanose:] blau-rote F"arbung von Haut und Schleimh"auten infolge einer Abnahme des
  Sauerstoffgehalts im Blut.
\end{description}


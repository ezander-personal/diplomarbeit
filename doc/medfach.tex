\chapter{Medizinische Fachbegriffe} 

Die meisten der hier erläuterten medizinischen Fachbegriffe stammen aus ``Pschyrembel --
Klinisches Wörterbuch'' \cite{Pschyrembel}.  Für Frühgeborene spezifische Änderungen
oder Erweiterungen der Definitionen sind aus \autor(Poets)  \cite{Poets93} und
\autor(Hoch) und \autor(Bergmann) \cite{Hoch96} ergänzt worden.

\begin{description}
\item[Bradykardie:] langsame Schlagfolge des Herzens mit einer Pulsfrequenz unter 60/min,
  bei Frühgeborenen schon ab unter 90-100/min.
\item[Gestationsalter:] Schwangerschaftsdauer, Reifezeichen des Neugeborenen.
\item[Hypoxämie:] niedriger Sauerstoffpartialdruck im arterielle Blut ($\mathrm{pO_2}<70$
  mmHg). Bei Neugeborenen ein Abfall auf unter 40 -- 45 mmHg bzw.\  unter 20\% des
  Basalwertes. 
\item[Nervus Vagus:] Wurzel des Parasympathikus (Teil des vegetativen Nervensystems), steuert
  Erholungsprozesse von Herz, Lungen, Magen-Darm-Trakt und einigen weiteren Systemen.
\item[Pulsoxymetrie:] transkutane (unblutige) Messung der arteriellen
  Sauerstoffsättigung.
\item[QRS-Komplex:] Phase der Erregungsausbreitung in den Herzkammern. Äußert sich im
  EKG durch kurzen, sehr starken Ausschlag. 
\item[REM-Schlaf:] Abkürzung für engl.:\  rapid eye movement. Schlafphase mit raschen
  Augenbewegungen und erhöhter Herz- und Atemfrequenz.
\item[thorakal:] zum Brustkorb (Thorax) gehörig.
\item[Zyanose:] blau-rote Färbung von Haut und Schleimhäuten infolge einer Abnahme des
  Sauerstoffgehalts im Blut.
\end{description}


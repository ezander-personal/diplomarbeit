\addchap{Schlu"s} 

Die in die Zeitreihenanalyse gesetzten Erwartungen f"ur eine Charakterisierung der
Atmungstypen Fr"uhgeborener und deren Beschreibung durch seltsame Attraktoren,
insbesondere im Hinblick auf eine m"ogliche Fr"uherkennung von Atemstillst"anden, konnte
in der vorliegenden Arbeit nicht erf"ullt werden. Es konnte zwar gezeigt werden, da"s
beide Atmungstypen nicht durch stochastische Prozesse zu beschreiben sind, und ihnen somit
ein determistischer Charakter eigen ist, andererseits konnte die genaue Struktur der
Dynamik nicht identifiziert werden. Hierf"ur lassen sich mehere m"ogliche Gr"unde angeben

\begin{itemize}
\item Die Atmung, auf die reine Lungenaktivit"at reduziert, stellt kein autonomes System
  dar. In der Dynamik spielen noch weitere Einfl"usse eine Rolle die aus dem TI-Signal
  nicht rekonstruierbar sind. Insofern ist eine der Grundlagen f"ur die Verwendung der
  Verz"ogerungskoordinatenabbildung verletzt, da in diesem Fall selbst beliebig viele verz"ogerte
  Werte nicht den kompletten Systemzustand wiedergeben k"onnen.  Eventuell m"u"ste hier
  der Zustand der atemkompetenten Neuronengruppen in eine Analyse miteinbezogen werden.
  
\item In die gleiche Richtung wie der vorige Punkt geht die Annahme, da"s
  Kontrollparameter existieren, die w"ahrend der Aufnahme der Zeitreihe variieren k"onnen.
  Dies "au"sert sich in zeitlichen "Anderungen der nat"urlichen Dichteverteilungen in den
  Phasenraumrekonstruktionen. Hierdurch sind Verfahren wie Dimensionsalgorithmen nicht
  mehr anwendbar, da sie auf station"aren Dichten beruhen.
  
\item Das gemessene Signal unterliegt einer Reihe weiterer nicht atemkorrelierter
  Einfl"usse wie beispielweise der Herzt"atigkeit. Ein Signal, in dem solche Einfl"usse
  ausgeschlossen werden k"onnen (z.B. Volumenstrommessungen), mag bessere Ergebnisse
  liefern als die thorakale Impedanz.
  
\item Im Gegensatz zu anderen physiologischen Vorg"angen wie der Herzt"atigkeit kann die
  Atmung leicht bewu"st beeinflu"st werden. So k"onnen Vorg"ange wie Schlucken,
  Laut"au"serungen oder Innehalten die Aussagekraft des Signals bez"uglich der
  Atmungsdynamik stark beeintr"achtigen.
\end{itemize}





\subsection{Singular Value Decomposition}
\label{chapsvd}
Eine weitere Methode der Attraktorrekonstruktion stammt von \autor(Broomhead) und
\autor(King) \cite{broomhead-king}. Diese sogenannte \begriff(Singular Value
Decomposition) (SVD) hat gegen"uber der Verz"ogerungskoordinateneinbettung haupts"achlich
zwei Vorteile. Zum einen mu"s die Einbettungsdimension nicht vorher festgelegt oder "uber
andere Verfahren ermittelt werden. Zum anderen kann durch die SVD ein dem Signal
eingepr"agtes additives Rauschen vermindert werden.

%Grundidee
Die Grundidee ist folgende. Wir betten den Attraktor in einen hochdimensionalen Raum
$\R^\embed$ der Dimension $\embed$ ein. Dann spannt der so rekonstruierte Attraktor
$\attr$ in der Regel nur einen $\minembed$-dimensio\-nalen Unterraum $\subspace$ des
$\R^\embed$ auf, wobei die Dimension dieses Unterraums im allgemeinen sehr viel niedriger
als die des Einbettungsraumes ist.  Dieser Unterraum $\subspace$ wird nun bestimmt und der
Attraktor $\attr$ hieraus in den niedrigdimensionalen $\R^\minembed$ projiziert.

%Neue Basis finden
Hauptbestandteil des Verfahrens ist es, eine neue, orthonormale Basis $\folge(\vec
c,1,\embed)$ des Einbettungsraumes $\R^\embed$ zu bestimmen. Diese soll so bestimmt sein,
da"s die ersten $\minembed$ Vektoren der Basis den Unterraum $\subspace$ aufspannen m"ogen
\eqn{\subspace = \mathop{\mathrm{span}}(\folge(\vec c,1,\minembed))} Da $\subspace$ nach
Voraussetzung auch von den Rekonstruktionsvektoren $\folge(\x,1,N)$ aufgespannt wird,
lassen sich die $\vec c_i\mraise{\rvert}{-0.2}_{i=1\dots\minembed}$ offensichtlich als
Linearkombination der $\x_j$ darstellen \eqnl[svdkomp]{\vec c_i = \sum_{k=1}^N
  \frac{1}{\sigma_i} s_{ik} \x_k} Hierbei seien die $\vec s_i = (s_{i1},\dots,s_{iN})^\Tr$
Vektoren aus $\R^N$. Da die $\vec c_i$ linear unabh"angig sind, gilt dies auch f"ur die
$\vec s_i$\footnote{Sonst h"atte die Matrix $\mat S=(\vec s_1,\dots,\vec s_\minembed)$ den
  Rang $\rang(S)<\minembed$ und die $\vec c_i$ k"onnten nach \eqnref{svdkomp} nicht mehr
  linear unabh"angig sein.}. Wir nehmen weiterhin an, da"s die letzteren orthogonalisiert
sind. Au"serdem seien ihr Betrag durch den Faktor $1/\sigma_i$ auf eins normiert.  Wir
definieren nun die \begriff(Trajektorienmatrix) $\mat X$ durch\footnote{In
  \cite{broomhead-king} wird die Trajektorienmatrix durch $\mat X = N^{-1/2}(\x_1, \dots,
  \x_N)^\Tr$ definiert. Die hier getroffene Definition l"a"st die Ergebnisse unver"andert,
  f"uhrt jedoch zu einfacheren Gleichungen.}  \eqn{\mat X = N^{-1/2}(\x_1, \dots, \x_N)}
wobei der Faktor $N^{-1/2}$ der Normierung dient.  \eqnref{svdkomp} kann dann in der
folgenden Form geschrieben werden \eqnl[svdbase]{\sigma_i \vec c_i = \mat X \vec s_i}
%Strukturmatrix
Multiplizieren wir die linke und rechte Seite von \eqnref{svdbase} mit ihrer jeweiligen
Transponierten, so erhalten wir unter Ausnutzung der Orthonormalit"at der $\vec c_i$
\eqnl[svdbla1]{\sigma_i \sigma_j \delta_{ij} = \vec s_i^\Tr \tmat X \mat X s_j}
Die $N\times N$ Matrix $\gmat \Theta = \tmat X \mat X$, aufgrund ihres Aufbaus auch
\begriff(Strukturmatrix) genannt, ist reell und symmetrisch. Ihre
Eigenvektoren bilden also eine vollst"andige, orthonormale Basis des $\R^N$. 
Eine L"osung der obigen Gleichung erh"alt man, indem man f"ur $\vec s_i$ 
den $i$ten Eigenvektor von $\gmat \Theta$ sowie f"ur $\sigma_i$ die Wurzel des entsprechenden Eigenwerts w"ahlt
\eqnl[svdeigen1]{\gmat \Theta = \sigma_i^2 \vec s_i}
Dies best"atigt man leicht durch Einsetzen in \eqnref{svdbla1} und Ausnutzung der Orthogonalit"at der
Eigenvektoren symmetrischer Matrizen. 

%Kovarianzmatrix
Die Bestimmung des Unterraumes $\subspace$ erfordert also die
Diagonalisierung der Strukturmatrix $\gmat \Theta$. Da $\gmat \Theta$
eine $N\times N$-Matrix ist, wird dies f"ur gro"se $N$ jedoch
sehr rechenaufwendig. Die Diagonalisierung einer $N\times N$-Matrix ben"otigt \order{N^3}
Schritte. 
 Da nach \eqnref{svdbase} maximal $\embed$ Eigenwerte
$\sigma_i^2$ gr"o"ser als Null sein k"onnen und $\gmat \Theta$ somit maximal den Rang
$\embed$ hat, lohnt es sich nach einem schnelleren Weg zu suchen. 
Multiplizieren wir \eqnref{svdbase} von links mit der
\begriff(Kovarianzmatrix) $\gmat \Xi = \mat X \tmat X$, so erhalten wir mit \eqnref{svdeigen1}
\eqna{\gmat \Xi\sigma_i \vec c_i &=& \mat X \tmat X \mat X \vec s_i \nonumber\\
&=& \mat X \sigma_i^2 \vec s_i}
F"ur $\sigma_i\neq 0$ ergibt sich dann mit \eqnref{svdbase} 
\eqnl[svdeigen2]{\gmat \Xi \vec c_i = \sigma_i^2 \vec c_i}
Die neue Basis $\vec c_i$ erhalten wir  
also einfacher durch Diagonalisierung der Kovarianzmatrix $\gmat \Xi$. Da $\gmat \Xi$ eine
$\embed\times\embed$-Matrix und $\embed\ll N$ ist, kann \eqnref{svdeigen2} mit sehr viel
weniger Rechenaufwand als \eqnref{svdeigen1} gel"ost werden.

%Bedeutung der Eigenwerte
Der vom Attraktor aufgespannte Unterraum $\subspace$ ist durch die Eigenvektoren $\vec
c_i$ gegeben, deren zugeh"orige Eigenwerte $\sigma_i$ nicht verschwinden. Um die genaue
Bedeutung der $\sigma_i$ m"ussen wir uns nun Gedanken machen.

Um die Bedeutung der Eigenwerte zu erhellen, m"ussen wir erst die Struktur des neuen
Phasenraumes analysieren. Die Basis des Rekonstruktionsraumes wird gebildet durch die
Eigenvektoren der Kovarianzmatrix $\vec c_i$. Die Darstellung der Rekonstruktionsvektoren
$\vec x'_i$ bez"uglich dieser neuen Basis erhalten wir "uber
\comment{\eqn{\vec y_j=\left( \begin{array}{c} \x_j \vec c_1 \\ \x_j \vec c_2 \\ \vdots \\ \x_j \vec c_\embed
\end{array} \right)}}
\eqnl[svdrecvec]{\vec x'_i = \sum_{j=1}^{n} (\vec x_i \cdot \vec c_j) \vec e'_j } 
Die mittlere Ausdehnung $\delta'_k$ des rekonstruierten Attraktors in Richtung eines Basisvektors
$\vec e'_k$ betr"agt
\eqn{ \delta'_k = \left( \sum_{i=1}^N (\x'_i \cdot \vec e'_k)^2 \right)^{1/2} }
Durch Einsetzen von \eqnref{svdrecvec} und der Definition der Trajektorienmatrix folgt daraus
\eqna{ \delta'_k &=& \left( \tmat X \vec c_k \cdot \tmat X \vec c_k  \right)^{1/2}
\nonumber \\
&=& \left( \vec c_k \cdot \gmat \Xi \vec c_k  \right)^{1/2} \nonumber \\ 
&=& \sigma_k }
%&=& \left( \sum_{i=1}^N (\x_i \cdot \vec c_k)^2 \right)^{1/2}  \\ 
%&=& \left( (\mat X \vec c_k)^2 \right)^{1/2}  }
Die Eigenwerte $\sigma_k$ geben also an, wie weit der Attraktor in Richtung $\vec c_k$
im Mittel aufgespannt wird. Die Verteilung der Attraktorpunkte kann man sich in etwa vorstellen als ein
$\embed$-Ellispsoid dessen Halbachsen durch die Eigenwerte $\sigma_k$ der Kovarianzmatrix
gegeben sind. 

%Rauschen
Bevor wir auf die Anwendung des Verfahrens kommen, m"ussen wir uns mit dem Einflu"s von
Rauschen auf die Singular Value Decomposition besch"aftigen, da selbst in ``rauschfreien''
Systemen aufgrund endlicher speicher- und rechengenauigkeit Rauschen erzeugt wird. 
Wir betrachten eine Zeitreihe mit additivem, gau"sverteiltem Rauschen $x_i = \bar x_i +
\xi_i$. 
Ein "Uberstrich symbolisiert hier wie im folgenden die deterministische Komponente. 
Wir m"ussen nun den Einflu"s des Rauschens auf die Kovarianzmatrix betrachten. 
Aus der Definition der Kovarianzmatrix folgt
\def\sumin{{\sum\limits_{i=1}^N}}
\eqnl[svdkovdef]{\gmat \Xi = \frac{1}{N}\left( \begin{array}{cccc}
 \sumin x_i x_i & \sumin x_i x_{i+1} & \dots & \sumin x_i x_{i+\embed-1} \\
 \vdots         & \vdots             & \ddots & \vdots \\
 \sumin x_{i+\embed-1} x_i & \sumin x_{i+\embed-1} x_{i+1}& \dots & \sumin x_{i+\embed-1} x_{i+\embed-1}  
\end{array} \right)}
\comment{Da die Rauschterme $\xi_i$ unkorreliert sind, ergibt sich f"ur die Elemente von $\gmat
\Xi$ bei gro"sen Datenmengen $N$}
Die $x_i$ ersetzen wir nun durch die Summe aus deterministischer Komponente $\bar x_i$ und 
Rauschkomponente $\xi_i$. Da die Rauschterme $\xi_i$ untereinander unkorreliert sind, 
ergibt die Summation "uber $\xi_i\xi_{i+k}$ bei gro"sen Datenmengen $N$ Null. Da die $\xi_i$ und die $x_i$ auch
unkorreliert sind, ergibt die Summation "uber $\xi_i x_{i+k}$ ebenfalls Null. Wir erhalten 
so f"ur die Elemente von $\gmat\Xi$
\eqn{\gmat \Xi_{kl} = \frac{1}{N}\left( \underbrace{\sumin x_{i+k} x_{i+l}}_{= \overline{\gmat \Xi}_{kl}} + 
\underbrace{\sumin x_{i+k} \xi_{i+l}}_{\approx 0} + 
\underbrace{\sumin \xi_{i+k} x_{i+l}}_{\approx 0} + 
\underbrace{\sumin \xi_{i+k} \xi_{i+l}}_{\approx N <\xi^2>\delta_{kl}} \right) }
und somit
\eqn{\gmat \Xi = \overline{\gmat \Xi}\; + <\!\xi^2\!>\unity_\embed}
Die Eigenwerte der Kovarianzmatrix $\sigma_i^2$ werden also einheitlich um $<\!\xi^2\!>$
erh"oht. 
Die Wirkung additiven Rauschens besteht darin, die L"ange der Hauptachsen des oben beschriebenen
Ellipsoids gleichm"a"sig um den Betrag $<\!\xi^2\!>$ zu vergr"o"sern. Auch die Richtungen,
f"ur die ohne Rauschen $\sigma_i=0$ galt, werden nun von Trajektorien besucht. Es wird
also der ganze Phasenraum aufgespannt.

In Richtung der Achsen mit $\sigma_i^2 \approx <\!\xi^2\!>$ wird die Dynamik allerdings
vollst"andig durch die Rauschterme dominiert. 
F"ur eine gute Rekonstruktion des Attraktors wird also nur der Teilraum
$\subspace=\mathop{\mathrm{span}}\{\vec c_1,\dots,\vec c_{\minembed}\}$ benutzt, f"ur den
die entsprechenden Eigenwerte gr"o"ser als die St"arke des Rauchens ist.

Wir k"onnen nun das Verfahren zusammenfassen.
\begin{enumerate}
\item Aus der Zeitreihe wird zu gegebenem $\embed$ die Kovarianzmatrix $\gmat \Xi$
berechnet.
Aufgrund der Symmetrie der Matrix $\gmat \Xi_{ij}=\gmat \Xi_{ji}$ reicht es aus, nur die
rechte, obere H"alfte der Matrix zu berechnen. Zus"atzlich ergibt sich aus der Definition
\eqnref{svdkovdef} die Beziehung
$\gmat\Xi_{i+1,j+1}=\gmat\Xi_{ij}+\frac{1}{N}(x_{N+i}x_{N+j}-x_ix_j)$. Hierdurch kann die Anzahl der
ben"otigten Multiplikationen und Additionen betr"achtlich gesenkt werden.
\item Die Kovarianzmatrix wird diagonalisiert und Eigenwerte sowie Eigenvektoren nach
absteigender Gr"o"se der Eigenwerte sortiert. 
Hierf"ur stehen verschiedene Algorhitmen in Form fertiger Bibliotheken zur Verf"ugung
\cite{numerical-recipes}. 
\item Die Gr"o"se des aufgepr"aten Rauschens $\xi_0$ wird abgesch"atzt. 
F"ur diese Absch"atzung existiert kein allgemein g"ultiges Verfahren.
Sie erfolgt nach den besonderen Gegebenheiten und der Herkunft der
Zeitreihe. 
Hinweise auf die Gr"o"se von $\xi_0$ sind gegeben durch die interne Rechengenauigkeit und die 
Me"sgenauigkeit bei der Aufnahme der Zeitreihe (z.B. Quantisierungsrauschen).
Liegen hier"uber keine oder ungen"ugende Angaben vor, kann $\xi_0$ auch "uber die
Eins"attigung der $\sigma_i$ f"ur gro"se $i$ abgesch"atzt werden.
\item Die Rekonstruktionspunkte k"onnen nun gem"a"s \eqnref{svdrecvec} berechnet werden,
wobei allerdings nur die ersten $\minembed$ Komponenten verwendet werden. 
Statt alle $\minembed$ Komponenten zu verwenden, kann allerdings alternativ auch nur die
erste (nun gefilterte) Komponente "uber die Verz"ogerungskoordinatenabbiltung wieder
eingebettet werden. 
Auf diese Weise dient die SVD nur einer adaptiven Rauschfilterung und herk"ommliche
Zeitreihenanalyseverfahren k"onnen wieder angewandt werden.
Ein weitere M"oglichkeit liegt darin, die ersten $n\leq\minembed$ f"ur eine multivariante
Verz"ogerungskoordinatenabbildung zu benutzen.
Darauf soll in diesem Rahmen jedoch nicht n"aher eingegangen werden.
\end{enumerate}

\subsubsection{Anwendung des Verfahrens}
Das Verfahren soll nun am Beispiel des Lorenzattraktors angewendet werden.
Nach Integration
der Differentialgleichungen wurde aus der $x$-Komponente des Zustandvektors eine Zeitreihe
gebildet.
Nach Bildung der Kovarianzmatrix zu $\embed=7$ wurden die Eigenwerte und Eigenvektoren
gebildet.
Die Ergebnisse der Berechnungen  zeigen \psref{svdval} und \psref{svdvec}. 
\epsfigsingle{svd/simple/lorsvdval}
{Logarithmus der Eigenwerte $\sigma_i$ der Kovarianzmatrix $\gmat \Xi$ f"ur den
Lorenzattraktor mit Einbettungsdimension $\embed=7$.
}
{svdval}{-0.2cm}
\epsfigtriplehigh{svd/simple/lorsvdvec1}{svd/simple/lorsvdvec2}{svd/simple/lorsvdvec3}
{Die ersten drei Eigenvektoren der Kovarianzmatrix f"ur den Lorenzattraktor
($\embed=7$). Aufgetragen ist f"ur jeden Eigenvektor die $i$-te Komponente "uber $i$.}
{svdvec}{-0.2cm}

In \psref{svdval} ist deutlich zu sehen, da"s ab $i\geq 4$ die Eigenwerte $\sigma_i$ auf
einem nahezu konstanten Wert bleiben. Dies liegt daran, da"s ab $i=4$ die Eigenwerte nur
noch durch Rauschen dominiert werden. Dieses ist allerdings nicht k"unstlich aufaddiert
worden sondern, bedingt durch die Genauigkeit mit der die Werte der Zeitreihe
zwischengespeichert wurden\footnote{Die Werte wurden mit einer Genauigkeit von 6 Stellen
zwischengespeichert, so da"s $\xi_0\simeq 10^{-6}$ und $\ln(\xi_0)\simeq -13,8$ ist.}. 

Da nur die ersten drei Eigenwerte deutlich "uber dem Rauschpegel liegen, sind auch nur die
ersten drei Eigenvektoren von $\gmat\Xi$ in \psref{svdvec}
dargestellt. Skalarmultiplikation des ersten Vektors mit aufeinanderfolgenden Werten der
Zeitreihe entspricht ungef"ahr einer Mittelung "uber diese Werte. Bei dem zweiten bzw.\
dritten Vektor entspricht die Skalarmultiplikation ungef"ahr einer Mittelung "uber die
erste bzw.\  zweite Zeitableitung. Dies hat Konsequenzen auf die wir jedoch sp"ater
genauer eingehen werden.

Wir wollen nun die Singular Value Decomposition bei Vorhandensein additiven Rauschens
untersuchen. 
Als Beispiel m"oge hier wieder eine Zeitreihe des Lorentzattraktor dienen. 
Auf diese wurde gau"ssches Rauschen in verschiedener St"arke addiert (Angaben in 
Prozent der Varianz von $x$). 
F"ur die so entstandenen, verrauschten Zeitreihen wurden jeweils die
Kovarianzmatrix $\gmat \Xi$ und ihre Eigenwerte $\sigma_i^2$ bestimmt (s. \psref{svdvalnoise}).
\afterpage{
\epsfigsingle{svd/noise/lorsvd}
{Logarithmus der Eigenwerte der Kovarianzmatrix $\gmat \Xi$ f"ur verschiedene
Rauschpegel (0\%;0,2\%;0,5\%;1,0\%;2,0\%;5,0\% )}
{svdvalnoise}{-0.2cm}
}

Man erkennt deutlich, da"s die Eigenwerte $\sigma_i$ ab $i=4$ nur durch die St"arke des
Rauschens bestimmt sind. 
Die Verh"altnisse der Eigenwerte sind durch die relative St"arke der
Rauschamplitude gegeben. 

Verschiedene Rekonstruktionen des Lorenzattraktors aus der Zeitreihe mit 5\% Rauschen sind 
in \psref{svdrecnoise} abgebildet. Man sieht deutlich wie der Signal-Rausch-Abstand durch
die SVD Rekonstruktion verbessert wird. 

\afterpage{
\epsfigtriple{svd/noise/lorrec50_1}{svd/noise/lorsrec50_1}{svd/noise/lorrecs50_1}
{Rekonstruktionen des Lorenzattraktors bei 5\% Rauschen. 
Oben durch einfache Verz"ogerungskoordinatenabbildung, in der Mitte durch normale SVD
Rekonstruktion, Unten durch Verz"ogerungskoordinateneinbettung der ersten Komponente der
SVD Rekonstruktion. 
}
{svdrecnoise}{-0.2cm}
}

Die Vorteile der Singular Value Decomposition zur Rauschverminderung sind klar
ersichtlich. Auf der anderen Seite kann die Rekonstruktion "uber Verz"ogerunsgkoordinaten
bei Verwendung der durch Redundanzanalyse gewonnenen Verz"ogerungszeit $\delay_R$
\naja(bessere) Einbettungen liefern. Dies zeigte \autor(A. M. Fraser), indem er ein
sogenanntes \begriff(Verzerrungsfunktional) (engl. distortion functional)
einf"uhrte. Dieses mi"st, wie gut die Lage eines Punktes im Originalphasenraum aus der
Kenntnis seines rekonstruierten Bildes bestimmt ist\cite{fraser}. Das Funktional mi"st,
mit Frasers Worten, wie \naja(diffeomorph) die Rekonstruktion ist bzw.\  welche
Rekonstruktion \naja(diffeomorpher) zum Original ist. Das Ergebnis dieses Vergleichs ist,
da"s die durch Redundanzanlyse gewonnenen Rekonstruktionen den SVD Rekonstruktionen
"uberlegen sind. Wir werden also im folgenden das kombinierte Verfahren verwenden. Die
Zeitreihe wird durch SVD eingebettet, die erste, rauschgefilterte Komponente wieder
extrahiert und "uber Redundanzanalyse wieder eingebettet. Dies steht auch in Einklang mit
des Ergebnissen von \autor(Mees \etal) , die zwar die rauschmindernden
Eigenschaften der SVD best"atigen, die M"oglichkeit den vom Attraktor aufgespannten
Unterraum zu identifizieren jedoch zweifelhaft erscheinen lassen\cite{Mees87}.


\comment{
\afterpage{
\epsfigsix{svd/noise/lorsrec00_1}{svd/noise/lorrec00_1}{svd/noise/lorsrec00_2}{svd/noise/lorrec00_2}
{svd/noise/lorsrec00_3}{svd/noise/lorrec00_3}
{Rekonstruktionen ohne rauschen}{svdrecnonoise}{-0.2cm}
}
}

\comment{
\afterpage{
\epsfigsix{svd/noise/lorsrec50_1}{svd/noise/lorrec50_1}{svd/noise/lorsrec50_2}{svd/noise/lorrec50_2}
{svd/noise/lorsrec50_3}{svd/noise/lorrec50_3}
{Rekonstruktionen mit rauschen 5proz}{svdrecnoise}{-0.2cm}
}
}










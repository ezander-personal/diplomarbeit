\documentclass[a4paper,12pt]{scrreprt}

% Befehle, die in Standard, aber nicht in den KOMA classes zur Verf"ugung stehen
\def\frontmatter{}
\def\mainmatter{}
\def\backmatter{}

% Deutsche Layouts und Trenntabellen
\usepackage{german}

% Zum einbinden von PostScript-Grafiken
\usepackage{graphicx}
\usepackage{psfrag}

% Bindet die AMS-fonts ein um Symbole f"ur reelle und complexe Zahlen
% zur Verf"ugung zu haben
\usepackage{amsfonts}

% Definiert \layout Befehl: coole Sache um Seitenlayout zu "uberpr"ufen
\usepackage{layout}

% Definiert befehl (\newtheroem) um theorem-umgebungen zu erstellen
\usepackage{theorem}

% Einige packages die ich vielleicht noch brauchen werde
% indentfirst, fancyheadings, float, subfigure, psfig

% Allgemeines zeug und trennungen
\usepackage{trennungen}
\usepackage[themabf,newint,normsect]{elmar}


%%%%%%%%%%%%%%%%%%%%%%%%%%%%%%%%%%%%%%%%%%%%%%%%%%
%%%%%%%%%%%%%%%%%%%%%%%%%%%%%%%%%%%%%%%%%%%%%%%%%%

\begin{document}

%%%%%%%%%%%%%%%%%%%%%%%%%%%%%%%%%%%%%%%%%%%%%%%%%%
%% Titelseite

\frontmatter

\titlepage
\subject{Diplomarbeit angefertigt am Institut f"ur Theoretische Physik
I\\Wilhelm-Klemm-Str.\  9}
\title{Grundlagen der Zeitreihenanalyse und ihre Anwendung auf die Atmungsdynamik Fr"uhgeborener}
\author{Vorgelegt von \\ Elmar Zander}
\date{\today}

\maketitle

%%%%%%%%%%%%%%%%%%%%%%%%%%%%%%%%%%%%%%%%%%%%%%%%%%
%% Inhaltsverzeichnis
\tableofcontents


%%%%%%%%%%%%%%%%%%%%%%%%%%%%%%%%%%%%%%%%%%%%%%%%%%
%% Hauptteil

\mainmatter

%\addchap{Einleitung}

Komplexes Verhalten tritt in nahezu allen Bereichen der Natur in Erscheinung. Während die
Physik diesen Phänomenen früher relativ ratlos gegenüberstand, konnten in den letzten
Jahren durch die Erkenntnisse der Chaostheorie auf diesem Gebiet große Fortschritte
erzielt werden. Die Theorie nichtlinearer dynamischer Systeme leistete hier Wesentliches,
indem sie die oftmals auftretenden komplexen Dynamiken mathematisch-physikalisch erfaßbar
und charakterisierbar machte.  Hierbei ist besonders das Konzept \begriff(seltsamer Attraktoren)
hervorzuheben.  Seltsame Attraktoren bilden die Menge der im Langzeitverhalten
erreichbaren Phasenraumpunkte dissipativer chaotischer Systeme und besitzen im allgemeinen
eine sehr verwickelte fraktale Struktur. Damit einher geht ein Verlust der Langzeitvorhersage
durch die Eigenschaft dieser Systeme, kleine Störungen exponentiell zu verstärken.  Die
Erkenntnis, daß solche Strukturen auch bei deterministischen Systemen mit wenigen
Freiheitsgraden auftreten können, hat in vielen Gebieten der Naturwissenschaften, auch
außerhalb der Physik, die Frage aufgeworfen, ob die beobachteten Verhaltensweisen und
Strukturen durch diese Konzepte zu erklären seien. (Kapitel~1.1.)


In den empirischen Naturwissenschaften liegen häufig nur unzureichende Daten über den
genauen Zustand des beobachteten Systems vor, oftmals in Form \begriff(skalarer
Zeitreihen) einer gemessenen Zustandsvariable.  Eine entscheidende Voraussetzung für die
Charakterisierung dieser Zeitreihen mittels der Methoden der nichtlinearen Dynamik ist
die Möglichkeit, aus diesen wieder Attraktoren rekonstruieren zu können.  Dies geschieht
über das Konzept der sogenannten \begriff(Verzögerungskoordinaten), welches durch eine
mathematische Theorie abgesichert werden konnte. Die Voraussetzungen der mathematischen
Theorie sind in der experimentellen Praxis jedoch nicht erfüllbar. Neben der ständigen
Anwesenheit von Rauschen ist vor allem die zeitliche Begrenztheit der zur Verfügung
stehenden Zeitreihen ein Problem. Aufgrund dieser Einschränkungen ist eine gute Wahl der
Einbettungsparameter vonnöten.  Desweiteren müssen Möglichkeiten zur Verminderung des
Rauschens gefunden werden. (Kapitel~1.2.)

Für die Charakterisierung seltsamer Attraktoren steht eine Vielzahl verschiedener Maßzahlen zur
Verfügung. Die drei am häufigsten herangezogenen sind die fraktale Dimension, die
Kolmogorov-Entropie und der Lyapunov-Exponent. Diese beschreiben respektive die Anzahl
der effektiven Freiheitsgrade, die Rate der Informationsproduktion sowie die exponentielle
Separation der Trajektorien des Systems. Da die Komplexität eines Systems meistens mit
der Anzahl seiner Freiheitsgrade korreliert ist, wendet sich diese Arbeit besonders den
verschiedenen Begriffen fraktaler Dimension zu.  Im Bereich der nichtlinearen
Zeitreihenanalyse wird zumeist die Korrelationsdimension verwendet, die durch ihre schnelle
und stabile Berechenbarkeit durch den Grassberger-Procaccia-Algorithmus wesentliche
Bedeutung erlangt hat. Probleme stellen sich hierbei hauptsächlich durch die endliche
Datenmenge, weißes und Digitalisierungsrauschen und sogenannte Randeffekte.  (Kapitel~1.3.)

Damit die Verwendung der beschriebenen Methoden überhaupt einen Sinn macht, ist es
notwendig zu wissen, daß der beobachteten Zeitreihe ein deterministisches System zugrunde
liegt.  Zur Lösung des Problems existieren eine Vielzahl von Vorschlägen, von denen hier
der Methode der Surrogatdaten auf der Basis statistischer Hypothesentests der Vorzug gegeben
wird.  Diese bieten ein breites Fundament, um Fragen nach der Struktur und
zugrunde liegendem Determinismus experimenteller Zeitreihen beantworten zu können. Durch
die Verfügbarkeit hoher Rechenleistungen in modernen Computern lassen sich eine Vielzahl
verschiedener Hypothesen durch Vergleich der Originalzeitreihen mit sogenannten
Surrogatdaten testen. (Kapitel~1.4.)

Die Ergebnisse dieser Arbeit sollen auf das konkrete Beispiel der Atmungsdynamik
Frühgeborener angewandt werden. Das Ziel
hierbei ist es zu klären, ob die vorgestellten Methoden aus der Zeitreihenanalyse bei der
Untersuchung dieser Problematik von Nutzen sein können.
Konkret geht es hierbei um folgende Fragen:
\begin{itemize}
\item Ist die den verschiedenen Atmungstypen zugrunde liegende Dynamik durch ein
  deterministisches System beschreibbar ?
\item Können für die verschiedenen Atmungstypen, regelmäßiges und periodisches Atmen, die
  fraktale Dimension berechnet und  die Existenz eines seltsamen Attraktors belegt werden ?
\item Lassen sich signifikante Änderungen der Dimension vor dem Auftreten von
  Atemstillstän\-den (Apnoen) feststellen ?
\end{itemize}
Eine Bejahung dieser Fragen könnte zu einer möglichen Früherkennung von
Atemstillständen führen, was für die Frühgeborenenmedizin von größtem Nutzen wäre.
(Kapitel~2)







\clearpage
\chapter{Grundlagen der Zeitreihenanalyse}


%%%%%%%%%%%%%%%%%%%%%%%%%%%%%%%%%%%%%%%%%%%%%%%%%%%%%%%%%%%%%%%%%%%%%%%%%%%%%%%%%%%%%%%%%%%%%%%%%%%%
%% Dynamische Systeme
%%%%%%%%%%%%%%%%%%%%%%%%%%%%%%%%%%%%%%%%%%%%%%%%%%%%%%%%%%%%%%%%%%%%%%%%%%%%%%%%%%%%%%%%%%%%%%%%%%%%

\section{Seltsame Attraktoren}
\label{chapdynsystems}

Das Konzept seltsamer Attraktoren hat in den letzten beiden Jahrzehnten eine ständig
wachsende Bedeutung bei der Erklärung von Phänomenen aus den unterschiedlichsten
Forschungsgebieten gefunden. Gemeinsam ist vielen dieser Phänomene, daß die untersuchten
Systeme prinzipiell durch gewöhnliche Differentialgleichungen der Form
\eqnl[evolut]{\dot\x = \vec F(\x)}
beschrieben werden können. Die Geschwindigkeit $\dot\x$, mit der sich der Zustand des
Systems entwickelt, und damit auch die ganze zukünftige Entwicklung\footnote{Unter
  hinlänglich allgemeinen Forderungen an das Vektorfeld $\vec F$, nämlich der
  Lipschitz-Stetigkeit.}, ist determiniert durch den aktuellen Zustand $\x$. Man sagt
auch, das Vektorfeld $\vec F$ erzeuge einen Fluß $\flow^t$:
\eqn{\x(t) = \flow^t(\x_0) .}
Der Fluß beschreibt die zeitliche Entwicklung des Systems in Abhängigkeit vom
Anfangszustand $\x_0$.  Die Bewegung findet in einem Raum statt, der als
\begriff(Phasenraum) $\phase$ bezeichnet wird. Jedem Freiheitsgrad des Systems entspricht
im allgemeinen eine Koordinate des Phasenraumes\footnote{Bei Hamilton-Systemen wird die
  Anzahl der Freiheitsgrade $f$ als die Anzahl der konjugierten Orte und Impulse
  definiert. In diesem Fall ist die Dimension des Phasenraumes gleich $2f$.}. Die Anzahl
der Freiheitsgrade und damit die Dimension des Phasenraumes ist in vielen Fällen endlich,
kann jedoch bei \naja(ausgedehnten) Systemen\footnote{Gemeint sind hier beispielsweise
  hydrodynamische Systeme, deren Zustand durch Dichte- , Temperatur- und
  Geschwindigkeitsfelder beschrieben wird. Die hier auftretenden partiellen
  Differentialgleichungen können oftmals durch eine Modenentwicklung (z.B. nach
  \autor(Ritz-Galerkin)) auf eine \eqnref{evolut} entsprechende Form gebracht werden.}
auch abzählbar oder überabzählbar unendlich werden.

Bei realen Systemen ist es die Regel, daß Energie durch Prozesse wie Reibung
verlorengeht. Bei solchen \begriff(dissipativen Systemen) kann man die Beobachtung
machen, daß sich die Dynamik nach einer transienten Phase auf eine 
niedrigdimensionale Untermannigfaltigkeit $\M\subset\phase$ des Phasenraumes
reduziert. Die Energiedissipation eines Systems ist gleichbedeutend mit einer Kontraktion
des Phasenraumvolumens $V$ beliebiger Teilmengen $\set B\subset\phase$ unter dem Fluß $\flow^t$:
\eqn{\abl{}{t}V(\flow^t(\set B))\lt 0.}
Dies kann auch über das erzeugende Vektorfeld $\vec F$ ausgedrückt werden. Die Divergenz
des Vektorfeldes muß in den erreichbaren Teilen des Phasenraumes negativ sein:
\eqn{\mathop{\mathrm{div}}\vec F \lt 0.}
Hierfür ist es nicht notwendig, daß die Phasenraumvolumina in alle Richtungen gestaucht
werden. Es können auch bestimmte Richtungen expandieren, solange die Kontraktion in
die anderen Richtungen überwiegt.

Bei vielen Systemen klingt die transiente Phase der Bewegung recht schnell ab. In einem solchen
Fall interessiert man sich meist für das \begriff(Langzeitverhalten) des Systems, und
beschränkt sich bei der Untersuchung auf die \begriff(Grenzmenge)\footnote{Die positive
  Grenzmenge $\limit^+$ eines Systems ist definiert als die Menge aller Punkte $\y\in\phase$, für
  die Folgen $t_0\lt t_1\lt t_2\dots$ existieren, so daß
  $\lim\limits_{i\to\infty}\flow^{t_i}(\x)=\y$ für mindestens ein $\x\in\phase$.}
$\limit^+$, d.h.\ den Bereich des Phasenraumes, in dem sich die Dynamik im Grenzfall
unendlich langer Zeiten abspielt.

Lange herrschte die Auffassung, daß sich die einzig möglichen Grenzmengen aus
Punkt\-attraktoren, Grenzzyklen oder attraktiven $k$-Tori ($k\geq2$) zusammensetzen.
Diesen ist gemeinsam, daß sie periodisch (in den ersten beiden Fällen) bzw.\ periodisch
oder quasiperiodisch (im letzten Fall) sind. Sie besitzen ein diskretes
Fourier-Spektrum.  Die Entstehung von Turbulenz wurde verstanden durch
sukzessive Generation immer neuer Fourier-Moden.  Das komplexe Verhalten turbulenter
Systeme würde also als Überlagerung sehr vieler inkommensurabler Frequenzen betrachtet
und wäre somit qualitativ nicht verschieden vom Verhalten einfacher Systeme. Bei der
Untersuchung eines (stark vereinfachten) Modells der Konvektion von Luft in der
Atmosphäre stieß \autor(Lorenz) 1963 \cite{Lorenz63} jedoch auf ein System, das
keinerlei Periodizität und somit ein kontinuierliches Fourier-Spektrum aufwies
(siehe~\psref{attrlor}). Diese \begriff(deterministischen nichtperiodischen Flüsse)
erhielten später von \autor(Ruelle) und \autor(Takens) \cite{Ruelle71a} den
einprägsamen Namen \begriff(seltsame Attraktoren).


\epsfigdouble{attraktoren/lornew}{attraktoren/fournew}
{Oben eine typische Trajektorie des Lorenz-Systems. Unten das Leistungsspektrum $P(f)$ der
  $x$-Komponente.}{attrlor}{-0.2cm} 

Eine mathematisch exakte Definition seltsamer Attraktoren ist ein schwieriges
Problem. Eine endgültige Fassung, auf die sich alle geeinigt hätten, ist bis heute nicht
gelungen \cite{Pawelzik91}. Die Schwierigkeit liegt darin, die Definition sowohl
mathematisch exakt zu halten, als auch so umfassend, daß alle wesentlichen Eigenschaften
erfaßt werden. Ein grundlegenderes Problem ist jedoch, daß selbst über die
Eigenschaften seltsamer Attraktoren keine Einigkeit zwischen allen beteiligten Wissenschaftlern
besteht. Wir wollen uns hier deshalb auf einige für diese 
Arbeit wichtige Eigenschaften beschränken:
\begin{itemize}
\item Seltsame Attraktoren sind \begriff(mischend).  Ein Attraktor $\attr$ heißt genau
  dann mischend, wenn es für beliebige, bezüglich $\attr$ offene\footnote{Eine Menge
    $\A$ heißt \begriff(offen bezüglich) einer Menge $\M$, wenn $\A$ ganz in $\M$ liegt
    und es zu jedem $x\in\A$ eine reelle Zahl $\eps>0$ gibt, so daß der Schnitt der
    $\eps$-Umgebung $\U_\eps(x)$ mit der Menge $\M$ ganz in $\A$ liegt: $\U_\eps(x) \cap
    \M \subset \A$.  }, nichtleere Mengen $\set I,\set J\subset\,\attr$ mindestens einen
  Punkt $\x\in\set I$ gibt, so daß $\flow^t(\x)\in\set J$ für mindestens ein $t>0$.

\comment{Eine Menge $\A$ heißt \begriff(offen bezüglich) einer Menge $\M$, wenn es eine
  offene Menge $\tilde\A$ gibt, so daß $\A=\tilde\A\cap\M$. Der Begriff \naja(offen
  bezüglich) stellt eine Verallgemeinerung der Vorstellung dar, daß sich alle Punkte
  einer offenen Menge \naja(innerhalb) derselben befinden.}

Eine wichtige Folgerung daraus ist, daß fast alle Punkte
des Attraktors unter dem Fluß $\flow^t$ jedem anderen Attraktorpunkt beliebig nahe
kommen. Diese Eigenschaft ist notwendig für die Definition ergodischer Maße
auf dem Attraktor. 

\item Zwei zu einem gegebenen Zeitpunkt $t_0$ eng beieinander liegende Punkte auf dem Attraktor
werden unter dem Fluß $\flow^t$ mit der Zeit exponentiell voneinander getrennt. Man spricht von einer 
\begriff(exponentiellen Separation) der Trajektorien oder auch \begriff(sensibler
Abhängigkeit von den Anfangsbedingungen). Dies findet Ausdruck in der Existenz positiver
Lyapunov-Exponenten.

\item \begriff(Periodische Orbits) liegen dicht auf dem Attraktor. Aufgrund der
  exponentiellen Separation sind diese Orbits jedoch alle instabil. Der Träger eines
  seltsamen Attraktors ist darstellbar als Abschluß seiner instabilen periodischen Orbits
  (IPOs).
  
  Diese Tatsache kann im Rahmen der Zeitreihenanalyse zur effizienten Bestimmung von
  charakteristischen Größen wie Lyapunov-Exponenten und fraktalen Dimensionen ausgenutzt
  werden. Für die Berechnung muß nicht die ganze zur Verfügung stehende Datenmenge
  herangezogen werden, da oftmals Mittelungen über die IPOs ausreichend sind.  Ein
  Verfahren zur Extraktion instabiler periodischer Orbits aus experimentellen Zeitreihen
  findet sich in \cite{Pawelzik91,Pawelzik91a}.
\end{itemize}
Die aufgezählten Eigenschaften sind nicht voneinander unabhängig. \autor(Banks \etal)
konnten zeigen, daß die exponentielle Separation der Trajektorien aus den beiden anderen
Eigenschaften gefolgert werden kann \cite{Banks92}. Da die Eigenschaften des Durchmischens
als auch das Dichtliegen periodischer Orbits topologische Invarianten sind, gilt
dies somit ebenfalls für die exponentielle Separation der Trajektorien. Unter
topologischen Abbildungen bleiben also die dynamischen und geometrischen Eigenschaften
seltsamer Attraktoren unberührt.

Oft werden seltsame Attraktoren auch über ihre fraktale Struktur definiert
\cite{Peitgen92}. Diesem Ansatz soll hier nicht gefolgt werden, da er nur geometrische,
jedoch keine dynamischen Aspekte des Attraktors beschreibt. Es existieren Attraktoren, die
seltsam (nach obiger Definition), aber nicht fraktal sind (beispielsweise der
\naja(Attraktor) von \begriff(Arnolds Katzenabbildung)\footnote{Arnolds Katzenabbildung
  ist eine Abbildung eines 2-Torus auf sich selbst \cite{Arnold68}. Die Dynamik ist
  chaotisch. Da das System volumenerhaltend ist, ist der ganze Torus der Attraktor
  \cite{Eckmann-ruelle}.  }). Demgegenüber ist der
\begriff(Feigenbaum-Attraktor)\footnote{Die logistische Abbildung ist definiert durch
  $f_\mu(x)=\mu x(1-x)$ mit $\mu\in[0,4]$. Sie hat attraktive periodische Punkte der
  Periode $2^n$, wobei $n$ gegen unendlich läuft für $\mu$ gegen $\mu_\infty=3,57\dots$
  . Der für $\mu=\mu_\infty$ entstehende Attraktor wird als Feigenbaum-Attraktor
  bezeichnet. Er besitzt eine fraktale Struktur, ist jedoch nicht chaotisch, da keine
  sensitive Abhängigkeit von den Anfangsbedingungen besteht \cite{Eckmann-ruelle}.}
fraktal, aber nicht seltsam.  Die meisten Attraktoren (Lorenz-Attraktor,
Rössler-Attraktor \dots) sind jedoch sowohl seltsam als auch fraktal.


\section{Rekonstruktionsverfahren}

Die Theorie seltsamer Attraktoren kann oftmals gute Erklärungsansätze für das
Ver\-ständnis dynamischer Systeme liefern. So kann beispielsweise das Lorenz-System als
Modell für bestimmte Konvektionszellen (sogenannte Rayleigh-B\'enard-Zellen) verwendet
werden\footnote{\autor(Lorenz) benutzte zur Aufstellung seiner Gleichungen das
  \autor(Rayleigh)sche strömungsdynamische Modell für bestimmte rechteckige
  Flüssigkeitszellen. Die im Lorenz-Modell auftretenden Phasenraumvariablen sind die
  zeitabhängig gemachten Amplituden spezieller Lösungen des Rayleigh-Modells
  \cite{Peitgen92}.}. Durch Variation eines Parameters kann der 
Übergang dieser Zellen von laminarem zu turbulentem Verhalten studiert werden. Die vielen
Näherungen, die für dieses System gemacht werden, erlauben zwar keine direkten
Voraussagen über reale Konvektionszellen, die Dynamik kann jedoch prinzipiell verstanden
werden.

Anders liegt der Fall bei biologischen, medizinischen oder auch komplizierteren
hydrodynamischen Systemen. Hier tauchen eine Reihe von Problemen auf:
\begin{itemize}
\item Die relevanten Phasenraumvariablen sind oft nicht bekannt. Gerade bei
medizinischen und biologischen Systemen ist dies sehr oft der Fall.
\item Die Anzahl der Freiheitsgrade ausgedehnter Systeme (z.B. meteorologischer oder
hydrodynamischer) ist in der Regel abzählbar oder gar überabzählbar unendlich. 
\item Selbst bei Kenntnis aller Phasenraumvariablen sind die für die Dynamik zuständigen
Evolutionsgleichungen unbekannt.
\end{itemize}
Daten, die in experimentellen Situationen gewonnen werden, beschränken sich so meistens
auf Meßreihen einer oder weniger Größen, von denen angenommen wird, daß  sie die
Dynamik des Systems charakterisieren. Stellt sich nun heraus, daß die so gewonnene
Zeitreihe einen nicht trivialen\footnote{\begriff(Trivial) bedeutet in diesem
  Zusammenhang, daß sich die Zeitreihe nicht als Überlagerung weniger Frequenzen
  darstellen läßt, d.h.\  sie hat kein diskretes
Fourier-Spektrum.} zeitlichen Verlauf aufweist, ergeben sich daraus einige Fragen:
\begin{itemize}
\item Liegt der Zeitreihe ein deterministisches System zugrunde, oder ist das erratische
Verhalten eine Folge additiven Rauschens?
\item Läßt sich das System durch einen seltsamen Attraktor beschreiben? Wenn ja, wie
können wir diesen aus den vorhandenen Daten rekonstruieren?
\item Wie können für den rekonstruierten Attraktor charakteristische Größen wie
fraktale Dimensionen oder Lyapunov-Exponenten bestimmt werden?
\end{itemize}
Wir wollen uns zuerst mit der zweiten Frage beschäftigen, wobei wir im folgenden annehmen
(zumindest als Arbeitshypothese), dem System liege ein seltsamer Attraktor zugrunde.
Die erste und die dritte Frage werden wir in späteren Abschnitten angehen.

\subsection{Verzögerungskoordinaten (MOD)}

Den ersten Ansatz zur Lösung des Problems der Attraktorrekonstruktion  lieferten
\linebreak \autor(Packard \etal) 1980
mit dem Konzept der \begriff(Verzögerungskoordinaten), auch kurz als
MOD (engl.: method of delays) bezeichnet \cite{Packard80}. Um das Verfahren 
zu veranschaulichen, verwenden wir ein System dessen Dynamik uns bereits bekannt ist. An
diesem soll eine einzige Observable\footnote{D.h.\   eine beliebige glatte Funktion
$v:\phase\to\R$ der Phasenraumvariablen.} (numerisch) gemessen werden, die uns als Zeitreihe
dient. Hieraus soll die Dynamik wieder rekonstruiert und mit der ursprünglichen Dynamik
verglichen werden.

%Als dynamisches System wählen wir 
Das Rössler-System \cite{Roessler76} ist durch das folgende System von
Differentialgleichungen bestimmt
\eqna{
\dot x &=& -z-x \nonumber \\
\dot y &=& x + a y \nonumber \\
\dot z &=& b + z(x-c) 
}
Das System ist in der dritten Gleichung nichtlinear und wird für den Parametersatz
$a=0.38$, $b=0.3$, $c=4.5$ chaotisch. Der Attraktor besitzt eine fraktale
Struktur\footnote{Durch Analyse der Gleichungen kann man bei diesem System sehr schön den Chaos
  erzeugenden \metapher(Streck-und-Falt)-Prozeß erkennen \cite{Peitgen92}.  Die Struktur
  entspricht lokal dem kartesischen Produkt einer 2-dimensionalen Mannigfaltigkeit und einer
  Cantor-Menge.}.

Aus den Differentialgleichungen wird durch numerische Integration ein diskreter
Orbit $\folge(\x,1,N)$ erzeugt\footnote{Verwendet wird ein Runge-Kutta-Verfahren vierter Ordnung mit
  Schrittweite $\sample=0.009$. Nach einer Transienzzeit von $1000\sample$, nach der sich
  das System dem Attraktor hinreichend genähert hat, wird der Orbit in diskreten
  Schritten $\sample$ aufgezeichnet.  Die Anzahl der berechneten Orbitpunkte beträgt
  $N=20000$.}
 (siehe \psref{rekroe} oben).  Als Observable $v$ dient die $x$-Komponente der
Punkte $\x$: $v(\x)=x$, so daß wir eine Zeitreihe $v_i = v(\x(i\sample))$ erhalten. 
Die Zeitreihe wurde zur Verdeutlichung ihres diskreten Charakters durch Punkte
dargestellt (siehe \psref{rekroe} unten).  



\epsfigdouble{rekonstruktion/roenew}{rekonstruktion/timenew}
{Oben eine Trajektorie des Rössler-Attraktors aus einer numerischen
Integration. Unten die gemessene Observable $v$.}{rekroe}{-0.2cm}

Die Informationen über die Werte der Koordinaten $y$ und $z$ stehen nun nicht mehr zur
Verfü\-gung. Nach der Idee von \autor(Packard \etal) kann jedoch der Zustand eines
$n$-dimensio\-na\-len Systems zu einer gegebenen Zeit durch jeden Satz von $n$
unabhängigen, sonst aber beliebigen Koordinaten spezifiziert werden \cite{Packard80}.
Der Zustand des Rössler-Systems zu einer Zeit $t_i$ sollte also statt durch
$(x_i,y_i,z_i)$ ebenso durch $(x_i,\dot x_i,\ddot x_i)$ oder $(x_i,x_{i+1},x_{i+2})$
beschrieben werden können.

Wir wollen nun anhand einer einfachen Überlegung 
plausibel machen, daß in den Ver\-zögerungskoordinaten
tatsächlich die gleiche Information steckt wie in den originalen Phasenraumkoordinaten.
Wenn die Zeit $\sample$ zwischen zwei Messungen klein ist, gilt näherungsweise:
\eqnal[recidea1]{\dot x_i &\simeq& (x_{i+1} - x_i) / \sample \nonumber \\
\ddot x_i &\simeq& (x_{i+2} - 2x_{i+1} + x_i) / \sample^2 ;}
andererseits gilt für die erste und zweite Zeitableitung
\eqnal[recidea2]{ 
\dot x_i &=& f_1(x_i, y_i, z_i) \nonumber \\
\ddot x_i &=& \pabl{F_1}{x} F_1(x_i,y_i,z_i) + \pabl{F_1}{y} F_2(x_i,y_i,z_i) +
\pabl{F_1}{z} F_3(x_i,y_i,z_i) ,} 
wobei die $F_j$ die Komponenten des erzeugenden Vektorfeldes sind (siehe \eqnref{evolut}).
Unter im betrachteten Zusammenhang recht allgemeinen Bedingungen an die $F_j$,
lassen sich $x_i$, $\dot x_i$ und $\ddot x_i$ wieder nach den Phasenraumvariablen $x_i$,
$y_i$ und $z_i$ auflösen.  Da andererseits nach \eqnref{recidea1} die Ableitungen durch die
Verzögerungskoordinaten bestimmt sind, können wir den kompletten Systemzustand
$x_i,y_i,z_i$ aus den verzögerten Werten der Zeitreihe $x_{i},x_{i+1},x_{i+2}$
\metapher(zurückholen). Wenn diese \naja(Äquivalenz) zwischen Original- und
Verzögerungskoordinaten gegeben ist, spricht man von einer \begriff(Einbettung) des
Attraktors (näheres dazu später), und bezeichnet den Raum, in den die Zeitreihe
eingebettet wird, (in diesem Fall der $\R^3$) als \begriff(Einbettungsraum).


Die vorstehenden Überlegungen können auch auf beliebige Observable $v$ sowie auf höherdimensionale
Einbettungsräume $\R^\embed$ erweitert werden. Weiterhin können größere
Zeit\-ab\-stände $k\sample$ zwischen den Verzögerungskoordinaten gewählt werden.  Man
erhält so als Rekonstruktionsvektoren die Folge $(v_{i},
v_{i+k},\dots,v_{i+(\embed-1)k})$. Dieses Verfahren entspricht einer
Abbildung $\diffeo_{k,\embed,v}:\phase\to\R^\embed$ aus dem Original- in den
Rekonstruktionsphasenraum, welche durch
\eqn{\diffeo_{k,\embed,v}(\x) = (v(\x),v(\flow^{k\sample}(\x)), \dots, v(\flow^{(\embed-1)k\sample}(\x)))}
gegeben ist. Man bezeichnet diese Abbildung  als \begriff(Verzögerungskoordinatenabbildung).

Wir wollen das Verfahren nun bei dem oben erwähnten Rössler-System anwenden.  Da wir die
Anzahl der Freiheitsgrade des Rössler-Systems kennen, lassen wir es bei der uns bekannten
und \naja(ausreichenden) Einbettungsdimension $\embed=3$. Für die Verzögerung\footnote{
  Die in Einheiten der \begriff(Sampling Time) $\sample$ gemessene
  \begriff(Verzögerungszeit) $\delay=k\sample$ wird als \begriff(Verzögerung) $k$
  bezeichnet. Da Verzögerung und Verzögerungszeit {\em immer} über diese Relation
  eindeutig verknüpft sind ($\sample$ ist konstant), wird im folgenden je nach Kontext
  der besser geeignete der beiden Begriffe benutzt.} wählen wir $k=30$.

%\afterpage
{\epsfigsingle{rekonstruktion/recnew}
{Rekonstruktion des Rössler-Attraktors aus der Zeitreihe in \psref{rekroe} (unten) zur
  Einbettungsdimension $\embed=3$ mit Verzögerung $k=30$. Die Rekonstruktionspunkte
  wurden zur besseren Darstellung durch gerade Linien verbunden.}{rekrek}{-0.5cm}}

Die Rekonstruktion in \psref{rekrek} ähnelt einer verzerrten Kopie des
Originalattraktors in \psref{rekroe}. Insofern leistet die
Verzögerungskoordinatenabbildung schon gute Dienste. Die rein visuelle Ähnlichkeit
reicht aber für eine genauere Analyse der Dynamik nicht aus. Es stellen sich mehrere
Fragen, die noch zu beantworten sind
\begin{myitemize}
\item Sind die Dynamik des rekonstruierten und des Originalattraktors zueinander
konjugiert? Mit anderen Worten: Ist die Verzögerungskoordinatenabbildung $\diffeo_{k,\embed,v}$
ein Diffeomorphismus?
\item Sind fraktale Dimensionen und Lyapunov-Exponenten unter der Rekonstruktion durch 
die Verzögerungskoordinatenabbildung invariant?
\item Wie sind die Einbettungsparameter $d$ und $k$ zu wählen, wenn das ursprüngliche
System nicht bekannt ist?
\end{myitemize}
Diese Fragen sollen in den nächsten Abschnitten beantwortet werden.




%%%%%%%%%%%%%%%%%%%%%%%%%%%%%%%%%%%%%%%%%%%%%%%%%%%%%%%%%%%%%%%%%%%%%%%%%%%%%%%%%%%%%%%%%%%%%%%%%%%%
%% Einbettungen
%%%%%%%%%%%%%%%%%%%%%%%%%%%%%%%%%%%%%%%%%%%%%%%%%%%%%%%%%%%%%%%%%%%%%%%%%%%%%%%%%%%%%%%%%%%%%%%%%%%%

\subsubsection{Einbettungen}

Bevor wir uns mit den Einbettungstheoremen beschäftigen, soll erst einmal der Begriff der
\begriff(Einbettung) selbst geklärt werden. Bei einer Einbettung handelt es sich immer um
eine Abbildung (einer Menge) aus einem Phasenraum in einen anderen Phasenraum. Nun sollen unter
dieser Abbildung (und auch unter der Umkehrabbildung) keine Punkte kollabieren, d.h. es
sollen keine verschiedenen Originalpunkte auf den selben Bildpunkt abgebildet werden.
Solche Abbildungen, die die topologischen Eigenschaften von Punktmengen invariant lassen
bezeichnet man als \begriff(Homöomorphismen). Weiterhin sollen durch die Einbettung auch
keine Tangentenrichtungen kollabieren, was beispielsweise für die Bestimmung von
Lyapunov-Exponenten von Bedeutung ist.  Zusätzlich wird also die stetige
Differenzierbarkeit der Abbildung und der Umkehrabbildung gefordert. Eine
Einbettung muß somit ein $\sm^1$-Diffeomorphismus sein.




Wir wollen uns nun mit dem Problem beschäftigen, unter
welchen Voraussetzungen die Verzögerungskoordinatenabbildung $\diffeo_{k,\embed,v}$ eine
Einbettung im obigen Sinne ist. Den ersten Ansatz zur Beantwortung dieser Frage lieferte
\autor(Takens) 1980 \cite{Takens80}.  \comment{Sein erstes Theorem soll hier vollständig zitiert
werden}
\comment{
\begin{theorem}
Sei $\M$ eine kompakte Mannigfaltigkeit der Dimension $\mandim$. Für Paare $(\flow,v)$,
wobei  \linebreak[4] $\flow:\M\to\M$ ein  glatter Diffeomorphismus und $v:\M\to\R$ eine
glatte Funktion ist, ist es eine generische Eigenschaft, daß die 
Abbildung $\diffeo_{(\flow,v)} : \M \to\R^{2\mandim+1}$, definiert durch 
\eqnl[takmod]{\diffeo_{(\flow,v)}(\x) = (v(\x),v(\flow^1(\x)), \dots, v(\flow^{2\mandim}(\x))),}
eine Einbettung ist; \metapher(Glatt) bedeutet hier mindestens $\sm^2$.
Hierbei werden zusätzlich folgende Voraussetzungen an den Fluß $\flow$ gestellt:
\begin{myitemize}
\item Wenn $\x$ periodischer Punkt der Periode $k\le 2\mandim+1$ ist, sind alle Eigenwerte
von $\mathrm{D}\flow^k(\x)$ paarweise verschieden und verschieden von 1.
\item Für verschiedene Fixpunkte $\x^*$ von $\flow$, sind auch die $v(\x^*)$
verschieden\footnote{Der Satz gilt entsprechend für durch $\sm^2$-Vektorfelder erzeugte Flüsse,
wobei sich die beiden Voraussetzungen leicht auf das Vektorfeld übertragen lassen.}.
\end{myitemize}
\end{theorem}
}

\begin{theorem}
  Sei $\M$ eine kompakte Mannigfaltigkeit der Dimension $\mandim$. Für Paare $(\vec
  F,v,\delay)$, wobei $\vec F$ ein glattes Vektorfeld, $v:\M\to\R$ eine glatte Funktion
  und $\delay>0$ eine reelle Zahl ist\footnotemark, ist es eine generische Eigenschaft, daß die
  Abbildung $\diffeo_{(\vec F,v,\delay)}: \M \to\R^{2\mandim+1}$, definiert durch
  \eqnl[takmod]{\diffeo_{(\vec F,v,\delay)}(\x) = (v(\x),v(\flow^\delay(\x)), \dots,
    v(\flow^{2\mandim\delay}(\x))),} eine Einbettung ist, wobei $\flow^t$ der durch $\vec
  F$ erzeugte Fluß ist; \metapher(Glatt) bedeutet hier mindestens $\sm^2$.  Über das
  Vektorfeld $\vec F$ werden folgende zusätzliche Annahmen gemacht:
\begin{myitemize}
\item Wenn $\vec F(\x)=0$ ist, dann sind alle Eigenwerte von $\mathrm{D}\flow^\delay(\x)$ paarweise verschieden und verschieden von 1.
\item Kein periodischer Orbit von $\vec F$ hat eine Periode $n\delay\, (n\in\N)$ mit $n\leq2\mandim+1$.
\end{myitemize}
\end{theorem}
\footnotetext{\autor(Takens) formulierte und bewies das Theorem für die Verzögerungszeit 
  $\delay=1$. Da die Zeit jedoch immer entsprechend umskaliert werden kann,
  erschien es mir sinnvoll, das Theorem gleich für beliebige Verzögerungszeiten
  $\delay>0$ zu formulieren.}

Die Abbildung $\diffeo_{(\vec F,v,\delay)}$ im obigen Theorem entspricht der
Verzögerungskoordinatenabbildung $\diffeo_{k,\embed,v}$.  Da wir über das Vektorfeld
$\vec F$ keine Kenntnis haben, können wir die zusätzlichen Annahmen des Theorems nicht
verifizieren. Nach \autor(Takens) sind diese jedoch auch unter generischen Bedingungen
erfüllt. 

Das Theorem versichert uns also, daß für generische Vektorfelder $\vec F$,
Meßfunktionen $v$ und Verzögerungszeiten $\delay>0$, die
Verzögerungskoordinatenabbildung $\diffeo_{(\vec F,v,\delay)}$ eine Einbettung ist. Da im
Experiment nur in diskreten Zeitschritten $\sample$ gemessen werden kann, steht uns jedoch
nicht, wie vorausgesetzt, eine kontinuierliche Funktion $v(t)$, sondern nur die diskrete
Meßreihe $v_i=v(i\sample)$ zur Verfügung. Wir möchten nun wissen, ob auch die
Grenzmenge der diskreten Folge konjugiert zu der des Originalsystems ist. Dies wird durch
ein Korollar zu Takens' viertem Theorem beantwortet.

\begin{corollar}
Sei $\M$ eine kompakte Mannigfaltigkeit der Dimension $\mandim$. Wir betrachten Viertupel,
bestehend aus einem Vektorfeld $\vec F$, einer Funktion $v$, einem Punkt $\x$ und einer
positiven reellen Zahl $\delay$. Für generische $(\vec F, v, \x, \delay)$ ist die
positive Grenzmenge $\limitp(\x)$ \naja(diffeomorph) zu der Grenzmenge der folgenden
Sequenz im $\R^{2m+1}$
\eqn{ \left\{ \left( v(\flow^{i\delay}(\x)),v(\flow^{(i+1)\delay}(\x)), \dots
,v(\flow^{(i+2m)\delay}(\x))    \right) \right\}^\infty_{i=0} }
\naja(Diffeomorph) heißt hier: es gibt eine glatte Einbettung von $\M$ nach $\R^{2m+1}$,
die $\limitp(\x)$ bijektiv auf die Grenzmenge dieser Punktfolge abbildet.
\end{corollar}

Wir können nun schließen, daß für generische Verzögerungen $k$ und Meßfunktionen
$v$ die Verzögerungskoordinatenabbildung $\diffeo_{k,\embed,v}$ eine Einbettung des
Attraktors liefert, sofern nur $\embed\geq2m+1$ ist\footnote{Zur Bestimmung von $m$ siehe
  Abschnitt \ref{chapparams}~.}.
Der Begriff \naja(generisch) ist jedoch relativ schwach und sagt nichts über die
Wahrscheinlichkeit aus, daß dies tatsächlich der Fall ist, wenngleich man dies gerne
meinen möchte. 

Dies soll genauer erläutert werden.  Der Ausdruck \naja(generisch) beschreibt die Häufigkeit
des Auftretens bestimmter Eigenschaften bei Elementen einer Menge $\A$.
Eine \label{generisch} Eigenschaft heißt bereits dann generisch auf $\set A$, wenn
eine \begriff(residuale Teilmenge)\footnote{Eine Menge $\set R\subset\set A$ heißt
\begriff(residuale) Teilmenge von $\set A$, wenn $\set R$ abzählbarer Durchschnitt
offener, dichter Teilmengen von $\set A$ ist. Residuale Mengen sind selbst wieder dicht, jedoch nicht notwendig offen. }
 $\set R$ von $\set A$ existiert, so
daß alle Elemente von $\set R$ diese Eigenschaft aufweisen \cite{Liebert91}.
Zu jedem Element aus $\set A$ findet man also in
jeder endlichen, beliebig kleinen Umgebung ein Element aus $\set R$, das diese
Eigenschaft aufweist. Dies sagt jedoch nichts über die
Wahrscheinlichkeit, ein Element dieser Menge zufällig zu treffen. Es existieren Beispiele,
in denen diese Wahrscheinlichkeit sogar null wird. So ist beispielsweise
die Menge $\Omega_{\text{stab}}$ der Parameterwerte $\omega\in[0,2\pi]$, für die die
eindimensionale Kreisabbildung
\eqn{g_{\omega,k}(x)=x+\omega+k\sin(x) \nonumber}
stabile Orbits besitzt, eine residuale Teilmenge von $[0,2\pi]$. Für $k\to0$ verschwindet
das Lebesgue-Maß von $\Omega_{\text{stab}}$, die Wahrscheinlichkeit, ein
$\omega$ aus $[0,2\pi]$ zufällig so zu wählen, daß $g_{\omega,k}$ stabile Orbits besitzt (d.h.\
$\omega\in\Omega_{\text{stab}}$), geht demnach gegen null \cite{Sauer91}.


Für den Experimentator ist die Zusicherung aus Takens' Korrolar somit nicht ausreichend,
da nach den obigen Ausfüh\-rungen, Generizität nichts über die Wahrscheinlichkeit, daß
hier wirklich eine Einbettung vorliegt, aussagt. Wir möchten sicher sein, daß die
Verzögerungskoordinatenabbildung mit der Wahrscheinlichkeit eins eine Einbettung ist.

Um auszudrücken, eine Eigenschaft treffe mit Wahrscheinlichkeit eins auf die Elemente
einer Menge $\set A$ zu, sagen wir, die Eigenschaft sei \begriff(prävalent) auf $\set A$. 
Da die Definition diese Begriffs auch für überabzählbar dimensionale Mengen sinnvoll
sein soll, kann er nicht über das verschwindende Lebesgue-Maß der Komplementärmenge
definiert werden, da ein solches hier nicht existiert. Die folgende Definition ist
entnommen aus \autor(Sauer \etal) \cite{Sauer91}:

\begin{definition}
Eine Borel-Teilmenge $\set A$ eines normierten Vektorraumes $\set V$ ist \begriff(prävalent), wenn
es einen endlich dimensionalen Untervektorraum $\set E$ aus $\set V$ gibt, so daß für alle $v$ aus
$\set V$ gilt, $v+e\in \set A$ für fast alle $e$ aus $\set E$.
\end{definition}
Den Unterraum $\set E$ bezeichnet man als \begriff(Testraum) (engl.: probe space). Die Prävalenz
einer Eigenschaft  kann man sich nun folgendermaßen vorstellen. Sei irgendein
Punkt $v$ aus $V$ vorgegeben, dann kann man von da aus in jede beliebige Richtung aus $E$
\naja(wandern) und trifft mit Wahrscheinlichkeit eins auf einen Punkt aus $S$. Da mit $E$
auch jeder Untervektorraum $E'$, der $E$ enthält, ein Testraum ist, ist leicht
einzusehen, daß die Prävalenz einer Eigenschaft für endlich dimensionale Vektorräume zu
der üblichen Definition von \naja(für fast alle) bzw.\ \naja(mit Wahrscheinlichkeit
eins) äquivalent ist.


Aufgrund des oben beschriebenen Mankos von Takens' Theorem bewiesen \autor(Sauer),
\autor(Yorke) und \autor(Casdagli) ein erweitertes Einbettungstheorem \cite{Sauer91}, das
sogenannte \begriff(Fractal Delay Embedding Prevalence Theorem).

\begin{theorem}
  Sei $\flow^t$ ein (durch ein Vektorfeld $\vec F$ erzeugter) Fluß auf einer offenen Teilmenge $\set U$ des $\R^k$ und sei $\set A$
  eine kompakte Teilmenge von $\set U$ der Kapazität $\fracdim$ \comment{(siehe Abschnitt
    \ref{chapcapacity})}. Sei $\embed>2\fracdim$ ganzzahlig und $\delay>0$. $\set A$ enthalte
  höchstens eine endliche Anzahl Fixpunkte, keine periodischen Orbits der Periode
  $\delay$ oder $2\delay$, und höchstens endlich viele periodische Orbits der Periode
  $3\delay,4\delay,\dots,\embed\delay$, und die Jakobimatrizen der Wiederkehrabbildungen der
  periodischen Orbits haben paarweise verschiedene Eigenwerte. Dann
  gilt für fast alle glatten Meßfunktionen $v:\set U\to\R$, daß die
  Verzögerungskoordinatenabbildung $\diffeo_{(\vec F,v,\delay)}:\set U\to\R^\embed$
  (siehe~\eqnref{takmod}) eine Einbettung von $\set A$ nach  $\diffeo_{(\vec
    F,v,\delay)}(\set A)$ 
  ist.
\end{theorem}
\comment{
: \comment{ eine Einbettung von $\set A$ liefert.}
\begin{myitemize}
\item eineindeutig auf $\set A$ ist.
\item auf jeder kompakten Teilmenge $\set C$ einer in $\set A$ enthaltenen 
  glatten Mannigfaltigkeit eine \naja(Immersion) ist.
\end{myitemize}
\end{theorem}
}
Die wesentlichen Unterschiede zu Takens' Theorem sind erstens, daß statt glatter
Mannigfaltigkeiten der Dimension $\mandim$ kompakte Mengen der Kapazität $\fracdim$
(welche erstere beinhalten)betrachtet werden, und zweitens, daß generisch durch die
stärkere Eigenschaft prävalent ersetzt wird.




\subsubsection{Verfahren zur Wahl der Einbettungsparameter}
\label{chapparams}
In den Einbettungstheoremen im vorigen Abschnitt wurde implizit vorausgesetzt, daß wir
unendliche, rauschfreie Zeitreihen zur Verfügung haben. Dies ist jedoch bei Daten aus
realen Experimenten niemals der Fall. Für verrauschte, endliche Zeitreihen ist eine
vernünftige Wahl der Verzögerungszeit $\delay$, die in den Einbettungstheoremen (bis auf
wenige Ausnahmen) beliebig sein kann, wesentlich. Außerdem ist die Dimension des
einzubettenden Attraktors unbekannt und somit auch die Anzahl der benötigten
Verzögerungskoordinaten. Die folgenden Abschnitte werden sich mit diesen Punkten
beschäftigen.


\paragraph{Die Einbettungsdimension}

Nach  dem Einbettungstheorem von \autor(Sauer \etal)  ist für Einbettungsdimensionen\footnote{Mit
  Einbettungsdimension ist hier und im folgenden die Dimension unseres
  Rekonstruktionsphasenraumes gemeint, gleichgültig ob es sich bei der Abbildung in
  diesen Raum um eine Einbettung handelt oder nicht.} $\embed$ größer als $2\fracdim$,
wobei $\fracdim$ die Kapazität des Attraktors ist, sichergestellt, daß die
Verzögerungskoordinatenabbildung eine Einbettung ist. Nun ist bei experimentellen
Zeitreihen die Kapazität nicht a priori bekannt. Wir müßten die Zeitreihe erst
einbetten, um die Kapazität bestimmen zu können.

Es stellt sich außerdem die Frage, ob nicht schon kleinere Einbettungsdimensionen
$\embed\lt 2\fracdim$ Einbettungen liefern. Wie wir am Beispiel des Rössler-Attraktors
gesehen haben, reicht hier eine Einbettungsdimension von $\embed=3$, während Sauers
Theorem wegen $D_0\simeq2,07$ eine Einbettungsdimension von $\embed=5$ fordert. Die
Bedingung $\embed>2\fracdim$ ist nur notwendig, um absolut sicher zu gehen, daß die
Abbildung eine Einbettung ist.

Um zu einer gegebenen Zeitreihe die optimale Einbettungsdimension zu bestimmen, sind eine
ganze Reihe von Verfahren entwickelt worden. Ein paar von diesen wollen wir im folgenden
vorstellen.
\begin{myitemize}
\item \rem(Falsche nächste Nachbarn:) Diese Methode beruht darauf, daß unter
  Einbettungen, die Nachbarschaftsbeziehungen zwischen benachbarten Punkten nicht
  ver\-än\-dert werden \cite{Kennel92}. Wenn zwei Orbitpunkte im Originalphasenraum benachbart
  sind, so sind sie es auch im Einbettungsraum. Wenn nun $\embed$ eine ausreichende
  Einbettungsdimension ist, gilt dies auch für $\embed+1$. Zwei Punkte die im $\R^\embed$
  benachbart sind, sollten also auch im $\R^{\embed+1}$ benachbart sein. Sind sie es
  nicht, kann die Abbildung in den $\R^\embed$ keine Einbettung gewesen sein.
  
  Man bestimmt nun zu einer Einbettungsdimension $\embed$ zu jedem Punkt $\x$ die
  \begriff(näch\-sten Nachbarn), die innerhalb einer $\eps$-Umgebung  von $\x$ liegen. Die
  Nachbarpunkte, die beim Übergang zur Einbettungsdimension $\embed+1$ aus der
  $\eps$-Umgebung von $\x$ \metapher(entweichen), werden als \begriff(falsche nächste
  Nachbarn) bezeichnet. Für ausreichende Einbettungsdimensionen wird das Verhältnis
  zwischen falschen und \metapher(echten) nächsten Nachbarn sehr klein, und wir
  können $\embed$ als Einbettungsdimension annehmen.
  
\item \rem(Attraktorvolumen:) Bei diesem Verfahren wird als erstes ein Maß für das
  \metapher(Volumen) des rekonstruierten Attraktors definiert. $\buzvol{k,\embed}$ ist das
  mittlere von je $\embed+1$ Attraktorpunkten aufgespannte Volumen zur Verzögerung $k$ .
  Das Volumen, das von den $\embed+1$ Attraktorpunkten $\folge(\x,0,\embed)$ aufgespannt
  wird, beträgt $\abs{\det\left(\x_1-\x_0,\dots,\x_\embed-\x_0\right)}$.  Für
  ausreichende Einbettungsdimensionen $d$ ist der Ausdruck
  $\log\buzvol{k,\embed+1}-\log\buzvol{k,\embed}$ für alle $k$ in guter Näherung
  konstant. Falls $d$ jedoch keine ausreichende Einbettungsdimension ist, kommt es für
  bestimmte $k$ zu Überschneidungen von Trajektorien und damit zu einem \naja(zu
  kleinem) mittleren Volumen. Falls $d+1$ nun
  ausreichend ist, treten diese Überschneidungen nicht mehr auf. Dies äußert sich in
  einem Peak in der Auftragung von $\log\buzvol{k,\embed+1}-\log\buzvol{k,\embed}$ über
  $k$. Treten diese Peaks für ein bestimmtes $\embed$ nicht mehr auf, so ist $\embed$
  eine ausreichende Einbettungsdimension \cite{Buzug94,Buzug90a}.

\item \rem(Singular Value Decomposition:) Hierbei
  wird zuerst eine Einbettung in einen hochdimensionalen Rekonstruktionsraum $\R^\embed$
  vorgenommen \cite{Broomhead-king}. Für niedrigdimensionale Dynamiken wird sich diese
  jedoch auf einen Unterraum $\subspace$ der Dimension $\minembed$ beschränken. Dieser
  Unterraum wird nun ermittelt und der Attraktor hieraus in den $\R^\minembed$ projiziert.
  Das Verfahren wird ausführlicher in Abschnitt~\ref{chapsvd} diskutiert.
\end{myitemize}
Die Reihe der Verfahren ließe sich beliebig fortsetzen. Ein gute Übersicht
findet sich bei \autor(Buzug) \cite{Buzug94}. 

Wir wollen nun eines der noch nicht
aufgeführten Verfahren genauer betrachten, da es sich u.a.\  gut dazu eignet,
Schwierigkeiten und Probleme bei der Analyse experimenteller chaotischer Systeme zu
demonstrieren. Bei diesem von \autor(Packard \etal) \cite{Packard80} entwickelten Verfahren stellen wir
uns den Attraktor eingebettet in eine $\mandim$-dimensionale Mannigfaltigkeit $\M$ vor.
Diese Mannigfaltigkeit sei ihrerseits eingebettet in den euklidischen Vektorraum
$\R^\embed$.  Schnitte von $\M$ mit $(\embed-1)$-di\-men\-sio\-nalen Hyperflächen
erzeugen nun im allgemeinen $(\mandim-1)$-dimensionale Mannigfaltigkeiten\footnote{Dies
  ist nur dann nicht der Fall, wenn der Schnitt leer ist oder die Hyperfläche tangential
  zu der Mannigfaltigkeit liegt.  Das erste werden wir im folgenden durch die Wahl der
  Hyperflächen ausschließen. Letzteres ist für prävalente Mengen von
  Mannigfaltigkeiten und Hyperflächen nicht der Fall.}.  Wenn wir die
Mannigfaltigkeit nun mit $\mandim$ paarweise orthogonalen Hyperflächen schneiden, wird
die Schnittmenge auf einen Punkt reduziert. Dies offeriert eine Möglichkeit, die Dimension
der Mannigfaltigkeit $\M$, in der der Attraktor $\attr\subset\M$ liegt, zu bestimmen.

Als Schnittflächen betrachten wir die zu den Koordinatenachsen orthogonalen Teil\-räume
des $\R^\embed$. Die Tatsache, daß ein Punkt innerhalb der Schnittmenge von $\mandim'$
dieser Teil\-räume mit der Mannigfaltigkeit $\M$ liegt, bringt uns Kenntnis über
$\mandim'$ seiner Koordinaten. Genau $\embed-\mandim'$ der Koordinaten sind unbestimmt,
wobei von diesen allerdings nur $\mandim-\mandim'$ Koordinaten unabhängig sind, da die
Mannigfaltigkeit ja $\mandim$-dimensional ist. Wenn wir also $\mandim$ Schnitte
betrachten, sind dadurch alle Koordinaten eines Punktes aus $\M$ festgelegt.

Diese Tatsache kann nun durch \begriff(bedingte Wahrscheinlichkeiten) ausgedrückt werden.
Wir legen $\mandim'$ Koordinaten $x^0_1,\dots,x^0_{\mandim'}$ fest. Die
Rekonstruktionspunkte, deren erste $\mandim'$ Koordinaten gleich den $x^0_i$ sind, bilden
die Schnittmenge des Attraktors mit den durch $\mathcal{S}_i=\left\{ \x \vert
  x_i=x^0_i\right\}$ gegebenen Hyperflächen. Die Wahrscheinlichkeit, daß ein Punkt aus
dieser Schnittmenge als $(\mandim'+1)$.\  Komponente den Wert $x$ aufweist, bezeichnen wir
mit $\Prob_{\mandim'}(x)=\Prob(x\vert x_1=x^0_1,\dots,x_{\mandim'}=x^0_{\mandim'})$. Mit
den anfänglichen Überlegungen, daß die genauen Koordinaten erst durch $\mandim$
Schnitte festgelegt sind, läßt sich nun schließen, daß die Verteilung von
$\Prob_{\mandim'}(x)$ für $\mandim'<\mandim$ ausgedehnt, für $\mandim'=\mandim$ dagegen
singulär werden muß.

Für die Berechnung der Verteilungen müssen wir aufgrund der endlichen Datenmenge (und
auch der endlichen Genauigkeit) von der exakten Gleichheit der Koordinaten abgehen und
Wahrscheinlichkeitsverteilungen $\Prob(i\vert i^0_1,\dots,i^0_{\mandim'})$ betrachten.
Hierbei ist der Wertebereich von $x$ in Intervalle $I_i=[x_i,x_{i+1}[$
aufgeteilt\footnote{Da wir hier Verzögerungskoordinaten betrachten, ist der Wertebereich
  für alle Koordinaten gleich, und es macht Sinn, die Intervalle für alle Koordinaten
  gleich zu wählen.}.\@ $\Prob(i\vert i^0_1,\dots,i^0_{\mandim'})$ ist die
Wahrscheinlichkeit, daß die $(\mandim'+1).$ Koordinate im Intervall $I_i$ liegt, unter der
Bedingung, daß die ersten $\mandim'$ Koordinaten in den Intervallen
$I_{i^0_1},\dots,I_{i^0_{\mandim'}}$ liegen.

\noafterpage{
  \epsfigdouble{packdim/packdim1}{packdim/packdim2}{ Im Text beschriebene
    Wahrscheinlichkeitsverteilungen für $m'=1$ (oben) bzw.\ $m'=2$ (unten). 
    Die Verzögerungszeit betrug $\delay=32\sample$. }
  {pakdim}{-0.2cm}

  \epsfigdouble{packdim/packdimb1}{packdim/packdimb2}{
    Wahrscheinlichkeitsverteilung wie in \psref{pakdim} aber mit Verzögerungszeit $\delay=35\sample$.
    }{pakdimb}{-0.2cm} 
}


Das Ergebnis einer solchen Berechnung ist in \psref{pakdim} am Beispiel des
Rössler-Attraktors dargestellt. Die Anzahl der Intervalle beträgt $200$, die
Vergleichskoordinaten wurden zu $x^0_1=0.0$ und $x^0_2=5.0$ (entspricht $i^0_1=124,
i^0_2=196$) gewählt. Für die Verzögerungszeit $\delay=32\sample$ 
sieht man deutlich, daß die Verteilung für $\mandim'=2$ sehr scharf wird (\psref{pakdim}
unten) und somit $\mandim=2$ gilt. Dies steht im
Einklang mit der lokalen Struktur des Rössler-Attraktors, nämlich einer Cantor-Menge
zweidimensionaler Schichten. Die Tatsache, daß $\Prob$ auch neben dem Maximalwert nicht
verschwindet, ist  bedingt durch die endliche Rechengenauigkeit und durch
die \emph{eben nicht} exakt zweidimensionale Struktur des Rössler-Attraktors.

In \psref{pakdimb} sind die Wahrscheinlichkeitsverteilungen für
$\delay=35\sample$ bei sonst gleichen Parameterwerten dargestellt. Die Verteilungen zeigen 
weder für $\mandim'=1$ noch für $\mandim'=2$ ein einzelnes scharfes Maximum. Man erkennt
hieran deutlich ein
Problem dieses Verfahrens. Für geringfügig andere Verzögerungszeiten\footnote{Daß diese Abweichung in Bezug auf
  die Bestimmbarkeit optimaler Verzögerungszeiten wirklich \begriff(geringfügig) ist,
  wird sich in Abschnitt~\ref{chapdelay} zeigen.} wird die
Verteilung $\Prob(i\vert i_1,i_2)$ nicht mehr singulär, sondern zeigt mehrere Maxima. Wir
müßten für diese Verzögerungszeit schließen, daß der Attraktor mindestens
dreidimensional ist.

Diese Tatsache wäre nicht so interessant, wenn die abweichenden Resultate bei
unterschiedlichen Parameterwerten nicht ein für viele Verfahren der Zeitreihenanalyse
auftretendes Phänomen wäre. Es stellen sich hier mehrere Fragen:
\begin{myitemize}
\item Inwiefern ist dieses Ergebnis nur ein Artefakt unserer speziellen Parameterwahl? In
  dem hier betrachteten Fall taucht für die meisten $\delay$ eine breite Verteilung auf.
  Ist nun eher dem Ergebnis für das spezielle $\delay$, welches $\mandim=2$ impliziert,
  oder den Ergebnissen für andere Verzögerungszeiten, welche eher $\mandim>2$ nahelegen,
  zu trauen? In diesem Fall ist dies durch unsere Systemkenntnis natürlich leicht zu
  entscheiden, aber wie sieht es bei unbekannten Systemen aus ?
\item Damit aussagekräftige Verteilungen erzeugt werden können, muß in der
  betrachteten Schnittmenge eine \naja(ausreichende) Zahl von Rekonstruktionspunkten
  liegen. Diese Zahl nimmt jedoch mit $(\Delta x)^{\mandim'}$ ab. Wir müssen also
  entweder mit einer geringeren Genauigkeit $(\Delta x)$ arbeiten, was häufig nicht
  akzeptabel ist, oder die Datenmenge entsprechend erhöhen. Dies ist jedoch bei
  experimentellen, insbesondere bei biologischen oder medizinischen, Systemen oft
  nicht möglich.
\end{myitemize}

Das erste Problem der Parameterwahl wird in abgewandelter Form noch öfter auftreten. Das
zweite, die exponentielle Zunahme der erforderlichen Datenmenge, wird in Abschnitt
\ref{chapcorrdim} genauer behandelt.

\paragraph{Die Verzögerungszeit}
\label{chapdelay}

Nach dem Einbettungstheorem von \autor(Sauer \etal) ist die Wahl der Verzögerungszeit bis auf wenige, in der Regel
erfüllte Ausnahmen beliebig. Bei endlichen vielen, eventuell verrauschten Daten ist die Wahl
einer \naja(guten) Verzögerungszeit jedoch entscheidend für eine erfolgreiche
Phasenraumrekonstruktion. In \psref{rekzeit} sind drei verschiedene Rekonstruktionen des
Rössler-Attraktors aus der Zeitreihe in \psref{rekroe} (unten) zu den Verzögerungszeiten
$\delay=3\sample$, $30\sample$ und $200\sample$ abgebildet.

\afterpage{
\epsfigtriplebot{zeit/rectslow}{zeit/rectsmed}{zeit/rectshigh} {Rekonstruktion des
  Rössler-Attraktors zu verschiedenen Verzögerungszeiten $\delay=3\sample$ (oben),
  $\delay=30\sample$  (unten links) und $\delay=200\sample$ (unten rechts). 
  Die Rekonstruktionen wurden im $\R^3$ vorgenommen,
  und hieraus für die Abbildung in den $\R^2$ projiziert.  
}{rekzeit}{-0.2cm}
}

Für kleine Verzögerungen $k=\delay/\sample$ werden die Koordinaten annähernd gleich:
\eqn{x_i\simeq x_{i+k}\simeq x_{i+2k};} die Rekonstruktionspunkte konzentrieren sich auf
die Hauptdiagonale des $\R^3$ (siehe \psref{rekzeit} oben). Für einen
Dimensionsalgorithmus, der nur mit endlicher Genauigkeit arbeiten kann, wäre dies kaum zu
unterscheiden, von einer eindimensionalen Struktur. In \psref{rekzeit} (unten rechts) ist
die Verzögerungszeit dagegen sehr hoch gewählt.  Der Attraktor scheint wesentlich
komplizierter als der des Originalsystems. Bei rauschfreien Daten mag das noch nicht ganz
so gravierend sein. Bei Vorhandensein von Rauschen werden die Koordinaten der
rekonstruierten Punkte jedoch mit steigendem $\delay$ zunehmend unkorreliert, Trajektorien
durchkreuzen sich, und der Attraktor spannt den ganzen Phasenraum auf. Aufgrund dieser
Probleme benötigen wir also Verfahren, um vernünftige Verzögerungszeiten bestimmen zu
können.


Methoden zur Bestimmung der Verzögerungszeit gibt es sehr viele.  Man kann sie grob
einteilen in Verfahren, die die Fensterlänge, d.h. den zeitlichen Abstand
$\delay_w=(\embed-1)\delay$ der ersten und der letzten Komponente der
Rekonstruktionsvektoren, und in Verfahren, die direkt die Verzögerungszeit bestimmen.  Zu
den ersteren zählt beispielsweise der Vorschlag von \autor(Broomhead) und \autor(King)
\cite{Broomhead-king}, $\delay_w$ als das Inverse einer das Frequenzspektrum begrenzenden
Frequenz\footnote{Eine genaue Definition oder ein Verfahren zur Bestimmung dieser Frequenz
  wird allerdings nicht angegeben.}  (engl.: band-limiting frequency) $f_\tmax$ zu wählen
oder die Methode von \autor(Hilborn) und \autor(Ding) \cite{Hilborn-ding}, die, um
Überfaltungen des Attraktors zu vermeiden, $\delay_w$ umgekehrt proportional zum
größten Lyapunov-Exponenten des Systems wählen\footnote{Wie der größte
  Lyapunov-Exponent ohne vernünftige vorherige Rekonstruktion bestimmt werden soll,
  bleibt jedoch im Unklaren.}. Die Verfahren, die hier besprochen werden sollen, sind von
der zweiten Art und berechnen die Verzögerungszeit direkt.

Bevor mit der Besprechung dieser Verfahren begonnen wird, kurz ein Kommentar zur Güte der
jeweils bestimmten Verzögerungszeiten. In vielen Publikationen ist zu lesen, dieses oder
jenes Verfahren sei in irgendeiner Weise anderen Methoden überlegen. Zum Teil werden
sogar ``Gütefunktionen'' definiert, die die Überlegenheit der Methode quantitativ,
anhand bestimmter Merkmale der Rekonstruktion, belegen sollen. Nach meiner Erfahrung macht
so ein Vorgehen wenig Sinn. Die ``beste'' Verzögerungszeit ist oft davon abhängig, was
man genau mit der Zeitreihe anstellen möchte, d.h. ob man z.B.  Dimensionen oder
Lyapunov-Exponenten berechnen möchte oder eventuell Poincar\'e-Plots erstellen will. Oft
liefern die Verfahren gute Ansatzpunkte für die Wahl der Verzögerungszeit, die genaue
Festlegung ist dann jedoch oft eine Sache von \naja(Trial and Error).
  

\subparagraph{Nulldurchgang der Autokorrelationsfunktion.} Ausgangspunkt für die
Entwicklung der Verzögerungskoordinatenabbildung war die Idee, daß die Dynamik des
Systems durch beliebige \emph{unabhängige} Koordinaten dargestellt werden kann. Es ist
daher sinnvoll, die Verzögerung so zu wählen, daß die Koordinaten möglichst
unabhängig werden. In der Signaltheorie wird ein Abweichungsmaß für zwei Signale $X$
und $Y$ über deren \begriff(Kreuzkorrelation) $C(X,Y)$ definiert:
\eqnl[crosskorr]{C(X,Y)=\lim_{T\to\infty}\frac{1}{2T}\int_{-T}^{+T} (x(t)-\bar
  x)(y(t)-\bar y) \mathrm{dt}.} 
 Hierbei sind $x(t)$ und $y(t)$ die Werte der Signale zur
Zeit $t$ und $\bar x$ und $\bar y$ ihre Mittelwerte\footnote{Die hier definierte
  Kreuzkorrelation heißt in der Signaltheorie eigentlich \begriff(Kreuzkovarianz). Beide
  Größen unterscheiden sich dadurch, daß bei der Kreuzkorrelation die Subtraktion der
  Mittelwerte {\em nicht} vorgenommen wird \cite{Lueke92}. Das Resultat ist, daß beide
  Funktionen sich um eine Konstante -- das Produkt der Mittelwerte der Signale --
  unterscheiden. Entsprechendes gilt für die Autokorrelationsfunktion. Wir wollen hier
  jedoch der in der Zeitreihenanalyse üblichen Bezeichnungsweise folgen. }.  Die
Kreuzkorrelation wird maximal, wenn die Signale proportional, und null, wenn sie
orthogonal zueinander sind. $C(X,Y)=0$ impliziert auch die lineare Unabhängigkeit der
Signale\footnote{Dies ist leicht einzusehen, wenn man die Signale $X$ und $Y$ als Elemente des
  Vektorraumes quadratintegrabler Funktionen mit dem Skalarprodukt $\langle X,Y \rangle=C(X,Y)$ auffaßt.}.
Bei der Verzögerungskoordinatenabbildung ist das Signal $Y$ nun genau das um die Zeit
$\delay$ verschobene Signal $X$ d.h.\ $y(t)=x(t+\delay)$.  Die rechte Seite von
\eqnref{crosskorr} geht damit über in die Definition der \begriff(Autokorrelationsfunktion):

\eqnl[autokorr1]{\ac(X,\delay)=\lim_{T\to\infty}\frac{1}{2T}\int_{-T}^{+T} (x(t)-\bar
  x)(x(t+\delay)-\bar x) \mathrm{dt} .} 

Für endliche diskrete Zeitreihen $\{x_i\}_{i=1\dots N}$ der Länge $N$ und Zeiten
$k\sample$ kann die Autokorrelationsfunktion genähert werden durch
\eqnl[autokorr2]{\ac(k\sample)=\frac{1}{N-k}\sum_{i=1}^{N-k} (x_i-\bar x)(x_{i+k}-\bar x)} 
mit
\eqnl[acmean]{\bar x = \frac{1}{N}\sum_{i=1}^N x_i .}

Für Zeiten $\delay$, die kein ganzzahliges Vielfaches der Sampling Time $\sample$ sind,
wird die Autokorrelationsfunktion linear approximiert durch:
\eqnl[autokorr3]{\ac(\delay)= \ac(k\sample) + \frac{\delay-k\sample}{\sample} \big(  \ac((k+1)\sample)
  - \ac(k\sample) \big)   ,}
wobei $k$ so bestimmt wird, daß $k\sample<\tau<(k+1)\sample$ gilt.  Die
aufeinanderfolgenden Koordinaten werden maximal unabhängig, wenn die Verzögerungszeit
$\delay=\tilde\delay_\ac$ so gewählt wird, daß $\ac(\tilde\delay_\ac)=0$ gilt.  Die
Autokorrelationsfunktion hat in der Regel mehrere Nullstellen. Um eine Überfaltung des
Attraktors zu vermeiden, sollte die Verzögerungszeit jedoch möglichst klein sein.  Aus
diesem Grund wählen wir den ersten Nulldurchgang der Autokorrelationsfunktion als
Verzögerungszeit $\delay=\tilde\delay_\ac$.  Falls $\tilde\delay_\ac$ kein ganzzahliges Vielfaches
der Sampling Time $\sample$ ist (was im allgemeinen  der Fall ist), muß stattdessen die Verzögerungszeit
$\delay_\ac=k_\ac\sample$ benutzt werden, die am nächsten bei $\tilde\delay_\ac$ liegt.

Die Implementierung des entsprechenden Algorithmus nach \eqnref{autokorr2} und
\eqnref{acmean} ist direkt durch Bildung der jeweiligen Summen durchführbar. Die Laufzeit
beträgt $\order{N^2}$. Dies läßt sich jedoch beschleunigen durch Anwendung des
\begriff(Wiener-Khintchine-Theorems). Demnach ist die Autokorrelationsfunktion eines
Signals $X$ die Fourier-Transformierte des Leistungsspektrums $P_X(f)$. Wir benötigen
also nur zwei Fourier-Transformationen, deren Laufzeit bei Verwendung der
Fast-Fourier-Transformation (FFT) jeweils nur \linebreak[4]  $\mathcal{O}(N\log N)$  beträgt.  Algorithmen zur
Berechnung 
der FFT finden sich in nahezu jeder Mathematikbibliothek (siehe z.B. \cite{Numerical-recipes}), so daß
hier nicht näher darauf eingegangen wird. Beispiele für die
Autokorrelationsfunktion und die zur jeweiligen Verzögerung $k_\ac$ rekonstruierten
Attraktoren sind in \psref{acfigsa} und in \psref{acfigsb}  dargestellt.  

\noafterpage{
  \epsfigdouble{autocorr/roeac}{autocorr/roerec2}{
    Autokorrelationsfunktion $\ac(\delay)$ (oben) und der zur Verzögerungszeit $\delay_\ac=1.29$
    rekonstruierte Attraktor (unten) für das Rössler-System. 
}{acfigsa}{-0.2cm} 
\epsfigdouble{autocorr/lorac}{autocorr/lorrec2} {
    Autokorrelationsfunktion $\ac(\delay)$ (oben) und der zur Verzögerungszeit $\delay_\ac=2.50$
    rekonstruierte Attraktor (unten) für das Lorenz-System. 
}{acfigsb}{-0.2cm} 
}

Für den Rössler-Attraktor gelingt die Rekonstruktion mit der so gewählten
Verzögerungszeit ganz passabel (siehe \psref{acfigsa} unten). Der Attraktor wirkt zwar
leicht überfaltet, die Messung von Dimensionen o.ä.\ ist hier aber trotzdem gut
möglich. Beim Lorenz-Attraktor ist die Verzögerungszeit erkennbar zu groß. Der
Attraktor ist stark überfaltet und spannt fast den gesamten Phasenraum auf (siehe
\psref{acfigsb} unten). Dimensionsalgorithmen werden hier schwerlich auf vernünftige
Werte konvergieren. Die zu hoch bestimmte Verzögerungszeit liegt in der Struktur des
Lorenz-Attraktors begründet. Die Orbits \naja(kreisen) meist mehrere Umläufe um einen
der instabilen Fixpunkte des Systems. Während dieser Zeit sind die Koordinaten daher
stark korreliert, da in dem einen Flügel $x$ konstant positiv und in dem anderen konstant
negativ ist. Die hier bestimmte Verzögerungszeit entspricht also ungefähr der mittleren
Aufenthaltszeit des Systems auf einem der Flügel.

Aufgrund dieser Schwächen gibt es Ansätze, statt des Nulldurchgangs das erste lokale
Minimum oder den Abfall auf ein $e$-tel des Anfangswertes $A(0)$ der Autokorrelationsfunktion zu betrachten. Diese
Ansätze bringen zwar zum Teil bessere Ergebnisse, sind jedoch theoretisch nicht zu
begründen. Wir wollen sie hier deshalb beiseite lassen und ein allgemeineres Verfahren
besprechen.

\subparagraph{Redundanzanalyse.} \label{chapredundancy} Wie wir gesehen haben stellt die lineare
Un\-ab\-hän\-gig\-keit zweier Koordinaten nicht das optimale Kriterium für eine gute
Verzögerungszeit dar. Die Probleme resultieren hauptsächlich daraus, daß wir es mit
\emph{nichtlinearen} Systemen zu tun haben. Wir suchen eine allgemeinere Unabhängigkeit
der Koordinaten.


\comment{Sei $X$ eine beliebige Zufallsvariable und $\Prob_X(i)$ die Wahrscheinlichkeit
  bei einer Messung von $X$ einen Wert im Intervall $[x_i,x_{i+1}[$ zu erhalten. Dann
  beträgt die mittlere Information einer Messung von $X$}

Um die \begriff(allgemeine) Unabhängigkeit zweier Koordinaten\footnote{Hiermit ist
  gemeint, daß zwischen den Werten der Koordinaten kein (wie auch immer gearteter)
  funktionaler Zusammenhang besteht.} zu untersuchen,
müssen\korrektur(naja) wir uns Begriffen der Informationstheorie, insbesondere dem
\begriff(Shannonschen Informationsmaß), zuwenden. Die Messung einer Observablen $X$ habe
$n$ verschiedene Ausgänge $x_i$, die mit den Wahrscheinlichkeiten $\Prob_X(i)$
auftreten\footnote{In der Informationstheorie betrachtet man im allgemeinen  Experimente, 
  die eine abzählbare Menge von Meßergebnissen liefern. Die möglichen Meßergebnisse
  werden in diesem Rahmen üblicherweise als \begriff(Ausgänge) der Messung bezeichnet.}.
Dann liefert eine Messung der Observablen $X$ im Mittel die
\begriff(Information)\footnote{In der Informationstheorie wird statt des natürlichen
  Logarithmus der Zweierlogarithmus benutzt. Die sich ergebende Einheit der Information
  ist $1\,\bit$. Wir benutzen hier, wie in der Physik üblich, den natürlichen
  Logarithmus. Die Einheit in der hier die Information gemessen wird ist das sogenannte
  $\nat$ (von engl.: natural). Beide Maße sind linear über $1\,\nat =\log_2
  e\,\bit\,$miteinander verknüpft.}
\eqn{H(X)=-\sum_i \Prob_X(i) \ln \Prob_X(i).} 
Die
mittlere Information ist maximal, wenn alle Ausgänge der Messung gleich wahrscheinlich
sind $\Prob_X(1)=\dots=\Prob_X(n)$. Sie wird umso kleiner, je stärker die
Wahrscheinlichkeiten auf wenige Ausgänge konzentriert sind.

Werden am betrachteten System zwei Messungen $X$ und $Y$ durchgeführt, so beträgt die
mittlere Information der kombinierten Messung:
\eqn{H(X,Y)=-\sum_{i,j} \Prob_{XY}(i,j) \ln\Prob_{XY}(i,j),} 
wobei $\Prob_{XY}(i,j)$ die Verbundwahrscheinlichkeit ist, daß bei der
Messung von $X$ der Wert $x_i$ und bei der Messung von $Y$ der Wert $y_j$ festgestellt
wird\footnote{Die Verbundwahrscheinlichkeit $\Prob_{XY}(i,j)$ ist zu unterscheiden von der
  bedingten Wahrscheinlichkeit $\Prob_{Y|X}(i,j)$, die angibt mit welcher
  Wahrscheinlichkeit an $Y$ der Wert $y_j$ gemessen wird, \emph{wenn} die Messung von $X$
  den Wert $x_i$ ergab. Beide Wahrscheinlichkeiten sind verknüpft über
  $\Prob_{XY}(i,j)=\Prob_{Y|X}(i,j)\Prob_X(i)$.}.  Die Information,
die die zusätzliche Messung von $Y$ liefert, ist daher im allgemeinen  kleiner als $H(Y)$,
nämlich:
\eqn{H(Y|X)=H(X,Y)-H(X).} 
Hierbei bezeichnet $H(Y|X)$ die mittlere Information der
Messung von $Y$ bei Kenntnis des Meßergebnisses von $X$. Diese zusätzliche Information
wird genau dann gleich $H(Y)$, wenn $X$ und $Y$ unabhängige Meßgrößen sind. In diesem
Fall gilt für die Verbundwahrscheinlichkeiten $\Prob_{XY}(i,j)=\Prob_X(i)\Prob_Y(j)$ und
wir erhalten:
\eqna{ H(X,Y)&=&-\sum_{i,j}\Prob_X(i) \Prob_Y(j) \{\ln \Prob_X(i)+\ln \Prob_Y(j)\} \nonumber \\
  &=& -\sum_i \Prob_X(i) \ln \Prob_X(i) - \sum_j \Prob_Y(j) \ln \Prob_Y(j) \nonumber \\
  &=& H(X)+H(Y),}
 somit $H(Y|X)=H(Y)$. In dem Fall, daß zwischen $X$ und $Y$ ein direkter
funktionaler Zusammenhang besteht, liefert die Messung von $Y$ gar keine zusätzliche
Information: 
\eqn{H(Y|X)=0.}

Wir definieren nun die \begriff(Redundanz) der kombinierten Messung $X,Y$. Redundanz
bedeutet im allgemeinen die Menge an überflüssiger Information. In diesem Fall ist dies die
Information, die sowohl in der Messung von $X$ als auch in der von $Y$ enthalten ist:
\eqnl[genredundancy]{R(X,Y)=H(X)+H(Y)-H(X,Y).}
 Die Redundanz $R$ wird genau dann minimal,
wenn die Meßgrößen maximal unabhängig voneinander sind. Die Redundanz wird auch
manchmal als \begriff(Transinformation) (engl.: mutual information) bezeichnet. Diese
Bezeichnung rührt daher, daß die Information $R(X,Y)$ von $X$ nach $Y$ \slang(übergeht)
oder \slang(fließt). Die Transinformation kann in diesem Kontext zur Beschreibung von
Informationsflüssen in ausgehnten System verwendet werden\cite{Pawelzik91}.  In höheren
Dimensionen ergeben sich jedoch Unterschiede zwischen der Redundanz und Transinformation
\cite{Prichard95}.



Die Observablen $X$ und $Y$ müssen nicht notwendig verschiedene Meßgrößen darstellen.
Sie können auch die zeitversetzte Messung \emph{einer} Größe bedeuten.  Dies bringt uns
auf das Verfahren zur Bestimmung der Verzögerungszeit.  Wir definieren die zweite
Observable $Y$ als den zu einer Zeit $t+\delay$ gemessenen Wert von $X$, während $X$ zur
Zeit $t$ gemessen wird. Damit geht \eqnref{genredundancy} über in
\eqnal[redundancy2]{R(X,\delay)&=&2H(X)-H(X,X_\delay) \nonumber \\
  &=& -2 \sum_i \Prob_X(i) \ln \Prob_X(i) + \sum_{i,j} \Prob_{X_\delay}(i,j) \log
  \Prob_{X_\delay}(i,j) ,} 
wobei $\Prob_{X_\delay}(i,j)$ die Wahrscheinlichkeit ist, daß
eine Messung an $X$ zu einer beliebigen Zeit $t$ den Wert $x_i$ und zur Zeit $t+\delay$
den Wert $x_j$ liefert\comment{Genauer haben wir es hier mit der bedingten
  Wahrscheinlichkeit $\Prob(x_j=x(t+\delay)|x_i=x(t))$ zu tun.}.  Da wir nach
Unabhängigkeit der Verzögerungskoordinaten gefragt haben, müssen wir also nur den Wert
von $\delay$ bestimmen, für den $R(X,\delay)$ minimal wird. Wie wir schon in den
Betrachtungen zur Autokorrelationsfunktion festgestellt haben, nimmt die Korrelation
zeitlich versetzter Messungen exponentiell ab. Die Redundanz $R(X,\delay)$ strebt also
gegen null für $\delay\to\infty$, erreicht ihr absolutes Minimum also für sehr große $\delay$. Da
wir sowohl an kleinen Verzögerungszeiten als auch an minimaler Redundanz interessiert
sind, wählen wir als Verzögerung das erste lokale Minimum von $R$.
\epsfigtriplebot{density/density1}{density/density2}{density/density3} {Die
  Verbundwahrscheinlichkeiten $\Prob_{X_\delay}(i,j)$ für das erste lokale Minimum
  ($\delay=0.162$, oben), das erste lokale Maximum ($\delay=0.234$, unten links) und das
  zweite lokale Maximum ($\delay=0.585$, unten rechts) der Redundanz $R(X,\delay)$ für
  das Lorenz-System.
}{reddensities}{-0.2cm}

Die Wahrscheinlichkeitsverteilungen $P_{X_\delay}(i,j)$ für das Lorenz-System bei
verschiedenen Verzögerungszeiten $\delay$ zeigt \psref{reddensities}. Flache Verteilungen
weisen auf niedrige Werte der Redundanz hin, während Verteilungen, die hauptsächlich in
kleinen Bereichen lokalisiert sind, hohe Werte der Redundanz anzeigen.  Entsprechend
erkennt man in \psref{reddensities} (unten links) für das erste lokale Maximum der
Redundanz, daß die Verteilung der Wahrscheinlichkeiten stark auf die Ränder konzentriert
ist.  In \psref{reddensities} (oben und unten rechts) für das erste bzw.\ zweite lokale
Minimum der Redundanz ist die Verteilung dagegen relativ flach. Die Verteilung für das
zweite lokale Minimum läßt jedoch kaum noch (die vom Lorenz-System bekannte) Struktur
erkennen.

Die Berechnung von $R(X,\delay)$ erfordert die Berechnung der Wahrscheinlichkeiten
$\Prob_X(i)$ und der Verbundwahrscheinlichkeiten $\Prob_{X_\delay}(i,j)$. Diese Berechnung
ist nicht ganz trivial. Da wir zum einen kontinuierliche Meßwerte und zum anderen eine
begrenzte Anzahl Meßpunkte haben, müssen wir den Meßraum geeignet partitionieren und
das Wahrscheinlichkeitsmaß dieser Partitionen bestimmen\footnote{Bei diskreten Meßwerten
  (d.h.\ endlich vielen Ausgängen des Experiments) wäre verständlicherweise keine
  Partitionierung notwendig. Bei unendlich vielen kontinuierlichen Meßwerten könnte man
  dagegen die entsprechende integrale Form von \eqnref{redundancy2} verwenden. Hierbei
  gehen die Summen in Integrale und die Wahrscheinlichkeiten in die entsprechenden Dichten
  über. Da wir jedoch nur endlich viele Meßwerte haben, ist diese Vorgehensweise hier
  nicht anwendbar.}  Als Partitionen wählen wir Intervalle $I^1_i=[\xi_i,\xi_{i+1}[$
bzw.\ Produkte von Intervallen $I^2_{i,j}=[\xi_i,\xi_{i+1}[\times[\xi_j,\xi_{j+1}[$. Die
Wahrscheinlichkeit $\Prob_X(i)$ ist dann gleich dem Anteil der Punkte $N_i$, der in das
Intervall $I^1_i$ fällt, d.h.\ $\Prob_X(i)=N_i/N$. Für die Verbundwahrscheinlichkeiten
gilt entsprechend $\Prob_{x,\delay}(i,j)=N_{ij}/N$.

Offen ist noch wie die Grenzen der Intervalle $I^1_i$ gewählt werden sollen. Es
existieren dazu verschiedene Ansätze.
\begin{itemize}
\item Der erste benutzt einfach äquidistante Grenzen $x_i$. Die Länge der Intervalle
  wird auf einen bestimmten Teil der Varianz\footnote{Die Spannweite der Meßwerte wäre
    hier ein schlechtes Maß, da bei experimentellen Systemen öfters \slang(Ausreißer)
    vorkommen, die die Spannweite stark verbreitern, sonst aber nicht viel beitragen.} der
  Meßwerte festgelegt. In den hier benutzten Beispielen wird die Intervallänge auf
  $1/50$ der Varianz der Daten eingestellt.
\item Eine weitere Methode arbeitet mit Intervallen $I^1_i$, die so festgelegt werden,
  daß die Wahrscheinlichkeiten $\Prob_X(i)$ für alle Intervalle nahezu gleich sind.
  Hierzu wird die Zeitreihe $x_i$ der Größe nach sortiert. Die sortierte Reihe werde mit
  $x_{(i)}$ bezeichnet.  Offensichtlich haben Intervalle $[x_{(i)},x_{(i+k)}[$ die
  Wahrscheinlichkeit $\Prob_X(x_{(i)}\leq x<x_{(i+k)})=k/N$. Da die Wahrscheinlichkeiten
  für alle Intervalle gleich sein sollen, legen wir für die Intervallgrenzen
  $\xi_i=x_{(\lfloor\frac{N}{n}(i-1)\rfloor+1)}$ fest, wobei $n$ die Anzahl der Intervalle
  ist \footnote{$\lfloor x \rfloor$ ist die größte ganze Zahl kleiner als $x$.}.
\item Ein von \autor(Fraser) und \autor(Swinney) vorgeschlagenes Verfahren bestimmt die
  Intervalle so, daß die Verbundwahrscheinlichkeiten $\Prob_{X_\delay}(i,j)$ für alle
  $(i,j)$ annähernd gleich werden \cite{Fraser-swinney}. Gestartet wird mit einer
  Partitionierung, die aus einem einzigen Element
  $[\xi^0_1,\xi^0_2[\times[\xi^0_1,\xi^0_2[$ besteht. Dieses Element wird nun so in vier
  rechteckige Elemente zerlegt, daß in jedem Element gleich viele Paare $(x_i,x_j)$ zu
  liegen kommen. Mit den durch diese Zerlegung entstandenen Elementen $\{
  [\xi^1_k,\xi^1_{k+1} [ \times [\xi^1_l, \xi^1_{l+1} [ \, \vert 1 \leq k , l \leq 2\}$
  wird nun in derselben Weise weiter verfahren. Nach $m$ Schritten erhält man so eine
  Partitionierung des Phasenraumes mit $4^m$ gleich wahrscheinlichen Elementen
  $\{[\xi^m_k,\xi^m_{k+1}[\times[\xi^m_l,\xi^m_{l+1}[\,\vert 1\leq k,l \leq 2^m\}$.
  Das Verfahren ist jedoch recht aufwendig und bringt meiner Einschätzung nach keine
  wesentlich besseren Ergebnisse.
\item \autor(Pawelzik) entwickelte ein Verfahren, um die Redundanz aus
  verallgemeinerten Korrelationsintegralen zu bestimmen \cite{Pawelzik91}.
  Es gilt $R(X,\delay,\eps)=2C_1(1,0,\eps)-C_1(2,\delay,\eps)$, wobei $C_1(d,\delay,\eps)$
  das verallgemeinerte Korrelationsintegral\footnote{Die verallgemeinerten
    Korrelationsintegrale $C_q(\eps)$ dienen (normalerweise) der Bestimmung der verallgemeinerten Dimensionen
    $D_q$, wobei $D_q = \lim_{\eps\to0}\log C_q(\eps)/\log\eps$ gilt. } zur Einbettungsdimension $d$ und 
  Verzögerungszeit $\delay$ ist \cite{Pawelzik-schuster}. $\eps$ entspricht der
  Länge der Intervalle $I_i^1$ bei äquidistanten Grenzen. 
\end{itemize}


Die Ergebnisse für die Messung der Verzögerungszeit $\delay_R$ über das erste lokale
Minimum der Redundanz zeigen \psref{redresulta} (oben) und \psref{redresultb}
(oben). Während sich die Qualität der Rekonstruktion beim
Rössler-Attraktor nicht wesentlich verbessert hat (siehe \psref{redresulta} unten), ist
das Resultat beim Lorenz-Attraktor (siehe \psref{redresultb} unten)
deutlich besser als bei der vorherigen Rekonstruktion mit $\delay_\ac$ (siehe \psref{acfigsb} unten). Die Verzögerungszeiten sind
bei beiden Systemen kleiner geworden: beim Rössler-System von $1.29$ auf $1.16$, beim
Lorenz-System sogar von $2.50$ auf $0.162$.  

Die über die
Redundanzanalyse bestimmten Verzögerungszeiten $\delay_R$ sind immer kleiner oder gleich den
über den Nulldurchgang der Autokorrelationsfunktion bestimmten $\delay_\ac$. Man kann
sich dies in etwa so plausibel machen, daß die bei der Verzögerungszeit $\delay_\ac$
vorliegende lineare Unabhängigkeit der Koordinaten auch immer eine allgemeine
Unabhängigkeit impliziert.

\noafterpage{
  \epsfigdouble{redundancy/roemut}{redundancy/roerec2}
  {Redundanz $R(\delay)$ (oben) und der zum ersten lokalen Minimum von $R(\delay)$
    rekonstruierte Attraktoren (unten) für das Rössler-System ($\delay_R=1.16$).
    }{redresulta}{-0.2cm} 
  \epsfigdouble{redundancy/lormut}{redundancy/lorrec2}
  {Redundanz $R(\delay)$ (oben) und der zum ersten lokalen Minimum von $R(\delay)$
    rekonstruierte Attraktoren (unten) für das Lorenz-System ($\delay_R=0.162$).
    }{redresultb}{-0.2cm} 
} 

\comment{
\noafterpage{
  \epsfigdouble{autocorr/roecmp}{autocorr/lorcmp} {Vergleich der Autokorrelationsfunktion
    $\ac(\delay)$ mit der Redundanz $R(\delay)$. Die Verzögerungszeiten sind für die
    Redundanz kleiner als die für die Autokorrelationsfunktion.  }{acfigs3}{-0.2cm} }
}
\clearpage












\subsection{Singular Value Decomposition}
\label{chapsvd}
Eine weitere Methode der Attraktorrekonstruktion stammt von \autor(Broomhead) und
\autor(King) \cite{Broomhead-king}. Diese sogenannte \begriff(Singular Value
Decomposition) (SVD) hat gegenüber der Ver\-zögerungskoordinateneinbettung hauptsächlich
zwei Vorteile. Zum einen muß die Einbettungsdimension nicht vorher festgelegt oder über
andere Verfahren ermittelt werden. Zum anderen kann durch die SVD ein dem Signal
eingeprägtes additives Rauschen vermindert werden.

%Grundidee
\comment{Die Grundidee ist folgende. Wir betten den Attraktor in einen hochdimensionalen Raum
$\R^\embed$ der Dimension $\embed$ ein. Dann spannt der so rekonstruierte Attraktor
in der Regel nur einen $\minembed$-dimensio\-nalen Unterraum
$\subspace=\mathop{\mathrm{span}}(\folge(\x,1,N))$  des
$\R^\embed$ auf\footnote{$N$ ist die Anzahl der Rekonstruktionsvektoren,
  und hängt mit der Anzahl der Meßpunkte $\tilde N$ über $N=\tilde N-d+1$ zusammen. Im allgemeinen
  gilt: $d\ll N$.}, wobei die Dimension dieses Unterraumes im allgemeinen niedriger
als die des Einbettungsraumes ist: $d'<d$.  Dieser Unterraum $\subspace$ wird nun bestimmt und der
Attraktor hieraus in den niedrigdimensionalen $\R^\minembed$ projiziert.
}

Die Grundidee ist folgende: Der Attraktor wird in einen euklidischen Vektorraum
$\R^\embed$ eingebettet, dessen Dimension $d$ so hoch gewählt wird, daß man mit großer
Sicherheit davon ausgehen kann, daß das Einbettungstheorem 2 erfüllt ist. Erwartet man
beispielsweise, daß der Attraktor eine Kapazität $D_0\leq5$ hat, könnte man für $d$
irgendeinen Wert größer oder gleich 10 (z.B. $d=20$) wählen. Durch die
Verzögerungskoordinatenabbildung erhält man aus der Zeitreihe, die aus den $\tilde N$
Meßwerten $\folge(x,1,\tilde N)$ bestehen möge, die $N=\tilde N-d+1$
Rekonstruktionsvektoren $\vec x_k = (\folge(x,k,k+d-1))^\Tr$.  Der so rekonstruierte
Attraktor spannt einen Unterraum $\subspace=\mathop{\mathrm{span}}(\folge(\x,1,N))$ des
$\R^\embed$ auf, dessen Dimension $\minembed=\mathop{\mathrm{dim}}\subspace$ kleiner oder
gleich $\embed$ ist. Man bestimmt eine Basis des Unterraumes $\subspace$, und kann nun
(durch Kenntnis dieser Basis) den in $\subspace$ liegenden Attraktor in den euklidischen
Vektorraum $\R^\minembed$ einbetten. Hierdurch verringert sich die Anzahl der für die
Angabe eines Attraktorpunktes nötigen Komponenten von $\embed$ auf $\minembed$.  Der
$\R^\minembed$ ist nach Konstruktion der Einbettungsraum mit der minimal ausreichenden
Einbettungsdimension.


%Neue Basis finden
Hauptbestandteil des Verfahrens ist es, eine neue Orthonormalbasis $\folge(\vec
c,1,\embed)$ des Einbettungsraumes $\R^\embed$ zu bestimmen. Diese soll so beschaffen sein,
daß die ersten $\minembed$ Vektoren der Basis den Unterraum $\subspace$ aufspannen:
\eqn{\subspace = \mathop{\mathrm{span}}(\folge(\vec c,1,\minembed)).} 
Da $\subspace$ nach Voraussetzung auch von den Rekonstruktionsvektoren $\folge(\x,1,N)$
aufgespannt wird, lassen sich die $\vec c_i\forall(i,1,\minembed)$ offensichtlich als
Linearkombination der $\x_k$ darstellen:
\eqnl[svdkomp]{\vec c_i = \sum_{k=1}^N \frac{1}{\sigma_i\sqrt N} s_{ik} \x_k,} 
wobei die Darstellung wegen $\minembed\ll N$ allerdings nicht eindeutig ist.  Die $s_{ik}$
bilden eine $\minembed\times N$-Matrix, deren Zeilenvektoren durch
$\vec s_i^\Tr = (s_{i1},\dots,s_{iN})$ gegeben sind (die $\vec s_i$ sind also $N$-komponentige Spaltenvektoren).  Da die $\vec c_i\forall(i,1,\minembed)$
nach Voraussetzung orthonormal (und somit auch linear unabhängig) sind, muß die
$s_{ik}$-Matrix mindestens 
  den Rang $\minembed$ haben; die $\vec s_i$ müssen daher auch
linear unabhängig sein.  Wir nehmen weiterhin an, daß die letzteren orthogonalisiert
sind. Durch den Faktor $1/\sigma_i\sqrt N$ wird erreicht, daß der Betrag der $\vec c_i$
{\em und} der $\vec s_i$ auf eins normiert werden kann. Der Faktor $1/\sqrt N$ macht die
$\sigma_i$ unabhängig von der Anzahl $N$ der Rekonstruktionsvektoren, wie später
einzusehen sein wird.  Wir definieren nun die als \begriff(Trajektorienmatrix) bezeichnete
$N\times \embed$-Matrix $\mat X$ durch:
\eqnl[svdtrdef]{\mat X = N^{-1/2}(\x_1, \dots, \x_N)^\Tr.}
\eqnref{svdkomp} kann dann in der folgenden Form geschrieben werden: 
\eqnl[svdbase]{\sigma_i \vec c_i = \tmat X \vec s_i.}

%Strukturmatrix
Daraus folgt unter Ausnutzung der vorausgesetzten Orthonormalität der $\vec c_i$:
\eqnl[svdbla1]{\sigma_i \sigma_j \delta_{ij} = \vec s_i^\Tr \mat X \tmat X \vec s_j.}
Die $N\times N$ Matrix $\gmat \Theta = \mat X \tmat X$, \comment{aufgrund ihres Aufbaus} auch
als \begriff(Strukturmatrix) bezeichnet, ist reell und symmetrisch. Ihre
Eigenvektoren bilden also eine vollständige, orthonormale Basis des $\R^N$. 
Eine Lösung der vorstehenden Gleichung erhält man, indem man für $\vec s_i$ 
den $i$ten Eigenvektor von $\gmat \Theta$ sowie für $\sigma_i$ die Wurzel des entsprechenden Eigenwerts wählt:
\eqnl[svdeigen1]{\gmat \Theta \vec s_i = \sigma_i^2 \vec s_i.}
Dies bestätigt man leicht durch Einsetzen in \eqnref{svdbla1} und Ausnutzung der Orthogonalität der
Eigenvektoren symmetrischer Matrizen. Von den $N$ Lösungen von \eqnref{svdeigen1}
besitzen nur $\minembed$ von null verschiedene Eigenwerte, da der Rang von $\mat X$
und somit auch der Rang von $\gmat\Theta$ gleich $\minembed$ ist. \comment{Nur diese Lösungen
sind für die Bestimmung der $\vec c_i\mraise{\rvert}{-0.2}_{i=1\dots\minembed}$ nach
\eqnref{svdbase} zu gebrauchen.}

\comment{Dies ist im allgemeinen  nicht die einzige Lösung 
  von \eqnref{svdbla1}. Zur Bestimmung des Unterraumes $\subspace$ reicht jedoch die
  Kenntnis {\em einer} Lösung. }

%Kovarianzmatrix
Die Bestimmung des Unterraumes $\subspace$ über \eqnref{svdeigen1} erfordert die
Diagonalisierung der Strukturmatrix $\gmat \Theta$. Da $\gmat \Theta$ eine $N\times
N$-Matrix ist, wird dies für große $N$ sehr rechenaufwendig. Die Diagonalisierung einer
$N\times N$-Matrix benötigt \order{N^3} Schritte.  Da von den $N$ Eigenwerten
$\sigma_i^2$ jedoch nur $\minembed$ nicht verschwinden, lohnt es sich nach einem
schnelleren Weg zu suchen.  Multiplizieren wir \eqnref{svdbase} von links mit der
\begriff(Kovarianzmatrix) $\gmat \Xi = \tmat X \mat X$, so erhalten wir mit
\eqnref{svdeigen1}:
\eqna{\gmat \Xi\sigma_i \vec c_i &=& \tmat X \mat X \tmat X \vec s_i \nonumber\\
&=& \tmat X \sigma_i^2 \vec s_i .}
Für $\sigma_i\neq 0$ ergibt sich dann mit \eqnref{svdbase}:
\eqnl[svdeigen2]{\gmat \Xi \vec c_i = \sigma_i^2 \vec c_i .}
Die neue Basis $\vec c_i$ erhalten wir  
also einfacher durch Diagonalisierung der Kovarianzmatrix $\gmat \Xi$. Da $\gmat \Xi$ eine
$\embed\times\embed$-Matrix und $\embed\ll N$ ist, kann \eqnref{svdeigen2} mit sehr viel
weniger Rechenaufwand als \eqnref{svdeigen1} gelöst werden.

%Bedeutung der Eigenwerte
Der vom Attraktor aufgespannte Unterraum $\subspace$ ist durch die Eigenvektoren $\vec
c_i$ gegeben, deren zugehörige Eigenwerte $\sigma_i^2$ nicht verschwinden. Um die genaue
Bedeutung der $\sigma_i$ zu erhellen, müssen wir erst die Struktur des neuen
Einbettungsraumes $\R^\minembed$ analysieren. Dieser Raum, in den die
Rekonstruktionsvektoren aus dem Unterraum $\subspace$ abgebildet werden, habe die
Orthonormalbasis $\vec e'_j\mraise{\rvert}{-0.2}_{j=1\dots\minembed}$.  Die in den 
$\R^\minembed$ eingebetteten Rekonstruktionsvektoren seien mit $\x'_i$ bezeichnet. Damit
die Abbildung von $\subspace$ nach $\R^\minembed$ orthogonal ist, soll die Projektion eines Rekonstruktionsvektors $\vec x'_i$
auf einen Basisvektor $\vec e'_j$ genauso groß sein, wie die Projektion von $\x_i$ auf
$\vec c_j$.
\eqnl[svdrecvec]{\vec x'_i = \sum_{j=1}^{\minembed} (\vec x_i \cdot \vec c_j) \vec e'_j .} 

\comment{
Um die Bedeutung der Eigenwerte zu erhellen, müssen wir erst die Struktur des neuen
Phasenraumes analysieren. Die Basis des Rekonstruktionsraumes wird gebildet durch die
Eigenvektoren der Kovarianzmatrix $\vec c_i$. Die Darstellung der Rekonstruktionsvektoren
$\vec x'_i$ bezüglich dieser neuen Basis erhalten wir über:
\eqnl[svdrecvecbla]{\vec x'_i = \sum_{j=1}^{\minembed} (\vec x_i \cdot \vec c_j) \vec e'_j .} 
}

Die mittlere Ausdehnung $\delta'_k$ des rekonstruierten Attraktors in Richtung eines Basisvektors
$\vec e'_k$ beträgt
\eqna{ \delta'_k &=& \left( \frac1N \sum_{i=1}^N (\x'_i \cdot \vec e'_k)^2 \right)^{1/2} \nonumber\\
&=&  \left( \frac1N \sum_{i=1}^N (\x_i \cdot \vec c_k)^2 \right)^{1/2} .}
Mit der  \eqnref{svdtrdef} folgt daraus:
\eqna{ \delta'_k &=& \left( \mat X \vec c_k \cdot \mat X \vec c_k  \right)^{1/2}
\nonumber \\
&=& \left( \vec c_k \cdot \gmat \Xi \vec c_k  \right)^{1/2} \nonumber \\ 
&=& \sigma_k .}
%&=& \left( \sum_{i=1}^N (\x_i \cdot \vec c_k)^2 \right)^{1/2}  \\ 
%&=& \left( (\tmat X \vec c_k)^2 \right)^{1/2}  }
Die Eigenwerte $\sigma_k$ geben also an, wie weit sich der Attraktor im Mittel in Richtung
$\vec e'_k$ bzw.\ $\vec c_k$ erstreckt. Die Verteilung der Attraktorpunkte kann man sich
in etwa vorstellen als ein $\minembed$-Ellipsoid dessen Halbachsen durch die Eigenwerte
$\sigma_k$ der Kovarianzmatrix gegeben sind.

%Rauschen
Bevor wir auf die Anwendung des Verfahrens kommen, müssen wir uns mit dem Einfluß von
Rauschen auf die Singular Value Decomposition beschäftigen, da selbst in ``rauschfreien''
Systemen aufgrund endlicher Speicher- und Rechengenauigkeit Rauschen erzeugt wird. 
Wir betrachten eine Zeitreihe mit additivem, gaußverteiltem Rauschen $x_i = \bar x_i +
\xi_i$. 
Ein Überstrich symbolisiert hier die deterministische Komponente. 
Wir müssen nun den Einfluß des Rauschens auf die Kovarianzmatrix betrachten. 
Aus der Definition der Kovarianzmatrix folgt:
\def\sumin{{\sum\limits_{i=1}^N}}
\eqnl[svdkovdef]{\gmat \Xi = \frac{1}{N}\left( \begin{array}{cccc}
 \sumin x_i x_i & \sumin x_i x_{i+1} & \dots & \sumin x_i x_{i+\embed-1} \\
 \vdots         & \vdots             & \ddots & \vdots \\
 \sumin x_{i+\embed-1} x_i & \sumin x_{i+\embed-1} x_{i+1}& \dots & \sumin x_{i+\embed-1} x_{i+\embed-1}  
\end{array} \right) .}
\comment{Da die Rauschterme $\xi_i$ unkorreliert sind, ergibt sich für die Elemente von $\gmat
\Xi$ bei großen Datenmengen $N$}
Die $x_i$ ersetzen wir nun durch die Summe aus deterministischer Komponente $\bar x_i$ und 
Rauschkomponente $\xi_i$. Da die Rauschterme $\xi_i$ untereinander unkorreliert sind, 
ergibt die Summation über $\xi_i\xi_{i+k}$ für $k\neq0$ bei großen Datenmengen $N$ null. Da die $\xi_i$ und die $x_i$ auch
unkorreliert sind, verschwindet die Summation über $\xi_i x_{i+k}$ ebenfalls. Wir erhalten 
so für die Elemente von $\gmat\Xi$:
\eqn{\gmat \Xi_{kl} = \frac{1}{N}\left( \underbrace{\sumin x_{i+k} x_{i+l}}_{= \overline{\gmat \Xi}_{kl}} + 
\underbrace{\sumin x_{i+k} \xi_{i+l}}_{\approx 0} + 
\underbrace{\sumin \xi_{i+k} x_{i+l}}_{\approx 0} + 
\underbrace{\sumin \xi_{i+k} \xi_{i+l}}_{\approx N <\xi^2>\delta_{kl}} \right), }
und somit:
\eqn{\gmat \Xi = \overline{\gmat \Xi}\; + <\!\xi^2\!>\unity_\embed .}
Die Eigenwerte der Kovarianzmatrix $\sigma_i^2$ werden also einheitlich um $<\!\xi^2\!>$
erhöht. 
Die Wirkung additiven Rauschens besteht darin, die Länge der Hauptachsen des oben beschriebenen
Ellipsoids gleichmäßig um den Betrag $\xi_0 = \sqrt{<\!\xi^2\!>}$ zu vergrößern. Auch die Richtungen,
für die ohne Rauschen $\sigma_i=0$ galt, werden nun von Trajektorien besucht. Es wird
somit der ganze $\embed$-dimensionale Phasenraum aufgespannt: $\subspace=\R^\embed$.

In Richtung der Achsen mit $\sigma_i\approx \xi_0$ wird die Dynamik allerdings
vollständig durch die Rauschterme dominiert. 
Für eine gute Rekonstruktion des Attraktors wird also nur der Teilraum
$\subspace=\mathop{\mathrm{span}}\{\vec c_1,\dots,\vec c_{\minembed}\}$ benutzt, für den
die entsprechenden Eigenwerte größer als die Stärke des Rauschens ist.

Wir können nun das Verfahren zusammenfassen.
\begin{enumerate}
\item Aus der Zeitreihe wird zu gegebenem $\embed$ die Kovarianzmatrix $\gmat \Xi$
  berechnet.  Aufgrund der Symmetrie dieser Matrix reicht es
  aus, nur die rechte obere Hälfte der Matrix inklusive der Diagonalen zu berechnen. Zusätzlich ergibt sich
  \eqnref{svdkovdef} die Beziehung
  $\gmat\Xi_{i+1,j+1}=\gmat\Xi_{ij}+\frac{1}{N}(x_{N+i}x_{N+j}-x_ix_j)$. Hierdurch kann
  die Anzahl der benötigten Multiplikationen und Additionen beträchtlich gesenkt werden.
\item Die Kovarianzmatrix wird diagonalisiert, und die erhaltenen Eigenwerte sowie
  Eigenvektoren werden nach absteigender Größe der Eigenwerte sortiert.  Hierfür stehen
  fertige Algorithmen in vielen Mathematikbibliotheken zur Verfügung (siehe z.B.~ \cite{Numerical-recipes}).
\item Die Größe des überlagerten Rauschens $\xi_0$ wird abgeschätzt.  Für diese
  Abschätzung existiert kein allgemeingültiges Verfahren;  sie erfolgt nach den
  besonderen Gegebenheiten und der Herkunft der Zeitreihe.  Hinweise auf die Größe von
  $\xi_0$ sind gegeben durch die interne Rechengenauigkeit und die Meßgenauigkeit bei der
  Aufnahme der Zeitreihe (z.B. Digitalisierungsrauschen).
  
  Liegen hierüber keine oder ungenügende Angaben vor, so kann $\xi_0$ auch anders
  abgeschätzt werden.  $\sigma_i$ strebt für große $i$ im allgemeinen gegen einen Grenzwert
  $\sigma_\tmin$, der durch das Rauschen bestimmt ist (siehe \psref{svdvalnoise}). Für
  $\xi_0$ wird dann dieser Grenzwert benutzt.
\item Die Rekonstruktionsvektoren können nun gemäß \eqnref{svdrecvec} berechnet werden.
  Statt alle $\minembed$ Komponenten zu verwenden, kann alternativ auch nur die
  erste (nun gefilterte) Komponente $\x_i\cdot\vec c_1$ über die
  Verzögerungskoordinatenabbildung wieder eingebettet werden.  Auf diese Weise dient die
  SVD nur einer adaptiven Rauschfilterung, und herkömmliche Zeitreihenanalyseverfahren
  können wieder angewandt werden.  Ein weitere Möglichkeit liegt darin, die ersten
  $n\leq\minembed$ für eine multivariante Verzögerungskoordinatenabbildung zu benutzen
  \cite{Fraedrich-wang}.  Darauf soll in diesem Rahmen jedoch nicht näher eingegangen
  werden.
\end{enumerate}

% \subsubsection{Anwendung des Verfahrens}
Das Verfahren soll nun am Beispiel des Lorenz-Attraktors demonstriert werden.  Nach
Integration der Differentialgleichungen wurde aus der $x$-Komponente des Zustandvektors
eine Zeitreihe gebildet.  Nach Berechnung der Kovarianzmatrix zu $\embed=7$ wurden die
Eigenwerte und Eigenvektoren ermittelt.  Die Ergebnisse der Rechnungen zeigen
\psref{svdval} und \psref{svdvec}.
\epsfigsingle{svd/simple/lorsvdval}
{Logarithmus der Eigenwerte $\sigma_i$ der Kovarianzmatrix $\gmat \Xi$ für den
Lorenz-Attraktor mit der Einbettungsdimension $\embed=7$.
}
{svdval}{-0.2cm}
\epsfigtripletop{svd/simple/lorsvdvec1}{svd/simple/lorsvdvec2}{svd/simple/lorsvdvec3}
{Die ersten drei Eigenvektoren der Kovarianzmatrix für den Lorenz-Attraktor
($\embed=7$). Für jeden Eigenvektor $\vec c_j$ ist die $k$-te Komponente
$\vec c_{j,k}$ über den Index $k$ aufgetragen.}
{svdvec}{-0.2cm}

In \psref{svdval} ist deutlich zu sehen, daß ab $i\geq 4$ die Eigenwerte $\sigma_i$ auf
einem nahezu konstanten Wert bleiben. Dies liegt daran, daß ab $i=4$ die Eigenwerte nur
noch durch Rauschen dominiert werden. Dieses ist allerdings nicht künstlich addiert
worden, sondern bedingt durch die Genauigkeit, mit der die Werte der Zeitreihe
zwischengespeichert wurden\footnote{Die Werte wurden mit einer Genauigkeit von 6 Stellen
zwischengespeichert, so daß $\xi_0\simeq 10^{-6}$ und $\ln(\xi_0)\simeq -13,8$ ist.}. 
Der Rauschpegel kann durch $\xi_0\simeq e^{-14}\sigma_1$ abgeschätzt werden.
Da nur die ersten drei Eigenwerte deutlich über $\xi_0$ liegen, sind auch nur die
ersten drei Eigenvektoren von $\gmat\Xi$ in \psref{svdvec}
dargestellt. 

Für die Berechnung der Rekonstruktionsvektoren $\x'_i$ im $\R^\minembed$ muß eine
Skalarmultiplikation der $\x_i$ mit den $\vec c_j$ durchgeführt werden. Die $\x_i$
bestehen nach Konstruktion aus aufeinanderfolgenden Werten der Zeitreihe. Es kann nun
gezeigt werden, daß die skalare Multiplikation mit $\vec c_1$ ungefähr einer Mittelung dieser 
Werte entspricht. Die skalare Multiplikation mit $\vec c_j$ entspricht i.a.\ einer gemittelten 
$(j-1)$-ten Ableitung. Um dies plausibel zu machen, sollen der Mittelwert und die 
numerischen Ableitungen für $\embed=3$ explizit aufgeführt werden:
\def\myvec(#1,#2,#3){\left( \begin{array}{c}#1\\#2\\#3\end{array}\right)}
\def\myvectr(#1,#2,#3){(#1,#2,#3)^\Tr}
\def\mysample{\sample}
%\def\mysample{}
\eqn{\begin{array}{lll} 
\bar x_i =  (x_{i-1} + x_{i} + x_{i+1} )/3  &=&
\myvec(x_{i-1},x_i,x_{i+1})\cdot\myvec(1/3,1/3,1/3) \\ 
{\bar{\dot x}}_i = (-x_{i-1} + x_{i+1})/2\mysample &=& \myvec(x_{i-1},x_{i},x_{i-1})\cdot\myvec(-1/2\mysample,0,1/2\mysample)\\ 
{\bar{\ddot x}}_i = ( x_{i-1} -2 x_{i} + x_{i+1} )/\mysample^2 &=& \myvec(x_{i-1},x_{i},x_{i-1})\cdot\myvec(1/\mysample^2,-2/\mysample^2,1/\mysample^2)
\end{array} .
}
Die Vektoren, die mit $\myvectr(x_{i-1},x_{i},x_{i-1})$ skalar multipliziert
werden, sind den $\vec c_j$ strukturell ähnlich; auch für $\embed>3$ wiesen die
entsprechenden Vektoren stets diese Form auf. Daraus wird ersichtlich, daß die Komponente
$\x_i\cdot\vec c_j$ in etwa der zeitlichen Ableitung $\abls{^{(j-1)}x_{i+\lfloor d/2 \rfloor}}{t^{(j-1)}}$ 
entspricht.
Auf die hieraus folgenden Konsequenzen werden wir  jedoch später
genauer eingehen.

Wir wollen nun die Singular Value Decomposition bei Vorhandensein additiven Rauschens
untersuchen. 
Als Beispiel möge hier wieder eine Zeitreihe des Lorenz-Attraktors dienen. 
Zu dieses wurde Gaußsches Rauschen in verschiedener Stärke addiert (Angaben in 
Prozent der Varianz von $x$). 
Für die so entstandenen verrauschten Zeitreihen wurden jeweils die
Kovarianzmatrix $\gmat \Xi$ und ihre Eigenwerte $\sigma_i^2$ bestimmt (siehe \psref{svdvalnoise}).
\noafterpage{
\epsfigsingle{svd/noise/lorsvd}
{Logarithmus der Eigenwerte der Kovarianzmatrix $\gmat \Xi$ für verschiedene
Rauschpegel (0\%;0,2\%;0,5\%;1,0\%;2,0\%;5,0\% von unten nach oben)}
{svdvalnoise}{-0.2cm}
}

Man erkennt deutlich, daß die Eigenwerte $\sigma_i$ ab $i=4$ nur durch die Stärke des
Rauschens bestimmt sind. 
Die Verhältnisse der Eigenwerte sind durch die relative Stärke der
Rauschamplitude gegeben. 

Verschiedene Rekonstruktionen des Lorenz-Attraktors aus der Zeitreihe mit 5\% Rauschen sind
in \psref{svdrecnoise} abgebildet. Hierbei wurden unterschiedliche Verfahren angewendet.
In der oberen Abbildung wurde einfach durch die Verzögerungskoordinatenabbildung das 2-Tupel
$(x_i,x_{i+k_R})$ rekonstruiert, wobei die Verzögerung $k_R=19$ durch Redundanzanalyse
gewonnen wurde. Unten links wurden die ersten beiden Komponenten
$(x'_{i,1},x'_{i,2})$ der SVD-Rekonstruktion eingebettet, während unten rechts die erste Komponente
$x'_{i,1}$ mit der um $k_R$ verzögerten Komponente $x'_{i+k_R,1}$ eingebettet wurde.
Man sieht deutlich, wie der Signal-Rausch-Abstand durch die SVD-Rekonstruktion verbessert
wird.

\afterpage{
\epsfigtriplebot{svd/noise/lorrec50_1}{svd/noise/lorsrec50_1}{svd/noise/lorrecs50_1}
{Rekonstruktionen des Lorenz-Attraktors bei 5\% Rauschen. 
Oben durch einfache Verzögerungskoordinatenabbildung, unten links durch normale
SVD-Rekonstruktion, unten rechts durch Verzögerungskoordinateneinbettung der ersten Komponente
der SVD-Rekonstruktion. 
}
{svdrecnoise}{-0.2cm}
}

Die Vorteile der Singular Value Decomposition zur Rauschverminderung sind klar
ersichtlich. Auf der anderen Seite kann die Rekonstruktion über Verzögerungskoordinaten
bei Verwendung der durch Redundanzanalyse gewonnenen Verzögerungszeit $\delay_R$
\naja(bessere) Einbettungen liefern. Dies zeigte \autor(Fraser), indem er ein
sogenanntes \begriff(Verzerrungsfunktional) (engl.: distortion functional)
einführte \cite{Fraser}. Dieses mißt, wie gut die Lage eines Punktes im Originalphasenraum aus der
Kenntnis seines rekonstruierten Bildes bestimmt ist. Das Funktional mißt
-- mit Frasers Worten --, wie \naja(diffeomorph) die Rekonstruktion ist bzw.\  welche
Rekonstruktion \naja(diffeomorpher) zum Original ist. Das Ergebnis dieses Vergleichs ist,
daß die durch Redundanzanalyse gewonnenen Rekonstruktionen den SVD-Rekonstruktionen
überlegen sind. Wir werden also im folgenden das kombinierte Verfahren verwenden. Die
Zeitreihe wird durch SVD eingebettet, die erste, rauschgefilterte Komponente wird
extrahiert und über Redundanzanalyse wieder eingebettet. Dies steht auch in Einklang mit
den Ergebnissen von \autor(Mees \etal) , die zwar die rauschmindernden
Eigenschaften der SVD bestätigen, die Möglichkeit, den vom Attraktor aufgespannten
Unterraum zu identifizieren, jedoch zweifelhaft erscheinen lassen\cite{Mees87}.





\newpage
\newpage
%\section{Quantitative Charakterisierung seltsamer Attraktoren}
\section{Fraktale Dimensionen}
Bei der Untersuchung chaotischer System ist es wichtig ein Ma"s f"ur die Komplexit"at der
Dynamik zu haben. Ein solches Komplexit"atsma"s ist durch die fraktale Dimension des Attraktors
gegeben. Bei integrablen Systemen ist die Dimension des Attraktors gleichzeitig auch die Anzahl der 
Freiheitsgrade. Bei chaotischen Systemen dagegen wird die Anzahl der Freiheitsgrade durch
Dissipation (s. Abschnitt \ref{chapdynsystems}) stark reduziert auf wenige effektive
Freiheitsgrade. Die Dimension des Attraktor ist i.allg. weit geringer als die Anzahl der
Freiheitsgrade. Dar"uber hinaus l"a"st sie sich  die Dimension seltsamer Attraktoren nicht
mehr durch ganze Zahlen 
beschreiben. Beispielsweise ist der Attraktor des H\'enonsystems \naja(mehr) als eine
Kurve, jedoch \naja(weniger) als eine Fl"ache. Seine Dimension liegt irgendwo zwischen
eins und zwei. Es sollen im folgenden meherere Verfahren zur Berechnung der Dimension
solcher fraktalen Objekte beschrieben werden. Dabei liegt das Hauptaugenmerk auf der
numerischen Berechenbarkeit durch das Verfahren. 

\subsection{Hausdorffdimension}
Von den verschiedenen Dimensionsbegriffen soll als erster der grundlegendste, n"amlich der 
der \begriff(Hausdorffdimension) eingef"uhrt werden.
Der Grundgedanke ist der folgende. Wenn ein geometrisches Objekt, beispielsweise eine
Fl"ache, mit einem Ma"s niedrigerer Dimensionalit"at, z.B. durch Geradenst"ucke, ausgemessen 
wird, ergibt sich ein unendlicher Wert bez"uglich dieses Ma"ses, da wir unendlich viele
Geradenst"ucke brauchen um die Fl"ache zu "uberdecken. Wird dagegen mit einem Ma"s h"oherer Dimension gemessen,
z.B. indem die Fl"ache durch W"urfel ausgemessen wird, ergibt der Wert Null, da eine
Fl"ache kein Volumen besitzt. Nur wenn mit einem 
Ma"s der gleichen Dimension gemessen wird, resultiert ein endlicher\footnote{Vorausgesetzt 
die Menge ist kompakt. Dies ist im weiteren nicht wesentlich, da es nur darauf ankommt, da"s
eine Sprungstelle von Unendlich auf Null existiert.} Wert. Aus diesem Sprungverhalten kann auf die Dimension des Objekts 
geschlossen werden.

Nun m"ussen die oben benutzten Begriffe genauer, mathematisch spezifiziert werden. 
Sei $\B$ eine beliebige, nichtleere Teilmenge des $\R^n$, deren Dimension wir messen m"ochten. Um das
$d$-dimensionale Ma"s  der Menge zu bestimmen, "uberdecken wir sie mit Teilmengen 
$\set C_{\eps,i}$ des $\R^n$ deren Durchmesser $\norm{\set C_{\eps,i}}$ kleiner als eine obere
Schranke $\eps$ sein soll. Ein solche "Uberdeckung bezeichnen wir mit $\mathcal
C_\eps$. Wir definieren nun 
\eqnl[hausdorffmass1]{\hdm(\B,d,\eps) = \inf_{\mathcal C_\eps} \sum_{\set C_{\eps,i} \in \mathcal C_\eps}\norm{\set C_{\eps,i}}^d }
Auf diese Weise betrachten wir alle m"oglichen "Uberdeckungen von $\B$, durch Mengen,
deren Durchmesser h"ochstens $\eps$ ist, und \naja(versuchen) die Summe der $d$-ten Potenzen der
Durchmesser zu minimieren. Wenn $\eps$ kleiner wird reduziert sich die Anzahl der
m"oglichen "Uberdeckungen und $\hdm(\B,d,\eps)$ strebt einem Grenzwert entgegen.
\eqnl[hausdorffmass2]{\hdm(\B,d) = \lim_{\eps\to 0}\hdm(\B,d,\eps)}
Der Grenzwert $\hdm(\B,d)$ existiert f"ur alle $\B$ und $d$ und hei"st das
$d$-dimensionale \begriff(Hausdorffma"s) der Menge $\B$. F"ur ganzzahlige $d$ entspricht
es, bis auf einen konstanten Faktor, dem $d$-dimensionalen Lebesguesma"s der Menge
$\B$. Nach dem weiter oben gesagten, sollte nun eine Sprungstelle $D_H$ existieren, so
da"s
\eqnl[hausdorffmass3]{\hdm(\B,d) = \left\{ \begin{array}{ll}
					  \infty, & d<D_H\\
					  0,      & d>D_H
					  \end{array}\right. }
Diese Sprungstelle existiert immer\footnotemark. Das Hausdorffma"s $\hdm(\B,d)$ kann f"ur
$d=D_H$ einen unendlichen oder einen endlichen Wert gr"o"ser Null annehmen\footnotemark. Wir definieren als
\begriff(Hausdorffdimension) der Menge $\B$
\eqnl[hausdorffdim]{D_H(\B) = \inf\{d\vert \hdm(\B,d)=0 \} }
\footnotetext{Dies folgt aus den Skalierungseigenschaften von $\hdm(\B,d,\eps)$. Es gilt
$\hdm(\B,t,\eps)\leq\eps^{t-d}\hdm(\B,d,\eps)$. L"a"st man $\eps$ gegen Null laufen, kann
mit \eqnref{hausdorffmass2} die Existenz der Sprungstelle gefolgert werden. } 
 \footnotetext{Nicht jedoch den Wert Null, wie in manchen Publikationen f"alschlich
behauptet, au"ser im trivialen Fall $\B=\emptyset$. Hierf"ur ist die Hausdorffdimension
jedoch nicht definiert.}

Die Hausdorffdimension ist die mathematisch am \naja(saubersten) definierte. Ihr
Anwendungsbereich liegt vor allem in der Theorie. F"ur die Verwendung in
Computeralgorithmen ist sie dagegen schlecht geeignet. Dies liegt haupts"achlich an
Einbeziehung beliebiger "Uberdeckungen, was mit Computermethoden nicht zu realisieren
ist. F"ur numerische Berechnungen ist es sinnvoller die Menge der m"oglichen
"Uberdeckungen einzuschr"anken. Eine dieser Einschr"ankungen f"uhrt uns auf den Begriff
der Kapazit"at.



\subsection{Kapazit"at}
\label{chapcapacity}
Bei der auf \autor(Kolmogorov) zur"uckgehenden \begriff(Kapazit"at) wird die Klasse der
$\eps$-"Uberdeckungen von $\B$ beschr"ankt auf "Uberdeckungen $\mathcal C^K_\eps$, die als Teilmengen nur
$n$-dimensionale W"urfel mit Durchmesser $\eps$ enthalten. Der Summand in
\eqnref{hausdorffmass1} ist nun konstant gleich $\eps^d$. Das auf die "Uberdeckungen $\mathcal C^K_\eps$
beschr"ankte Ma"s $\kpm$ ist also nur abh"angig von der Anzahl der Mengen, die zur "Uberdeckung
von $\B$ minimal gebraucht werden. Bezeichnen wir diese Anzahl mit $N(\eps)$, so gilt f"ur das
Ma"s $\kpm$
\eqn{\kpm(\B,d,\eps)=N(\eps)\eps^d}
Auch dieses Ma"s hat eine Sprungstelle bei einem bestimmten $d=D_K$ auf. Diese l"a"st sich 
jedoch weit einfacher ermitteln als bei der Hausdorffdimension in
\eqnref{hausdorffdim}. Offensichtlich kann $\kpm(\B,d,\eps)$ nur dann ein endlichen Wert
annehmen, wenn $N(\eps)$ mit $(1/\eps)^d$ skaliert. Daher definieren wir
\eqnl[capacity]{D_K(\B) = \lim_{\eps\to0} \frac{\log N(\eps)}{\log(1/\eps)}}
Dies ist die Kapazit"at der Menge $\B$. Da die Bestimmung der Kapazit"at nach
\eqnref{capacity} nur darauf beruht, da"s die \naja(K"astchen), die zur "Uberdeckung der
Menge $\B$ ben"otigt werden, gez"ahlt werden, spricht man auch von der
\begriff(boxcounting-) oder \begriff(K"astchenz"ahldimension). F"ur typische Attraktoren
wird erwartet, da"s Kapazit"at und Hausdorffdimension "ubereinstimmen
\cite{farmer-ott-yorke}. Es k"onnen jedoch Mengen konstruiert werden f"ur die das nicht der 
Fall ist.\footnote{Beispielsweise hat die Menge $\B=\{1/i\vert i\in\N\}$ die 
Hausdorffdimension $D_H=0$ und die Kapzit"at $D_K=1/2$ \cite{Leven89}. Dies ist im "ubrigen einer der
Schwachpunkte der Kapzit"at, da"s sie abz"ahlbaren Mengen endliche Dimension zuordnen kann.}

K"astchenz"ahlalgorithmen sind auf Computern sehr einfach zu implementieren. Der
Phasenraum braucht nur in K"astchen der Kantenl"ange $\eps$ eingeteilt werden\footnotemark. Dies
geschieht auf dem Computer, indem in Feld $F$ mit $(L/\eps)^n$ Eintr"agen angelegt
werden, wobei $L$ die lineare Ausdehnung des Attraktors ist. Jedem dieser Eintr"age wird
nun genau ein K"astchen des Phasenraumes 
zugeordnet. F"ur jeden Rekonstruktionspunkt wird der Eintrag $i$ ermittelt und $F(i)$ um
eins erh"oht. Die Anzahl der Eintr"age, f"ur die $F(i)\neq 0$, entspricht (ungef"ahr) der
Anzahl $N(\eps)$. Hieraus kann dann "uber \eqnref{capacity} die Kapazit"at abgesch"atzt
werden. Dieses direkte Verfahren ist jedoch sehr Speicher- und Zeitaufwendig, da 
sowohl Speicherbedarf als auch Rechendauer mit der Ordnung \order{(L/\eps)^n} anwachsen.
\footnotetext{Dies entspricht nicht genau dem bei der Definiton der Kapzit"at gemachten
Ansatz, da hier ein festes Gitter mit Gitterkonstante $\eps$, statt einer "Uberdeckung
durch beliebige K"astchen der Kantenl"ange $\eps$ benutzt wird. Generisch ist der
Grenzwert in \eqnref{capacity} jedoch bei beiden Ans"atzen gleich.  }


Ein wesentlich schnelleres und speicherschonenderes Verfahren soll im folgenden
vorgestellt werden \cite{Junglas}.
\comment{Das Verfahren, da"s sich aus der Definition der Kapazit"at f"ur eine computergest"utzte
Berechnung ableitet, ist sehr einfach und erfolgt  in folgenden Schritten \cite{Junglas}:}
\begin{itemize}
\item Jeder Punkt der Menge\footnote{Da jede auf einem
Computer darstellbare Menge abz"ahlbar sein mu"s, gehen wir hier wie auch im folgenden bei 
rechnerischen Verfahren von abz"ahlbaren Mengen aus.} $\x\in\B\,\subset\,\R^n$ wird auf einen
Punkt $\y\in\Z^n$, den sogenannten \begriff(Indexraum),  abgebildet. Hierzu wird jede
der Komponenten von $\x$ durch $\eps$ geteilt und der gebrochene Teil abgeschnitten
d.h.\  $y_i=\lfloor x_i/\eps \rfloor$. Durch diese Abbildung wird jedem Element $\x$ der
Menge das $\x$ enthaltende K"astchen mit den Indizes $y_1,\dots,y_n$ zugeordnet.
\item Die Menge der $\y_i$ wird nun geordnet. Dazu ist es notwendig eine Ordnungsrelation
auf $\Z^n$ zu definieren. Die genaue Definition dieser Relation ist hier unwesentlich. Sie mu"s nur den
mathematischen Anforderungen an eine Ordnungsrelation entsprechen. Wir definieren: $y_i$
ist genau dann kleiner als $y_j$, wenn ein $k\in\{1,\dots,n\}$ existiert, so da"s
alle $(\y_i)_m=(\y_j)_m$ f"ur alle $m<k$ und $(\y_i)_m<(\y_j)_m$ f"ur $m=k$ gilt.
\item Nach der Konstruktion im ersten Schritt ist $N(\eps)$ identisch mit der Anzahl
verschiedener $\y_i$, da dies genau der Anzahl von $\B$ belegter K"astchen
entspricht. Durch die Sortierung in Schritt zwei kann diese Anzahl sehr schnell abgez"ahlt 
werden. Division von $N(\eps)$ durch $\log(1/\eps)$ liefert eine Absch"atzung f"ur $D_0$.
\end{itemize}
Der zeitaufwendigste Teil ist die Sortierung der Punkte mit \order{n N\log N} Schritten
w"ahrend der erste und dritte Teil des Verfahrens nur \order{n N} Schritte ben"otigen. 
Die Berechnungszeit w"achst also, im Gegensatz zu manchen gegenteiligen Behauptungen,
nur linear mit $n$ und nicht exponentiell. Demgegen"uber w"achst jedoch die
ben"otigte Datenmenge $N$, wie wir sp"ater noch sehen werden, exponentiell oder
"uberexponentiell\footnote{In der Literatur existieren hier verschiedene Absch"atzungen
f"ur $N$.} mit $D_K$. Dies ist jedoch ein gemeinsames Charakteristikum
aller Dimensionsberechnungen.

Bei der Berechnung der Kapazit"at stellt sich jedoch ein anderes Problem. Aufgrund der
endlichen Datenmenge werden manche K"astchen nicht mitgez"ahlt, obwohl sie f"ur
$N\to\infty$ auch von Trajektorien besucht w"urden. Dies f"uhrt zu relativ gro"sen Fehlern 
bei der Anwendung dieses Verfahrens. 

K"astchen werden bei der Kapzit"at
entweder gez"ahlt oder nicht gez"ahlt. Besser w"are, bei endlichen Datenmengen, statt die
K"astchen  einfach zu z"ahlen, sie entsprechend ihrer Wahrscheinlichkeit zu
gewichten. Dies f"uhrt uns zu den sogenannten \begriff(probabilistischen) Dimensionen, der 
Informations- und der Korrelationsdimension, sowie den verallgemeinerten Dimensionen.



\subsection{Informationsdimension}
Die \begriff(Informationsdimension) w"ahlt einen v"ollig anderen Zugang zum Begriff der Dimension,
als die beiden vorangegangenen. Um einen Punkt in einem $n$-dimensionalen Raum festzulegen 
werden genau $n$ reelle Zahlen ben"otigt. Anstatt anzugeben, wie wieviele reelle Zahlen
hierzu ben"otigt werden, kann auch die Menge an Information angegeben
werden, die ben"otigt wird um die Position des Punktes mit einer Genauigkeit $\eps$
festzulegen. Diese Information betr"agt $I(\eps)=-n\ln(\eps)$, wobei wir hier wieder die
\naja(physikalischere) Einheit der Information in $\nat's$ gew"ahlt haben. Ist nun bekannt, da"s
sich der Punkt in einer $D_I$-dimensionalen Teilmenge $\B$ des $\R^n$ befindet, reduziert 
sich die notwendige Information auf $I(\eps)=-D_I\ln(\eps)$. Andererseits bringt uns die 
Informationstheorie eine Ausdruck f"ur die mittlere Information, die die Messung eines
Punktes aus $\B$ liefert\footnote{Vorrausgesetzt auf $\B$ ist ein Wahrscheinlichkeitsma"s
definiert.}. "Uberdecken wir die Menge mit K"astchen der Kantenl"ange $\eps$ und sei $\Prob_i$ 
das Wahrscheinlichkeitsma"s des $i$-ten K"astchens $\bar I(\eps)=-\sum_i \Prob_i\ln \Prob_i$. F"ur 
$\eps\to 0$ k"onnen wir beide Ausdr"ucke gleichsetzen und nach $D_I$ aufl"osen.
\eqnl[informationdim]{D_I(\B)=\lim_{\eps\to 0}\frac{\sum_i \Prob_i\ln \Prob_i}{\ln\eps}}
Dies ist die Informationsdimension der Menge $B$. Die Informationsdimension hat in der
letzten Zeit gegen"uber der noch zu besoprechenden Korrelationsdimension wieder vermehrt
Anwendung gefunden. Dies liegt daran, da"s sie "uber sogenannte
\begriff(N"achst-Nachbar-Algorithmen) gut berechenbar gemacht wurde\cite{Badii85}. Sei
$\delta(k)$ der Abstand eines Referenzpunktes zu seinem $k$-ten n"achsten Nachbarn, dann
gilt
\eqn{D_I=-\frac{\log N}{<\log \delta(k)>}}
wobei $<\log \delta(k)>$ der Erwartungswert von $\delta(k)$ ist. Dieser wird durch
Mittelung "uber geeignete Referenzpunkte berechnet. F"ur die Vorteile dieser Methode
gegen"uber den Korrelationsintegralen sie auf \cite{Liebert91} verwiesen.





\subsection{Korrelationsdimension}
\label{chapcorrdim}


Die momentan g"angigste Methode der Dimensionsbestimmung ist die Berechnung der
\begriff(Korrelationsdimension) nach \autor(Grassberger) und \autor(Procaccia)
\cite{Grassberger-procaccia}. Die Korrelationsdimension ist definiert durch:
\eqnl[cdimdef]{D_C = \lim_{\eps\to 0} \frac{\log(C(\eps))}{\log(\eps)},}
wobei $C(\eps)$ das sogenannte \begriff(Korrelationsintegral) darstellt: 
\eqnl[cintdef]{C(\eps) = \frac{1}{N(N-1)}\sum_{i\neq j}\Theta(\eps-\norm{\x_i-\x_j}).}
$C(\eps)$ \naja(z"ahlt) wieviel Paare von Punkten existieren, die einen Abstand
$\norm{\x_i-\x_j}$ kleiner als $\eps$ haben. Wir wollen nun zeigen, da"s die "uber
\eqnref{cdimdef} definierte Korrelationsdimension mit der generalisierten Dimension $D_2$ 
"ubereinstimmt\footnote{Dies soll kein formaler Beweis werden, nur eine Beweisskizze.}.

Die verallgemeinerte Dimension $D_2$ ist gegeben durch:
\eqnl[d2def]{\corrdim =  \lim_{\eps\to 0} \frac{\log\left(\sum_i \Prob_i^2\right)}{\log(\eps)}.} 
$\Prob_i$ ist die Wahrscheinlichkeit einen Attraktorpunkt in der $i$-ten Box der
gew"ahlten Partitionierung zu finden. $\Prob_i^2$ ist somit die Wahrscheinlichkeit zwei
beliebige, aber verschiedene Punkte gleichzeitig in der Box~$i$ anzutreffen:
\eqn{\Prob_i^2 =  \frac{1}{N(N-1)}\sum_{j,k} I_i(\x_j) I_i(\x_k).}
Hierbei ist $I_i$ die Indikatorfunktion der Box~$i$.  Die Summation von $\Prob_i^2$ "uber
alle Boxen ergibt nun die Wahrscheinlichkeit, irgend zwei Punkte gleichzeitig in einer
beliebigen Box anzutreffen.  Diese kann angen"ahert werden durch die Wahrscheinlichkeit,
zwei Punkte in einer Entfernung kleiner als der Boxdurchmesser anzutreffen.
Diese ist nun gegeben durch Verh"altnis aller Punktepaare mit einem
Abstand kleiner als $\eps$ zur Gesamtanzahl aller Punktepaare, d.h.
\eqn{\sum_i \Prob_i^2 \simeq \frac{1}{N(N-1)} \, \# \left\{ (i,j) \, \vert \, i \neq j \land  \norm{\x_i-\x_j}\leq \eps \right\} .} 
Die linke Seite der Gleichung kann aber wieder durch das Korrelationsintegral
\eqnref{cintdef} ausgedr"uckt werden. Es kann nun weiterhin gezeigt werden, da"s die
gemachten N"aherungen f"ur $\eps\to0$ verschwinden, und wir erhalten somit:
\eqn{D_2=D_C.} 
Die Korrelationsdimension ist also gleich der verallgemeinerten Dimension $D_2$ und dient
uns so als untere Absch"atzung f"ur die  Kapazit"atsdimension. 

Aufgrund der Unabh"angigkeit von $D_q$ von der speziellen Form der Partitionierung
(siehe \eqnref{gendim}) ist das Korrelationsintegral (im Grenzfall $\eps\to0$) unabh"angig von
der Wahl der Norm. F"ur die numerische Berechnung des Korrelationsintegrals ist es daher
aus Gr"unden der Geschwindigkeit sinnvoll, statt der euklidischen Norm $\norm{\cdot}_2$ die
Maximumsnorm $\norm{\cdot}_\infty$ zu verwenden. Da die Definition des
Korrelationsintegrals in \eqnref{cintdef} symmetrisch bez"uglich $i$ und $j$ ist, gen"ugt
die Berechnung desselben f"ur $i<j$:
\eqnl[cintdef2]{C(\eps) = \frac{2}{N(N-1)}\sum_{i<j}\Theta(\eps-\norm{\x_i-\x_j}_\infty).}



%%%%%%%%%%%%%%%%%%%%%%%%%%%%%%%%%%%%%%%%%%%%%%%%%%%%%%%%%%
\subsubsection{Numerische Berechnung des Korrelationsintegrals}
Bedeutung erlangt hat die Korrelationsdimension vor allem wegen der schnellen und
relativ genauen Berechenbarkeit des Korrelationsintegrals. 
Die Berechnung des Korrelationsintegrals geschieht nun in den folgenden Schritten.
\begin{enumerate}
\item Der minimale und der maximale Abstand $\rmin$ und $\rmax$ zweier Punkte der
Zeitreihe werden bestimmt. Unter Verwendung der Maximumsnorm kann f"ur den maximalen
Abstand $\rmax=\max\limits_{i,j}{\lvert x_i-x_j \rvert}$ gew"ahlt werden bzw.\  f"ur den
minimalen $\rmin=\min\limits_{i,j}{\lvert x_i-x_j \rvert}$. 
\item Der Bereich $[\rmin,\rmax[$ wird in $m$ Intervalle $[r_l,r_{l+1}[$ mit
$r_0=\rmin$ und $r_{m-1}=\rmax$ aufgeteilt. Die Aufteilung sollte logarithmisch (d.h.\
$r_{l+1}/r_l=\rho=\const$) erfolgen, um dem Skalierungsverhalten des Korrelationsintegrals
Rechnung zu tragen.
\item Den Intervallen wird ein Array $K$ zugeordnet, so da"s dem Intervall $[r_l,r_{l+1}[$
der Wert $K_l$ entspricht. Das Array $K$ dient dazu, bei der sp"ateren Kalkulation die
Anzahl der Punktepaare, deren Abstand in dem entsprechenden Intervall liegt, aufzunehmen.
\item F"ur alle $i<j$ wird der Abstand $r_{ij}=\norm{\x_i-\x_j}_\infty$
berechnet. Dasjenige $K_l$ mit $r_l\leq r_{ij}\lt r_{l+1}$ wird um eins erh"oht.

Die Berechnung des Index $l$ aus dem Abstand $r_{ij}$ ist einer der zeitkritischen Teile 
des Algorithmus. Der Index $l$ kann "uber $l=\lfloor log(r_{ij} / \rmax ) / log(\rho)
\rfloor + m$ berechnet werden. Da die Berechnung des Logarithmus sehr langsam ist, wird
hier folgenderma"sen verfahren. Statt des nat"urlichen Logarithmus wird der
Zweierlogarithmus benutzt. W"ahlt man nun das Abstandsverh"altnis $\rho$, so da"s
$\rho=2^{1/k}$ mit $k\in\N$ gilt, kann die obige Formel umgeschrieben werden zu $l=\lfloor
log_2\left( (r_{ij} / \rmax )^k \right) \rfloor + m$. Die Funktion $\lfloor
log_2(\cdot) \rfloor$ kann aufgrund der internen Zahlendarstellung von Computern sehr
schnell berechnet werden.
\item Aus den $K_l$ wird "uber $C(r_l)=\sum\limits_{m=0}^{l-1} K_m$ das Korrelationsintegral bestimmt. 
\end{enumerate}
Am effektivesten ist das Verfahren, wenn das Korrelationsintegral in einem Durchlauf f"ur verschiedene 
Einbettungsdimensionen $d=1\dots d_\tmax$ bestimmt wird. In diesem Fall mu"s das
eindimensionale Array $K$ durch ein zweidimensionales ersetzt werden, dessen zweite
Komponente die Einbettungsdimension spezifiziert. Schritt 4 ist folgenderma"sen
abzu"andern, da"s zuerst der Abstand $r_{ij,1}=\abs{\x_{i,1}-\x_{j,1}}$ f"ur die
Einbettungsdimension $d=1$ berechnet wird. Die folgenden Abst"ande folgen dann bei Verwendung
der Maximumsnorm aus $r_{ij,d}=\max(r_{ij,d-1},\abs{\x_{i,d}-\x_{j,d}})$. Der Index $l$
mu"s nur dann neu berechnet werden, wenn $r_{ij,d}\gt r_{ij,d-1}$ ist.

\epsfigsingle{corrint/perfect/corrint700b}
{Korrelationsintegral f"ur eine Zeitreihe ($N=7\times 10^5$)des R"ossler-Attraktor mit $\rho=2^{1/10}$,
$m=102$ und $d=1\dots 8$ (von oben nach unten). Die Abst"ande sind einheitlich auf
$\rmax=1$ skaliert worden. Aufgrund des Skalierungsverhaltens des
Korrelationsintegrals erfolgt die Darstellung doppelt logarithmisch.} 
{corrintperf}{-0.2cm}

\psref{corrintperf} zeigt eine Berechnung des Korrelationsintegrals f"ur eine aus dem
R"ossler-System \cite{Roessler76} gewonnenen Zeitreihe mit $7\times 10^5$ Punkten 
f"ur Einbettungsdimensionen $d=1\dots 8$. In der Abbildung sind deutlich zwei Bereiche zu
unterscheiden. F"ur $\log r>-1$ geht $\log C(r)$ gegen 0. Der Grund liegt einfach
darin, da"s der Attraktor nur einen begrenzten Raumbereich aufspannt und f"ur hinreichend
gro"ses $r$ alle Paare von Rekonstruktionspunkte  in der Summe von \eqnref{cintdef2}
erfa"st sind. Dieser Bereich nennt sich auch \begriff(S"attigungsbereich) des
Korrelationsintegrals. Den Abstand, ab dem S"attigung auftritt, bezeichnen wir mit
$\rsaett$. F"ur $\log r<-1$ zeigt das Korrelationsintegral das 
 erwartete Skalierungsverhalten $C(r)\propto r^\nu$, wobei die Konstante
$\nu$ f"ur unterschiedliche Einbettungsdimensionen verschiedene Werte annimmt. Da der
R"ossler-Attraktor die Korrelationsdimension $\corrdim\sim 2.07>2$ besitzt, skaliert das $C(r)$
f"ur Einbettungsdimensionen $d\leq 2<\corrdim$ mit $r^d$, da in diesem Fall der gesamte Phasenraum
aufgespannt wird. F"ur Einbettungsdimensionen $d>\corrdim$ skaliert das Korrelationsintegral
mit $r^{\corrdim}$. Dieser Bereich, auf dessen Bestimmung bei der Korrelationsanalyse das
Hauptaugenmerk liegt, hei"st \begriff(Skalierungsbereich)
\footnote{F"ur die Dimensionsberechnung ben"otigen wir also nur $d>\corrdim$. Die Bedingung
$d>2D_H$ (nach Takens Theorem) ist hier nicht notwendig zu erf"ullen, da es bei
Dimensionsberechnungen nicht auf die Eineindeutigkeit von $\diffeo$ ankommt. Dieses Resultat geht
auch hervor aus dem Projektionssatz f"ur fraktale Mengen \cite{Falconer93}.}.

\subsubsection{Berechnung der Korrelationsdimension aus dem Korrelationsintegral}
\paragraph{Ableitung des Korrelationsintegrals}
F"ur die numerische Berechnung der Korrelationsdimension ist die Definition
\eqnref{cdimdef} schlecht geeignet, da die Gleichheit nur im Grenzfall $r\to 0$ gilt. F"ur 
endliche $r$ gilt wegen $C(r)=k r^{\corrdim}:$
\eqn{\mcorrdim(r)=\corrdim + \frac{\log k}{\log r}.}
Aufgrund des zweiten Summanden ist die Konvergenz gegen $\corrdim$ logarithmisch langsam.
Besser geeignet ist die Berechnung von $\corrdim$ "uber die Ableitung, da hier der Einflu"s
des Proportionalit"atsfaktors wegf"allt:
\eqn{\mcorrdim(r)=\abl{\log C(r)}{\log r}=\frac{\abls{C(r)}{r}}{C(r)/r} .}
Numerisch bestimmt wird die Ableitung an einer Stelle $r_i$, indem durch $2k+1$
Nachbarpunkte\footnote{D.h. die Punkte $(\log r_{i-k}, \log C(r_{i-k}),\dots,(\log r_i,
\log C(r_i),\dots,(\log r_{i+k}, \log C(r_{i+k})$. } von $r_i$ eine Regressionsgerade
gelegt und deren Steigung ermittelt wird. Einen Graphen von $\mcorrdim(r)$ "uber $\log r$
zeigt \psref{corrslpperf}.

\epsfigsingle{corrint/perfect/corrslp700b}
{Ableitung des Logarithmus des Korrelationsintegrals aus \psref{corrintperf}. In die
Berechnung der Steigung wurden f"unf Nachbarpunkte (d.h. $k=2$) mit einbezogen. Der
Skalierungsbereich liegt ungef"ahr zwischen $\log(\rmin)=-5.7$ und $\log(\rmax)=-3.3$.}
{corrslpperf}{-0.2cm}
Die Berechnung der Korrelationsdimension "uber die Ableitung hat jedoch einige
Nachteile. Zum einen schwankt der Wert von $\mcorrdim(r)$ relativ stark f"ur
unterschiedliche $r$. Zum anderen wird ein gro"ser Teil Informationen, die das
Korrelationsintegral liefert \naja(verschwendet), da nur ein sehr begrenzter Teil f"ur
die Berechnung verwendet wird. \psref{corrslpperf} liefert trotzdem wichtige Informationen zur 
Berechnung der Korrelationsdimension. Aus der Abbildung 
kann gut abgelesen werden in welchem Bereich die Steigung des
Korrelationsintegrals konstant bleibt. Wir k"onnen hier"uber die Grenzen des
Skalierungsbereiches $\rsmin$ und $\rsmax$ bestimmen. 

\paragraph{Steigung und lineare Regression}
Eine verl"a"slichere Sch"atzung der Korrelationsdimension erhalten wir, indem wir die
Steigung "uber den gesamten Skalierungsbereich ermitteln. Als Grenzen des
Skalierungsbereiches k"onnen beispielsweise die im vorigen Abschnitt gewonnenen Werte
$\rsmin$ und $\rsmax$ genommen werden. F"ur die Korrelationsdimension ergibt sich dann:
\eqn{\mcorrdim=\frac{\log C(\rsmax)-\log C(\rsmin)}{\log \rsmax-\log(\rsmin)} .}
Die Ermittelung von $\rsmin$ und $\rsmax$ aus der Slopekurve geschieht im allgemeinen
manuell. H"aufig ist dies auch sinnvoll, da die Daten zuerst mit dem Auge begutachtet
werden sollten, bevor eine Aussage "uber die Korrelationsdimension einer Zeitreihe gemacht
wird. 

Die Steigung des Korrelationsintegrals im Bereich $I=[\rmin,\rmax[$ kann nat"urlich auch
durch Bestimmung einer Regressionsgeraden berechnet werden. Die Steigung dieser Geraden
ist gegeben durch
\eqn{\mcorrdim = \frac{\sum\limits_{i\in I} (\log C(r_i)-\overline{\log C(r)}(\log r_i
-\overline{\log r_i}))}{\sum\limits_{i\in I}(\log r_i-\overline{\log r})^2} ,}
wobei $\overline{\log C(r)}$ und $\overline{\log r_i}$ die Mittelwerte von $\log C(r)$
bzw.\ $\log r$ im betrachteten Intervall $I$ sind. Es mu"s jedoch angemerkt werden, da"s
eine der Voraussetzungen f"ur die Durchf"uhrung einer linearen Regression hier nicht
erf"ullt ist. Die Werte von $\log C(r)$ sind f"ur verschiedene Werte von $r$ nicht
voneinander unabh"angig. Zudem hat die Absch"atzung des Fehlers bei einer solchen Methode
kleinster Quadrate meist wenig mit dem wirklichen Fehler bei der Dimensionsberechnung zu
tun. Wir werden sp"ater eine Methode entwickeln, die diese Schwachstellen nicht teilt (siehe
Abschnitt \ref{chaptakensest}).

\paragraph{Korrelationskoeffizient}
F"ur die Bestimmung der Korrelationsdimension vieler Zeitreihen ist jedoch ein Verfahren
zur automatischen Bestimmung des Skalierungsbereiches angebracht. Verfahren dieser Art
sind vielf"altig entwickelt worden, wir wollen jedoch nur eines, welches hier auch
Anwendung gefunden hat, besprechen. Der Skalierungsbereich ist derjenige, in der die
Auftragung $\log C(r)$ "uber $\log r$ mit konstanter Steigung $\mcorrdim$
verl"auft. Es ist also Aufgabe des Verfahrens, aus dem $\log C(r)$-$\log r$-Graph den
\naja(geradesten) Teil herauszufinden. 

Daf"ur ist es notwendig, ein Ma"s daf"ur zu finden, wie \naja(gerade) der Graph in einem
bestimmten Bereich verl"auft \cite{Raidl}. Ein solches Ma"s ist der 
\begriff(Korrelationskoeffizient)\footnote{Der (\begriff(Pearsonsche))
\begriff(Korrelationskoeffizient) $r(X,Y)=\mathrm{Cov}(X,Y)/\sigma_X \sigma_Y$ beschreibt die G"ute 
der linearen Vorhersagbarkeit der Zufallsvariable $Y$ durch die Zufallsvariable $X$. Er ist 
eng gekoppelt mit dem Optimierungsproblem $Y$ m"oglichst gut (linear) aus $X$
vorherzusagen, d.h.\  den Vorhersagefehler $E(Y-(a+bX))^2$ zu minimieren. Es gilt
$\min(E(Y-(a+bX))^2) = \sigma_Y(1-r^2(X,Y))$.} 
$r(X,Y)$, wobei $X$ und $Y$ zwei beliebige Zufallsvariablen sein m"ogen. Er wird $1$ 
bzw.\  $-1$, wenn zwischen $X$ und $Y$ ein exakt linearer Zusammenhang
besteht. Bei schw"acheren Zusammenh"angen wird er betragsm"a"sig kleiner bzw.\  $0$, falls 
gar keiner besteht. Betrachten wir nun $\log C(r)$ und $\log r$ in einem Intervall
$I=[r_1,r_2[$ als Zufallsvariablen, so erhalten wir mit $L(I)=\abs{ r(\log r, \log
C)\rvert_{I}}$ ein Ma"s f"ur die Linearit"at des Korrelationsintegrals in diesem
Bereich:
\eqn{L(I) = \left| \frac{\frac{1}{n-1} \sum\limits_{r\in I}\left( \log r - \overline{\log
r}\right)\left( \log C(r) - \overline{\log C(r)} \right) } {\sqrt{\left[ \frac{1}{n-1}
\sum\limits_{r\in I}\left( \log r - \overline{\log r}\right) \right] \left[ \frac{1}{n-1}
\sum\limits_{r\in I}\left( \log C(r) - \overline{\log C(r)} \right) \right ]}} \right| .} 
Hierbei bezeichnen $\overline{\log r}$ und $\overline{\log C(r)}$ jeweils die
im Intervall $I$ gemittelten Gr"o"sen. Um den Skalierungsbereich zu finden wird nun $L(I)$ 
f"ur alle m"oglichen Intervalle berechnet, und dasjenige, welches das maximale $L$
liefert, als Skalierungsbereich benutzt. Hier sind jedoch zwei Einschr"ankungen zu
beachten. Zum einen sollte eine Mindestl"ange f"ur die Intervalle vorgegeben
sein. Ansonsten tendiert der Algorithmus dazu, sehr kleine Intervalle auszuw"ahlen. Zum
anderen m"ussen Unter- und Obergrenzen f"ur $r$ vorgegeben werden, da das
Korrelationsintegral im Bereich des Digitalisierungsrauschens und der S"attigung  auch
einen linearen Bereich (mit Steigung 0) besitzt. F"ur das Korrelationsintegral aus \psref{corrintperf} 
liefert das Verfahren die Werte $\log \rmin = -5,7$ und $\log \rmax = -3,6$. Diese stimmen gut 
mit denen "uberein, die man auch aus \psref{corrslpperf} absch"atzen w"urde.

\paragraph{Takens' Sch"atzer}
\label{chaptakensest}
Die Berechnung der Korrelationsdimension "uber eine Regressionsgerade durch ein auf
irgendeine Weise festgelegtes Intervall hat jedoch zwei Nachteile. Zum einen bleiben
Informationen "uber den Verlauf von $C(r)$ unterhalb von $\rmin$ ungenutzt. Gerade kleine
Werte von $r$ sollten jedoch nach \eqnref{cdimdef} Informationen "uber die
Korrelationsdimension liefern. Zum anderen wird nicht ber"ucksichtigt, da"s
aufeinanderfolgende Werte von $C(r)$ nicht voneinander unabh"angig sind. Einen Ausweg
hieraus bietet der sogenannte \begriff(Takens' Sch"atzer) (Takens' estimator) \cite{Takens85a}. 

Wir m"ussen hier einen wahrscheinlichkeitstheoretischen Ansatz f"ur das
Korrelationsintegral machen. Betrachten wir die Wahrscheinlichkeit $\Prob$, da"s zwei
Attraktorpunkte $\x_i$ und $\x_j$ einen Abstand $\rij$ kleiner als $r$ haben, so erhalten wir:
\eqn{\Prob(\norm{\x_i-\x_j}<r)=\frac{\sum\limits_{i\neq
      j}\Theta(r-\norm{\x_i-\x_j})}{N(N-1)}=C(r) .}

Die betrachtete Wahrscheinlichkeit ist also genau gleich dem Korrelationsintegral $C(r)$.
Da uns nur das Skalierungsverhalten unterhalb des S"attigungsbereiches interessiert,
legen wir eine Obergrenze $\rmax<\rsaett$ f"ur $r$ fest. Alle Punktepaare mit Abstand
$r\geq\rmax$ werden ignoriert und wir betrachten die bedingte Wahrscheinlichkeit $\tilde
\Prob$, da"s zwei dieser Punktepaare einen Abstand kleiner $r$ haben m"ogen:
\eqn{\tilde \Prob(\rij<r)=\Prob(\rij<r \vert \rij < \rmax) = \frac{C(r)}{C(\rmax)}   .}

Da $C(r)$ im Skalierungsbereich idealerweise wie $kr^\corrdim$ skaliert, verh"alt sich
$\tilde \Prob(\rij<r)$ idealerweise wie $(r/\rmax)^\corrdim$. 

Unsere Aufgabe ist nun den freien Parameter $D_2$ der Wahrscheinlichkeitsverteilung
$\tilde\Prob$ abzusch"atzen.  Dies geschieht in der Statistik "ublicherweise durch
Anwendung der \begriff(Maximum-Likelyhood-Regel).  Sie besagt, da"s als Sch"atzwert f"ur
einen unbekannten Parameter einer Wahrscheinlichkeitsverteilung ein solcher Wert des
Parameters verwendet wird, bei dessen Vorliegen der konkreten Stichprobe eine m"oglichst
gro"se Wahrscheinlichkeit zukommt.


Die vorliegende konkrete Stichprobe
besteht aus den gemessenen Abst"anden $r_{ij}<\rmax$. Aus Gr"unden der Vereinfachung
"andern wir hier die Indizierung, und betrachten die Folge $\folge(r,1,m)$, welche alle
$r_{ij}$ beinhalten soll.  Die Wahrscheinlichkeit $L$ eine solche Stichprobe zu erhalten,
ist das Produkt der Wahrscheinlichkeiten f"ur die Messung jedes einzelnen Abstands:
\eqn{L_\mathrm{disk}(\folge(r,1,m);D_2) = \prod_{i=1}^m\tilde\Prob(r=r_i) .}
Die so definierte \begriff(Likelyhood-Funktion) ist jedoch nur f"ur diskrete
Wahrscheinlichkeitsverteilungen verwendbar.  Da die vorliegende Verteilung kontinuierlich
ist, ist die Wahrscheinlichkeit f"ur die exakte Gleichheit $\tilde\Prob(r=r_i)$ null, und
somit ist auch die Likelyhood-Funktion identisch null. Es kann nun gezeigt werden, da"s f"ur
kontinuierliche Verteilungen die Wahrscheinlichkeitsdichte ma"sgeblich ist\footnote{Man
  betrachtet statt den Wahrscheinlichkeiten $\Prob(r=r_i)$ die Wahrscheinlichkeiten $\Prob(r_i\leq r <r_i+
  dr_i) = f(r_i)dr_i$. Das Maximum der Likelyhood-Funktion ist unabh"angig von
  der Wahl der $dr_i$, so da"s direkt die Wahrscheinlichkeitsdichte $f$ benutzt werden kann.  }.
Wir erhalten als Likelyhood-Funktion somit:
\eqnl[maxlike1]{L(\folge(r,1,m);D_2) = \prod_{i=1}^m f(r_{i}) ,}
wobei die Wahrscheinlichkeitsdichte durch:
\eqn{f(r) = \abl{\tilde \Prob}{r} =   \corrdim  (r/\rmax)^{\corrdim-1}  }
gegeben ist.  Der Sch"atzwert f"ur $D_2$ ist nun derjenige, f"ur den $L$ ein Maximum
annimmt. 

Da die Ableitung des Produkts in \eqnref{maxlike1} sowie die Berechnung derer
Nullstellen recht kompliziert ist, verwendet man einen \naja(Standardtrick). Da der
Logarithmus eine streng monoton wachsende Funktion ist, besitzen $L$ und $\ln
L$ die gleichen Extremstellen. Das Produkt geht hierbei in eine Summation "uber, und die
Berechnung des Maximums vereinfacht sich erheblich:
\eqna{\abl{\ln L(\corrdim)}{\corrdim}&=& \abl{}{\corrdim}\left( m\ln\corrdim + (\corrdim-1)\sum_{i=1}^m\ln (r_i/\rmax) \right)\\
&=& \frac{m}{\corrdim}+\sum_{i=1}^m\log (r_i/\rmax).}


\comment{
Diese Aussage ist folgenderma"sen zu verstehen. Die erzeugte Verteilung hat in Bezug auf
die theoretische Verteilung eine bestimmte Wahrscheinlichkeit $L$, die vom Parameter
$\corrdim$ abh"angt. Wir bestimmen $\corrdim$ nun so, da"s $L(\corrdim)$ ein Maximum
annimmt. Welche Wahrscheinlichkeit hat nun die gemessene Verteilung? Zun"achst einmal
ordnen wir die $\rij$ so an, da"s wir einen Folge $r_1,\dots,r_m$ erhalten. Die
Wahrscheinlichkeit genau diese Folge zu messen, ist offensichtlich gleich:
\eqn{\Prob(r=r_1)\cdot \Prob(r=r_2)\cdot\dots\cdot \Prob(r=r_m).}
Diese ist jedoch 0, da wir es mit einer kontinuierlichen Verteilung zu tun haben und
somit die Einzelwahrscheinlichkeiten 0 sind. Wir suchen daher die Wahrscheinlichkeit
die Verteilung in den Intervallen $[r_i,r_i+\d r_i[$ gemessen zu haben, wobei die $\d r_i$ 
beliebig sind.\comment{ -- die Intervalle sollten sich jedoch nicht "uberschneiden.}  Wir erhalten
damit f"ur die gesuchte Wahrscheinlichkeit
\eqn{L(\corrdim)=\prod_{i=1}^m \Prob(r_i\leq r < r_i + \d r_i)}
$L(\corrdim)$ ist die sogenannte \begriff(Maximum-Likelyhood-Funktion).
Bei der theoretischen Verteilung $\Prob(r<\eps)=k\eps^\corrdim$ erhalten wir 
$\Prob(\eps<r<\eps+\d\eps)=k\corrdim\eps^{\corrdim-1}\d\eps$ und somit
\eqn{L(\corrdim)=\prod_{i=1}^m k \corrdim  r_i^{\corrdim-1}\d r_i}
Um das Maximum der Maximum-Likelyhood-Funktion zu finden, logarithmieren wir die obige
Gleichung und leiten nach $\corrdim$ ab
\eqna{\abl{\log L(\corrdim)}{\corrdim}&=& \abl{}{\corrdim}\left( m\log\corrdim + (\corrdim-1)\sum_{i=1}^m\log r_i +
\log k + \sum_{i=1}^m\d r_i\right)\nonumber\\
&=& \frac{m}{\corrdim}+\sum_{i=1}^m\log r_i}

Hier wird nun auch ersichtlich warum die spezielle Wahl der $\d r_i$ beliebig
war. 
}
Nullsetzen der Gleichung ergibt den Sch"atzwert f"ur $\corrdim$:
\eqn{\corrdim=-\frac{m}{\sum_{i=1}^m \ln(r_i/\rmax)}=-\frac{1}{<\ln (r_i/\rmax)>} .}

Dies ist der sogenannte \begriff(Takens' Sch"atzer). \autor(Takens) zeigte, da"s
diese Sch"atzung erwartungstreu ist, was bei Maximum-Likelyhood-Methoden
nicht immer der Fall ist\footnote{F"ur eine \begriff(erwartungstreue) Sch"atzung ist
  der Erwartungswert des Sch"atzwerts gleich dem zu sch"atzenden Wert. Beispielsweise ist
  die Sch"atzung $s^2=\frac1{N-1}\sum_{i=1}^N(x_i-\bar x)^2$ eine erwartungstreue
  Sch"atzung f"ur die Varianz $\sigma^2$ der Stichprobe $(\folge(x,1,N))$, da
  $<s^2>=\sigma^2$ gilt. Die "uber die Maximum-Likelyhood-Methode gewonnene Sch"atzung
  ${s^*}^2=\frac1{N}\sum_{i=1}^N(x_i-\bar x)^2$ ist nur \begriff(asymptotisch
  erwartungstreu), da nur f"ur den Grenzwert $\lim_{N\to\infty}{s^*}^2> = \sigma^2$ gilt.
  Maximum-Likelyhood-Sch"atzungen sind immer mindestens asymptotisch erwartungstreu. }.
Weiterhin ist diese Sch"atzung die \begriff(wirksamste), da die Varianz des
Sch"atzwertes f"ur diese Sch"atzung ein Minimum annimmt. 

Da das Korrelationsintegral, wie oben gesehen eine Wahrscheinlichkeitsverteilung f"ur die
$r_i$, darstellt, kann der Erwartungswert $<\ln (r_i/\rmax)>$ jetzt "uber $C(r)$ ausgedr"uckt
werden. Die Wahrscheinlichkeit, bei der gemessenen Verteilung einen Wert zwischen $r$ und
$r+\d r$ zu messen, betr"agt $\frac{\d  C(r)}{\d r}\frac{\d r}{C(\rmax)}$. Wir erhalten
also f"ur ein gegebenes $\rmax$:
\eqn{\corrdim(\rmax) = \int\limits_0^\rmax \abl{C(r)}{r}\frac{\ln (r/\rmax)}{C(\rmax)}\d r .}
Da wir das Korrelationsintegral jedoch nur f"ur diskrete Werte $r_i$ bestimmen k"onnen,
geht die obige Gleichung "uber in:
\eqn{\corrdim(r_n) = \sum\limits_{i=1}^{n-1} \frac{C(r_i+1)-C(r_i)}{C(r_n)}\ln(r_i/r_n), }
wobei $r_n=\rmax$ gesetzt wurde.
Ein Vergleich dieser Methode mit der im vorigen Abschnitt beschriebenen zeigt, da"s der
Takens' Sch"atzer bei bekannten Systemen im allgemeinen bessere Werte liefert (siehe \psref{corrdimcomp}).
%\inkorrektur(Fehlerabsch"atzung)

\epsfigdouble{corrint/perfect/corrdim700b}{corrint/perfect/takdim700b} { Vergleich der
  Bestimmung der Korrelationsdimension $\corrdim$ in Abh"angigkeit von der
  Einbettungsdimension $\embed$ durch Regressionsgeraden (links) bzw.\ Takens' Sch"atzer
  (rechts) f"ur das Korrelationsintegral aus \psref{corrintperf}. Die "uber Takens'
  Sch"atzer bestimmten Werte der Korrelationsdimension ($\corrdim=2,07\pm0.01$) $\corrdim$
  stimmen deutlich besser mit den theoretischen Werten "uberein.  }  {corrdimcomp}{-0.2cm}

\subsubsection{Fehler bei der Korrelationsanalyse}
Das Korrelationsintegral in \psref{corrintperf} ist nahezu optimal im Sinne des erwarteten 
theoretischen Verlaufs. Der Skalierungsbereich 
erstreckt sich "uber einige Gr"o"senordnungen ($\rmax/\rmin\simeq 54)$ und die Steigung innerhalb
dieses Bereichs ist nahezu konstant (siehe \psref{corrslpperf}). Der Grund liegt vor allem
darin, da"s f"ur die Berechnung eine gro"se Anzahl Datenpunkte ($7\times10^5$) vorlag und
mit nahezu rauschfreien Daten gearbeitet wurde. In experimentellen Situationen liegt
beides meist nicht vor. Die Fehler, die hieraus (und auch aus anderen Quellen) resultieren,
sollen im folgenden diskutiert werden.




\paragraph{Endliche Datenmenge}
Ein Faktor, der die verl"a"sliche Bestimmung der Korrelationdimension wesentlich
beeinflu"st, ist die Menge der verf"ugbaren Daten. Das Korrelationsintegral hat einen
Wertebereich von $2/N(N-1)$ bis $1$. Da $C(r)$ f"ur $r<\rmax$ mit $(r/\rmax)^\corrdim$ skaliert, tragen
bei Abst"anden der Gr"o"senordnung $\rmax(2/N^2)^{1/\corrdim}$ nur noch wenige Punktepaare 
zum Korrelationsintegral bei. Daraus resultieren in diesem Bereich starke statistische
Schwankungen von $\mcorrdim(r)$. Aus diesem Verhalten leiteten \autor(Eckmann und Ruelle)
\cite{Eckmann-ruelle2} eine Gleichung f"ur die minimal erforderliche Datenmenge f"ur
Dimensionsberechnungen ab. Bezeichnet man das Verh"altnis von gr"o"sten zu kleinsten
Skalen mit $\rho$, so ergibt sich f"ur die minimale Datenmenge:
\eqnl[corrdimnminer]{N_\tmin=\sqrt{2\rho^\corrdim}}
oder andererseits bei fester Datenmenge die maximal berechenbare Dimension:
\eqnl[corrmaxdim]{D_{2,\tmax}=\frac{2\log N}{\log \rho}.}
Nun mu"s f"ur eine vern"unftige Absch"atzung der Dimension das Skalenverh"altnis $\rho$
hinreichend gro"s sein. Eckmann und Ruelle geben hier ein minimales Verh"altnis von
$\rho_\tmin=10$ an, so da"s man etwa bei $N=1000$ Datenpunkten maximal eine
Korrelationsdimension $\corrdim\leq 6$ sinnvoll bestimmen kann. Diese Grenze scheint mir
allerdings sehr hoch, da sich bei $N=1000$ schon die Dimensionen von Attraktoren mit
$\corrdim\simeq 2$ nur sehr schlecht bestimmen lassen. Andererseits kann das Ergebnis gut
als wirkliche obere Grenze f"ur die Dimensionsberechnung angesehen werden. 




\paragraph{Kanten und endliche Ausdehnung des Attraktors}
Wie wir bereits in \psref{corrintperf} gesehen haben, geht das Korrelationsintegral ab einem
bestimmten Wert $\rsaett$ in einen S"attigungsbereich "uber. Dies ist eine Konsequenz der
endlichen Ausdehnung des Attraktors. Effekte, die aus dem S"attigungsverhalten resultieren, 
sind schon auf L"angen\-skalen weit unterhalb der linearen Ausdehnung des Attraktors
erkennbar. Sobald eine wesentliche Anzahl Punkte einen Abstand von weniger als $r$ vom
\naja(Rand) des Attraktors hat, weicht das Skalierungsverhalten des Korrelationsintegrals
stark vom theoretischen Verlauf ab.

Der Wert $r_s$, f"ur den das Skalierungsverhalten durch den \begriff(Kanteneffekt)
(engl.: edge effect) abbricht, h"angt sehr von
der Dimension und der Geometrie des Attraktors ab. 
F"ur einen $m$-dimensionalen Hyperkubus der Kantenl"ange 1 kann das 
Korrelationsintegral jedoch exakt berechnet werden\footnotemark:
\footnotetext{Wie bereits gezeigt wurde gilt f"ur das Korrelationsintegral
$C(r)=\Prob(\rij<r)$. Bei einem homogenen Hyperkubus sind die einzelnen Komponenten des
Abstandsvektors voneinander statistisch unabh"angig und es gilt bei Verwendung der
Maximumsnorm $\Prob(\rij)=\Prob((\vec r_{ij,1} < r) \land\dots\land (\vec r_{ij,m} < r)) = \Prob_1(\vec
r_{ij,1} < r)\cdot\dots\cdot \Prob_1(\vec r_{ij,m} < r)$. Da au"serdem die Verteilungen der
einzelnen Komponenten gleich sind folgt $\Prob(\rij)=\Prob_1(\vec r_{ij,1} < r)^m$. F"ur die
Verteilung $\Prob_1$ gilt nun $\Prob_1(\rij<r)=\int_0^1 \{\min(1,r'+r)-\max(0,r'-r)\}\d r'=
2r-r^2$. Somit folgt die Behauptung.
}
\eqnl[corrinthyper]{C(r)=(2r-r^2)^m.}
Unter der vereinfachenden Annahme, da"s sich der Attraktor "ahnlich einem Hyperkubus
der Dimension $\corrdim$ und Kantenl"ange $\rmax$ verh"alt, kann die Korrelationsdimension entsprechend
\eqnref{corrinthyper} durch Auftragung von $\log C(r)$ "uber $\log(2r/\rmax-(r(\rmax)^2)$
abgesch"atzt werden. Die sich hieraus ergebenden Korrekturen  
sind jedoch "au"serst gering (wenige Promille).

Die Eigenschaft des Korrelationsintegrals ab einer bestimmten Obergrenze in S"attigung zu
gehen wurde von \autor(Nerenberg und Essex) benutzt, um eine Untergrenze der
ben"otigten Datenmenge zur Berechnung der Korrelationsdimension seltsamer Attraktoren
 zu bestimmen \cite{Nerenberg-essex}. Ihre Berechnung beruht darauf, da"s der Skalierungsbereich
von unten durch die Menge der verf"ugbaren Daten beschr"ankt wird. Kennzeichnend hierf"ur
ist der charakteristische Abstand n"achster Nachbarn $r_n$. Von oben ist der Skalierungsbereich
beschr"ankt durch den durchschnittlichen Abstand der Punkte zum Rand des Attraktors
$r_s$. Fallen beide zusammen, verschwindet der Skalierungsbereich und eine
Dimensionsbestimmung ist nicht mehr m"oglich. Beide Werte werden nun wiederum f"ur einen
homogenen Hyperkubus der Dimension $m$ und Kantenl"ange 1 berechnet\footnotemark:
\eqna{r_n &=& \frac{1}{2(m+1)}\nonumber\\
r_s &=& N^{-1/m} .}
Gleichsetzen der beiden Grenzen $r_n$ und $r_s$ ergibt:
\eqnl[corrnminnea]{N_\tmin(m)=\{2(m+1)\}^m .}
Zur Berechnung der Dimension eines Attraktors der Korrelationsdimension $\corrdim$ sind nach
dieser Absch"atzung $N_\tmin(\lceil \corrdim \rceil)$ Datenpunkte n"otig\footnote{$\lceil x \rceil$
bedeutet hier, die n"achste ganze Zahl, gr"o"ser oder gleich $x$.}. Zum Vergleich mit der
weiter oben angegebenen Formel von \autor(Eckmann und Ruelle) setzen wir das Verh"altnis
von $r_s$ zu $r_n$ gleich $\rho$. Damit erhalten wir:
\eqnl[corrnminneb]{N_\tmin(m,\rho)=\{2(m+1)\rho\}^m.}
Dies ergibt beispielsweise f"ur Attraktoren der Dimension 2 eine minimale Datenmenge von
ca.\  4000 Punkten bzw.\  500000 Punkte f"ur die Dimension 3. Diese Absch"atzung deckt
sich eher mit meinen Erfahrungen bei Dimensionsberechnungen. 





\paragraph{Wei"ses und Digitalisierungsrauschen}
Der Skalierungsbereich des Korrelationsintegrals ist nach unten, au"ser durch die Menge der 
verf"ugbaren Daten, durch Rauschen beschr"ankt. Die Auswirkungen beider Arten von
Rauschen -- wei"ses und Digitalisierungsrauschen -- m"ussen zun"achst getrennt behandelt
werden. 

Wir betrachten dem Signal "uberlagertes, wei"ses Rauschen der St"arke $\xi$. F"ur Abst"ande 
$r>\xi$ spielt das Rauschen nur eine untergeordnete Rolle. Die Abst"ande zwischen den
Attraktorpunkten werden nicht wesentlich beeinflu"st und das Korrelationsintegral skaliert 
wie bei rauschfreien Daten. 
Im Bereich $r<\xi$ wird das Signal jedoch stark vom Rauschen dominiert. Die
Wahrscheinlichkeit in der Umgebung eines Attraktorpunktes einen weiteren Punkt zu finden
verh"alt sich also wie bei einem reinen Rauschsignal. Da Rauschen jedoch immer den ganzen
Phasenraum aufspannt, skaliert das Korrelationsintegral hier mit der Einbettungsdimension
$d$, d.h.\  $C(r)\propto r^d$. Dieser Effekt ist dargestellt in \psref{corrnoise} links.

Bei Digitalisierungsrauschen verh"alt sich die Sache etwas anders. Sei $\eps$ der
Diskretisierungslevel. Dann ist jeder Me"swert und somit auch jeder Abstand zwischen
Attraktorpunkten  ein ganzzahliges Vielfaches von $\eps$ \footnotemark. Das Korrelationsintegral
verl"auft daher nicht kontinuierlich sondern in Spr"ungen bei jedem $r=n\eps$ ($n\in\N$)
(siehe \psref{corrnoise} rechts). 
Zus"atzlich werden verschiedene Attraktorpunkte auf ein und denselben Punkt
abgebildet, so da"s Punkte mit $\rij=0$ mit endlicher Wahrscheinlichkeit auftreten. Ein
Algorithmus zur Berechnung der Korrelationsdimension mu"s mit diesen Punkten umgehen
k"onnen.
\footnotetext{Vorausgesetzt der Abstand wird "uber die Maximums- oder die Absolutnorm
bestimmt.}

Ein von \autor(Theiler) hergeleitetes Modell ergibt f"ur das Skalierungsverhalten des
Korrelationsintegrals eines $m$-dimensionalen Gitters ergibt $C(r)\propto(r+\eps/2)^m$. Er 
schl"agt daher vor bei einer diskretisierten Zeitreihe $\log C(r)$ gegen $\log(r+\eps/2)$
aufzutragen\footnotemark. Diese Korrektur hat, nach meiner Erfahrung, jedoch wenig
Einflu"s auf tats"achliche Dimensionsbestimmungen, sondern dient mehr dazu Singularit"aten 
bei der Berechnung von $\log r$ zu vermeiden.
\footnotetext{Sei $r_\eps$ das n"achste Vielfache von $\eps$ kleiner als $r$. Dann wird
$r$ mit 50 prozentiger Wahrscheinlichkeit auf $r_\eps$ bzw auf $r_\eps+\eps$ abgebildet. F"ur 
das Korrelationsintegral der diskretisierten Daten $C'(r)$ folgt daraus
$C'(r)=\{C(r_\eps)+C(r_\eps+\eps)\}/2$. N"aherung und ausnutzung des urspr"unglichen
Skalierungsverhaltens f"uhrt auf den oben angegebenen Term.}

Um Rauschen zu vermindern k"onnen verschiedene Filtertechniken angewandt werden. Die
einfachsten hiervon sind Tief- oder Bandpa"sfilter. Diese haben jedoch den Nachteil, da"s
ihre Anwendung dem System einen Freiheitsgrad hinzuf"ugt und somit auch die gemessene
Dimension erh"ohen kann (siehe Abschnitt \ref{chapcorrdimfiltered}). Zudem ist im
allgemeinen nicht direkt ersichtlich ab welcher Oberfrequenz gefiltert werden sollte. Das
Abschneiden aller Frequenzen oberhalb einer bestimmten Grenzfrequenz birgt auch die
Gefahr, da"s eventuell f"ur die Dynamik wesentliche Informationen verloren gehen.

Das Verfahren, welches hier zum Einsatz kommt beruht auf der in Abschnitt \korrektur(SVD)
beschriebenen Methode der Singular Value Decomposition. Hierbei verwenden wir allerding
nur die erste Komponente der Rekonstruktion, welche "uber die
Verz"ogerungskoordinatenabbildung dann wieder eingebettet wird (\begriff(Reembedding)).
Das Ergebnis des Verfahrens sieht man in \psref{corrfilter}. Die oben erw"ahnten Nachteile 
anderer Verfahren teilt diese Methode nicht, da sie sich automatisch dem System anpa"st
und das Problem der Dimensionserh"ohung hier prinzipiell nicht auftreten kann.

\epsfigdouble{corrint/errors/noise/corrint02b}{corrint/errors/discrete/corrint05b}
{Links das Korrelationsintegral zu einer Zeitreihe des Lorenz-Systems dem wei"ses
Rauschen "uberlagert ist. Die Varianz des Rauschsignals betr"agt 0,2 Prozent der Varianz des
Originalsignals. Rechts wurde dasselbe Signal diskretisiert in Schritten von 0,5 Prozent
der Varianz des Signals.
}{corrnoise}{-0.2cm}

\epsfigfour{corrint/errors/noise/svd/corrint10}{corrint/errors/noise/svd/corrint10sf}
{corrint/errors/noise/svd/corrslp10}{corrint/errors/noise/svd/corrslp10sf}
{Links das Korrelationsintegral und die Slopekurve des ungefilterten Signals aus
\psref{corrnoise}. Rechts die entsprechenden Kurven f"ur das SVD-gefilterte
Signal. Deutlich zu erkennen ist der gr"o"sere Skalierungsbereich im rechten Bild.
}{corrfilter}{-0.2cm}

\epsfigdouble{corrint/errors/discrete/corrint05b}{corrint/errors/discrete/corrint5bsf}
{Links das Korrelationsintegral der diskretisierten Zeitreihe aus \psref{corrnoise}. Auf
  der rechten Seite das Korrelationsintegral des SVD-gefilterten Signals. Die Aufweitung des
  Skalierungsbereiches ist betr"achtlich.
}{corrdiscrete}{-0.2cm}

\paragraph{Gefilterte Zeitreihen}
\label{chapcorrdimfiltered}
Wie im vorigen Abschnitt bereits erw"ahnt kann die Filterung von Zeitreihen "uber einfache 
\begriff(Tief-) oder \begriff(Bandpa"sfilter) die Dimension des Systems erh"ohen \cite{Badii-politi}. Wir wollen dazu einen
einfachen Tiefpa"s erster Ordnung betrachten. Sei $x$ das Originalsignal und $z$ das
gefilterte Signal. Dann kann die Zeitabh"angigkeit des gefilterten Signals durch die
folgende Differentialgleichung beschrieben werden:
\eqnl[corrlowpass]{\dot z(t) = -\eta z(t) + x(t).}
Hierbei ist $\eta$ die Grenzfrequenz des Filters. Offensichtlich wird die Zahl der
Freiheitsgrade des Systems durch Anwendung dieses Filters um eins erh"oht. Da"s durch den
Filter auch die Dimension des Systems erh"oht werden kann, zeigten \autor(Badii \etal)
(1987). Die Gleichung f"ugt dem System einen neuen Lyapunov-Exponenten
$\lambda_f=-\eta$ hinzu. Dies f"uhrt je nach Gr"o"se von $\lambda_f$ in Bezug auf die
anderen Lyapunov-Exponenten des Systems zu einer Erh"ohung der Lyapunov-Dimension
$D_L$. Unter Annahme der G"ultigkeit der  Kaplan-Yorke Vermutung $D_L=D_1$ f"uhrt dies
auch zu einer Zunahme der Informationsdimension $D_1$. Da"s auch die weiteren
verallgemeinerten Dimensionen erh"oht werden, kann hier nur vermutet werden, ist jedoch
wahrscheinlich.

Um dem entgegen zu wirken, ist von \autor(Chennaoui \etal) ein Verfahren entwickelt worden 
um den unbekannten Filterparameter $\eta$ zu bestimmen \cite{Chennaoui}. Durch Inversion der Gleichung
\eqnref{corrlowpass} kann dann, mit bekanntem $\eta$, die originale Zeitreihe wieder
extrahiert werden. Das Verfahren beruht jedoch auf der Berechnung der
Informationsdimension und funktioniert auch nur f"ur Filter der oben beschriebenen
Art. F"ur \begriff(akausale) Filter\footnote{Akausale Filter sind Filter, deren
Sprungantwort $h(t)$ bereits f"ur $t<0$ ungleich 0 ist. Der \begriff(ideale) Tiefpa"s
ist ein Beispiel f"ur solch ein Filter. Akausale Filter sind zwar in Echtzeit nicht zu
realisieren, ihrer Anwendung auf komplett vorliegende Zeitreihen steht jedoch nichts
entgegen.} konnte \autor(Mitschke) (1989) allerdings zeigen, da"s dieses Problem nicht 
auftaucht \cite{Mitschke}.

\paragraph{Autokorrelation}
\label{corrdimtheiler}
Aus deterministischen Systemen gewonnene Signale sind grunds"atzlich autokorreliert. Es
existiert eine Autokorrelationszeit $\tau_\ac$, so da"s f"ur Zeiten $\tau<\tau_\ac$ sind
$x(t)$ und $x(t+\tau)$ stark miteinander korreliert. Falls die Autokorrelationszeit gro"s 
gegen sie Sampling Time $\sample$ ist, kann im Korrelationsintegral eine anormale Stufe
auftreten. Nach einem Vorschlag von \autor(Theiler) l"a"st sich dies vermeiden, indem das
Korrelationsintegral nur "uber Punktepaare  gebildet wird, die zeitlich mindestens
$W\sample>\tau_\ac$ auseinanderliegen \cite{Theiler}. Das so korrigierte Korrelationsintegral lautet dann:
\eqnl[cintdefac]{C(\eps) = \frac{2}{(N-W+1)(N-W)}\sum_{i<=j-W}\Theta(\eps-\norm{\x_i-\x_j}_\infty).}
F"ur $W=1$ geht dies wieder in die urspr"ungliche Form von \eqnref{cintdef2} "uber. Bei
den hier untersuchten Zeitreihen trat dieser Effekt niemals deutlich auf. Da diese
Korrektur jedoch die Anzahl, der in die Berechnung eingehenden Punktepaare, nicht
wesentlich herabsetzt, wurde sie in allen Berechnungen des Korrelationsintegrals zur
Sicherheit vorgenommen.

\paragraph{Lakunarit"at}
\korrektur(Broggi)
Die fraktale Struktur einer Menge kann au"ser durch ihre Dimension auch durch die
sogenannte \begriff(Lakunarit"at) (lat.\  lacuna = Loch, H"ohle), einem von
\autor(B. Mandelbrot) \cite{Mandelbrot82} gepr"agten Begriff, charakterisiert werden. 
Bei Fraktalen gleicher Dimension ist dasjenige mit der h"oheren Lakunarit"at st"arker
texturiert und erscheint fraktal-"ahnlicher. Ein Beispiel f"ur zwei \begriff(Cantor-Mengen)
unterschiedlicher Lakunarit"at zeigt \psref{lacunarity}.
%
\epsfigdouble{corrint/errors/lacunarity/sevena}{corrint/errors/lacunarity/sevenb}
{Die ersten f"unf Konstruktionschritte zweier Cantor-Mengen. In jedem Schritt werden die
Intervalle in sieben gleich gro"se Teilintervalle unterteilt von denen im linken jeweils
das dritte, vierte und f"unfte und im rechten das zweite, vierte und sechste Teilintervall
gestrichen wird. Die Kapazit"at der entstehenden fraktalen Mengen ist 
gleich $(D^l_0=D^r_0=\ln4/\ln 7)$. Von beiden Fraktalen hat das linke jedoch eine h"ohere Lakunarit"at.
}{lacunarity}{-0.2cm}

Die Lakunarit"at eines Fraktals hat Auswirkungen auf die Dimensionsbestimmung. Die
Relation $C(r)\propto r^\corrdim$ ist hier falsch, da $C(r)$ eine stufenweise Funktion
(\begriff(Teufelstreppe)) ist \cite{Broggi88}. 
Die Auswirkungen sind Oszillationen im Korrelationsintegrals, d.h.\  $C(r)$ ist nun proportional zu
$k(r)r^\corrdim$, wobei $k(r)$ im allgemeinen periodisch in $\log r$ ist
(siehe \psref{corrintlac}). Der Einflu"s der  Lakunarit"at kann minimiert werden, wenn bei
der Dimensionbestimmung "uber volle Perioden von $k(r)$ gemittelt wird.
\autor(Grassberger) zeigte andererseits, da"s die Oszillationen im Grenzfall $r\to 0$
ausged"ampft werden\cite{Grassberger88}.
%
\epsfigfour{corrint/errors/lacunarity/canacint}{corrint/errors/lacunarity/canbcint}
{corrint/errors/lacunarity/canacslp}{corrint/errors/lacunarity/canbcslp}
{Korrelationsintegrale und Steigungskurven f"ur die Cantor-Mengen aus \psref{lacunarity}
(Die linke Menge ist durch \gpmarkb, die rechte durch \gpmarkd{} gekennzeichnet.). Die
Korrelationsdimension ergibt sich zu $\mcorrdim=0,710$ bzw.\  $\mcorrdim=0,718$, was sehr gut mit dem
theoretischen Wert $\corrdim=\ln 4 /\ln 7\simeq 0,712$ "ubereinstimmt\footnotemark.
}{corrintlac}{-0.2cm}
\footnotetext{Da es sich hier
um homogene Fraktale handelt gilt $D_q=\const$ also auch $\corrdim=D_0$.}.

\comment{
\epsfigsingle{bla}
{Alle Gnuplot Symbole \gpmarka, \gpmarkb, \gpmarkc, \gpmarkd, \gpmarke, \gpmarkf, \gpmarkg, \gpmarkh
}{gpsymbols}{-0.2cm}
}




\clearpage
\section{Tests}
Eine der Fragen, die sich bei der Rekonstruktion von Attraktoren aus experimentellen
Zeitreihen stellte, war ob es sich hierbei wirklich um ein deterministisches System
handelt.  Man k"onnte hier m"oglicherweise versuchen, "uber die Dimension des
rekonstruierten Attraktors zu argumentieren. Die Dimension eines deterministischen
Systems hat immer einen endlichen Wert. Dagegen spannen die Rekonstruktionen
stochastischer Signale immer den ganzen Phasenraum auf. Die berechnete Dimension
konvergiert nicht mit steigender Einbettungsdimension. K"onnen also "uber die
Dimensionsberechnungen deterministische von stochastischen Systemen unterschieden werden?

Die Antwort ist leider \naja(Nein). Wie \autor(A. R. Osborne) und \autor(A. Provencale)
nachwiesen, k"onnen auch stochastische Systeme mit Leistungsspektren
$P(\omega)\propto\omega^{-\alpha}$, gegen eine endliche Dimension
konvergieren\cite{Osborne89a}. F"ur $1<\alpha<3$ erhielten sie die Korrelationsdimension $D_2=2/(\alpha-1)$.  Dies
liegt an zeitlichen Korrelationen aufeinanderfolgender Werte in der Zeitreihe. Diese
k"onnen zwar durch die Methode von \autor(Theiler) (s. Abschnitt \ref{corrdimtheiler})
vermieden werden, andererseits existieren noch andere Effekte, die eine Konvergenz der
Korrelationsdimension bei wei"sem oder farbigem Rauschen bewirken k"onnen \cite{Kennel92b}.

\comment{
  \epsfigfour{surrogate/noise/noise}{surrogate/noise/fourier}{surrogate/noise/corrint}{surrogate/noise/corrdim}
  {Links oben Zeitreihe $1/f^2$ Rauschen. Rechts oben Fourierspektrum. Links unten
    Korrelationsintegral. Rechts unten Korrelationsdimension, s"attigt bei ca.\ $5,0\dots
    5,5$ }{einsfnoise}{-0.2cm} }

\subsection{Statitische Hypothesentests}
Es sind noch weitere M"oglichkeiten vorgeschlagen worden, Zeitreihen hinsichtlich eines
zugrundeliegenden deterministischen Systems zu untersuchen (beispielsweise der
Determinismustest von \autor(Kaplan) und \autor(Glass) \cite{Kaplan-glass}).  Diese sind
jedoch in der Anwendung oft sehr beschr"ankt.  Die umfassendste und mathematisch
fundierteste M"oglichkeit diesem Problem zu begegnen, liegt im Feld statistischer
Hypothesentests.  Hypothesentests besitzen zugleich den Vorteil, sich nicht nur auf die
Frage nach einem zugrunde liegenden Determinismus zu beschr"anken.  Sie bieten ein
Ger"ust, um Fragen aller Art an die vorliegende Zeitreihen zu stellen, zum Beispiel
\begin{itemize}
\item Sind die Daten nicht-gau"sverteilt ?
\item Gibt es zeitliche Korrelationen in der Zeitreihe ?
\item Existiert eine nichtlineare Struktur ?
\item Sind die Daten durch eine chaotische Dynamik erzeugt ?
\end{itemize}
Um eine dieser Frage, im Rahmen eines Hypothesentests, zu beantworten, wird zuerst eine
\begriff(Nullhypothese) $\nullhyp$ aufgestellt, welche einer Verneinung eben dieser
entspricht.  Die Nullhypothese w"are beispielsweise im ersten Fall, {\em da"s} die Daten
gau"sverteilt sind.

Eine Nullhypothese kann weder bewiesen noch widerlegt werden. Man versucht hingegen,
die Nullhypothese abzulehnen, d.h. zu zeigen, da"s es unwahrscheinlich ist, da"s die Daten
mit der Hypothese in Einklang stehen. Eine Nullhypothese mu"s daher mit einem Modell
verkn"upft werden, welches beschreibt, wie die Daten erzeugt worden sein k"onnen.
Anders gesagt: der Nullhypothese $\nullhyp$ wird ein Proze"s bzw.\ eine Klasse von
Prozessen $\process$ zugeordnet, die mit $\nullhyp$ in Einklang stehen
(beispielsweise die Menge aller Prozesse, die gau"sverteilte Daten
erzeugen). Um die Nullhypothese abzulehnen, wird nun gezeigt werden, da"s die
Wahrscheinlichlichkeit, da"s die realen Daten durch einen Proze"s aus $\process$
erzeugt worden sind, sehr gering ist. 

Hierzu wird eine \begriff(Teststatistik) $T$ berechnet, wobei $T$ irgendeine skalare
Funktion der Daten ist.  F"ur Prozesse aus $\process$ kann man erwarten, da"s die Werte
innerhalb eines bestimmten Bereichs liegen. Dieser Bereich hei"st \begriff(Annahme-) oder
\begriff(Akzeptanzbereichs) der Nullhypothese.  Liegt der $T$-Wert der realen Daten
au"serhalb des Akzeptanzbereichs, wird die Nullhypothese abgelehnt, andernfalls wird sie
angenommen.  Man sagt hier auch, der Test h"atte versagt, die Nullhypothese abzulehnen, da
das ja i.allg.\ das Ergebnis ist, das man haben m"ochte\footnotemark. \footnotetext{Da die
  Teststatistik dazu dienen soll, die realen Daten von den mit der Nullhypothese
  konsistenten Prozessen $\process$ zu \naja(unterscheiden), bezeichnet man $T$ auch als
  \begriff(Diskriminator).}

\subsubsection{Fehler 1. und 2. Art}
Bei der Annahme oder Ablehnung einer Nullhypothese k"onnen verschiedener Art Fehler
auftreten (s. \psref{taberrors}). Der
erste m"ogliche Fehler ist, da"s die Nullhypothese abgelehnt wird, obwohl sie eigentlich wahr ist.
Man spricht hier von einem \begriff(Fehler 1. Art). Die Wahrscheinlichkeit $\alpha$, mit
der Fehler 1. Art auftreten, kann frei bestimmt werden. Dies geschieht, indem als
Annahmebereich des Tests das $(1-\alpha)$\begriff(-Konfidenzintervall) der Teststatistik
f"ur die betrachteten Prozesse gew"ahlt wird. Das $(1-\alpha)$-Konfidenzintervall ist der
Bereich der $T$-Werte, f"ur den mit Wahrscheinlichkeit $1-\alpha$ Realisierungen von
Prozessen aus $\process$ in diesem Bereich liegen\korrektur(einfacher formulieren).  Man
spricht bei einem Test mit vorgegebenem $\alpha$ auch von einem \begriff(Niveau
$\alpha$-Test) bzw.\ von einem \begriff(Test zum Signifikanzniveau $\alpha$). Anstatt das
Signifikanzniveau von vorneherein festzulegen, wird ab und zu auch der $p$-Wert eines Tests
angegeben. Dies ist der kleinste Wert von $\alpha$, f"ur den die Nullhypothese gerade noch
abgelehnt w"urde.

Bei Annahme der Nullhypothese, wenn sie tats"achlich falsch ist, spricht man von einem
Fehler 2. Art.  Die Wahrscheinlichkeit f"ur das Auftreten solcher Fehler wird mit $\beta$
bezeichnet. Die komplement"are Wahrscheinlichkeit $1-\beta$ gibt an, wie \naja(gut) der
Test in der Lage ist, die Nullhypothese f"ur mit ihr inkonsistenten Daten abzulehnen. Man
bezeichnet $1-\beta$ daher auch als die \begriff(G"ute) des Tests.  Da die G"ute eines
Tests davon abh"angt, wie nicht-konsistent die wirklichen Daten mit der Nullhypothese
sind, kann $\beta$ im Gegensatz zu $\alpha$ nicht vorgegeben werden.  Allerdings h"angt
die G"ute des Tests von $\alpha$ ab -- je h"oher das Signifikanzniveau $\alpha$ des Test
ist, desto geringer ist $\beta$. Es ist andererseits nicht sinnvoll, um die G"ute
$1-\beta$ gro"s zu machen, ein sehr hohes $\alpha$ zu w"ahlen. Man w"urde sich die h"ohere
G"ute des Tests mit einer geringeren Signifikanz\footnotemark, d.h.\ Aussagekraft, des
Test erkaufen.  
\footnotetext{Hier mu"s unterschieden werden zwischen den Begriffen
  Signifikanz und Signifikanzniveau. Ein niedriges Signifikanzniveau bedeutet eine hohe
  Signifikanz und umgekehrt.}

\epsfigcommon{M"ogliche Fehler bei der Entscheidungsfindung durch einen Hypothesentest.}{taberrors}{0cm}{
\newcommand{\rb}[1]{\raisebox{1.5ex}[-1.5ex]{#1}}
\begin{tabular}{|c|c|c|}
\hline
& & \\
 & \rb{$\nullhyp$ ist wahr} & \rb{$\nullhyp$ ist falsch} \\ \hline
& & \\
$\nullhyp$ wird & falsche Entscheidung &    \\
abgelehnt & Fehler 1. Art & \rb{richtige Entscheidung} \\ 
& & \\ 
\hline
 & & \\
$\nullhyp$ wird &   & falsche Entscheidung  \\
angenommen &  \rb{richtige Entscheidung} & Fehler 2. Art \\
& & \\
\hline
\end{tabular}
}



\comment{
\begin{center}
\newcommand{\rb}[1]{\raisebox{1.5ex}[-1.5ex]{#1}}
\begin{tabular}{c|c|c}
 & $\nullhyp$ ist wahr & $\nullhyp$ ist falsch \\ \hline
& & \\
$\nullhyp$ wird & falsche Entscheidung &    \\
abgelehnt & Fehler 1. Art & \rb{richtige Entscheidung} \\ 
& & \\ 
\hline
 & & \\
$\nullhyp$ wird &   & falsche Entscheidung  \\
angenommen &  \rb{richtige Entscheidung} & Fehler 2. Art \\
& & \\
\end{tabular}
\end{center}
}

Hieraus wird auch ersichtlich, warum wir die Nullhypothese negativ formulieren und uns
daran gelegen ist, sie abzulehnen.  Die Wahrscheinlichkeit daf"ur, da"s unsere
Entscheidung korrekt ist, l"a"st sich genau angeben. W"are die Entscheidung falsch,
handelte sich es ja um einen Fehler 1. Art, der mit der Wahrscheinlichkeit $\alpha$
auftritt. Wir haben also mit Wahrscheinlichkeit $1-\alpha$ die richtige Entscheidung
getroffen.  K"onnen wir die Nullhypothese dagegen nicht ablehnen, so kann nichts dar"uber
gesagt werden, mit welcher Wahrscheinlichkeit die Entscheidung zur Annahme der
Nullhypothese korrekt ist. Die Wahrscheinlichkeit f"ur Fehler 2. Art h"angt stark von den
Daten selber ab, als auch vom Umfang der Daten. Man kann die G"ute i.allg.\ nur f"ur den
Test {\em bestimmter} Daten gegen die Nullhypthese in Abh"angigkeit vom Datenumfang
angeben.


\subsubsection{Einfache Nullhypothesen}
Bei der Konstruktion eines Tests ist zu beachten, da"s zwei verschiedene Typen von
Nullhypothesen existieren: \begriff(einfache) und \begriff(zusammengesetzte). Bei
einfachen Nullhypothesen besteht die Menge der mit $\nullhyp$ konsistenten Prozesse
$\process$ aus nur einem Element, w"ahrend sie bei zusammengesetzten Hypothesen aus
meheren bis m"oglicherweise "uberabz"ahlbar unendlich vielen besteht.

Es soll nun ein Beispiel f"ur eine einfache Nullhypothese etwas genauer betrachtet werden.
Die Nullhypothese sei, da"s die Daten gau"sverteilt sind, mit einem vorher festgelegten
Mittelwert $\mu_0$ und festgelegter Varianz $\sigma_0$. Als Diskriminator $T$
k"onnen wir f"ur diese Nullhypothese ein h"oheres Moment der Verteilung w"ahlen, beispielsweise:
\eqnl[teststatistik1]{T=\frac{1}{N}\sum_{i=1}^N x_i^4 .} 
Prinzipiell h"atte jede beliebige Funktion der $N$ Argumente $X=(x_1,\dots,x_n) $ gew"ahlt
werden k"onnen. Allerdings h"angt die G"ute des Tests stark von der Teststatistik $T$
ab\footnotemark.  \footnotetext{In der Tat ist das hier gew"ahlte $T$ nicht die optimale
  Wahl, da nicht-gau"sf"ormige Verteilungen existieren, die das gleiche vierte Moment wie
  eine Gau"sverteilung besitzen. Dies ist f"ur die folgenden Betrachtungen jedoch ohne
  Belang.}


Wir berechnen nun eine Anzahl $B$ von sogenannten \begriff(Surrogatdaten) oder kurz
\begriff(Surrogaten) $\{X_k, k=1,\dots,B\}$. Der Proze"s zur Erzeugung der Surrogatdaten
mu"s mit der Nullhypothese konsistent sein. Da es bei diesem Test nur auf die Verteilung
der Daten und nicht auf zeitliche Korrelationen ankommt, k"onnen wir einen Gau"sproze"s
$\gauss(\mu_0,\sigma_0^2)$, der unabh"angige, normalverteilte Zufallszahlen mit
Erwartungswert $\mu_0$ und Varianz $\sigma_0^2$ erzeugt, w"ahlen. Wir berechnen nun die
Teststatistik $T$ sowohl f"ur die Surrogatdaten als auch f"ur die realen Daten. Die
$T$-Werte seien mit $\{T_k,k=1,\dots,B\}$ bzw.\ $T_0$ bezeichnet.

Es gibt zwei M"oglichkeiten die Nullhypothese $\nullhyp$ abzulehnen. Die erste erfolgt
durch eine sogenannte \begriff(Ranganalyse) \cite{Prichard-theiler3}.  Hierzu bilden wir
aus der Menge der $T$-Werte einschlie"slich $T_0$ eine aufsteigend sortierte Liste.
Wollen dir die Nullhypothese nun auf dem $\alpha$-Signifikanzniveau ablehnen, mu"s $T_0$
unter den $(B+1)\alpha/2$ kleinsten oder den $(B+1)\alpha/2$ gr"o"sten Werten der
sortierten Liste sein. Zu beachten ist, da"s $B+1$ mindestens gleich $2/\alpha$ sein mu"s.
Im allgemeinen wird $B+1$ als ein Vielfaches von $2/\alpha$ gew"ahlt. F"ur ein "ubliches
Signifikanzniveau von $\alpha=0,05$ w"are $B=39$, die minimale Anzahl zu erzeugender 
Surrogatdaten.

Statt der eben vorgestellten Ranganalyse kann auch auch der $p$-Wert f"ur die Ablehnung der
Nullhypothese berechnet werden. Hierzu wird aus der Teststatistik der Surrogatdaten der
Mittelwert $\bar T$ und die Standardabweichung $\sigma_T$ berechnet. Wir k"onnen nun "uber
\eqn{\mathcal{S} = \frac{\abs{T_0-\bar T}}{\sigma_T}}
ein Ma"s f"ur die Signifikanz der Abweichung von Original- zu Surrogatdaten definieren
\cite{Theiler92b}. Unter der Annahme, da"s die $T$-Werte der Surrogatdaten normalverteilt
sind\footnote{Diese Annahme ist i.allg. vern"unftig und kann auch durch numerische
  Experimente best"atigt werden.}, kann der $p$-Wert "uber $p =
\mathrm{erfc}(\mathcal{S}/\sqrt2)$ berechnet werden\footnote{Es gilt
  $\mathrm{erfc}(x)=1-\mathrm{erf}(x)$, wobei $\mathrm{erf}$ das Gau"ssche Fehlerintegral
  $\mathrm{erf}(x)=\frac2{\sqrt{2}}\int_0^xe^{-t^2} dt$ ist. }.  Dies ist die
Wahrscheinlichkeit eine Abweichung gr"o"ser oder gleich $\mathcal{S}$ zu erhalten, obwohl
die Nullhypothese wahr ist.

\comment{\footnotetext{Bei sehr
  gro"sem $B$ oder bekannter $T$-Verteilung h"atten wir auch den $\alpha$-Konfidenzbereich
  $[T_{\alpha,\tmin},T_{\alpha,\tmax}]$ berechnen k"onnen. Liegt $T_0$ au"serhalb des
  Kondidenzbereichs, kann die Nullhypothese abgelehnt werden.}
}

\subsubsection{Zusammengesetzte Nullhypothesen}
Das vorangegangene Beispiel ist aufgrund der Beschr"ankung auf ein bestimmtes $\mu_0$
bzw.\ $\sigma_0$ recht praxisfern und -- au"ser zu Demonstrationszwecken -- eher
uninteressant. Bei vorliegenden Daten wollen wir die Nullhypothese pr"ufen, ob die Daten
allgemein gau"sverteilt mit unbekanntem, beliebigem Mittelwert $\mu$ und Varianz
$\sigma^2$ sind. Dies ist allerdings eine zusammengesetzte Nullhypothese. Die mit
$\nullhyp$ konsistenten Prozesse, sind alle Gau"sprozesse mit beliebigem $\mu$ und
$\sigma^2$. Es w"are nun offensichtlich weder sinnvoll noch praktikabel, die Teststatistik
\eqnref{teststatistik1} f"ur alle m"oglichen Gau"sprozesse $\gauss(\mu,\sigma^2)$ zu
berechnen. Um die Anzahl der betrachteten Prozesse einzuschr"anken, existieren zwei
verschiedene Ans"atze, die wir im folgenden diskutieren und vergleichen wollen
\cite{Prichard-theiler3}.

\paragraph{Typische Realisierungen}
Eine M"oglichkeit den Bereich der Prozesse einzuengen besteht in der Beschr"ankung auf
\begriff(typische Realisierungen). Man berechnet hierzu $\hat\mu$ und $\hat\sigma$ der Originaldaten
und erzeugt dann die Surrogatdaten durch Gau"sprozesse $\gauss(\mu,\sigma^2)$ mit
$\mu=\hat\mu$ und $\sigma=\hat\sigma$. Ein Problem hierbei ist, da"s f"ur die
Surrogatdaten der empirische Mittelwert und die Standardabweichung im allgemeinen ungleich
$\hat\mu$ bzw.\ $\hat\sigma$ sind\footnotemark. Die Verteilung der Daten ist f"ur Realisierungen von 
Prozessen mit verschiedenem $\hat\mu$ und $\hat\sigma$ unterschiedlich. Dies hat zur Folge, da"s die Teststatistik relativ
breit streut und die G"ute des Tests sehr schlecht wird. Dem l"a"st sich
abhelfen, indem wir statt der Teststatitstik \eqnref{teststatistik1} eine
\begriff(zentrale Teststatistik) (engl. pivotal test statistic) verwenden. Eine zentrale
Teststatistik ist dadurch definiert, da"s sie f"ur
alle Realisierungen der betrachteten Prozesse die gleiche Verteilung aufweist. 
Betracheten wir statt des vierten Moments (\eqnref{teststatistik1}) das vierte, zentrale
und normierte Moment:
\eqnl[teststatistik2]{T'=\frac{1}{N}\sum_{i=1}^N \left(\frac{x_i-\hat\mu}{\hat\sigma} \right)^4.} 
Die so definierte Teststatistik ist unabh"angig vom Mittelwert und der empirischen
Standardabweichung der Surrogat- bzw.\ Originaldaten. Die $T'$-Verteilung ist f"ur alle
Prozesse gleich und die Statistik somit zentral.
\footnotetext{Ein Gau"sproze"s $\gauss(\mu,\sigma^2)$ erzeugt Zufallszahlen
  mit Erwartungswert $\mu$ und Standardabweichung $\sigma$. Bei der Realisierung $X$ eines
  solchen Prozesses k"onnen der Mittelwert $\hat\mu$ und die empirische
  Standardabweichung $\hat\sigma$ der erzeugten Daten hiervon abweichen.  Die Werte f"ur
  die Realisierung eines solchen Prozesses sollen daher durch ein Dach "uber der Variablen
  unterschieden werden. F"ur sehr gro"se $N$ konvergieren $\hat\mu$ und $\hat\sigma$ gegen
  Erwartungswert $\mu$ und Standardabweichung $\sigma$.}


%Die Ablehnung der Nullhypothese entspricht einem Fehler 1. Art.
%Diese Fehler sollten nach Voraussetzung mit der Wahrscheinlichkeit $0,05$ auftreten, da

Die beiden Teststatistiken sollen nun bez"uglich ihres empirischen Signifikanzniveaus und ihrer
G"ute verglichen werden (s. \psref{typicalreal}). 
F"ur die Berechnung
des empirischen Signifikanzniveaus wurden $\gauss(0,1)$ verteilte Daten verschiedenen Umfangs $N$
gegen die Nullhypothese getestet. Zu jeder Testreihe wurden $B=39$ Surrogatdatenreihen
durch einen Gau"sproze"s $\gauss(\hat\mu,\hat\sigma^2)$ erzeugt, wobei $\hat\mu$ bzw.\ 
$\hat\sigma$ Mittelwert und Standardabweichung der Testreihe sind.  Das Signifikanzniveau
$\alpha$ f"ur diesen Test wurde auf $0,05$ festgelegt. Diese H"aufigkeit $\hat\alpha$ mit
der Fehler 1. Art, d.h.\ Ablehnungen der Nullhypothese, auftreten wurde berechnet, indem
der Test f"ur $M=10000$ Testdaten durchgef"uhrt wurde und die Anzahl der Ablehnung der
Nullhypothese durch $M$ geteilt wurde. Die eingezeichneten Fehlerbalken der L"ange
$\sqrt{\alpha(1-\alpha)/M}$, ergeben sich aus der Annahme, da"s die Anzahl der Ablehnungen
der Nullhypothese einer Binominalverteilung $\mathrm{Bin}(\alpha,M)$ folgt\footnotemark.
Um die G"ute der beiden Teststatistiken zu bestimmen wurden als Testreihen gleichverteilte
Daten im Intervall $[-1,1]$ verwendet. Da die Ablehnung der Nullhypothese hier korrekt
ist, stellt ein Versagen der Ablehnung einen Fehler zweiter Art dar. Die relative
H"aufigkeit der Ablehnungen der Nullhypothese  $1-\hat\beta$ ist also die G"ute des Tests 
gegen gleichverteilte Daten. 

 \epsfigdouble{surrogate/typcon/typicalsize}{surrogate/typcon/typicalpower} {Numerisch
   bestimmtes Signifikanzniveau $\hat\alpha$ (links) und G"ute $1-\hat\beta$ (rechts)
   aufgetragen "uber der Datensatzgr"o"se $N$. Die Surrogatdaten wurden als typische
   Realisierungen erstellt. Die gestrichelte Kurve kennzeichnet die jeweiligen Werte f"ur
   die nichtzentrale Teststastik $T$, die durchgezogene f"ur die zentrale Teststatistik $T'$.  }
 {typicalreal}{-0.2cm}

\footnotetext{Der beschriebene Proze"s entspricht einem Urnenmodell mit Zur"ucklegen ohne
  Beachtung der Reihenfolge. Hierbei entspricht das Ziehen einer Kugel der ersten Art der
  Ablehnung der Nullhypothese. Dieses Ereignis tritt mit Wahrscheinlichkeit $\alpha$ auf,
  das komplement"are Ereignis mit Wahrscheinlichkeit $1-\alpha$. Ein solches Urnenmodell
  wird durch die Binominalverteilung $\mathrm{Bin}(\alpha,M)$ beschrieben. Der
  Erwartungswert f"ur die Anzahl der Ablehnungen der Nullhypothese betr"agt daher $\alpha
  M$, die Standardabweichung $\sqrt{\alpha(1-\alpha)M}$. Der relative Fehler ist
  $\sqrt{\alpha(1-\alpha)/M}$.  }

Die H"aufigkeit mit der Fehler 1. Art auftreten, ist bei der nichtzentralen Statistik
deutlich geringer als $0,05$. Auf der einen Seite mag dies begr"u"senswert erscheinen, da
besagte Fehler seltener vorkommen, auf der anderen ist $\alpha$ jedoch ein festgelegter
Parameter und die Teststatistik sollte nicht beliebig davon abweichen. Ung"unstiger ist
aber noch das die G"ute des Tests bei vergleichbarem Datenumfang weitaus schlechter ist,
als bei der zentralen Statistik. Die zentrale Teststatistik kann schon bei einem
Datenumfang $N=100$ die gleichverteilten Daten fast mit Wahrscheinlichkeit $1$ von
normalverteilten unterscheiden. F"ur die nichtzentrale Statistik gelingt dies erst bei
ca.\  $N=560$.


\paragraph{Eingeschr"ankte Realisierungen}
Die im vorherigen Abschnitt dargestellte Methode der Hypothesentests mit zentralen
Teststatistiken funktionierte f"ur dieses Beispiel sehr gut. In allgemeineren F"allen
stellt sie jedoch eine starke Einschr"ankung dar. F"ur kompliziertere Nullhypothesen und
darauf zugeschnittene Teststatistiken ist es ein schwieriges Unterfangen, die
Teststatistik zentral zu machen.  Man denke beispielsweise an die Nullhypothese, da"s die
Daten einem linearen, autokorrelierten stochastischen Proze"s $\arma(p,q)$ entstammen,
wobei als Teststatistik die Vorhersagezeit $\tau_p$ oder die Korrelationsdimension $D_2$
verwendet werden sollen. Die Teststatistiken so umzuformulieren, da"s sie zentral
bez"uglich der betrachteten Prozesse werden, d"urfte unm"oglich sein. Ein Ansatz der hier
weiterhilft, ist die Methode der \begriff(eingeschr"ankten Realisierungen).

Die Unzul"anglichkeit der nichtzentralen Teststatistik aus dem vorangegangen Beispiel
resultierte aus den Schwankungen des Mittelwerts $\hat\mu$ und der Standardabweichung $\hat\sigma$ 
f"ur die typischen Realisierungen. Bei der Methode der eingeschr"ankten Realisierungen
wird dies ausgeschlossen indem nur Surrogatdaten verwendet werden, bei denen Mittelwert
und Standardabweichung exakt mit denen der Testreihe "ubereinstimmen. In diesem Fall ist
das sehr einfach zu erreichen. Wir berechnen wieder eine Surrogatreihe
$X_k=(x_{k,1},\dots,x_{k,N})$ "uber einen Gau"sproze"s $\gauss(\hat\mu,\hat\sigma)$. Nun
berechnen wir Mittelwert $\hat\mu_k$ und Standardabweichung $\hat\sigma_k$ der erzeugten
Daten und skalieren sie um:
\eqn{x'_{k,l} = \hat\mu_0 + (x_{k,l}-\hat\mu_k)\hat\sigma_0/\hat\sigma_k .}
Die so erzeugten Daten haben exakt gleichen Mittelwert und Standardabweichung wie die
Testdaten.

\psref{constrainedreal} zeigt die Ergebnisse f"ur Signifikanzniveau und G"ute des Tests
mit eingeschr"anken Realisierungen. Die Unterschiede bei den relativen H"aufigkeiten der
Fehler 1. Art sind statistisch bedingt. Sie liegen im Rahmen der Standardabweichung
$\Delta\hat\alpha=\sqrt{\alpha(1-\alpha)/M}$ von $\hat\alpha$. Bez"uglich der G"ute
$1-\hat\beta$ sind beide Statistiken nicht zu unterscheiden. Auch im Vergleich zu den
typischen Realisierungen mit zentraler Teststatistik ist kein Unterschied zu erkennen.
Die Methode der eingeschr"ankten Realisierungen macht uns unabh"angig von dem Zwang,
eine zentrale Statistik zu verwenden.

Da"s beide Teststatistiken die gleichen Ergebnisse erbringen, kann auch theoretisch
begr"undet werden.  Die Berechnung der nichtzentralen Statistik $T$ f"ur die
eingeschr"ankte Realisierung $X'_k$ ergibt:
\eqna{T''&=&\frac1N \sum_{i=1}^N (x'_{k,i})^4\nonumber\\
&=& \frac1N\left\{ \frac{\hat\sigma_0^4}{\hat\sigma_k^4} \sum_{i=1}^N (x'_{k,i}-\hat\mu_k)^4 +
\frac{4\hat\mu_0\hat\sigma_0^3}{\hat\sigma_k^3} \sum_{i=1}^N (x'_{k,i}-\hat\mu_k)^3 \right\} +
6\hat\mu_0^2\hat\sigma_0^2+\hat\sigma_0^4 .
}
Diese Teststatistik ist unabh"angig von $\hat\mu_k$ und $\hat\sigma_k$ und somit
bez"uglich der $X_k$ auch wieder zentral. 

\epsfigdouble{surrogate/typcon/constrainedsize}{surrogate/typcon/constrainedpower}
{Numerisch
   bestimmtes Signifikanzniveau $\hat\alpha$ (links) und G"ute $1-\hat\beta$ (rechts)
   aufgetragen "uber der Datensatzgr"o"se $N$. Die Surrogatdaten wurden als eingeschr"ankte
   Realisierungen erstellt. Die gestrichelte Kurve kennzeichnet die jeweiligen Werte f"ur
   die nichtzentrale Teststastik $T$, die durchgezogene f"ur die zentrale Teststatistik $T'$. 
}
{constrainedreal}{-0.2cm}


\subsubsection{Nichtlinearit"atstests}

\paragraph{ARMA-Prozesse}
Zur"uckkehrend zu unserer Eingangs gestellten Frage, ob eine vorliegende Zeitreihe deterministisch 
sei, k"onnen wir nun versuchen hierf"ur geeignete Nullhypothesen und Teststatistiken
aufzustellen. Da die Nullhypothese einer Verneinung der Frage, entspricht m"ussen wir also 
ein allgemeines Modell f"ur ein nicht-deterministisches System benutzen. Ein solches
Modell kann durch sogenannte \begriff(ARMA-Prozesse) beschrieben werden. Hierbei steht
das AR f"ur \begriff(autoregressiv) und MA f"ur \begriff(moving average). ARMA-Prozesse $\arma(p,q)$
beschreiben lineare, autokorrelierte stochastische Systeme, welche "uber eine Abbildung 
\eqn{ x_{i} = a_0 + \sum_{k=1}^p a_k x_{i-k}+\sum_{k=0}^q b_k \eps_{i-k} }
modelliert werden k"onnen \cite{Theiler92b}. Hierbei ist $x_i$ das Signal zur Zeit $t_i$, $\eps_i$ sind
$\gauss(0,1)$ verteilte, unkorrelierte Rauschterme .

Wir betrachten zun"achst die einfachste Form eines ARMA-Prozesses, einen
\begriff(Ornstein-Uhlenbeck-Proze"s) $\arma(1,0)$:
\eqnl[ornstein]{ x_{i} = a_0 +  a_1 x_{i-1}+ b_0 \eps_{i}. }
Um Surrogatdaten, die diesem Proze"s entsprechen zu erstellen, mu"s der Mittelwert
$\hat\mu$, die Standardabweichung $\hat\sigma$ und die Autokorrelationsfunktion $\ac(k)$
f"ur $k=1$ aus den Originaldaten berechnet werden. Die Parameter k"onnen dann
folgenderma"sen angepa"st werden:
\eqna{a_0&=&\hat\mu(1-\ac(1)^2) \\ a_1&=&\ac(1) \\b_0&=&\hat\sigma\sqrt{1-\ac(1)^2}.}
Die Surrogatdaten erh"alt man indem man \eqnref{ornstein} f"ur einen Startwert $x_0$ mit
diesen Parametern iteriert. F"ur die Berechnung der $\eps_i$ benutzt man einen
Zufallszahlengenerator, der normalverteilte Zufallszahlen mit Mittelwert 0 und Varianz 1
liefert.  

Die hierdurch erzeugten Surrogatdaten sind offensichtlich typische Realisierungen, da
$\hat\mu$, $\hat\sigma$ und $\ac(1)$ f"ur die Surrogatdaten sicherlich von den Werten der
Originaldaten abweichen werden.  Dies bringt uns wieder das Problem eine zentrale
Teststatistik finden zu m"ussen.  Zudem haben wir hier einen sehr einfachen ARMA-Proze"s
betrachtet und die Aufgabe, die Parameter zu fitten, wird bei Prozessen h"oherer Ordnung
$\arma(p,q)$ immer schwieriger.  Weiterhin ist es bei diesem Proze"s nicht offensichtlich,
wie eingeschr"ankte Realisierungen erzeugt werden k"onnen.


\paragraph{FT-Surrogate}
Dies soll nun f"ur reine autoregressive Prozesse gezeigt werden\footnotemark. F"ur
Prozesse $\arpr(q)$ (bzw.\ $\arma(q,0)$) gilt \footnotetext{F"ur gro"se $q$ sind AR- und
  ARMA-Modelle im wesentlichen "aquivalent. Wir werden daher im folgenden nur AR-Modelle
  betrachten \cite{Theiler92b}.}:
\eqn{ x_{i} = a_0 + \sum_{k=1}^q a_k x_{i-k}+b_0 \eps_{i}. }
Wie sich durch Bildung der Erwartungswerte $<x_ix_{i+k}>$ bzw.\  $<x_i>$ zeigen l"a"st,
sind die Koeffizienten $a_k$ und $b_0$ nur von Mittelwert $\hat\mu$, Standardabweichung 
$\hat\sigma$ und Autokorrelation $\ac(k)$ f"ur Zeiten $k=1,\dots,q$ der Originalzeitreihe abh"angig. Zur
Erzeugung eingeschr"ankter Realisierungen mu"s also die Autokorrelationsfuntion erhalten
bleiben. "Aquivalent dazu ist, nach Wiener-Khintchine, da"s das Leistungsspektrum der
Originaldaten mit dem der Surrogatdaten "ubereinstimmt. Dies liefert uns eine
einfache Methode eingeschr"ankte Realisierungen von Surrogatdaten zu erzeugen. Die
Originalzeitreihe wird fouriertransformiert. \comment{und "uber Betragsbildung das Leistungsspektrum 
berechnet.} Die komplexen Amplituden der auftretenden Frequenzen $f=-\frac12, \dots, -\frac1N, 0,
\frac1N, \dots, \frac12$ seien mit  $\alpha(f)$ bezeichnet. Da die Originalzeitreihe rein
reell ist gilt $\alpha(f)=\alpha(-f)^*$. Nun wird ein neues (Surrogat-) Frequenzspektrum
erzeugt mit $\alpha'(f)=\abs{\alpha(f)}e^{i\phi(f)}$. Auch die Surrogatdaten wieder rein
reell sein sollen mu"s $\phi(-f)=-\phi(f)$ und insbesondere $\phi(0)=0$ gew"ahlt
werden. Nach inverser Fouriertransformation der $\alpha'(f)$ hat man eine
Surrogatzeitreihe mit gleichem Leistungsspektrum und Autokorrelationsfunktion wie die Originalzeitreihe.
Der beschriebene Proze"s wird auch als \begriff(Phasenrandomisierung), die enstandenen
Surrogatdaten als FT-Surrogate bezeichnet.


\paragraph{AAFT-Surrogate}
Durch Phasenrandomisierung erzeugte FT-Surrogate weisen im allgemeinen eine gau"sf"ormige
Verteilung auf. Dies kann "uber den zentralen Grenzwertsatz begr"undet werden. Jeder Wert
der Surrogatreihe $x'_k$ enth"alt $N$ Summanden der Form
$\abs{\alpha(f)}\cos(\phi(f)+kf)$. Die einzelnen Summanden sind statistisch unabh"angig
und besitzen eine beschr"ankte Wahrscheinlichkeitsverteilung. F"ur gro"se $N$ konvergiert
die Summe dieser Verteilungen gegen eine Normalverteilung. 

Im allgemeinen werden wir nun Testdaten vorfinden, die keine Normalverteilung besitzen.
Ein Test dieser Daten gegen die FT-Surrogate daher mit hoher Wahrscheinlichkeit ein
negatives Resultat erbringen\footnote{Ein negatives Resultat bedeutet eine Ablehnung der
  Nullhypothese, in unserem Sinne eigentlich positiv.}. Damit ist jedoch nicht
zwingend ein Beweis daf"ur erbracht, da"s die Daten keinem linearen, stochastischen Proze"s
entstammen. Das Signal k"onnte beispielsweise einem solchen Proze"s entspringen und bei
oder vor der Messung einer nichtlinearen Transformation, sagen wir $h$, unterworfen sein.
Das Ursprungssignal $y$ w"are dann normalverteilt und das Me"ssignal $x=h(y)$ folgt einer
beliebigen nicht gau"sf"ormigen Verteilung. \footnote{Im Prinzip k"onnte man dies schon als
  nichtlineares System bezeichnen. Die zugrundeliegende Dynamik, an der wir ja interessiert sind, ist
  jedoch linear.}

Eine entsprechende Nullhypothese w"are demnach, da"s den Daten ein stochastischer Proze"s
zugrunde liegt, der durch eine nichtlineare Funktion gefiltert worden ist. Surrogatdaten, die
mit dieser Nullhypothese konsistent sind, bezeichnet man als \begriff(AAFT-Surrogate),
wobei AA f"ur ``Amplituden angepa"st'' (engl. amplitude adjusted) steht.  
Das Verfahren l"auft in  folgendem Schritten ab.
\begin{itemize}
\item Es wird eine Reihe von normalverteilten Daten $y_i$ wird erstellt. Diese Reihe wird
  in der Weise  sortiert, da"s wenn $x_j$ der $k$-kleinste Wert der Reihe $x_i$ ist, dann
  auch $y_j$ der $k$-kleinste Wert der Reihe $y_i$ ist. Etwas bildlicher ausgedr"uckt
  kann man sagen, die Reihe $y_i$ \naja(folge) der Originalreihe $x_i$.
  
  Ein entsprechender Algorithmus arbeitet wie folgt.  Eine normalverteilte Zeitreihe $z_i$
  wird "uber einen geeigneten Zufallszahlengenerator erzeugt. Diese wird nach
  aufsteigenden Werten sortiert $z_1\leq\dots\leq z_N$.  Nun werden aus den Originaldaten
  Wertepaare $(x_i,i)$ erzeugt, die wiederum nach aufsteigenden $x_i$ sortiert werden. Wir
  erhalten eine Reihe $(x'_j,n_j)$ mit $x'_1\leq\dots\leq x'_N$ und $x_{n_j}=x'_j$. Die
  $n_j$ werden nun mit den $z_j$ zu Wertepaaren $(z_j,n_j)$ verkn"upft, welche nun nach
  den $n_j$ aufsteigend sortiert werden. Die sortierten Paaren $(z'_j,j)$ enhalten nun die
  gesuchte, sortierte Reihe $y_j=z'_j$, da nach Konstruktion gilt $y_i<y_j\Leftrightarrow
  x_i<x_j$.
\item Aus der Zeitreihe $y_i$ wird "uber die Methode der Phasenrandomisierung eine
  Surrogatreihe $y'_i$ gebildet. 
\item Die Originalreihe $x_i$ wird nun so geordnet, da"s sie der Reihe $y'_i$ folgt. Die
  enstandene Surrogatreihe $x'_i$ hat, da sie nur umsortiert wurde, die gleiche
  Verteilung, wie die Originaldaten. Dar"uber hinaus haben die zugrundeliegenden
  Zeitreihen $y_i$ und $y'_i$ das gleiche Leistungsspektrum und die gleiche Autokorrelationsfunktion.
\end{itemize}
Dies ist die Methode, die auch im folgenden grunds"atzlich verwendet wird. Ein Test mit
FT-Surrogaten, macht wenig Sinn, da im Falle, da"s die Originaldaten nicht normalverteilt
sind, die Nullhypothese mit hoher Wahrscheinlichkeit abgelehnt wird. Der Grund f"ur die
Ablehnung, wird in diesem Fall jedoch eher an der Verteilung der Daten, als an einer 
eventuell zugrundeliegenden deterministischen Struktur liegen. Sind die Originaldaten dagegen
normalverteilt, so besteht kein Unterschied zwischen den FT- und den AAFT-Surrogatdaten.


\paragraph{Teststatistiken}
Wie im einf"uhrenden Abschnitt bereits dargestellt wurde, ist die spezielle Wahl der
Teststatistik bei einem Hypothesentest relativ unwichtig. Was den eigentlichen Kern des
Tests ausmacht, ist die Wahl des Modells und die entsprechende Generierung der
Surrogatdaten. Nichtsdestotrotz k"onnen manche Teststatitiken besser geeignet sein, eine
Nullhypothese abzulehnen, als andere. Eine Teststatistik sollte immer auf Merkmale
ausgerichtet sein, die \naja(inhaltlich) mit der Nullhypothese in Zusammenhang stehen.
Beispielsweise macht es f"ur einen Determinismustest "uber AAFT-Surrogate wenig Sinn, die
Original- und Surrogatzeitreihen auf statistische Eigenschaften hin zu untersuchen.
Geeigneter sind da Merkmale, die f"ur deterministische Systeme kennzeichnend sind. In der
Regel wird eine der drei folgenden Statistiken benutzt.
\begin{itemize}
\item \rem(Der mittlere Vorhersagefehler $\eps$.) Eine wesentliche Eigenschaft deterministischer 
  Systeme ist, da"s sie sich f"ur eine begrenzte Zeit vorhersagen lassen. Sei der Zustand
  des Systems $\x$ zu Zeit $t<t_0$ bekannt, so l"a"st sich der Zustand zur Zeit
  $t_0+\Delta t$ mit einer gewissen, von der Dynamik des Systems und der Qualit"at der
  Daten abh"angigen Genauigkeit vorhersagen. Diese Vorhersage geschieht im allgemeinen
  durch Anpassung lokaler, linearer Modelle an die Dynamik. Der mittlere Unterschied zwischen vorhergesagtem Zustand
  $\x_\mathrm{pred}(t_0+\Delta)$ und tats"achlichem Zustand $\x(t_0+\Delta)$ ist bei
  deterministischen Systemen deutlich geringer als bei stochastischen, und kann so als
  Teststatistik dienen. 
\item \rem(Der gr"o"ste Lyapunovexponent.) Lyapuovexponenten beschreiben das
  Auseinanderdriften benachbarter Trajektorien im Phasenraum. Sie kennzeichnen die
  Eigenschaft chaotischer Systeme der sensitiven Abh"angigkeit von den
  Anfangsbedingungen. W"ahrend lineare, deterministische Systeme keine positiven
  Lyapunovexponenten besitzen, exitiert f"ur ein chaotisches System mindestens ein
  positiver. F"ur stochastische Systeme divergieren sie. 
\item \rem(Fraktale Dimensionen.) Wie im Kapitel "uber fraktale Dimensionen bereits
  beschrieben, besitzen deterministische Systeme eine endliche Dimension, w"ahrend sie
  f"ur stochastische i.allg.\ mit steigender Einbettungsdimension divergiert. F"ur
  stochastische Systeme mit $\omega^{-\alpha}$ Leistungsspektrum weisen auch die Surrogate
  ein solches Leistungsspektrum auf, soda"s die Nullhypothese auch hier nicht f"alschlich
  abgelehnt w"urde. Als Teststatistik w"ahlt man hier im allgemeinen die
  Korrelationsdimension aufgrund ihrer einfachen Berechenbarkeit.
\end{itemize}
In unseren Untersuchungen wurde grunds"atzlich die Korrelationsdimension als Teststatistik 
benutzt. Dies liegt zum einen daran, da"s sie einfacher zu berechnen ist als die beiden
anderen Vorschl"age. F"ur den ersten ist das fitten lokaler, linearer Modelle f"ur
verschiedene Bereiche des Phasenraums notwendig, was ein komplizierter und
rechenaufwendiger Proze"s ist. Die Berechnung von Lyapunovexponenten ist dagegen nicht
sehr robust gegen Rauschen. Zum anderen spricht nichts gegen die Korrelationsdimension,
wenn sie in der Lage ist, die Nullhypothese abzulehnen. Im Falle des Versagens k"onnten
dann allerdings andere Statistiken verwendet werden. 

\paragraph{Beispiele}
Die Methode soll nun an zwei Beispielen getestet werden. Zum einen werden Testdaten
f"ur einen Ornstein-Uhlenbeck-Proze"s generiert. Ein Test f"ur diese Daten sollte zu
keiner Ablehnung der Nullhypothese f"uhren. Zum anderen werden aus dem Lorenzsystem
extrahierte Zeitreihen getestet, bei denen der Test zu einer Ablehnung f"uhren sollte.

F"ur den Ornstein-Uhlenbeck-Prozess wurde eine zuf"alliger Anfangswert $x_0'$
gew"ahlt. Dann wurde \eqnref{ornstein} $9192$ mal iteriert, wobei als Zeitreihe die
letzten $N=8192$ Werte der berechneten Reihe verwendet wurden\footnote{Die ersten 1000
  Werte werden bei jeder Generierung von Zeitreihen weggelassen, um ein eventuell noch
  anhaltendes transientes Verhalten auszuschlie"sen. Der Wert 8192 begr"undet sich darin,
  da"s f"ur die Anwendung der schnellen Fouriertransformation (FFT)ganzzahlige  Potenzen
  von 2 erforderlich sind. }. Einen Ausschnitt der Zeitreihe zeigt \psref{ornsteinfig}
links. Danach wurden $B=39$ AAFT-Surrogatzeitreihen erstellt, von denen eine in
\psref{ornsteinfig} rechts zu sehen ist. Bereits optisch ist kein gravierender,
qualitativer Unterschied feststellbar. F"ur die Original- und Surrogatzeitreihen wurden
die Korrelationsintegral f"ur die Einbettungsdimensionen $d=1,\dots,5$ und die Verz"ogerungszeit
$k=4$ (s. \psref{ornsteincorrint}) berechnet. Die Korrelationsdimension
$D_2$ wurde "uber Takens' Sch"atzer mit oberer Grenze $\ln\rmax=-2,5$ abgesch"atzt.

F"ur das Lorenzsystem wurde in der gleichen Weise eine Zeitreihe mit $N=8192$
Datenpunkten erstellt. Um die Robustheit des Verfahrens gegen Rauschen zu testen, wurde
auf diese Zeitreihe 10, 30 und 50 Prozent Rauschen addiert. In den Abbildungen
\psref{lorentsurr1} und \psref{lorentsurr2} ist zu erkennen, da"s die Nullhypothese bei
alleiniger Einbeziehung  der Einbettungsdimension $d=1$ bei keiner der Zeitreihen abgelehnt werden
kann. Dies ist auch nicht weiter verwunderlich, da ein System mit einer Dimension gr"o"ser 
als der Einbettungsdimension sich bez"uglich des Korrelationsintegrals wie ein
stochastisches System verh"alt. F"ur gr"o"sere Einbettungsdimensionen $d\geq2$ kann die
Nullhypothese dagegen bis zu einer St"arke des Rauschens von 30 Prozent abgelehnt werden,
was zu erwarten war. Bei 50 Prozent Rauschen oder mehr ist die Nullhypothese erst ab der
Einbettungsdimension $d=4$ abzulehnen. In einem solchen Fall sollte man jedoch i.allg.\   vorsichtig sein und
eventuell noch andere Teststatistiken hinzuziehen.

\clearpage
{
\def\psposmode{\psseparate}
\epsfigdouble{surrogate/ornstein/orig}{surrogate/ornstein/surr}{
Links ein Auschnitt aus einer Zeitreihe generiert durch eine Ornstein-Uhlenbeck-Proze"s mit den
Parametern $a_0=0$, $a1=0,9$ und $b_0=0,43$ (entspricht $\mu=0$, $\sigma^2=1$ und
$\ac(1)=0.9$). Rechts ein gleich langer Auschnitt aus einer Surrogatzeitreihe. 
}
{ornsteinfig}{-0.2cm}
\epsfigdouble{surrogate/ornstein/corrint}{surrogate/ornstein/surcint1}{
Korrelationsintegrale der Zeitreihen aus \psref{ornsteinfig} f"ur Einbettungsdimensionen
$\embed=1,\dots,5$. Von Bereichen sehr kleiner $r$ abgesehen, weisen die
Korrelationsintegrale keine signifikanten Unterchiede auf.
}
{ornsteincorrint}{-0.2cm}
\epsfigdouble{surrogate/ornstein/variance}{surrogate/ornstein/pvalue}{
Links abgebildet ist die Verteilung der Dimensionsberechnungen normalisiert auf Mittelwert 
und Standardabweichung der Surrogatdaten \gpmarkb.  Die entsprechenden Werte der
Originalzeitreihe sind durch \gpmarkf gekennzeichnet.
}
{ornsteinsigni}{-0.2cm}
}


\epsfigsix{surrogate/lorentz/noise0/orig}{surrogate/lorentz/noise10/orig}
{surrogate/lorentz/noise0/variance}{surrogate/lorentz/noise10/variance}
{surrogate/lorentz/noise0/pvalue}{surrogate/lorentz/noise10/pvalue}
{Oben Ausschnitte aus einer Zeitreihe des Lorenzsystems ohne (links) bzw.\  mit 10 Prozent
  Rauschen (rechts). In der Mitte sind die Verteilung der normalisierten
  Korrelationsdimensionen und unten die entsprechenden $p$-Werte abgebildet. Die
  Nullhypothese kann in beiden F"allen abgelehnt werden.
}{lorentsurr1}{-0.2cm}

\epsfigsix{surrogate/lorentz/noise30/orig}{surrogate/lorentz/noise50/orig}
{surrogate/lorentz/noise30/variance}{surrogate/lorentz/noise50/variance}
{surrogate/lorentz/noise30/pvalue}{surrogate/lorentz/noise50/pvalue}
{Die gleichen Abbildungen wie in \psref{lorentsurr1} f"ur Zeitreihen mit 30 bzw.\  50
  Prozent Rauschen. F"ur 50 Prozent Rauschen ist eine Ablehnung der Nullhypothese anhand
  der $p$-Werte schon zweifelhaft.
}{lorentsurr2}{-0.2cm}




%%%%%%%%%%%%%%%%%%%%%%%%%%%%%%%%%%%%%%%%%%%%%%%%%%
%%%%%%%%%%%%%%%%%%%%%%%%%%%%%%%%%%%%%%%%%%%%%%%%%%

\comment{
\subsection{Statistische Testverfahren}

\subsection{Anwendung in der Zeitreihenanalyse}

\subsection{Modelle}

\subsubsection{ARMA-Modelle}

\subsubsection{Surrogatdaten}

\paragraph{Phasenrandomisierung}

\paragraph{Amplitudenangepa"ste Phasenrandomisierung }

\subsection{Diskriminatoren}

\subsection{Beispiele}

\subsubsection{Wei"ses und farbiges Rauschen}

\subsubsection{Rauschfreie und verrauschte deterministische Systeme}
}









%\chapter{Anwendung auf die Atmungsdynamik Frühgeborener}

\section{Zur Atmung von Fr"uh- und Reifgeborenen}


\subsection{Problemstellung}

Die Fortschritte auf dem Gebiet der intensivmedizinischen Betreuung w"ahrend der letzten
20 Jahre erm"oglichen das "Uberleben kleiner ($<$1500 gr. Geburtsgewicht) und sehr kleiner
($<$1000 gr. Geburtsgewicht) Fr"uhgeborener und schwer erkrankter reifer Neugeborener. In
der auf die Zeit der Intensivbehandlung folgenden Zeit der Stabilisierung und ``Aufzucht''
sind Schwestern und "Arzte mit verschiedenen durch die Unreife bzw. Instablilit"at dieser
Patienten bedingten Problemen konfrontiert. Hierunter erfordert die Instabilit"at der
Atmung unter allen Reifealtern vordringliches Augenmerk: beim eigenst"andig atmenden
Fr"uh- und Reifgeborenen kann es ohne vorausgehende Zeichen zum Atemstillstand kommen.
Dieser f"uhrt zum Sauerstoffmangel (\begriff(Hypox"amie))und konsekutiv zu einem Abfall
der Herzfrequenz (\begriff(Bradykardie)) auf kritische, die Kreislaufversorgung gef"ahrdende
Verh"altnisse. Neben der Gefahr eines akuten Herz-Kreislaufversagens bringen h"aufige
Atemstillst"ande die Gef"ahrdung des sich entwickelnden Gehirns durch latenten
Sauerstoffmangel mit sich. Die Kinder werden durch Ableitung des EKG und des
atemsynchronen thorakalen Impedanzsignals (TI) fortlaufend "uberwacht. Besonders gef"ahrdete
Patienten werden zus"atzlich durch Messung der Sauerstoffs"attigung des Blutes "uberwacht
(\begriff(Pulsoxymetrie)).

\subsection{Generation des Atemrhythus und Atemrhythmen Neugeborener}

Heute wird vom Konzept eines lokalisierten Rhythmusgenerators der Atmung abgesehen. Statt
dessen ist experimentell ein verteiltes Netzwerk atemkompetenter Nervenzellen belegt. Zur
Zeit wird von sechs jeweils verschiedenen Atemphasen zugeordneten Nervenzellgruppen im
Hirnstamm ausgegangen. Ihre gemeinsame Endstrecke ist der das Zwerchfell stimulierende
\begriff(Nervus Phrenicus). Die respiratorisch kompetenten Gruppen unterliegen ihrerseits
ein Anzahl peripherer (Lungendehnung, $\mathrm{O}_2$-Gehalt des Blutes) und zentraler
Einfl"u"se (Wachheit), welche sie integrativ verarbeiten (\autor(Richter)). F"ur die
Gruppe der Neugeborenen, welche hier interessiert, werden im wesentlichen die folgenden
Atemrhythmen unterschieden, wobei diese Einteilung im wesentlichen ph"anonemologisch
erfolgt: Schlafassoziiert: REM \begriff(periodisches) Atmen, NON-REM
\begriff(regelm"a"siges) Atmen, aber auch \begriff(periodisches) Atmen.

\subsection{Stand der Forschung}

Mit erheblichem Aufwand wird versucht, die Bedingungen, welche zu Atemstillst"anden
f"uhren, zu identifizieren. Artikel von \autor(Poets) hier einarbeiten.  Nur wenige
Arbeiten gehen bisher der Frage nach dem dynamischen Charakter der Fr"uhgeborenenatmung
nach. \autor(Pilgrim \etal) geben f"ur das periodische Atmung zur"uckhaltend einen
deterministischen Charakter mit niedrigdimensionalem Attraktor an. Eine Aussage zum
dynamischen Charakter findet sich in tierexperimentellen Arbeiten von \autor(Sammon) und
\autor(Eldridge). Erstgenannter konnte f"ur den vagalen Einflu"s auf die Ruheatmung von
narkotisierten Ratten den Dimensionswechsel eines niedrigdimensionalen Attraktors
nachweisen. In einer sp"ateren theoretischen Arbeit formulierte \autor(Sammon) dann den
Atemzyklus als heteroklines, periodisches Orbit. \autor(Eldridge) konnte - ebenfalls an
einem Experiment mit Ratten - zeigen, da"s die experimentell gefundene Abh"angigkeit der
Reaktion auf eine St"orung des Atemzyklus von der Phasenlage durch Modellierung mittels
eines periodisch erregten \begriff(Van-der-Pol-Oszillators) genau nachgebildet werden
konnte.  Insgesamt fehlen bisher f"ur das menschliche Neugeborene deutliche Hinweise auf
einen deterministischen Charakter des Atemrhythmus. Von Interesse ist es deswegen, weil
ein deterministisches System auch mit nicht-statistischen Mitteln vorhergesagt werden
kann. Dar"uber hinaus spielt diese Frage auch in den Zusammenhang zwischen Atmung als
\begriff(conditio sine qua non) einerseits und den vielen Unterbrechungen des Atemrhythmus bei
S"augetieren (Schlucken, Innehalten, Sprechen etc.) andererseits ein Rolle.

\subsection{Signal Daten und Patienten}

F"ur die vorliegende Arbeit standen Aufzeichnungen des thorakalen Impedanzsignals von 8
Fr"uhgeborenen zur Verf"ugung. Diese entstammen der Routine"uberwachung. Die Messung der
thorakalen Impedanz registriert den atemabh"angig schwankenden Wechselstromwiderstand des
Brustkorbes anhand eines Me"sstroms, welcher "uber zwei Elektroden eingebracht wird (30 Khz,
3 uA). Sie bestimmt somit den Anteil des leitenden zum nichtleitenden Material im
Volumenleiter Thorax. Da nichtleitendes Atemgas periodisch ein und ausstr"omt, variiert
der gemessene Wechselstromwiderstand im wesentlichen mit der Atmung. Das Signal ist aber
wegen der Herzaktion und der sich aus dieser ergebenden "Anderung des leitenden Volumens im
Brustkorb auch an das Herzkreislaufsystem gekoppelt. Die Samplingrate betrug 100Hz. Die
Aufzeichnung erfolgte zusammen mit dem EKG "uber 24 Stunden. Nach Abschlu"s der Messung
wurden x Zeitreihen mit regelm"a"sigem, x Zeitreihen mit periodischem ausgew"ahlt. Apnoen.

\subsection{Ergebnisse}
Hallo \cite{Poets93}, \cite{Hoch96}. 


\begin{appendix}
\chapter{Medizinische Fachbegriffe} 
Die meisten der hier erl"auterten Fachbegriffe stammen aus ``Pschyrembel -- Klinisches
W"orterbuch'' \cite{Pschyrembel}. F"ur Fr"uhgeborene spezifische "Anderungen oder
Erweiterungen der Definitionenen sind aus \autor(Poets) (1993) \cite{Poets93} und
\autor(Hoch) und \autor(Bergmann) (1996) \cite{Hoch96} erg"anzt worden.

\begin{description}
\item[Apnoe:] Atemstillstand.
\item[Atmung, periodische:] Atmung mit Abwechselnd auftretenden mehreren tiefen Atemz"ugen
  und darauffolgender kurzer apnoischer Pause.
\item[Bradykardie:] langsame Schlagfolge des Herzens mit einer Pulsfrequenz unter 60/min,
  bei Fr"uhgeborenen schon ab unter 90-100/min.
\item[Epidemiologie:] Wissenschaftszweig, der sich mit der Verteilung von "ubertragbaren
  und nicht"ubertragbaren Krankheiten und deren physikalischen, chemischen, psychischen
  und sozialen Determinanten und Folgen in der Bev"olkerung befa"st.
\item[Gestationsalter:] Schwangerschaftsdauer, Reifezeichen des Neugeborenen.
\item[Hypox"amie:] niedriger Sauerstoffpartialdruck im arterielle Blut ($\mathrm{pO_2}<70$
  mmHg). Bei Neugeborenen ein Abfall auf unter 40 -- 45 mmHg bzw.\  unter 20\% des Basalwertes. 
\item[idiopathisch:] ohne erkennbare Ursache entstanden, Ursache nicht nachgewiesen.
\item[Konzeptionsalter:] Lebensalter mit Beginn der Empf"angnis.
\item[Neonatologie:] Teilgebiet der Kinderheilkunde, das sich mit Diagnose und Therapie von
  Erkrankungen des Neugeborenen befa"st.
\item[Pathophysiologie:] Lehre von den krankhaften Lebensvorg"angen im menschlichen
  Organismus.
\item[pathologisch:] krankhaft.
\item[Pulsoxymetrie:] transkutane (unblutige) Messung der arteriellen Sauerstoffs"attigung.
\item[thorakal:] zum Brustkorb geh"orig.
\item[Thorax:] Brustkorb.
\item[transkutan:] durch die Haut hindurch.
\end{description}
   
\end{appendix}

   
      


\section{Auswertung der Zeitreihen} 

\subsection{Fourier-Analyse}
Die gemessen Zeitreihen wurden zuerst einer Fourier-Analyse
unterzogen, wobei hier nur die Leistungsspektren $P(f)$ (d.h. der Betrag der Amplituden der
Fourier-Transfor\-mier\-ten) von Interesse sind. Die Leistungsspektren der beiden
ausgew"ahlten Zeitreihen zeigen \psref{medfourierr} (oben) und \psref{medfourierp}
(oben). Im Spektrum der regelm"a"sigen Atmung sind vier Peaks gut zu unterscheiden, von
denen allerdings nur die ersten drei durchgehend in den Leistungsspektren der
regelm"a"sigen Atmung auftauchen. Der 
erste liegt bei ca.\ 0,8 Hz und entspricht der mittleren Atemfrequenz des Kindes. Die
Atemfrequenzen Fr"uhgeborener liegen in der Regel bei ca.\ 
0,6-0,8 Hz, und k"onnen in Ausnahmef"allen auch bis 1,2 Hz reichen, ohne da"s dies als
pathologische Indikation zu werten ist. Die Atemfrequenz dieser Patienten liegt damit
deutlich h"oher als die von erwachsenen Individuen. Die beiden weiteren Peaks (bei 2,2 bzw. 4,3 Hz)
r"uhren von der Herzt"atigkeit her, was durch Vergleich mit den Spektren des gleichzeitig
aufgenommenen EKGs best"atigt wird (siehe \psref{medfourierr} unten). Auch die Herzfrequenz
Fr"uhgeborener liegt h"oher als die Erwachsener, wobei eine Frequenz von ca.\ 2 Hz als normal
anzusehen ist. Die deutlich zu erkennenden Oberfrequenzen werden vornehmlich hervorgerufen
durch die steilen Flanken des sogenannten QRS-Komplexes, eines im EKG-Signal auftretenden sehr
starken Ausschlags.

Bei der periodischen Atmung liegen die Peaks an etwa der gleichen Stelle. Im dargestellten
Spektrum (\psref{medfourierp} oben) erkennt man Peaks bei 0,7 und 2,1 Hz sowie
andeutungsweise bei 4,3 Hz.  Die Auspr"agung ist jedoch weniger deutlich als bei der
regelm"a"sigen Atmung. Auch hier k"onnen die Peaks mit Frequenzen "uber 2 Hz mit
entsprechenden Peaks des Leistungsspektrums des EKGs identifiziert werden.

Da die Peaks bei Frequenzen "uber 2 Hz im allgemeinen von der Herzt"atigkeit herr"uhren, ist es
h"aufige Praxis, auf die Zeitreihen Tiefpa"sfilter mit Grenzfrequenzen von 2 bis 2,5 Hz
anzuwenden. Da hierbei jedoch nicht klar ist, ob durch die Filterung eventuell f"ur die
Atmungsdynamik wesentliche Informationen verlorengehen, findet diese Vorgehensweise hier
keine Anwendung. 


\epsfigdouble
{anwendung/fourier/fr_r_3}
{anwendung/herzfreq/hfr_r}
{Oben das Leistungsspektrum der ausgew"ahlten Zeitreihe des thorakalen Impedanzsignals bei regelm"a"siger Atmung.
  Darunter zum Vergleich das entsprechende Spektrum  des gleichzeitig aufgenommen EKGs.
}
{medfourierr}{-0.5cm}

\epsfigdouble
{anwendung/fourier/fr_p_2}
{anwendung/herzfreq/hfr_p}
{Oben das Leistungsspektrum der ausgew"ahlten Zeitreihe des thorakalen Impedanzsignals bei
  periodischer Atmung.
  Darunter zum Vergleich das entsprechende Spektrum  des gleichzeitig aufgenommen EKGs.
}
{medfourierp}{-0.5cm}


\subsection{Bestimmung der Verz"ogerungszeit}
Die Verz"ogerungszeiten f"ur die Phasenraumrekonstruktionen wurden "uber das Verfahren der
Redundanzanalyse (siehe Seite \pageref{chapredundancy}ff) bestimmt. Die Redundanz $R(t)$
f"ur die beiden ausgew"ahlten Zeitreihen zeigt \psref{medredund}.  Die "uber das erste
lokale Minimum der Redundanz bestimmten Verz"ogerungszeiten $\delay_R$
lagen bei 0,27 -- 0,42 s f"ur regelm"a"sige und bei 0,24 -- 0,36 s f"ur periodische Atmung,
wobei der Unterschied nicht signifikant ist. Eine "Ubersicht "uber die gemessenen Werte
zeigt die untenstehende Tabelle. Da die Verz"ogerungszeiten eine recht geringe Streuung
aufweisen, wurde f"ur die meisten Auswertungen ein Standardwert von $\delay_R=0,32$ s
benutzt. 

\begin{center}
\begin{tabular}{|l||c|c|c|}
  \hline
  Atmungstyp & $\delay_{R,\tmin}/$s & $\delay_{R,\tmax}/$s & $\bar \delay_{R}/$s \\
  \hline
  regelm"a"sig   &  0,27 &  0,42   &  0,33$\pm$0,04 \\
  periodisch        &  0,24 &  0,36   &  0,31$\pm$0,03 \\
  \hline
\end{tabular}
\end{center}

\epsfigdouble
{anwendung/redund/mut_r_3}
{anwendung/redund/mut_p_1}
{Redundanzanalyse f"ur die Zeitreihen mit regelm"a"siger (oben) bzw.\  periodischer
  (unten) Atmung. Die Verz"ogerungszeiten
  liegen bei $\delay_R=0,33 s$ f"ur regelm"a"siges bzw.\ bei $\delay_R=0,27 s$ f"ur
  periodisches Atmen. 
}
{medredund}{-0.5cm}

\clearpage
\subsection{Versuch einer Phasenraumrekonstruktion}
Mit den so gewonnenen Verz"ogerungszeiten wurden Darstellungen der Phasenraumrekonstruktionen
erzeugt. Es sei darauf hingewiesen, da"s diese Abbildungen {\em keine} g"ultigen
Rekonstruktionen eventuell zugrunde liegender seltsamer Attraktoren darstellen, da die
verwendete Einbettungsdimension $\embed=2$ mit Sicherheit nicht ausreichend ist. Sie sollen
nur Hinweise auf die  Struktur der Dynamik und Anhaltspunkte f"ur die weitere
Verfahrensweise geben.

In der Rekonstruktion des regelm"a"sigen Atemsignals erkennt man deutlich eine toroidale
Struktur (siehe \psref{medrekonst} links oben). Diese ist allerdings gest"ort durch Rauschen
sowie ein gewisses \naja(Hin-und-her-Driften) der Trajektorien entlang der
Hauptdiagonalen; die Rekonstruktion erweckt den Eindruck eines Grenzzyklus, der entlang dieser Diagonalen hin
und her geschoben wird. Dies kann als Hinweis auf
eine Instationarit"at der Zeitreihe aufgefa"st werden. Die Rekonstruktion des periodischen Atmens
zeigt keine klare Struktur sondern eher ein \naja(Kn"auel) durcheinander laufernder
Trajektorien (siehe \psref{medrekonst} rechts oben).  Um das sichtbar starke Rauschen zu
unterdr"ucken wurde f"ur beide Signale eine Singular Value Decomposition durchgef"uhrt und
die erste Komponente der SVD-Rekonstruktion wieder eingebettet (siehe \psref{medrekonst}
unten). Die Trajektorien scheinen ein wenig gegl"attet zu sein, lassen jedoch auch nicht
mehr von der Struktur der Dynamik erahnen. 

\afterpage{
\epsfigfour
{anwendung/rects/rct_r_3}
{anwendung/rects/rct_p_1}
{anwendung/rects/rctsf_r_3}
{anwendung/rects/rctsf_p_1}
{Oben dargestellt sind die Rekonstruktionen eines regelm"a"sigen (links) bzw.\  periodischen Atemsignals (rechts)
  zur Einbettungsdimension $\embed=2$ und zur Verz"ogerungszeit $\delay_R=3,2$ s; unten
 die entsprechenden Rekonstruktionen der "uber SVD gefilterten Signale.
}
{medrekonst}{-0.5cm}
}

Die Instationarit"at der Zeitreihen l"a"st sich auch an "Anderungen der Verteilungen
$n(x)$ der Me"swerte "uber verschiedene Zeitr"aume beobachten, wobei die Gr"o"se $n(x)$
besagt wieviele Me"swerte in ein enges Intervall um den Me"swert $x$ fallen.
\psref{medinstat} zeigt jeweils die Verteilungen sowohl f"ur die gesamte als auch getrennt
f"ur die erste und zweite H"alfte der Zeitreihe. Zwischen den Verteilungen f"ur beide
H"alften bestehen signifikante Unterschiede. Sowohl die Lage des Maximums
als auch die Form der Verteilung variieren zwischen den beiden H"alften der Zeitreihen 
deutlich. Dies l"a"st sich auch quantitativ "uber die Mittelwerte und die
Streuungen der Verteilungen ausdr"ucken. Bei der Zeitreihe des regelm"a"sigen Atemsignals
betr"agt der Unterschied der Mittelwerte beider H"alften $1,2$ Prozent und der Unterschied
der Streuungen $9,5$ Prozent, wobei die Prozentangaben jeweils auf die Streuung
der gesamten Zeitreihe bezogen sind. Bei
der Zeitreihe aus periodischer Atmung betragen die entsprechenden Unterschiede $5,5$
Prozent bei den Mittelwerten und $55,8$ Prozent bei den Streuungen. Zum Vergleich: Bei
einer Zeitreihe des Lorenz-System mit vergleichbarer L"ange betrugen die gemessenen
Unterschiede nur $0,5$ bzw. $0,4$ Prozent.

Es sei an dieser Stelle noch darauf hingewiesen, da"s die bei diesen beiden
Zeitreihen beobachteten Instationarit"aten noch relativ gering sind und bei anderen
Zeitreihen weit gravierender ausfallen.

\afterpage{
\epsfigdouble
{anwendung/instat/cmphsregel}
{anwendung/instat/cmphsperiod}
{ Verteilungen $n$ der Me"swerte $x$ f"ur das regelm"a"sige Atmen (oben) und das periodische 
  Atmen (unten). Die Kurve f"ur die gesamte Zeitreihe (4 min) ist jeweils durchgezogen, f"ur die beiden 
  H"alften der Zeitreihe (je 2 min) jeweils gestrichelt dargestellt.
}
{medinstat}{-0.5cm}
}

\subsection{Bestimmung der Korrelationsintegrale und -dimensionen}
F"ur die Atemsignale wurden Korrelationsintegrale bis zu einer Einbettungsdimension
$\embed=8$ berechnet. Zwar sind bei dem gegebenen Datenumfang maximal
Korrelationsdimensionen bis $\corrdim\simeq 4$ berechenbar, was auch nur eine
Einbettungsdimension von $d=4$ erfordern w"urde, jedoch l"a"st die Berechnung  bis $d=8$
erkennen, ob die berechneten Integrale bzw.\  Dimensionen konvergieren oder nicht.

Die zu den Parameterwerten $d_\tmax=8$, $\delay=0,32$ s und $W=60$ berechneten
Korrelationsintegrale zeigt \psref{medcorrint}. In dem Korrelationsintegral der
regelm"a"sigen Atmung scheint ein Skalierungsbereich zwischen $\ln r$-Werten von  $-4$ und 
$-3$ zu liegen (siehe \psref{medcorrint} oben), bei der periodischen Atmung k"onnte man einen Skalierungsbereich zwischen
$-4,5$ und $-3$ vermuten (siehe \psref{medcorrint} unten).
Ob die Steigung des Korrelationsintegrals in diesen Bereichen wirklich konstant ist,
l"a"st sich jedoch besser an den entsprechenden Slopeplots erkennen erkennen. F"ur die
regelm"a"sige Atmung existiert (entgegen dem bei den Korrelationsintegralen gewonnenen Eindruck)
kein gr"o"serer Bereich von $r$-Werten
mit konstantem $\corrdim(r)$; weiterhin ist keine Konvergenz der Steigungen festzustellen
(siehe \psref{medcorrslp} oben). F"ur die
periodische Atmung scheint bei etwa $\ln r=-4$ Konvergenz aufzutreten (siehe
\psref{medcorrslp} unten). Dies ist jedoch eher
als Artefakt des bei h"oheren Einbettungsdimensionen auftretenden \begriff(Randeffekts)
zu deuten. Auch die "uber Takens' Sch"atzverfahren zum Maximalabstand $\ln\rmax=-2,5$ bestimmten
Dimensionen zeigen keine Konvergenz (siehe \psref{medcorrdim}). 


\epsfigdouble{anwendung/corrint/ci_r_3}{anwendung/corrint/ci_p_1}
{
Korrelationsintegrale der beiden ausgew"ahlten Zeitreihen f"ur regelm"a"sige (oben) und
periodische (unten) Atmung mit den Parameterwerten $\delay=0,32$ s und $W=60$ f"ur die
Einbettungsdimensionen $d=1,\dots,8$ (jeweils von oben nach unten).
}
{medcorrint}{-0.5cm}

\epsfigdouble{anwendung/corrint/cs_r_3}{anwendung/corrint/cs_p_1}
{
Slopekurven zu den Korrelationsintegralen aus \psref{medcorrint} f"ur die 
Einbettungsdimensionen $d=1,\dots,8$ (jeweils von unten nach oben).
}
{medcorrslp}{-0.5cm}

\epsfigdouble{anwendung/corrint/td_r_3}{anwendung/corrint/td_p_1}
{
"Uber Takens' Sch"atzverfahren berechnete Korrelationsdimension  zu den
Korrelationsintegralen aus \psref{medcorrint} mit dem Parameterwert $\ln\rmax=-2,5 $.
}
{medcorrdim}{-0.5cm}



Die Berechnungen wurden auch mit Variationen der Parameterwerte sowie mit verschiedenen
auf die Zeitreihen angewandten Filtertechniken
ausgef"uhrt. Die nachfolgend aufgez"ahlten Parametereinstellungen und Filtermethoden
wurden auf verschiedene Weisen kombiniert, f"uhrten jedoch zu keiner
Verbesserung der Ergebnisse.
\begin{description}
\item[Filterung:] Es wurden Zeitreihen mit und ohne SVD-Filterung verwendet, wobei f"ur
  die Singular Value Decomposition Einbettungsdimensionen zwischen $d=5$ und $d=30$
  gew"ahlt wurden. Weiterhin wurden Tiefpa"sfilterungen mit Grenzfrequenz bei 2,5 Hz
  durchgef"uhrt, um Beeinflussungen durch die Herzdynamik zu minimieren.
\item[Stationarit"at:] Biologische Zeitreihen sind, wie gesehen, oftmals instation"ar und weisen von
  der eigentlichen Dynamik unabh"angige langreichweitige Schwankungen (Bias) auf. 
  Um diese herauszufiltern wurden Fenster "uber die Zeitreihe gelegt, in denen dieser
  Grundwert berechnet und nachfolgend von den Signalwerten subtrahiert wurde.
\item[Verz"ogerungszeiten:] Kleine Verz"ogerungszeiten bringen bei der Berechnung der
  Korrelationsintegrale oftmals besseres Ergebnisse als die berechneten
  Verz"ogerungszeiten. Es wurden Verz"ogerungszeiten bis hinunter zu $\delay=0,1$ s
  getestet.
\item[Autokorrelationszeit:] Um Effekte durch Autokorrelationen auszuschlie"sen, wurde
  der Parameter f"ur die Autokorrelationsl"ange $W$ von 1 bis zu Vielfachen der
  Verz"ogerungszeit ausgetestet.
\end{description}
Da eine endliche fraktale Dimension Voraussetzung f"ur die Existenz eines seltsamen
Attraktors ist und die Korrelationsanalyse (die Korrelationsdimension ist eine untere
Absch"atzung f"ur die fraktale Dimension) keine konvergenten Ergebnisse erbrachte, k"onnen wir somit
nicht auf die Existenz eines solchen schlie"sen. 

F"ur den Atemtypus der regelm"a"sigen Atmung ist noch eine andere (sehr vorsichtige)
Deutung m"oglich. Es k"onnte sich hierbei um eine Grenzzyklus handeln, der jedoch stark
verrauscht ist. Dies st"unde sowohl im Einklang mit dem gemessenen Fourier-Spektrum, das
relativ klar hervortretende Peaks aufweist, als auch mit 2 bzw.\ 3-dimensionalen
Rekonstruktionen, die eine toroidale Struktur zeigen. Zudem weisen nicht-chaotische Systeme
einen verk"urzten Skalierungsbereich im Korrelationsintegral auf: Das Korrelationsintegral
skaliert nur "uber einen Bereich der Ordnung \order{N} statt der Ordnung \order{N^2} wie
im chaotischen Fall \cite{Theiler}.

\subsection{Vorhersagbarkeit von Apnoen}
Da die im vorigen Abschnitt berechneten Korrelationsintegrale keinen Skalierungsbereich
aufwiesen und auch keine Konvergenz der Slopekurven vorlag, konnte die Existenz eines
niedrigdimensionalen Attraktors mit bestimmter Korrelationsdimension $\corrdim$ f"ur die
Atmungsdynamik nicht nachgewiesen werden. Betrachtet man den Wert der f"ur eine bestimmte
Einbettungsdimension $d$ und eine bestimmte Verz"ogerungszeit $\delay$ berechneten
Korrelationsdimension $\corrdim$ jedoch nicht als wirkliche Dimension des zugrunde
liegenden dynamischen Systems sondern als ein Ma"s, da"s die relative Komplexit"at der
Dynamik in dem durch die Zeitreihe gegebenen Zeitabschnitt beschreibt, k"onnen die f"ur
verschiedene zeitlich versetzte Zeitreihen berechneten Werte "Anderungen dieses
Komplexit"atsmasses anzeigen und so eventuell eine Vorhersage von Atemstillst"anden
erm"oglichen.

Zum Test dieser M"oglichkeit wurde eine Zeitreihe, in der zwei Apnoen kurz
hintereinander auftraten,
verwendet (siehe \psref{medtestapnoe} oben). Die L"ange dieser Zeitreihe betrug 9 Minuten
(statt der "ublicherweise benutzen 4 min"utigen Zeitreihen), damit auch in  gr"o"seren
Zeitabst"anden vor Auftreten der Apnoen die Korrelationsdimensionen berechnet werden
konnten. Aus dieser Zeitreihe wurden St"ucke von jeweils 2 Minuten L"ange
extrahiert, wobei die Anfangszeiten dieser Zeitreihen jeweils um 15 Sekunden verschoben
wurden. Jede dieser Zeitreihen erstreckt sich somit "uber einen Zeitraum von $t_{i,a}=i
15\sek$ bis $t_{i,e}=i 15\sek + 120\sek$. Es wurden nun die
entsprechenden Korrelationsintegrale zur Einbettungsdimension $d=5$ berechnet und die Korrelationsdimensionen $D_{2,i}$ "uber Takens' 
Sch"atzverfahren mit dem Maximalabstand $\ln\rmax=-2,5$
bestimmt. Die Berechnung der Korrelationsdimension geschieht immer "uber einen bestimmten
Zeitraum, so da"s eine Auftragung der Korrelationsdimension "uber der Zeit nicht ganz
trivial ist.
Da zur Berechnung der Korrelationsdimension die Zeitreihe jedoch komplett vorliegen
mu"s, ist es sinnvoll bei dieser Auftragung jeweils das zeitliche Ende der Reihe als Wert auf
der Zeitachse zu w"ahlen, d.h. man betrachtet die Auftragung von $D_{2,i}$ "uber
$t_{i,e}$. Das Ergebnis einer solchen Berechnung zeigt \psref{medtestapnoe} (unten).

\epsfigdouble
{anwendung/apnoe/apn2}
{anwendung/apnoe/dimensionen}
{
Oben die verwendete Zeitreihe mit Apnoen bei $t=390\sek$ und $t=460\sek$. Unten die
zeitabh"angige Korrelationsdimension $D_2(t)$ f"ur $t\geq 120\sek$. F"ur $t<120\sek$ lie"s 
sich die Berechnung aufgrund der L"ange der Zeitreihen von 2 min nicht durchf"uhren.
}
{medtestapnoe}{-0.5cm}

Die Werte der Korrelationsdimension f"ur $t\leq390\sek$ schwanken in einem Bereich
zwischen ca.\ $4,0$ und ca.\ $4,3$, wobei in diesen Schwankungen kein bestimmter Trend auszumachen ist.
Ferner liegen die Werte in einem Bereich, der auch bei anderen Zeitreihen, in denen
keine Apnoen auftreten, typisch ist. Erst bei $t=405\sek$ zeigt sich ein deutlicher Abfall
der Korrelationsdimension auf ca.\ $2,1$, die sich nachfolgend langsam wieder zu h"oheren
Werten hin entwickelt.  Da der Atemstillstand jedoch schon bei $t=400\sek$ eintritt, ist
die Korrelationsdimension folglich f"ur die Vorhersage von Apnoen nicht zu gebrauchen.


\subsection{Determinismustest "uber AAFT-Surrogate}

Trotz der negativen Ergebnisse bei der Dimensionsanalyse k"onnen die f"ur bestimmte
Einbettungsdimensionen $d$ und f"ur einen bestimmten Maximalabstand $\rmax$ berechneten Werte der
Korrelationsdimension zum Vergleich mit entsprechenden Werten von Surrogatdaten im Rahmen
eines Determinismustests herangezogen werden. Da die Dichteverteilung der Me"swerte eine
deutlich nicht-gau"ssche Verteilung zeigen (siehe \psref{medinstat}) wurden als Surrogatdaten
AAFT-Surrogate verwendet. Die Anzahl der erstellten Datenreihen betrug in beiden F"allen
(d.h. f"ur regelm"a"sige bzw.\ periodische Atmung) 39. F"ur diese wurden mit den gleichen
Parametern ($\embed_\tmax=8, \delay=0,32, W=60)$ wie f"ur die Originalzeitreihen
Korrelationsintegrale berechnet. Aus diesen wurde "uber Takens' Sch"atzverfahren die
Korrelationsdimension $\corrdim$ in Abh"angigkeit von der Einbettungsdimension bestimmt. Die
Maximalabst"ande lagen bei $\rmax=-2,5$. Den Vergleich der Dimensionsberechnungen zwischen
Original- und Surrogatdaten zeigt \psref{medsurrodim}. Deutlich zu erkennen ist, da"s sich 
f"ur beide Atemtypen bei Einbettungsdimensionen $d\geq4$ die Korrelationsdimension der
Originaldaten stark von den Korrelationsdimensionen der Surrogatdaten unterscheidet.
Aus der Verteilung der Korrelationdimensionen wurde gem"a"s \eqnref{eqnsigni} die
Signifikanz $\mathcal S$ der Abweichungen berechnet (siehe \psref{medsurrosigni}).
Au"ser f"ur $d=7$ und $d=8$ bei der regelm"a"sigen Atmung und $d=1,\dots,3$ bei der
periodischen Atmung liegt die Signifikanz sehr hoch ($\mathcal{S}>5$). 
Die entsprechenden $p$-Werte (siehe \eqnref{eqnpvalue}) liegen in diesen F"allen unter
$10^{-7}$. 

Da die $p$-Werte angeben, mit welcher Wahrscheinlichkeit die Ablehnung der Nullhypothese
falsch ist, k"onnen wir mit "uber $99,9$ prozentiger Sicherheit davon ausgehen, da"s es
sich bei der Atmungsdynamik um kein durch ein ARMA-Modell beschreibbares stochastisches
System handelt. Selbst bei Zugrundelegung des vorher festlegegten Signifikanzniveaus
$\alpha=0,05$, das ja f"ur die minimale Anzahl der berechneten Surrogatzeitreihen
ausschlaggebend ist, liegt die Wahrscheinlichkeit, da"s der Atmung ein deterministischer
Charakter zugesprochen werden kann, bei mindestens $95$ Prozent.

\epsfigdouble{anwendung/surrogat/regel/cmpsurcdim}{anwendung/surrogat/period/cmpsurcdim}
{
Vergleich der berechneten Korrelationdimensionen zwischen Original- \gpmarkb\  und
Surrogatdaten \gpmarka\  f"ur regelm"a"sige Atmung (oben) und periodische Atmung (unten). 
}
{medsurrodim}{-0.5cm}

\epsfigdouble{anwendung/surrogat/regel/signi}{anwendung/surrogat/period/signi}
{
Signifikanz $\mathcal{S}$ der Abweichung der Korrelationsdimension der Originaldaten vom
Mittelwert der Korrelationsdimensionen der Surrogatdaten f"ur regelm"a"sige Atmung (oben)
und periodische Atmung (unten).
}
{medsurrosigni}{-0.5cm}

\comment{
{anwendung/surrogat/regel/pvalue}
{anwendung/surrogat/period/pvalue}
}










%\addchap{Schluß} 

In der vorliegenden Arbeit konnte gezeigt werden, daß die bei Frühgeborenen
haupt\-säch\-lich auftretenden Atemrhythmen (regelmäßige und periodische Atmung)  nicht durch stochastische Prozesse modellierbar
sind: Die Frühgeborenenatmung besitzt mit sehr hoher Wahrscheinlichkeit einen
deterministischen Charakter. Trotz dieses zugrunde liegenden Determinismus konnte die
genaue Struktur der Dynamik nicht identifiziert werden. Eine verläßliche Bestimmung der
Korrelationsdimension als Grundlage für eine mögliche Beschreibung des Atmungssystems
durch einen seltsamen Attraktor erwies sich als  unmöglich. Ferner ist die Korrelationdimension
als möglicher Indikator für eine Vorhersage von Atemstillständen nach den Resultaten dieser Arbeit 
nicht geeignet. Als Gründe dafür, daß  die verwendeten Methoden keinen Erfolg hatten, lassen sich mehrere
Punkte anführen:

\begin{itemize}
\item Die Atmung, auf die reine Lungenaktivität reduziert, stellt kein autonomes System
  dar. In der Dynamik spielen noch weitere Kontrollparameter eine Rolle, die aus dem TI-Signal
  nicht rekonstruierbar sind. Insofern ist eine der Grundlagen für die Verwendung der
  Verzögerungskoordinatenabbildung verletzt, da in diesem Fall selbst beliebig viele verzögerte
  Werte nicht den kompletten Systemzustand wiedergeben können.  Eventuell müßte hier
  der Zustand der atemkompetenten Neuronengruppen in eine Analyse miteinbezogen werden.
  
\item In die gleiche Richtung wie der vorige Punkt geht die Vermutung, daß
  weitere psychische und physische Einflüsse existieren, die während der Aufnahme der Zeitreihe variieren können.
  Dies äußert sich in zeitlichen Änderungen der natürlichen Dichteverteilungen in den
  Phasenraumrekonstruktionen. Hierdurch sind Verfahren wie Dimensionsalgorithmen nicht
  mehr anwendbar, da sie auf stationären Dichten beruhen.
  
\item Das gemessene Signal unterliegt einer Reihe weiterer nicht atemkorrelierter
  Einflüsse, wie beispielweise der Herztätigkeit. Ein Signal, in dem solche Einflüsse
  ausgeschlossen werden können (z.B. Volumenstrommessungen), mag bessere Ergebnisse
  liefern als die thorakale Impedanz.
  
\item Im Gegensatz zu anderen physiologischen Vorgängen, wie der Herztätigkeit, kann die
  Atmung leicht bewußt beeinflußt werden. So können Vorgänge, wie Schlucken,
  Lautäußerungen oder Innehalten, die Aussagekraft des Signals bezüglich der
  Atmungsdynamik stark beeinträchtigen.
\end{itemize}

Trotz der negativen Ergebnisse bei dem Versuch einer Systemidentifikation der
Atmungsdynamik ist durch den
Nachweis ihres deterministischen Charakters ein wichtiger Grundstein gelegt. Für
weitergehende Analysen sind jedoch besser geeignete Signale und bessere Filtertechniken
vonnöten (siehe beispielsweise \cite{Rao92}).  Eine weitere interessante Möglichkeit ist
die Modellierung chaotischer Systeme über selbstlernende neuronale Netze
(Back\-pro\-pa\-ga\-tion-Netze), die -- zumindest für einfache Systeme -- sehr gut in der 
Lage sind, die Dynamik aus kleinen, stark verrauschten Datenreihen zu extrahieren
\cite{Albano92}. Dieser gerade im Hinblick auf seine Robustheit 
gegenüber Rauschen vielversprechende Ansatz könnte bei der Analyse der Atmungsdynamik
eine große Hilfe sein.





%%\nocite{}
\nocite{Kantz97}
\nocite{Kaplan95}
\nocite{Schuster88}
\nocite{Lueke92}
\nocite{Ott94a}

%%%%%%%%%%%%%%%%%%%%%%%%%%%%%%%%%%%%%%%%%%%%%%%%%%
%% Anh"ange

\backmatter

\begin{appendix}
\chapter{Medizinische Fachbegriffe} 

Die meisten der hier erl"auterten medizinischen Fachbegriffe stammen aus ``Pschyrembel --
Klinisches W"orterbuch'' \cite{Pschyrembel}.  F"ur Fr"uhgeborene spezifische "Anderungen
oder Erweiterungen der Definitionenen sind aus \autor(Poets) (1993) \cite{Poets93} und
\autor(Hoch) und \autor(Bergmann) (1996) \cite{Hoch96} erg"anzt worden.

\begin{description}
\item[Apnoe:] Atemstillstand.
\item[Atmung, periodische:] Atmung mit Abwechselnd auftretenden mehreren tiefen Atemz"ugen
  und darauffolgender kurzer apnoischer Pause.
\item[Bradykardie:] langsame Schlagfolge des Herzens mit einer Pulsfrequenz unter 60/min,
  bei Fr"uhgeborenen schon ab unter 90-100/min.
\item[Epidemiologie:] Wissenschaftszweig, der sich mit der Verteilung von "ubertragbaren
  und nicht"ubertragbaren Krankheiten und deren physikalischen, chemischen, psychischen
  und sozialen Determinanten und Folgen in der Bev"olkerung befa"st.
\item[Gestationsalter:] Schwangerschaftsdauer, Reifezeichen des Neugeborenen.
\item[Hypox"amie:] niedriger Sauerstoffpartialdruck im arterielle Blut ($\mathrm{pO_2}<70$
  mmHg). Bei Neugeborenen ein Abfall auf unter 40 -- 45 mmHg bzw.\  unter 20\% des Basalwertes. 
\item[idiopathisch:] ohne erkennbare Ursache entstanden, Ursache nicht nachgewiesen.
\item[Konzeptionsalter:] Lebensalter mit Beginn der Empf"angnis.
\item[Neonatologie:] Teilgebiet der Kinderheilkunde, das sich mit Diagnose und Therapie von
  Erkrankungen des Neugeborenen befa"st.
\item[Pathophysiologie:] Lehre von den krankhaften Lebensvorg"angen im menschlichen
  Organismus.
\item[pathologisch:] krankhaft.
\item[Pr"avalenz:] Anzahl der Erkrankungsf"alle einer best.\  Erkrankung bzw.\
  H"aufigkeit eines best.\  Merkmals zu einem best.\  Zeitpunkt oder innerhalb einer
  best.\  Zeitperiode.
\item[Pulsoxymetrie:] transkutane (unblutige) Messung der arteriellen
  Sauerstoffs"attigung.
\item[QRS-Komplex:] Phase der Erregungsausbreitung in den Herzkammern.
\item[REM-Schlaf:] Abk"urzung f"ur engl.\  Rapid Eye Movement. Schlafphase mit raschen
  Augenbewegungen und erh"ohter Herz- und Atemfrequenz.
\item[thorakal:] zum Brustkorb geh"orig.
\item[Thorax:] Brustkorb.
\item[transkutan:] durch die Haut hindurch.
\item[Zyanose:] blau-rote F"arbung von Haut und Schleimh"auten infolge einer Abnahme des
  Sauerstoffgehalts im Blut.
\end{description}


\end{appendix}


%%%%%%%%%%%%%%%%%%%%%%%%%%%%%%%%%%%%%%%%%%%%%%%%%%
%% Literaturverzeichnis

\newpage
\addcontentsline{toc}{chapter}{Literaturverzeichnis}
%\selectlanguage{english}
\bibliography{bib/tsa}
%\bibliographystyle{myalpha}
%\bibliographystyle{plain}
\bibliographystyle{./bst/mysiam}

\end{document}
















\documentclass[a4paper,12pt]{scrreprt}

% Befehle, die in Standard, aber nicht in den KOMA classes zur Verf"ugung stehen
\def\frontmatter{}
\def\mainmatter{}
\def\backmatter{}

% Deutsche Layouts und Trenntabellen
\usepackage{german}

% Zum einbinden von PostScript-Grafiken
\usepackage{graphicx}
\usepackage{psfrag}

% Bindet die AMS-fonts ein um Symbole f"ur reelle und complexe Zahlen
% zur Verf"ugung zu haben
\usepackage{amsfonts}

% Definiert \layout Befehl: coole Sache um Seitenlayout zu "uberpr"ufen
\usepackage{layout}

% Definiert befehl (\newtheroem) um theorem-umgebungen zu erstellen
\usepackage{theorem}

% Einige packages die ich vielleicht noch brauchen werde
% indentfirst, fancyheadings, float, subfigure, psfig

% Allgemeines zeug und trennungen
\usepackage{trennungen}
\usepackage[themabf,newint,normsect]{elmar}


%%%%%%%%%%%%%%%%%%%%%%%%%%%%%%%%%%%%%%%%%%%%%%%%%%
%%%%%%%%%%%%%%%%%%%%%%%%%%%%%%%%%%%%%%%%%%%%%%%%%%

\begin{document}

%%%%%%%%%%%%%%%%%%%%%%%%%%%%%%%%%%%%%%%%%%%%%%%%%%
%% Titelseite

\frontmatter

\titlepage
\subject{Diplomarbeit angefertigt am Institut f"ur Theoretische Physik
I\\Wilhelm-Klemm-Str.\  9}
\title{Grundlagen der Zeitreihenanalyse und ihre Anwendung auf die Atmungsdynamik Fr"uhgeborener}
\author{Vorgelegt von \\ Elmar Zander}
\date{\today}

\maketitle

%%%%%%%%%%%%%%%%%%%%%%%%%%%%%%%%%%%%%%%%%%%%%%%%%%
%% Inhaltsverzeichnis
\tableofcontents


%%%%%%%%%%%%%%%%%%%%%%%%%%%%%%%%%%%%%%%%%%%%%%%%%%
%% Hauptteil

\mainmatter

%\addchap{Einleitung}

Komplexes Verhalten tritt in nahezu allen Bereichen der Natur in Erscheinung. W"ahrend die
Physik diesen Ph"anomenen fr"uher relativ ratlos gegen"uberstand, konnten in den letzten
Jahren durch die Erkenntnisse der Chaostheorie auf diesem Gebiet gro"se Fortschritte
erzielt werden. Die Theorie nichtlinearer dynamischer Systeme leistete hier Wesentliches,
indem sie die oftmals auftretenden komplexen Dynamiken mathematisch-physikalisch erfa"sbar
und charakterisierbar machte.  Hierbei ist besonders das Konzept \begriff(seltsamer Attraktoren)
hervorzuheben.  Seltsame Attraktoren bilden die Menge der im Langzeitverhalten
erreichbaren Phasenraumpunkte dissipativer chaotischer Systeme und besitzen im allgemeinen
eine sehr verwickelte fraktale Struktur. Damit einher geht ein Verlust der Langzeitvorhersage
durch die Eigenschaft dieser Systeme, kleine St"orungen exponentiell zu verst"arken.  Die
Erkenntnis, da"s solche Strukturen auch bei deterministischen Systemen mit wenigen
Freiheitsgraden auftreten k"onnen, hat in vielen Gebieten der Naturwissenschaften, auch
au"serhalb der Physik, die Frage aufgeworfen, ob die beobachteten Verhaltensweisen und
Strukturen durch diese Konzepte zu erkl"aren seien. (Kapitel~1.1.)


In den empirischen Naturwissenschaften liegen h"aufig nur unzureichende Daten "uber den
genauen Zustand des beobachteten Systems vor, oftmals in Form \begriff(skalarer
Zeitreihen) einer gemessenen Zustandsvariable.  Eine entscheidende Voraussetzung f"ur die
Charakterisierung dieser Zeitreihen mittels der Methoden der nichtlinearen Dynamik ist
die M"oglichkeit, aus diesen wieder Attraktoren rekonstruieren zu k"onnen.  Dies geschieht
"uber das Konzept der sogenannten \begriff(Verz"ogerungskoordinaten), welches durch eine
mathematische Theorie abgesichert werden konnte. Die Voraussetzungen der mathematischen
Theorie sind in der experimentellen Praxis jedoch nicht erf"ullbar. Neben der st"andigen
Anwesenheit von Rauschen ist vor allem die zeitliche Begrenztheit der zur Verf"ugung
stehenden Zeitreihen ein Problem. Aufgrund dieser Einschr"ankungen ist eine gute Wahl der
Einbettungsparameter vonn"oten.  Desweiteren m"ussen M"oglichkeiten zur Verminderung des
Rauschens gefunden werden. (Kapitel~1.2.)

F"ur die Charakterisierung seltsamer Attraktoren steht eine Vielzahl verschiedener Ma"szahlen zur
Verf"ugung. Die drei am h"aufigsten herangezogenen sind die fraktale Dimension, die
Kolmogorov-Entropie und der Lyapunov-Exponent. Diese beschreiben respektive die Anzahl
der effektiven Freiheitsgrade, die Rate der Informationsproduktion sowie die exponentielle
Separation der Trajektorien des Systems. Da die Komplexit"at eines Systems meistens mit
der Anzahl seiner Freiheitsgrade korreliert ist, wendet sich diese Arbeit besonders den
verschiedenen Begriffen fraktaler Dimension zu.  Im Bereich der nichtlinearen
Zeitreihenanalyse wird zumeist die Korrelationsdimension verwendet, die durch ihre schnelle
und stabile Berechenbarkeit durch den Grassberger-Procaccia-Algorithmus wesentliche
Bedeutung erlangt hat. Probleme stellen sich hierbei haupts"achlich durch die endliche
Datenmenge, wei"ses und Digitalisierungsrauschen und sogenannte Randeffekte.  (Kapitel~1.3.)

Damit die Verwendung der beschriebenen Methoden "uberhaupt einen Sinn macht, ist es
notwendig zu wissen, da"s der beobachteten Zeitreihe ein deterministisches System zugrunde
liegt.  Zur L"osung des Problems existieren eine Vielzahl von Vorschl"agen, von denen hier
der Methode der Surrogatdaten auf der Basis statistischer Hypothesentests der Vorzug gegeben
wird.  Diese bieten ein breites Fundament, um Fragen nach der Struktur und
zugrunde liegendem Determinismus experimenteller Zeitreihen beantworten zu k"onnen. Durch
die Verf"ugbarkeit hoher Rechenleistungen in modernen Computern lassen sich eine Vielzahl
verschiedener Hypothesen durch Vergleich der Originalzeitreihen mit sogenannten
Surrogatdaten testen. (Kapitel~1.4.)

Die Ergebnisse dieser Arbeit sollen auf das konkrete Beispiel der Atmungsdynamik
Fr"uhgeborener angewandt werden. Das Ziel
hierbei ist es zu kl"aren, ob die vorgestellten Methoden aus der Zeitreihenanalyse bei der
Untersuchung dieser Problematik von Nutzen sein k"onnen.
Konkret geht es hierbei um folgende Fragen:
\begin{itemize}
\item Ist die den verschiedenen Atmungstypen zugrunde liegende Dynamik durch ein
  deterministisches System beschreibbar ?
\item K"onnen f"ur die verschiedenen Atmungstypen, regelm"a"siges und periodisches Atmen, die
  fraktale Dimension berechnet und  die Existenz eines seltsamen Attraktors belegt werden ?
\item Lassen sich signifikante "Anderungen der Dimension vor dem Auftreten von
  Atemstillst"an\-den (Apnoen) feststellen ?
\end{itemize}
Eine Bejahung dieser Fragen k"onnte zu einer m"oglichen Fr"uherkennung von
Atemstillst"anden f"uhren, was f"ur die Fr"uhgeborenenmedizin von gr"o"stem Nutzen w"are.
(Kapitel~2)







%\clearpage
\chapter{Grundlagen der Zeitreihenanalyse}


%%%%%%%%%%%%%%%%%%%%%%%%%%%%%%%%%%%%%%%%%%%%%%%%%%%%%%%%%%%%%%%%%%%%%%%%%%%%%%%%%%%%%%%%%%%%%%%%%%%%
%% Dynamische Systeme
%%%%%%%%%%%%%%%%%%%%%%%%%%%%%%%%%%%%%%%%%%%%%%%%%%%%%%%%%%%%%%%%%%%%%%%%%%%%%%%%%%%%%%%%%%%%%%%%%%%%

\section{Seltsame Attraktoren}
\label{chapdynsystems}
Das Konzept seltsamer Attraktoren hat in den letzten beiden Jahrzehnten eine
st"andig wachsende Bedeutung bei der
Erkl"arung von Ph"anomenen aus den unterschiedlichsten Forschungsgebieten gefunden. Gemeinsam
ist allen diesen Ph"anomenen, da"s die untersuchten Systeme prinzipiell durch sogenannte
\begriff(Evolutionsgleichungen)
\eqnl[evolut]{\dot\x = \vec F(\x)}
beschrieben werden k"onnen. Die Geschwindigkeit $\dot\x$, mit der sich der Zustand des Systems entwickelt, und damit
auch die ganze zuk"unftige Entwicklung\footnote{Unter hinl"anglich allgemeinen Forderungen
an das Vektorfeld $\vec F$, n"amlich der Lipschitz-Stetigkeit.}, ist determiniert durch
den aktuellen Zustand $\x$. Man sagt auch, das Vektorfeld $\vec F$ erzeuge einen Flu"s
$\flow^t$:
\eqn{\x(t) = \flow^t(\x_0) .}
Der Flu"s beschreibt die zeitliche Entwicklung des Systems in Abh"angigkeit vom
Anfangszustand $\x_0$.
Die Bewegung findet in einem Raum statt, der als \begriff(Phasenraum) $\phase$
bezeichnet wird. Jedem  Freiheitsgrad des Systems entspricht genau eine Koordinate des
Phasenraumes. Die Anzahl der Freiheitsgrade und damit die Dimension des Phasenraumes ist in
vielen F"allen endlich, kann jedoch bei ausgedehnten Systemen\footnote{Gemeint sind hier
  beispielsweise hydrodynamische Systeme deren, 
  Zustand durch Dichte- und Temperaturfelder beschrieben wird.} auch "uberabz"ahlbar
unendlich werden\footnote{Dies ist auch bei eindimensionalen Systemen der Fall, die durch
sogenannte \begriff(Delay-Gleichungen) beschrieben werden, wie beispielsweise die
Mackey-Glass-Gleichung. }.


Bei realen Systemen ist es die Regel, da"s Energie durch Prozesse wie Reibung
verlorengeht. Bei solchen \begriff(dissipativen Systemen) kann man die Beobachtung
machen, da"s sich die Dynamik nach einer transienten Phase auf eine 
niedrigdimensionale Untermannigfaltigkeit $\M\subset\phase$ des Phasenraumes
reduziert. Die Energiedissipation eines Systems ist gleichbedeutend mit einer Kontraktion
des Phasenraumvolumens $V$ beliebiger Teilmengen $\set B\subset\phase$ unter dem Flu"s $\flow^t$:
\eqn{\abl{}{t}V(\flow^t(\set B))\lt 0.}

Dies kann auch "uber das erzeugende Vektorfeld $\vec F$ ausgedr"uckt werden. Die Divergenz
des Vektorfeldes mu"s in den erreichbaren Teilen des Phasenraumes negativ sein:
\eqn{\mathop{\mathrm{div}}\vec F \lt 0.}
Hierf"ur ist es nicht notwendig, da"s die Phasenraumvolumina in alle Richtungen gestaucht
werden. Es k"onnen auch bestimmte Richtungen expandieren, solange die Kontraktion in
die anderen Richtungen "uberwiegt.

Bei vielen Systemen klingt die transiente Phase der Bewegung recht schnell ab. In einem solchen
Fall interessiert man sich meist f"ur das \begriff(Langzeitverhalten) des Systems, und
beschr"ankt sich bei der Untersuchung auf die \begriff(Grenzmenge)\footnote{Die positive
  Grenzmenge $\limit^+$ eines Systems ist definiert als die Menge aller Punkte $\y\in\phase$, f"ur
  die Folgen $t_0\lt t_1\lt t_2\dots$ existieren, so da"s
  $\lim\limits_{i\to\infty}\flow^{t_i}(\x)=\y$ f"ur mindestens ein $\x\in\phase$.}
$\limit^+$, d.h.\ den Bereich des Phasenraumes, in dem sich die Dynamik im Grenzfall
unendlich langer Zeiten abspielt.

Lange herrschte die Auffassung, da"s sich die einzig m"oglichen Grenzmengen aus
Punkt\-attraktoren, Grenzzyklen oder attraktiven $k$-Tori ($k\geq2$) zusammensetzen.
Diesen ist gemeinsam, da"s sie periodisch (in den ersten beiden F"allen) bzw.\ periodisch
oder quasiperiodisch (im letzten Fall) sind. Sie besitzen ein diskretes
\begriff(Fourier-Spektrum).  Die Entstehung von Turbulenz wurde verstanden durch
sukzessive Generation immer neuer Fourier-Moden.  Das komplexe Verhalten turbulenter
Systeme w"urde also als "Uberlagerung sehr vieler inkommensurabler Frequenzen betrachtet
und w"are somit qualitativ nicht verschieden vom Verhalten einfacher Systeme. Bei der
Untersuchung eines (stark vereinfachten) Modells der Konvektion von Luft in der
Atmosph"are stie"s \autor(Lorenz) 1963 \cite{Lorenz63} jedoch auf ein System, das
keinerlei Periodizit"at und somit ein kontinuierliches Fourier-Spektrum aufwies
(siehe~\psref{attrlor}). Diese \begriff(deterministischen nichtperiodischen Fl"usse)
erhielten sp"ater von \autor(Ruelle) und \autor(Takens) \cite{Ruelle71a} den
einpr"agsamen Namen \begriff(seltsame Attraktoren).


\epsfigdouble{attraktoren/lornew}{attraktoren/fournew}
{Links eine typische Trajektorie des Lorenz-Systems. \comment{ zu den 
Standardparameterwerten $\sigma =10$, $r=28$, $b=8/3$} Rechts das Leistungsspektrum $P(\omega)$ der 
$x$-Komponente.}{attrlor}{-0.2cm}

Eine mathematisch exakte Definition seltsamer Attraktoren ist ein schwieriges
Problem. Eine endg"ultige Fassung, auf die sich alle geeinigt h"atten, ist bis heute nicht
gelungen \cite{Pawelzik91}. Die Schwierigkeit liegt darin, die Definition sowohl
mathematisch exakt zu halten, als auch so umfassend, da"s alle wesentlichen Eigenschaften
erfa"st werden. Ein grundlegenderes Problem ist jedoch, da"s selbst "uber die
Eigenschaften seltsamer Attraktoren keine Einigkeit zwischen allen beteiligten Wissenschaftlern
besteht. Wir wollen uns hier deshalb auf einige f"ur diese 
Arbeit wichtige Eigenschaften beschr"anken:
\begin{itemize}
\item Seltsame Attraktoren sind \begriff(mischend).  Ein Attraktor $\attr$ hei"st genau
  dann mischend, wenn es f"ur beliebige, bez"uglich $\attr$ offene\footnote{Eine Menge
    $\A$ hei"st \begriff(offen bez"uglich) einer Menge $\M$, wenn $\A$ ganz in $\M$ liegt
    und es zu jedem $x\in\A$ eine reelle Zahl $\eps>0$ gibt, so da"s der Schnitt der
    $\eps$-Umgebung $\U_\eps(x)$ mit der Menge $\M$ ganz in $\A$ liegt: $\U_\eps(x) \cap
    \M \subset \A$.  }, nichtleere Mengen $\set I,\set J\subset\,\attr$ mindestens einen
  Punkt $\x\in\set I$ gibt, so da"s $\flow^t(\x)\in\set J$ f"ur mindestens ein $t>0$.

\comment{Eine Menge $\A$ hei"st \begriff(offen bez"uglich) einer Menge $\M$, wenn es eine
  offene Menge $\tilde\A$ gibt, so da"s $\A=\tilde\A\cap\M$. Der Begriff \naja(offen
  bez"uglich) stellt eine Verallgemeinerung der Vorstellung dar, da"s sich alle Punkte
  einer offenen Menge \naja(innerhalb) derselben befinden.}

Eine wichtige Folgerung daraus ist, da"s fast alle Punkte
des Attraktors unter dem Flu"s $\flow^t$ jedem anderen Attraktorpunkt beliebig nahe
kommen. Diese Eigenschaft ist notwendig f"ur die Definition \begriff(ergodischer Ma"se)
auf dem Attraktor. 

\item Zwei zu einem gegebenen Zeitpunkt $t_0$ eng beieinander liegende Punkte auf dem Attraktor
werden unter dem Flu"s $\flow^t$ mit der Zeit exponentiell voneinander getrennt. Man spricht von einer 
\begriff(exponentiellen Separation) der Trajektorien oder auch \begriff(sensibler
Abh"angigkeit von den Anfangsbedingungen). Dies findet Ausdruck in der Existenz positiver
Lyapunov-Exponenten.

\item \begriff(Periodische Orbits) liegen dicht auf dem Attraktor. Aufgrund der
  exponentiellen Separation sind diese Orbits jedoch alle instabil. Der Tr"ager eines
  seltsamen Attraktors ist darstellbar als Abschlu"s seiner instabilen periodischen Orbits
  (IPOs).
  
  Diese Tatsache kann im Rahmen der Zeitreihenanalyse zur effizienten Bestimmung von
  charakteristischen Gr"o"sen wie Lyapunov-Exponenten und fraktalen Dimensionen ausgenutzt
  werden. F"ur die Berechnung mu"s nicht die ganze zur Verf"ugung stehende Datenmenge
  herangezogen werden, da oftmals Mittelungen "uber die IPOs ausreichend sind.  Ein
  Verfahren zur Extraktion instabiler periodischer Orbits aus experimentellen Zeitreihen
  findet sich in \cite{Pawelzik91,Pawelzik91a}.
\end{itemize}
Die aufgez"ahlten Eigenschaften sind nicht voneinander unabh"angig. \autor(J. Banks \etal)
konnten zeigen, da"s die exponentielle Separation der Trajektorien aus den beiden anderen
Eigenschaften gefolgert werden kann \cite{Banks92}. Da die Eigenschaften des Durchmischens
als auch das Dichtliegen periodischer Orbits topologische Invarianten sind, gilt
dies somit abenfalls f"ur die exponentielle Separation der Trajektorien. Unter
topologischen Abbildungen bleiben also die dynamischen und geometrischen Eigenschaften
seltsamer Attraktoren unber"uhrt.

Oft werden seltsame Attraktoren auch "uber ihre \begriff(fraktale) Struktur definiert
\cite{Peitgen92}. Diesem Ansatz soll hier nicht gefolgt werden, da er nur geometrische,
jedoch keine dynamischen Aspekte des Attraktors beschreibt. Es existieren Attraktoren, die
seltsam (nach obiger Definition), aber nicht fraktal sind (beispielsweise der
\naja(Attraktor) von \begriff(Arnolds Katzenabbildung)\footnote{Arnolds Katzenabbildung
  ist eine Abbildung eines 2-Torus auf sich selbst \cite{Arnold68}. Die Dynamik ist
  chaotisch. Da das System volumenerhaltend ist, ist der ganze Torus der Attraktor
  \cite{Eckmann-ruelle}.  }). Demgegen"uber ist der
\begriff(Feigenbaum-Attraktor)\footnote{Die logistische Abbildung ist definiert durch
  $f_\mu(x)=\mu x(1-x)$ mit $\mu\in[0,4]$. Sie hat attraktive periodische Punkte der
  Periode $2^n$, wobei $n$ gegen unendlich l"auft f"ur $\mu$ gegen $\mu_\infty=3,57\dots$
  . Der f"ur $\mu=\mu_\infty$ entstehende Attraktor wird als Feigenbaum-Attraktor
  bezeichnet. Er besitzt eine fraktale Struktur, ist jedoch nicht chaotisch, da keine
  sensitive Abh"angigkeit von den Anfangsbedingungen besteht \cite{Eckmann-ruelle}.}
fraktal, aber nicht seltsam.  Die meisten Attraktoren (Lorenz-Attraktor,
R"ossler-Attraktor \dots) sind jedoch sowohl seltsam als auch fraktal.


\section{Rekonstruktionsverfahren}

Die Theorie seltsamer Attraktoren kann oftmals gute Erkl"arungsans"atze f"ur das
Ver\-st"andnis dynamischer Systeme liefern. So kann beispielsweise das Lorenz-System als
Modell f"ur bestimmte Konvektionszellen (sogenannte Rayleigh-B\'enard-Zellen) verwendet
werden\footnote{\autor(Lorenz) benutzte zur Aufstellung seiner Gleichungen das
  \autor(Rayleigh)sche Modell der Str"omungsdynamik in bestimmten rechteckigen
  Fl"ussigkeitszellen. Er betrachtete die Amplituden spezieller L"osungen der
  Modellgleichungen als zeitabh"angig und setzte diese wieder in das Modell ein. Durch
  Streichung bestimmter Terme gelangte er so zu einem System von Differentialgleichungen
  f"ur die zeitabh"angigen Amplituden \cite{Peitgen92}.}. Durch Variation eines Parameters kann der
"Ubergang dieser Zellen von laminarem zu turbulentem Verhalten studiert werden. Die vielen
N"aherungen, die f"ur dieses System gemacht werden, erlauben zwar keine direkten
Voraussagen "uber reale Konvektionszellen, die Dynamik kann jedoch prinzipiell verstanden
werden.

Anders liegt der Fall bei biologischen, medizinischen oder auch komplizierteren
hydrodynamischen Systemen. Hier tauchen eine Reihe von Problemen auf:
\begin{itemize}
\item Die relevanten Phasenraumvariablen sind oft nicht bekannt. Gerade bei
medizinischen und biologischen Systemen ist dies sehr oft der Fall.
\item Die Anzahl der Freiheitsgrade ausgedehnter Systeme (z.B. meteorologischer oder
hydrodynamischer) ist in der Regel abz"ahlbar oder gar "uberabz"ahlbar unendlich. 
\item Selbst bei Kenntnis aller Phasenraumvariablen sind die f"ur die Dynamik zust"andigen
Evolutionsgleichungen unbekannt.
\end{itemize}
Daten, die in experimentellen Situationen gewonnen werden, beschr"anken sich so meistens
auf Me"sreihen einer oder weniger Gr"o"sen, von denen angenommen wird, da"s  sie die
Dynamik des Systems charakterisieren. Stellt sich nun heraus, da"s die so gewonnene
\begriff(Zeitreihe) einen nicht trivialen\footnote{\begriff(Trivial) bedeutet in diesem
  Zusammenhang, da"s sich die Zeitreihe nicht als "Uberlagerung weniger Frequenzen
  darstellen l"a"st, d.h.\  sie hat kein diskretes
Fourier-Spektrum.} zeitlichen Verlauf aufweist, ergeben sich daraus einige Fragen:
\begin{itemize}
\item Liegt der Zeitreihe ein deterministisches System zugrunde, oder ist das erratische
Verhalten eine Folge additiven Rauschens?
\item L"a"st sich das System durch einen seltsamen Attraktor beschreiben? Wenn ja, wie
k"onnen wir diesen aus den vorhandenen Daten rekonstruieren?
\item Wie k"onnen f"ur den rekonstruierten Attraktor charakteristische Gr"o"sen wie
fraktale Dimensionen oder Lyapunov-Exponenten bestimmt werden?
\end{itemize}
Wir wollen uns zuerst mit der zweiten Frage besch"aftigen, wobei wir im folgenden annehmen
(zumindest als Arbeitshypothese), dem System liege ein seltsamer Attraktor zugrunde.
Die erste und die dritte Frage werden wir in sp"ateren Abschnitten angehen.

\subsection{Verz"ogerungskoordinaten (MOD)}

Den ersten Ansatz zur L"osung des Problems der Attraktorrekonstruktion  lieferten
\linebreak \autor(Packard \etal) 1980
mit dem Konzept der \begriff(Verz"ogerungskoordinaten), auch kurz als
MOD (engl.: method of delays) bezeichnet \cite{Packard80}. Um das Verfahren 
zu veranschaulichen, verwenden wir ein System dessen Dynamik uns bereits bekannt ist. An
diesem soll eine einzige Observable\footnote{D.h.\   eine beliebige glatte Funktion
$v:\phase\to\R$ der Phasenraumvariablen.} (numerisch) gemessen werden, die uns als Zeitreihe
dient. Hieraus soll die Dynamik wieder rekonstruiert und mit der urspr"unglichen Dynamik
verglichen werden.

%Als dynamisches System w"ahlen wir 
Das R"ossler-System \cite{Roessler76} ist durch das folgende System von
Differentialgleichungen bestimmt
\eqna{
\dot x &=& -z-x \nonumber \\
\dot y &=& x + a y \nonumber \\
\dot z &=& b + z(x-c) 
}
Das System ist in der dritten Gleichung nichtlinear und wird f"ur den Parametersatz
$a=0.38$, $b=0.3$, $c=4.5$ chaotisch. Der Attraktor besitzt eine fraktale
Struktur\footnote{Durch Analyse der Gleichungen kann man bei diesem System sehr sch"on den Chaos
  erzeugenden \metapher(Streck-und-Falt)-Proze"s erkennen \cite{Peitgen92}.  Die Struktur
  entspricht lokal dem kartesischen Produkt einer 2-dimensionalen Mannigfaltigkeit und einer
  Cantor-Menge.}.

Aus den Differentialgleichungen wird durch numerische Integration ein diskreter
Orbit $\folge(\x,1,N)$\footnotemark erzeugt (siehe \psref{rekroe} oben).  Als Observable $v$ dient die $x$-Komponente der
Punkte $\x$: $v(\x)=x$, so da"s wir eine Zeitreihe $v_i = v(\x(i\sample))$ erhalten. 
Die Zeitreihe wurde zur Verdeutlichung ihres diskreten Charakters durch Punkte
dargestellt (siehe \psref{rekroe} unten).  \footnotetext{Verwendet wird ein Runge-Kutta-Verfahren vierter Ordnung mit
  Schrittweite $\sample=0.009$. Nach einer Transienzzeit von $1000\sample$, nach der sich
  das System dem Attraktor hinreichend gen"ahert hat, wird der Orbit in diskreten
  Schritten $\sample$ aufgezeichnet.  Die Anzahl der berechneten Orbitpunkte betr"agt
  $N=20000$.}


\epsfigdouble{rekonstruktion/roenew}{rekonstruktion/timenew}
{Oben eine Trajektorie des R"ossler-Attraktors aus einer numerischen
Integration. Unten die gemessene Observable $v$.}{rekroe}{-0.2cm}

Die Informationen "uber die Werte der Koordinaten $y$ und $z$ stehen nun nicht mehr zur
Verf"u\-gung. Nach der Idee von \autor(Packard \etal) kann jedoch der Zustand eines
$n$-dimensio\-na\-len Systems zu einer gegebenen Zeit durch jeden Satz von $n$
unabh"angigen, sonst aber beliebigen Koordinaten spezifiziert werden \cite{Packard80}.
Der Zustand des R"ossler-Systems zu einer Zeit $t_i$ sollte also statt durch
$(x_i,y_i,z_i)$ ebenso durch $(x_i,\dot x_i,\ddot x_i)$ oder $(x_i,x_{i+1},x_{i+2})$
beschrieben werden k"onnen.

Wir wollen nun anhand einer einfachen "Uberlegung 
plausibel machen, da"s in den Verz"ogerungskoordinaten
tats"achlich die gleiche Information steckt wie in den originalen Phasenraumkoordinaten.
Wenn die Zeit $\sample$ zwischen zwei Messungen klein ist, gilt n"aherungsweise:
\eqnal[recidea1]{\dot x_i &\simeq& (x_{i+1} - x_i) / \sample \nonumber \\
\ddot x_i &\simeq& (x_{i+2} - 2x_{i+1} + x_i) / \sample^2 ;}
andererseits gilt f"ur die erste und zweite Zeitableitung
\eqnal[recidea2]{ 
\dot x_i &=& f_1(x_i, y_i, z_i) \nonumber \\
\ddot x_i &=& \pabl{F_1}{x} F_1(x_i,y_i,z_i) + \pabl{F_1}{y} F_2(x_i,y_i,z_i) +
\pabl{F_1}{z} F_3(x_i,y_i,z_i) ,} 
wobei die $F_j$ die Komponenten des erzeugenden Vektorfeldes sind (siehe \eqnref{evolut}).
Unter im betrachteten Zusammenhang recht allgemeinen Bedingungen an die $F_j$,
lassen sich $x_i$, $\dot x_i$ und $\ddot x_i$ wieder nach den Phasenraumvariablen $x_i$,
$y_i$ und $z_i$ aufl"osen.  Da andererseits nach \eqnref{recidea1} die Ableitungen durch die
Verz"ogerungskoordinaten bestimmt sind, k"onnen wir den kompletten Systemzustand
$x_i,y_i,z_i$ aus den verz"ogerten Werten der Zeitreihe $x_{i},x_{i+1},x_{i+2}$
\metapher(zur"uckholen). Wenn diese \naja("Aquivalenz) zwischen Original- und
Verz"ogerungskoordinaten gegeben ist, spricht man von einer \begriff(Einbettung) des
Attraktors (n"aheres dazu sp"ater), und bezeichnet den Raum, in den die Zeitreihe
eingebettet wird, (in diesem Fall der $\R^3$) als \begriff(Einbettungsraum).


Die vorstehenden "Uberlegungen k"onnen auch auf beliebige Observable $v$ sowie auf h"oherdimensionale
Einbettungsr"aume $\R^\embed$ erweitert werden. Weiterhin k"onnen gr"o"sere
Zeit\-ab\-st"ande $k\sample$ zwischen den Verz"ogerungskoordinaten gew"ahlt werden.  Man
erh"alt so als Rekonstruktionsvektoren die Folge $(v_{i},
v_{i+k},\dots,v_{i+(\embed-1)k})$. Dieses Verfahren entspricht einer
Abbildung $\diffeo_{k,\embed,v}:\phase\to\R^\embed$ aus dem Original- in den
Rekonstruktionsphasenraum, welche durch
\eqn{\diffeo_{k,\embed,v}(\x) = (v(\x),v(\flow^{k\sample}(\x)), \dots, v(\flow^{(\embed-1)k\sample}(\x)))}
gegeben ist. Man bezeichnet diese Abbildung  als \begriff(Verz"ogerungskoordinatenabbildung).

Wir wollen das Verfahren nun bei dem oben erw"ahnten R"ossler-System anwenden.  Da wir die
Anzahl der Freiheitsgrade des R"ossler-Systems kennen, lassen wir es bei der uns bekannten
und \naja(ausreichenden) Einbettungsdimension $\embed=3$. F"ur die Verz"ogerung\footnote{
  Die in Einheiten der \begriff(Sampling Time) $\sample$ gemessene
  \begriff(Verz"ogerungszeit) $\delay=k\sample$ wird als \begriff(Verz"ogerung) $k$
  bezeichnet. Da Verz"ogerung und Verz"ogerungszeit {\em immer} "uber diese Relation
  eindeutig verkn"upft sind ($\sample$ ist konstant), wird im folgenden je nach Kontext
  der besser geeignete der beiden Begriffe benutzt.} w"ahlen wir $k=30$.

%\afterpage
{\epsfigsingle{rekonstruktion/recnew}
{Rekonstruktion des R"ossler-Attraktors aus der Zeitreihe in \psref{rekroe} (unten) zur
  Einbettungsdimension $\embed=3$ mit Verz"ogerung $k=30$. Die Rekonstruktionspunkte
  wurden zur besseren Darstellung durch gerade Linien verbunden.}{rekrek}{-0.5cm}}

Die Rekonstruktion in \psref{rekrek} "ahnelt einer verzerrten Kopie des
Originalattraktors in \psref{rekroe}. Insofern leistet die
Verz"ogerungskoordinatenabbildung schon gute Dienste. Die rein visuelle "Ahnlichkeit
reicht aber f"ur eine genauere Analyse der Dynamik nicht aus. Es stellen sich mehrere
Fragen, die noch zu beantworten sind
\begin{myitemize}
\item Sind die Dynamik des rekonstruierten und des Originalattraktors zueinander
konjugiert? Mit anderen Worten: Ist die Verz"ogerungskoordinatenabbildung $\diffeo_{k,\embed,v}$
ein Diffeomorphismus?
\item Sind fraktale Dimensionen und Lyapunov-Exponenten unter der Rekonstruktion durch 
die Verz"ogerungskoordinatenabbildung invariant?
\item Wie sind die Einbettungsparameter $d$ und $k$ zu w"ahlen, wenn das urspr"ungliche
System nicht bekannt ist?
\end{myitemize}
Diese Fragen sollen in den n"achsten Abschnitten beantwortet werden.




%%%%%%%%%%%%%%%%%%%%%%%%%%%%%%%%%%%%%%%%%%%%%%%%%%%%%%%%%%%%%%%%%%%%%%%%%%%%%%%%%%%%%%%%%%%%%%%%%%%%
%% Einbettungen
%%%%%%%%%%%%%%%%%%%%%%%%%%%%%%%%%%%%%%%%%%%%%%%%%%%%%%%%%%%%%%%%%%%%%%%%%%%%%%%%%%%%%%%%%%%%%%%%%%%%

\subsubsection{Einbettungen}

Bevor wir uns mit den Einbettungstheoremen besch"aftigen, soll erst einmal der Begriff der
\begriff(Einbettung) selbst gekl"art werden. Bei einer Einbettung handelt es sich immer um
eine Abbildung (einer Menge) aus einem Phasenraum in einen anderen Phasenraum. Nun sollen unter
dieser Abbildung (und auch unter der Umkehrabbildung) keine Punkte kollabieren, d.h. es
sollen keine verschiedenen Originalpunkte auf den selben Bildpunkt abgebildet werden.
Solche Abbildungen, die die topologischen Eigenschaften von Punktmengen invariant lassen
bezeichnet man als \begriff(Hom"oomorphismen). Weiterhin sollen durch die Einbettung auch
keine Tangentenrichtungen kollabieren, was beispielsweise f"ur die Bestimmung von
Lyapunov-Exponenten von Bedeutung ist.  Zus"atzlich wird also die stetige
Differenzierbarkeit der Abbildung und der Umkehrabbildung gefordert. Eine
Einbettung mu"s somit ein $\sm^1$-Diffeomorphismus sein.




Wir wollen uns nun mit dem Problem besch"aftigen, unter
welchen Voraussetzungen die Verz"ogerungskoordinatenabbildung $\diffeo_{k,\embed,v}$ eine
Einbettung im obigen Sinne ist. Den ersten Ansatz zur Beantwortung dieser Frage lieferte
\autor(Takens) 1980 \cite{Takens80}.  \comment{Sein erstes Theorem soll hier vollst"andig zitiert
werden}
\comment{
\begin{theorem}
Sei $\M$ eine kompakte Mannigfaltigkeit der Dimension $\mandim$. F"ur Paare $(\flow,v)$,
wobei  \linebreak[4] $\flow:\M\to\M$ ein  glatter Diffeomorphismus und $v:\M\to\R$ eine
glatte Funktion ist, ist es eine generische Eigenschaft, da"s die 
Abbildung $\diffeo_{(\flow,v)} : \M \to\R^{2\mandim+1}$, definiert durch 
\eqnl[takmod]{\diffeo_{(\flow,v)}(\x) = (v(\x),v(\flow^1(\x)), \dots, v(\flow^{2\mandim}(\x))),}
eine Einbettung ist; \metapher(Glatt) bedeutet hier mindestens $\sm^2$.
Hierbei werden zus"atzlich folgende Voraussetzungen an den Flu"s $\flow$ gestellt:
\begin{myitemize}
\item Wenn $\x$ periodischer Punkt der Periode $k\le 2\mandim+1$ ist, sind alle Eigenwerte
von $\mathrm{D}\flow^k(\x)$ paarweise verschieden und verschieden von 1.
\item F"ur verschiedene Fixpunkte $\x^*$ von $\flow$, sind auch die $v(\x^*)$
verschieden\footnote{Der Satz gilt entsprechend f"ur durch $\sm^2$-Vektorfelder erzeugte Fl"usse,
wobei sich die beiden Voraussetzungen leicht auf das Vektorfeld "ubertragen lassen.}.
\end{myitemize}
\end{theorem}
}

\begin{theorem}
  Sei $\M$ eine kompakte Mannigfaltigkeit der Dimension $\mandim$. F"ur Paare $(\vec
  F,v,\delay)$, wobei $\vec F$ ein glattes Vektorfeld, $v:\M\to\R$ eine glatte Funktion
  und $\delay>0$ eine reelle Zahl ist\footnotemark, ist es eine generische Eigenschaft, da"s die
  Abbildung $\diffeo_{(\vec F,v,\delay)}: \M \to\R^{2\mandim+1}$, definiert durch
  \eqnl[takmod]{\diffeo_{(\vec F,v,\delay)}(\x) = (v(\x),v(\flow^\delay(\x)), \dots,
    v(\flow^{2\mandim\delay}(\x))),} eine Einbettung ist, wobei $\flow^t$ der durch $\vec
  F$ erzeugte Flu"s ist; \metapher(Glatt) bedeutet hier mindestens $\sm^2$.  "Uber das
  Vektorfeld $\vec F$ werden folgende zus"atzliche Annahmen gemacht:
\begin{myitemize}
\item Wenn $\vec F(\x)=0$ ist, dann sind alle Eigenwerte von $\mathrm{D}\flow^\delay(\x)$ paarweise verschieden und verschieden von 1.
\item Kein periodischer Orbit von $\vec F$ hat eine Periode $n\delay\, (n\in\N)$ mit $n\leq2\mandim+1$.
\end{myitemize}
\end{theorem}
\footnotetext{\autor(Takens) formulierte und bewies das Theorem f"ur die Verz"ogerungszeit 
  $\delay=1$. Da die Zeit jedoch immer entsprechend umskaliert werden kann,
  erschien es mir sinnvoll, das Theorem gleich f"ur beliebige Verz"ogerungszeiten
  $\delay>0$ zu formulieren.}

Die Abbildung $\diffeo_{(\vec F,v,\delay)}$ im obigen Theorem entspricht der
Verz"ogerungskoordinatenabbildung $\diffeo_{k,\embed,v}$.  Da wir "uber das Vektorfeld
$\vec F$ keine Kenntnis haben, k"onnen wir die zus"atzlichen Annahmen des Theorems nicht
verifizieren. Nach \autor(Takens) sind diese jedoch auch unter generischen Bedingungen
erf"ullt. 

Das Theorem versichert uns also, da"s f"ur generische Vektorfelder $\vec F$,
Me"sfunktionen $v$ und Verz"ogerungszeiten $\delay>0$, die
Verz"ogerungskoordinatenabbildung $\diffeo_{(\vec F,v,\delay)}$ eine Einbettung ist. Da im
Experiment nur in diskreten Zeitschritten $\sample$ gemessen werden kann, steht uns jedoch
nicht, wie vorausgesetzt, eine kontinuierliche Funktion $v(t)$, sondern nur die diskrete
Me"sreihe $v_i=v(i\sample)$ zur Verf"ugung. Wir m"ochten nun wissen, ob auch die
Grenzmenge der diskreten Folge konjugiert zu der des Originalsystems ist. Dies wird durch
ein Korollar zu Takens' viertem Theorem beantwortet.

\begin{corollar}
Sei $\M$ eine kompakte Mannigfaltigkeit der Dimension $\mandim$. Wir betrachten Viertupel,
bestehend aus einem Vektorfeld $\vec F$, einer Funktion $v$, einem Punkt $\x$ und einer
positiven reellen Zahl $\delay$. F"ur generische $(\vec F, v, \x, \delay)$ ist die
positive Grenzmenge $\limitp(\x)$ \naja(diffeomorph) zu der Grenzmenge der folgenden
Sequenz im $\R^{2m+1}$
\eqn{ \left\{ \left( v(\flow^{i\delay}(\x)),v(\flow^{(i+1)\delay}(\x)), \dots
,v(\flow^{(i+2m)\delay}(\x))    \right) \right\}^\infty_{i=0} }
\naja(Diffeomorph) hei"st hier: es gibt eine glatte Einbettung von $\M$ nach $\R^{2m+1}$,
die $\limitp(\x)$ bijektiv auf die Grenzmenge dieser Punktfolge abbildet.
\end{corollar}

Wir k"onnen nun schlie"sen, da"s f"ur generische Verz"ogerungen $k$ und Me"sfunktionen
$v$ die Verz"ogerungskoordinatenabbildung $\diffeo_{k,\embed,v}$ eine Einbettung des
Attraktors liefert, sofern nur $\embed\geq2m+1$ ist\footnote{Zur Bestimmung von $m$ siehe
  Abschnitt \ref{chapparams}~.}.
Der Begriff \naja(generisch) ist jedoch relativ schwach und sagt nichts "uber die
Wahrscheinlichkeit aus, da"s dies tats"achlich der Fall ist, wenngleich man dies gerne
meinen m"ochte. 

Dies soll genauer erl"autert werden.  Der Audruck \naja(generisch) beschreibt die H"aufigkeit
des Auftretens bestimmter Eigenschaften bei Elementen einer Menge $\A$.
Eine \label{generisch} Eigenschaft hei"st bereits dann generisch auf $\set A$, wenn
eine \begriff(residuale Teilmenge)\footnote{Eine Menge $\set R\subset\set A$ hei"st
\begriff(residuale) Teilmenge von $\set A$, wenn $\set R$ abz"ahlbarer Durchschnitt
offener, dichter Teilmengen von $\set A$ ist. Residuale Mengen sind selbst wieder dicht, jedoch nicht notwendig offen. }
 $\set R$ von $\set A$ existiert, so
da"s alle Elemente von $\set R$ diese Eigenschaft aufweisen \cite{Liebert91}.
Zu jedem Element aus $\set A$ findet man also in
jeder endlichen, beliebig kleinen Umgebung ein Element aus $\set R$, das diese
Eigenschaft aufweist. Dies sagt jedoch nichts "uber die
Wahrscheinlichkeit, ein Element dieser Menge zuf"allig zu treffen. Es existieren Beispiele,
in denen diese Wahrscheinlichkeit sogar null wird. So ist beispielsweise
die Menge $\Omega_{\text{stab}}$ der Parameterwerte $\omega\in[0,2\pi]$, f"ur die die
eindimensionale Kreisabbildung
\eqn{g_{\omega,k}(x)=x+\omega+k\sin(x) \nonumber}
stabile Orbits besitzt, eine residuale Teilmenge von $[0,2\pi]$. F"ur $k\to0$ verschwindet
das Lebesgue-Ma"s von $\Omega_{\text{stab}}$, die Wahrscheinlichkeit, ein
$\omega$ aus $[0,2\pi]$ zuf"allig so zu w"ahlen, da"s $g_{\omega,k}$ stabile Orbits besitzt (d.h.\
$\omega\in\Omega_{\text{stab}}$), geht demnach gegen null \cite{Sauer91}.


F"ur den Experimentator ist die Zusicherung aus Takens' Korrolar somit nicht ausreichend,
da nach den obigen Ausf"uh\-rungen, Generizit"at nichts "uber die Wahrscheinlichkeit, da"s
hier wirklich eine Einbettung vorliegt, aussagt. Wir m"ochten sicher sein, da"s die
Verz"ogerungskoordinatenabbildung mit der Wahrscheinlichkeit eins eine Einbettung ist.

Um auszudr"ucken, eine Eigenschaft treffe mit Wahrscheinlichkeit eins auf die Elemente
einer Menge $\set A$ zu, sagen wir, die Eigenschaft sei \begriff(pr"avalent) auf $\set A$. 
Da die Definition diese Begriffs auch f"ur "uberabz"ahlbar dimensionale Mengen sinnvoll
sein soll, kann er nicht "uber das verschwindende Lebesgue-Ma"s der Komplement"armenge
definiert werden, da ein solches hier nicht existiert. Die folgende Definition ist
entnommen aus \autor(Sauer \etal) \cite{Sauer91}:

\begin{definition}
Eine Borel-Teilmenge $\set A$ eines normierten Vektorraumes $\set V$ ist \begriff(pr"avalent), wenn
es einen endlich dimensionalen Untervektorraum $\set E$ aus $\set V$ gibt, so da"s f"ur alle $v$ aus
$\set V$ gilt, $v+e\in \set A$ f"ur fast alle $e$ aus $\set E$.
\end{definition}
Den Unterraum $\set E$ bezeichnet man als \begriff(Testraum) (engl.: probe space). Die Pr"avalenz
einer Eigenschaft  kann man sich nun folgenderma"sen vorstellen. Sei irgendein
Punkt $v$ aus $V$ vorgegeben, dann kann man von da aus in jede beliebige Richtung aus $E$
\naja(wandern) und trifft mit Wahrscheinlichkeit eins auf einen Punkt aus $S$. Da mit $E$
auch jeder Untervektorraum $E'$, der $E$ enth"alt, ein Testraum ist, ist leicht
einzusehen, da"s die Pr"avalenz einer Eigenschaft f"ur endlich dimensionale Vektorr"aume zu
der "ublichen Definition von \naja(f"ur fast alle) bzw.\ \naja(mit Wahrscheinlichkeit
eins) "aquivalent ist.


Aufgrund des oben beschriebenen Mankos von Takens' Theorem bewiesen \autor(Sauer),
\autor(Yorke) und \autor(Casdagli) ein erweitertes Einbettungstheorem \cite{Sauer91}, das
sogenannte \begriff(Fractal Delay Embedding Prevalence Theorem). Wir geben hier die (etwas 
verst"andlichere) Version aus ``Coping with chaos'' \cite{Ott94} wieder:
\comment{
\begin{theorem}
  Ein kontinuierliches dynamisches System besitze eine kompakte, invariante glatte
  Mannigfaltigkeit $\M$ der Kapazit"at $\fracdim$, und sei $n>2\fracdim$. Sei $\delay$ die
  Verz"ogerungszeit.  $\set A$ enthalte h"ochstens eine endliche Anzahl Fixpunkte, keine
  periodischen Orbits der Periode $\delay$ oder $2\delay$, und h"ochstens endlich viele
  periodische Orbits der Periode $3\delay,4\delay,\dots,n\delay$, und die Jakobimatrizen der
  Wiederkehrabbildungen dieser periodischen Orbits haben verschiedene Eigenwerte.  Dann
  gilt f"ur fast alle glatten Me"sfunktionen $v:\set U\to\R$, da"s die
  Verz"ogerungskoordinatenabbildung $\diffeo(v,\flow,\delay):\set U\to\R^n$
  (siehe~\eqnref{takmod}) eine Einbettung ist.
\end{theorem}
}

\begin{theorem}
  Sei $\flow$ ein Flu"s auf einer offenen Teilmenge $\set U$ des $\R^k$ und sei $\set A$
  eine kompakte Teilmenge von $\set U$ der Kapazit"at $\fracdim$ \comment{(siehe Abschnitt
    \ref{chapcapacity})}. Sei $d>2\fracdim$ ganzzahlig und $\delay>0$. $\set A$ enthalte
  h"ochstens eine endliche Anzahl Fixpunkte, keine periodischen Orbits der Periode
  $\delay$ oder $2\delay$, und h"ochstens endlich viele periodische Orbits der Periode
  $3\delay,4\delay,\dots,d\delay$, und die Jakobimatrizen der Wiederkehrabbildungen der
  periodischen Orbits haben paarweise verschiedene Eigenwerte. Dann
  gilt f"ur fast alle glatten Me"sfunktionen $v:\set U\to\R$, da"s die
  Verz"ogerungskoordinatenabbildung $\diffeo(\vec F,\flow,\delay):\set U\to\R^d$
  (siehe~\eqnref{takmod}) eine Einbettung ist.
\begin{myitemize}
\item eineindeutig auf $\set A$ ist.
\item auf jeder kompakten Teilmenge $\set C$ einer in $\set A$ enthaltenen 
  glatten Mannigfaltigkeit eine Einbettung ist.
\end{myitemize}
\end{theorem}

Die wesentlichen Unterschiede zu Takens' Theorem sind erstens, da"s statt glatter
Mannigfaltigkeiten der Dimension $\mandim$ kompakte Mengen der Kapazit"at $\fracdim$
(welche erstere beinhalten)betrachtet werden, und zweitens, da"s generisch durch die
st"arkere Eigenschaft pr"avalent ersetzt wird.



%
\subsubsection{Verfahren zur Wahl der Einbettungsparameter}

In den Einbettungstheoremen im vorigen Abschnitt wurde implizit vorrausgesetzt, da"s wir
unendliche, rauschfreie Zeitreihen zur Verf"ugung haben. Dies ist jedoch bei Daten aus
realen Experimenten niemals der Fall. F"ur verrauschte, endliche Zeitreihen ist eine
vern"unftige Wahl der Verz"ogerungszeit $\delay$, die in den Einbettungstheoremen (bis auf
wenige Ausnahmen) beliebig sein kann, wesentlich. Au"serdem ist die Dimension des
einzubettenden Attraktors unbekannt und somit auch die Anzahl der ben"otigten
Verz"ogerungskoordinaten. Die folgenden Abschnitte werden sich mit diesen Punkten
besch"aftigen.


\paragraph{Die Einbettungsdimension}

Nach Takens' Einbettungstheorem ist f"ur Einbettungsdimensionen\footnote{Mit
  Einbettungsdimension ist hier und im folgenden die Dimension unseres
  Rekonstruktionsphasenraumes gemeint, gleichg"ultig ob es sich bei der Abbildung in
  diesen Raum um eine Einbettung handelt oder nicht.} $\embed$ gr"o"ser als $2\fracdim$,
wobei $\fracdim$ die Kapazit"at des Attraktors ist, sichergestellt, da"s die
Verz"ogerunsgkoordinatenabbildung eine Einbettung ist. Nun ist bei experimentellen
Zeitreihen die Kapazit"at nicht a priori bekannt. Wir m"u"sten die Zeitreihe erst
einbetten, um die Kapazit"at bestimmen zu k"onnen.

Es stellt sich au"serdem die Frage, ob nicht schon kleinere Einbettungsdimensionen
$\embed\lt 2\fracdim$ Einbettungen liefern. Wie wir am Beispiel des R"osslerattraktors
gesehen haben, reicht hier eine Einbettungsdimension von $\embed=3$, w"ahrend Takens'
Theorem eine Einbettungsdimension von $\embed=5$ erfordert. Die Bedingung
$\embed>2\fracdim$ ist nur notwendig, um absolut sicher zu gehen, da"s die Abbildung eine
Einbettung ist.

Um zu einer gegebenen Zeitreihe die optimale Einbettungsdimension zu bestimmen sind eine
ganze Reihe von Verfahren entwickelt worden. Ein paar von diesen wollen wir im folgenden
vorstellen.
\begin{myitemize}
\item \rem(Falsche n"achste Nachbarn:) Diese Methode beruht darauf, da"s unter
  Einbettungen, die Nachbarschaftsbeziehungen zwischen benachbarten Punkten nicht
  ver\-"an\-dert werden \cite{Kennel92}. Wenn zwei Punkte im Originalphasenraum benachbart
  sind, so sind sie es auch in der Einbettung. Wenn nun $\embed$ eine ausreichende
  Einbettungsdimension ist, gilt dies auch f"ur $\embed+1$. Zwei Punkte die im $\R^\embed$
  benachbart sind, sollten also auch im $\R^{\embed+1}$ benachbart sein. Sind sie es
  nicht, kann die Abbildung in den $\R^\embed$ keine Einbettung gewesen sein.
  
  Man bestimmt nun zu einer Einbettungsdimension $\embed$ zu jedem Punkt $\x$ die
  \begriff(n"achsten Nachbarn), die innerhalb einer Umgebung $\eps$ von $\x$ liegen. Die
  Nachbarpunkte, die beim "Ubergang zur Einbettungsdimension $\embed+1$ aus der
  $\eps$-Umgebung von $\x$ \metapher(entweichen), werden als \begriff(falsche n"achste
  Nachbarn) bezeichnet. F"ur ausreichende Einbettungsdimensionen sollte das Verh"altnis
  zwischen falschen und \metapher(echten) n"achsten Nachbarn sehr klein werden und wir
  k"onnen $\embed$ als Einbettungsdimsension annehmen.
  
\item \rem(Attraktorvolumen:) Bei diesem Verfahren wird als erstes ein Ma"s f"ur das
  \metapher(Volumen) des rekonstruierten Attraktors definiert. $\buzvol{k,\embed}$ ist das
  mittlere von je $\embed+1$ Attraktorpunkten aufgespannte Volumen zur Verz"ogerung $k$ .
  Das Volumen, das von den $\embed+1$ Attraktorpunkten $\folge(\x,0,\embed)$ aufgespannt
  wird, betr"agt $\abs{\det\left(\x_1-\x_0,\dots,\x_\embed-\x_0\right)}$. Da die Anzahl
  der Kombinationen von $d+1$ Punkten, exponentiell mit der Einbettungsdimension steigt,
  wird das mittlere Volumen durch eine gro"se $(>10^4)$ Zahl von Kombinationen
  abgesch"atzt. Falls es durch die Rekonstruktion zu "Uberschneidungen von Trajektorien
  kommt, schrumpft das aufgespannte Volumen. Wenn nun beim "Ubergang zu einer h"oheren
  Einbettungsdimension diese "Uberschneidung nicht mehr auftritt, "au"sert sich dies in
  einem Sprung des Volumenma"ses. In einer Auftragung von $\buzvol{k,\embed}$ kann man so
  die notwendige Einbettungsdimension ablesen \cite{Buzug90a} \cite{Buzug94}.
  \comment{\item \rem(Waberproduktanalyse:) \cite{Liebert89}\cite{Liebert91}}
\item \rem(Singular Value Decomposition:) Bei der \begriff(Singular Value Decomposition)
  wird zuerst eine Einbettung in einen hochdimensionalen Rekonstruktionsraum $\R^\embed$
  vorgenommen \cite{broomhead-king}. F"ur niedrigdimensionale Dynamiken wird sich diese
  jedoch nur in einem Unterraum $\subspace$ der Dimension $\minembed$ abspielen. Dieser
  Unterraum wird nun ermittelt und der Attraktor hieraus in den $\R^\minembed$ projiziert.
  Das Verfahren wird ausf"uhrlicher in Kapitel \ref{chapsvd} diskutiert.
\end{myitemize}
Die Reihe der Verfahren lie"se sich nahezu unendlich fortsetzen. Ein gute "Ubersicht
findet sich bei \autor(Buzug) \cite{Buzug94}. Wir wollen nun eins der noch nicht
aufgef"uhrten Verfahren genauer betrachten, da es sich u.a.\ gut dazu eignet,
Schwierigkeiten und Probleme bei der Analyse experimenteller, chaotischer Systeme zu
demonstrieren.

Bei diesem von \autor(Packard \etal) \cite{packard80} entwickelten Verfahren stellen wir
uns den Attraktor eingebettet in eine $\mandim$-dimensionale Mannigfaltigkeiten $\M$ vor.
Diese Mannigfaltigkeit sei wiederum eingebettet in den euklidischen Vektorraum
$\R^\embed$.  Schnitte von $\M$ mit $(\embed-1)$-di\-men\-sio\-nalen Hyperfl"achen
erzeugen nun im allgemeinen $(\mandim-1)$-dimensionale Mannigfaltigkeiten\footnote{Dies
  ist nur dann nicht der Fall, wenn der Schnitt leer ist oder die Hyperfl"ache tangential
  zu der Mannigfaltigkeit liegt.  Das erste werden wir im folgenden durch die Wahl der
  Hyperfl"achen ausschlie"sen. Letzteres ist generisch nicht der Fall.}.  Wenn wir die
Mannigfaltigkeit nun mit $\mandim$ paarweise orthogonalen Hyperfl"achen schneiden, wird
die Schnittmenge auf einen Punkt reduziert. Dies offeriert eine M"oglichkeit die Dimension
der Mannigfaltig $\M$, in der der Attraktor $\attr\subset\M$ liegt, zu bestimmen.

Als Schnittfl"achen betrachten wir die zu den Koordinatenachsen orthogonalen Teil\-r"aume
des $\R^\embed$. Die Tatsache, da"s ein Punkt innerhalb der Schnittmenge von $\mandim'$
dieser Teil\-r"aume mit der Mannigfaltigkeit $\M$ liegt, bringt uns Kenntnis "uber
$\mandim'$ seiner Koordinaten. Genau $\embed-\mandim'$ der Koordinaten sind unbestimmt,
wobei von diesen allerdings nur $\mandim-\mandim'$ Koordinaten unabh"angig sind, da die
Mannigfaltigkeit ja $\mandim$-dimensional ist. Wenn wir also $\mandim$ Schnitte
betrachten, sind dadurch alle Koordinaten eines Punktes aus $\M$ festgelegt.

Diese Tatsache kann nun durch \begriff(bedingte Wahrscheinlichkeiten) ausgedr"uckt werden.
Wir legen $\mandim'$ Koordinaten $x^0_1,\dots,x^0_{\mandim'}$ fest. Die
Rekonstruktionspunkte, deren erste $\mandim'$ Koordinaten gleich den $x^0_i$ sind, bilden
die Schnittmenge des Attraktors mit den durch $\mathcal{S}_i=\left\{ \x \vert
  x_i=x^0_i\right\}$ gegebenen Hyperfl"achen. Die Wahrscheinlichkeit, da"s ein Punkt aus
dieser Schnittmenge als $(\mandim'+1)$.\ Komponente den Wert $x$ aufweist, bezeichnen wir
mit $\Prob_{\mandim'}(x)=\Prob(x\vert x_1=x^0_1,\dots,x_{\mandim'}=x^0_{\mandim'})$. Mit
den anf"anglichen "Uberlegungen, da"s die genauen Koordinaten erst durch $\mandim$
Schnitte festgelegt sind, l"a"st sich nun schlie"sen, da"s die Verteilung von
$\Prob_{\mandim'}(x)$ f"ur $\mandim'<\mandim$ ausgedehnt, f"ur $\mandim'=\mandim$ dagegen
singul"ar werden mu"s.

F"ur die Berechnung der Verteilungen m"ussen wir aufgrund der endlichen Datenmenge (und
auch der endlichen Genauigkeit) von der exakten Gleichheit der Koordinaten abgehen und
Wahrscheinlichkeitsverteilungen $\Prob(i\vert i^0_1,\dots,i^0_{\mandim'})$ betrachten.
Hierbei ist der Wertebereich von $x$ in Intervalle $I_i=[x_i,x_{i+1}[$
aufgeteilt\footnote{Da wir hier Verz"ogerungskoordinaten betrachten, ist der Wertebereich
  f"ur alle Koordinaten gleich und es macht Sinn die Intervalle f"ur alle Koordinaten
  gleich zu w"ahlen.}.\@ $\Prob(i\vert i^0_1,\dots,i^0_{\mandim'})$ ist die
Wahrscheinlichkeit, da"s die $\mandim'+1.$ Koordinate im Intervall $I_i$ liegt, unter der
Bedingung, da"s die ersten $\mandim'$ Koordinaten in den Intervallen
$I_{i^0_1},\dots,I_{i^0_{\mandim'}}$ liegen.

Das Ergebnis einer solchen Berechnung ist in \psref{pakdim} am Beispiel des
R"osslerattraktors dargestellt. Die Anzahl der Intervalle (bins) betr"agt $200$, die
Vergleichskoordinaten wurden zu $x^0_1=0.0$ und $x^0_2=5.0$ (entspricht $i^0_1=124,
i^0_2=196$) gew"ahlt. F"ur die Verz"ogerungszeit $\delay=32\sample$ (\psref{pakdim} oben)
sieht man deutlich, da"s die Verteilung f"ur $\mandim'=2$ sehr scharf wird. Dies steht im
Einklang mit der lokalen Struktur des R"osslerattraktors, n"amlich einer Cantormenge
zweidimensionaler Schichten. Die Tatsache, da"s $\Prob$ auch neben dem Maximalwert nicht
verschwindet, ist einerseits bedingt durch die endliche Rechengenauigkeit als auch durch
die \emph{eben nicht} exakt zweidimensionale Struktur des R"osslerattraktors.

In \psref{pakdim} (unten) sind die Wahrscheinlichkeitsverteilungen f"ur
$\delay=35\sample$, sonst aber gleiche Parameter dargestellt. Man erkennt deutlich ein
Problem dieses Verfahrens. F"ur geringf"ugig\footnote{Das diese Abweichung in Bezug auf
  die Bestimmbarkeit optimaler Verz"ogerungszeiten wirklich \begriff(geringf"ugig) ist,
  wird sich in Kapitel \ref{chapdelay} zeigen.} andere Verz"ogerungszeiten wird die
Verteilung $\Prob(i\vert i_1,i_2)$ nicht mehr singul"ar, sondern zeigt mehrere Maxima. Wir
m"u"sten f"ur diese Verz"ogerungszeit schlie"sen, da"s der Attraktor mindestens
dreidimensional ist.

\noafterpage{
  \epsfigfour{packdim/packdim1}{packdim/packdim2}{packdim/packdimb1}{packdim/packdimb2} {
    Bedingte Wahrscheinlichkeitsverteilungen f"ur den R"osslerattraktor aus \psref{rekroe}
    mit $x^0_1=0.0$ und $x^0_2=5.0$. Die Verz"ogerungszeit wurde oben zu
    $\delay=32\sample$ und unten zu $\delay=35\sample$ gew"ahlt. Die Aufl"osung betr"agt
    $0,5$ Prozent der linearen Ausdehnung des R"osslerattraktors. Deutlich zu erkennen ist
    die empfindliche Abh"angigkeit des Ergebnisses vom Parameter $\delay$.
    }{pakdim}{-0.2cm} }

Diese Tatsache w"are nicht so interessant, wenn die abweichenden Resultate bei
unterschiedlichen Parametern nicht ein f"ur viele Verfahren der Zeitreihenanalyse
auftretendes Ph"anomen w"are. Es stellen sich hier mehrere Fragen.
\begin{myitemize}
\item Inwiefern ist dieses Ergebnis nur ein Artefakt unserer speziellen Parameterwahl? In
  dem hier betrachteten Fall taucht f"ur die meisten $\delay$ eine breite Verteilung auf.
  Ist nun eher dem Ergebnis f"ur das spezielle $\delay$, welches $\mandim=2$ impliziert,
  oder den Ergebnissen f"ur andere Verz"ogerungszeiten, welche eher $\mandim>2$ nahelegen,
  zu trauen? In diesem Fall ist dies durch unsere Systemkenntnis nat"urlich leicht zu
  entscheiden, aber wie sieht es bei unbekannten Systemen aus ?
\item Damit aussagekr"aftige Verteilungen erzeugt werden k"onnen, m"ussen in der
  betrachteten Schnittmenge eine \naja(ausreichende) Zahl von Rekonstruktionspunkten
  liegen. Diese Zahl nimmt jedoch mit $(\Delta x)^{\mandim'}$ ab. Wir m"ussen also
  entweder mit einer geringeren Genauigkeit $(\Delta x)$ arbeiten, was h"aufig nicht
  akzeptabel ist, oder die Datenmenge entsprechend erh"ohen. Dies ist jedoch bei
  experimentellen -- insbesondere bei biologischen oder medizinischen -- Systemen oft
  nicht m"oglich.
\end{myitemize}

Das erste Problem der Parameterwahl wird in abgewandelter Form noch "ofter auftreten. Das
zweite -- die exponentielle Zunahme der erforderlichen Datenmenge -- wird genauer
behandelt in Abschnitt \ref{chapcorrdim}.

\paragraph{Die Verz"ogerungszeit}
\label{chapdelay}

Nach Takens' Theoremen ist die Wahl der Verz"ogerungszeit bis auf wenige, in der Regel
erf"ullte Ausnahmen beliebig. Bei endlichen, eventuell verrauschten Daten ist die Wahl
einer \naja(guten) Verz"ogerungszeit entscheidend f"ur eine erfolgreiche
Phasenraumrekonstruktion. In \psref{rekzeit} sind drei verschiedene Rekonstruktionen des
R"osslerattraktors aus der Zeitreihe aus \psref{rekroe} zu den Verz"ogerungszeiten
$\delay=3\sample$, $30\sample$ und $200\sample$ abgebildet.

\epsfigtriple{zeit/rectslow}{zeit/rectsmed}{zeit/rectshigh} {Rekonstruktion des
  R"osslerattraktors zu verschiedenen Verz"ogerungszeiten $\delay=3\sample, 30\sample,
  200\sample$ (von links nach rechts). Man erkennt deutlich, da"s sich f"ur kleine
  $\delay$ der Attraktor auf die Raumdiagonale konzentriert, w"ahrend f"ur gro"se $\delay$
  "Uberfaltung des Attraktors eintritt. Die Rekonstruktionen wurden hier in den $\R^2$
  projiziert.  }{rekzeit}{-0.2cm}

Da f"ur kleine Verz"ogerungen $k=\delay/\sample$ die Koordinaten ann"ahernd gleich werden
$x_i\simeq x_{i+k}\simeq x_{i+2k}$, konzentrieren sich die Rekonstruktionspunkte in der
linken Abbildung auf die Hauptdiagonale des $\R^3$. F"ur einen Dimensionsalgorithmus, der
nur mit endlicher Genauigkeit arbeiten kann, w"are dies kaum zu unterscheiden, von einer
eindimensionalen Struktur. In der rechten Abbildung ist die Verz"ogerungszeit dagegen sehr
hoch gew"ahlt.  Der Attraktor scheint wesentlich komplizierter als der des
Originalsystems. Bei rauschfreien Daten mag das noch unwesentlich sein. Bei Vorhandensein
von dynamischem Rauschen werden die Koordinaten der rekonstruierten Punkte jedoch mit
steigendem $\delay$ zunehmend unkorreliert -- der Attraktor spannt den ganzen Phasenraum
auf. Aufgrund dieser Problemem ben"otigen wir also Verfahren, um vern"unftige
Verz"ogerungszeiten bestimmen zu k"onnen.

Bevor ich mit der Diskussion verschiedener Verfahren zur Bestimmung der Verz"ogerungszeit
beginne, m"ochte ich erst ich erst eine pers"onliche Bemerkung zu all diesen Methoden
machen. Es wird immer wieder geschrieben, dieses oder jenes Verfahren erbr"achte
\slang(bessere) Resultate (d.h. \slang(optimalere) Verz"ogerungszeiten) als irgendein
anderes. Teilweise sind sogar Ma"szahlen entwickelt worden, die die \slang(Optimalit"at)
verschiedener Verz"ogerungszeiten beweisen sollen. Da diese Kriterien immer auf dem Modell
beruhen, da"s der Autor von einer guten Verz"ogerungszeit hat, wird dadurch nat"urlich
immer die "Uberlegenheit des gerade vorgestellten Verfahrens bewiesen.

Meiner Meinung nach gibt es \slang(die) optimale Verz"ogerungszeit nicht. Welcher Wert
optimal ist, h"angt davon ab, was mit der Zeitreihe geschehen soll. Sollen m"oglichst gut
aufgespannte Poincar\'e-Plots erstellt werden? Sollen die Lyapunov-Exponenten des Systems
berechnet werden? Soll die Kapazit"at oder die Korrelationsdimension bestimmt werden? Die
vorgestellten Verfahren liefern in der Regel einen guten Ausgangspunkt f"ur verschiedene
weitergehende Berechnungen. Der Wert der endg"ultig gew"ahlt wird, ist dann aber meistens
eine Sache von \metapher(trial-and-error)\footnote{Bei der Berechnung der
  Korrelationsdimension bietet es sich zum Beispiel an, erst einen Wert f"ur $\delay$
  "uber das Verfahren der \begriff(marginalen Redundanz) zu bestimmen und dann aus Tests
  mit benachbarten Verz"ogerungszeiten diejenigen, die den gr"o"sten Skalierungsbereich
  liefert, zu w"ahlen.}.

\subparagraph{Nulldurchgang der Autokorrelationsfunktion.} Ausgangspunkt f"ur die
Entwicklung der Verz"ogerungskoordinatenabbildung war die Idee, da"s die Dynamik des
Systems durch beliebige, \emph{unabh"angige} Koordinaten dargestellt werden kann. Es w"are
daher sinnvoll, die Verz"ogerung so zu w"ahlen, da"s die Koordinaten m"oglichst
unabh"angig werden. In der Signaltheorie wird die \begriff(lineare Unabh"angigkeit) zweier
Signale $X$ und $Y$ "uber deren \begriff(Kreuzkorrelation) $C(X,Y)$ beschrieben.
\eqnl[crosskorr]{C(X,Y)=\lim_{T\to\infty}\frac{1}{2T}\int_{-T}^{+T} (x(t)-\bar
  x)(y(t)-\bar y) \mathrm{dt} } 
Hierbei sind $x(t)$ und $y(t)$ die Werte der Signale zur
Zeit $t$ und $\bar x$ und $\bar y$ die Mittelwerte der Signale \footnotemark. Die
Kreuzkorrelation wird maximal, wenn die Signale proportional zueinander sind, und Null,
wenn sie linear unabh"angig sind. Bei der Verz"ogerungskoordinatenabbildung ist das Signal
$Y$ nun genau das um die Zeit $\delay$ verschobene Signal $X$ d.h.\ $y(t)=x(t+\delay)$.
Die rechte Seite von \eqnref{crosskorr} geht damit "uber in die Definition der
Autokorrelationsfunktion $\ac$: \footnotetext{Teilweise wird die Kreuzkorrelation noch
  "uber den Kehrwert der Standardabweichungen von $X$ und $Y$ auf eins normiert. Dies ist
  f"ur die weiteren Ausf"uhrungen jedoch ohne Belang.}
\eqnl[autokorr1]{\ac(X,\delay)=\lim_{T\to\infty}\frac{1}{2T}\int_{-T}^{+T} (x(t)-\bar
  x)(x(t+\delay)-\bar x) \mathrm{dt} } F"ur diskrete, endliche Zeitreihen
$\{x_i\}_{i=1\dots N}\,$der L"ange $N$, erhalten wir
\eqnl[autokorr2]{\ac(k)=\frac{1}{N-k}\sum_{i_1}^{N-k} (x_i-\bar x)(x_{i+k}-\bar x)} 
mit
\eqnl[acmean]{\bar x = \frac{1}{N}\sum_{i=1}^N x_i}
 Die aufeinanderfolgenden Koordinaten
werden maximal unabh"angig, wenn die Verz"ogerung $k=k_\ac$ so gew"ahlt wird, da"s
$\ac(k_\ac)=0$ wird. Da wir eine "Uberfaltung des Attraktors vermeiden wollen, w"ahlen wir
den ersten Nulldurchgang\footnote{Da die durch \eqnref{autokorr2} definierte
  Autokorrelationsfunktion als Definitionsbereich die nat"urlichen Zahlen $\N$ hat,
  bestimmt man ein $k$, so da"s $\ac(k)>=0$ und $\ac(k+1)<0$ ist. $\ac$ wird im Intervall
  $[k,k+1[$ linear approximiert und der Nulldurchgang $k'$ der approximierten Funktion
  bestimmt. Als Nulldurchgang von $\ac$ wird dann dasjenige $k_\ac$ benutzt, das n"aher an
  $k'$ liegt.}  der Autokorrelationsfunktion als Verz"ogerung $k_\ac$. \comment{Zur
  Unterscheidung von beliebigen bzw.\ durch andere Verfahren gew"ahlte Verz"ogerungen
  benutzen wir f"ur diese die Bezeichnung $k_A$ und f"ur die Verz"ogerungszeit $\delay_A$}

Die Implementierung des entsprechenden Algorithmus nach \eqnref{autokorr2} und
\eqnref{acmean} ist direkt durch Bildung der jeweiligen Summen durchf"uhrbar. Die Laufzeit
betr"agt $\order{N^2}$. Dies l"a"st sich jedoch beschleunigen durch Anwendung des
\begriff(Wiener-Khintchine-Theorems). Demnach ist die Autokorrelationsfunktion eines
Signals $X$ die Fouriertransformierte des Leistungsspektrums $P_X(\omega)$. Wir ben"otigen
also nur zwei Fouriertransformationen deren Laufzeit, bei Verwendung der Fast Fourier
Transformation (FFT), jeweils nur $\order{N\log N}$ betr"agt.  Algorithmen zur Berechnung
der FFT finden sich in nahezu jeder Mathematikbibliothek \cite{numerical-recipes}, so da"s
hier nicht n"aher darauf eingegangen werden braucht. Beispiele f"ur die
Autokorrelationsfunktion und die zur jeweiligen Verz"ogerung $k_\ac$ rekonstruierten
Attraktoren sind in \psref{acfigs} dargestellt.  \noafterpage{
  \epsfigfour{autocorr/roeac}{autocorr/lorac}{autocorr/roerec2}{autocorr/lorrec2} {
    Autokorrelationsfunktion $\ac(k)$ (oben) und die zum Nulldurchgang von $\ac(k_A)$
    rekonstruierten Attraktoren (unten) f"ur das R"ossler- (links, $\delay_\ac=1.29$) und
    das Lorenzsystem (rechts, $\delay_\ac=2.50$). Die Rekonstruktion des Lorenzattraktors
    sehr stark "uberfaltet.  }{acfigs}{-0.2cm} }

F"ur den R"osslerattraktor gelingt die Rekonstruktion mit der so gew"ahlten
Verz"ogerungszeit ganz passabel. Der Attraktor zwar wirkt leicht "uberfaltet, die Messung
von Dimensionen o."a.\ ist hier aber trotzdem gut m"oglich. Beim Lorenzattraktor ist die
Verz"ogerungszeit erkennbar zu gro"s. Der Attraktor ist stark "uberfaltet und spannt fast
den gesamten Phasenraum auf. Dimensionsalgorithmen werden hier schwerlich auf vern"unftige
Werte konvergieren. Der Grund f"ur die zu hohe Verz"ogerung liegt in der Struktur des
Lorenzattraktors begr"undet. Die Orbits \naja(kreisen) meist mehrere Uml"aufe um einen der
instabilen Fixpunkte des Systems. W"ahrend dieser Zeit sind die Koordinaten jedoch linear
stark korreliert, da in dem einen Fl"ugel $x$ konstant positiv und in dem anderen konstant
negativ ist. Die hier bestimmte Verz"ogerungszeit entspricht also ungef"ahr der mittleren
Aufenthaltszeit des Systems auf einem der Fl"ugel.

Aufgrund dieser Schw"achen gibt es Ans"atze, statt des Nulldurchgangs das erste lokale
Minimum oder den Abfall auf $1/e$ der Autokorrelationsfunktion zu betrachten. Diese
Ans"atze bringen zwar zum Teil bessere Ergebnisse, sind jedoch theoretisch nicht zu
begr"unden. Wir wollen sie hier deshalb beiseite lassen und ein allgemeineres Verfahren
besprechen.

\subparagraph{Redundanzanalyse.}  Wie wir gesehen haben stellt die lineare
Un\-ab\-h"an\-gig\-keit zweier Koordinaten nicht das optimale Kriterium f"ur eine gute
Verz"ogerungszeit dar. Die Probleme resultieren haupts"achlich daraus, da"s wir es mit
\emph{nichtlinearen} Systemen zu tun haben. Wir suchen eine allgemeinere Unabh"angigkeit
der Koordinaten.


\comment{Sei $X$ eine beliebige Zufallsvariable und $\Prob_X(i)$ die Wahrscheinlichkeit
  bei einer Messung von $X$ einen Wert im Intervall $[x_i,x_{i+1}[$ zu erhalten. Dann
  betr"agt die mittlere Information einer Messung von $X$}

Um die \begriff(allgemeine) Unabh"angigkeit zweier Koordinaten zu untersuchen,
m"ussen\korrektur(naja) wir uns Begriffen der Informationstheorie, insbesondere dem
\begriff(Shannonschen Informationsma"s), zuwenden. Die Messung einer Observablen $X$ habe
$n$ verschiedene Ausg"ange $x_i$, die mit den Wahrscheinlichkeiten $\Prob_X(i)$ auftreten.
Dann liefert eine Messung der Observablen $X$ im Mittel die
\begriff(Information)\footnote{In der Informationstheorie wird statt des nat"urlichen
  Logarithmus der Zweierlogarithmus benutzt. Die sich ergebende Einheit der Information
  ist $1\,\bit$. Wir benutzen hier, wie in der Physik "ublich, den nat"urlichen
  Logarithmus. Die Einheit in der hier die Information gemessen wird ist das sogenannte
  $\nat$ (von engl. natural). Beide Ma"se sind linear "uber $1\,\nat =\log_2
  e\,\bit\,$miteinander verkn"upft.}  
\eqn{H(X)=-\sum_i \Prob_X(i) \ln \Prob_X(i)} 
Die
mittlere Information ist maximal, wenn alle Ausg"ange der Messung gleich wahrscheinlich
sind $\Prob_X(1)=\dots=\Prob_X(n)$. Sie wird umso kleiner, je st"arker die
Wahrscheinlichkeiten auf wenige Ausg"ange konzentriert sind.

Werden am betrachteten System zwei Messungen $X$ und $Y$ durchgef"uhrt, so betr"agt die
mittlere Information der kombinierten Messung 
\eqn{H(X,Y)=-\sum_{i,j} \Prob_{XY}(i,j) \ln\Prob_{XY}(i,j)} 
wobei $\Prob_{XY}(i,j)$ die Verbundwahrscheinlichkeit ist, da"s bei der
Messung von $X$ der Wert $x_i$ und bei der Messung von $Y$ der Wert $y_j$ festgestellt
wird\footnote{Die Verbundwahrscheinlichkeit $\Prob_{XY}(i,j)$ ist zu unterscheiden von der
  bedingten Wahrscheinlichkeit $\Prob_{Y|X}(i,j)$, die angibt mit welcher
  Wahrscheinlichkeit an $Y$ der Wert $y_j$ gemessen wird, \emph{wenn} die Messung von $X$
  den Wert $x_i$ ergab. Beide Wahrscheinlichkeiten sind verkn"upft "uber
  $\Prob_{XY}(i,j)=\Prob_{Y|X}(i,j)\Prob_X(i)$. Die Herleitung der nachfolgenden
  Ergebnisse gelingt genauso, ist jedoch ein wenig schwerf"alliger.}.  Die Information,
die die zus"atzliche Messung von $Y$ liefert, ist daher i.allg.\ kleiner als $H(Y)$,
n"amlich 
\eqn{H(Y|X)=H(X,Y)-H(X)} 
Hierbei bezeichnet $H(Y|X)$ die mittlere Information der
Messung von $Y$ bei Kenntnis des Me"sergebnisses von $X$. Diese zus"atzliche Information
wird genau dann gleich $H(Y)$ wenn $X$ und $Y$ unabh"angige Me"sgr"o"sen sind. In diesem
Fall gilt f"ur die Verbundwahrscheinlichkeiten $\Prob_{XY}(i,j)=\Prob_X(i)\Prob_Y(j)$ und
wir erhalten
\eqna{ H(X,Y)&=&-\sum_{i,j}\Prob_X(i) \Prob_Y(j) \{\ln \Prob_X(i)+\ln \Prob_Y(j)\} \nonumber \\
  &=& -\sum_i \Prob_X(i) \ln \Prob_X(i) - \sum_j \Prob_Y(j) \ln \Prob_Y(j) \nonumber \\
  &=& H(X)+H(Y)} somit $H(Y|X)=H(Y)$. In dem Fall, da"s zwischen $X$ und $Y$ ein direkter
funktionaler Zusammenhang besteht, liefert die Messung von $Y$ gar keine zus"atzliche
Information $H(Y|X)=0$.

Wir definieren nun die \begriff(Redundanz) der kombinierten Messung $X,Y$. Redundanz
bedeutet im allgemeinen die Menge an "uberfl"ussiger Information -- in diesem Fall die
Information, die sowohl in der Messung von $X$ als auch in der von $Y$ enthalten ist.
\eqnl[genredundancy]{R(X,Y)=H(X)+H(Y)-H(X,Y)}
 Die Redundanz $R$ wird genau dann minimal,
wenn die Me"sgr"o"sen maximal unabh"angig voneinander sind. Die Redundanz wird auch
manchmal als \begriff(Transinformation) (engl.\ Mutual Information) bezeichnet. Dies
Bezeichnung r"uhrt daher, da"s die Information $R(X,Y)$ von $X$ nach $Y$ \slang("ubergeht)
oder \slang(flie"st). Die Transinformation kann in diesem Kontext zur Beschreibung von
Informationsfl"ussen in ausgehnten System verwendet werden\cite{Pawelzik91}.  In h"oheren
Dimensionen ergeben sich jedoch Unterschiede zwischen der Redundanz und Transinformation
\cite{Prichard95}.



Die Observablen $X$ und $Y$ m"ussen nicht notwendig verschiedene Me"sgr"o"sen darstellen.
Sie k"onnen auch die zeitversetzte Messung \emph{einer} Gr"o"se bedeuten.  Dies bringt uns
auf das Verfahren zur Bestimmung der Verz"ogerungszeit.  Wir definieren die zweite
Observable $Y$ als den zu einer Zeit $t+\delay$ gemessenen Wert von $X$, w"ahrend $X$ zur
Zeit $t$ gemessen wird. Damit geht \eqnref{genredundancy} "uber in
\eqnal[redundancy2]{R(X,\delay)&=&2H(X)-H(X,X_\delay) \nonumber \\
  &=& -2 \sum_i \Prob_X(i) \ln \Prob_x(i) + \sum_{i,j} \Prob_{x,\delay}(i,j) \log
  \Prob_{x,\delay}(i,j) } 
wobei $\Prob_{x,\delay}(i,j)$ die Wahrscheinlichkeit ist, da"s
eine Messung an $X$ zu einer beliebigen Zeit $t$ den Wert $x_i$ und zur Zeit $t+\delay$
den Wert $x_j$ liefert\comment{Genauer haben wir es hier mit der bedingten
  Wahrscheinlichkeit $\Prob(x_j=x(t+\delay)|x_i=x(t))$ zu tun.}.  Da wir nach
Unabh"angigkeit der Verz"ogerungskoordinaten gefragt haben, m"ussen wir also nur den Wert
von $\delay$ bestimmen, f"ur den $R(X,\delay)$ minimal wird. Wie wir schon in den
Betrachtungen zur Autokorrelationsfunktion festgestellt haben, nimmt die Korrelation
zeitlich versetzter Messungen exponentiell ab. Die Redundanz $R(X,\delay)$ strebt also
gegen Null f"ur $\delay\to\infty$, erreicht ihr Minimum also f"ur sehr gro"se $\delay$. Da
wir nun sowohl an kleinen Verz"ogerungszeiten als auch an minimaler Redundanz interessiert
sind, w"ahlen wir als Verz"ogerung das erste lokale Minimum von $R$.
\epsfigtriple{redundancy/density1}{redundancy/density2}{redundancy/density3} {Die
  Verbundwahrscheinlichkeiten $\Prob_{X,\delay}(i,j)$ f"ur $\delay=0.162,0.234,0.585$. Die
  Verz"ogerungszeiten entsprechen dem ersten lokalen Minimum, dem ersten lokalen Maximum
  sowie dem zweiten lokalen Minimum der Redundanz $R(X,\delay)$. Man erkennt, da"s im
  mittleren Bild die Verteilung der Wahrscheinlichkeiten stark auf die R"ander
  konzentriert ist. Im rechten Bild ist die Verteilung sehr flach, l"a"st jedoch kaum noch
  Struktur erkennen.  }{reddensities}{-0.2cm}

Die Berechnung von $R(X,\delay)$ erfordert die Berechnung der Wahrscheinlichkeiten
$\Prob_X(i)$ und der Verbundwahrscheinlichkeiten $\Prob_{x,\delay}(i,j)$. Diese Berechnung
ist nicht ganz trivial. Da wir zum einen kontinuierliche Me"swerte und zum anderen eine
begrenzte Anzahl Me"spunkte haben, m"ussen wir den Me"sraum geeignet partitionieren und
das Wahrscheinlichkeitsma"s dieser Partitionen bestimmen\footnote{Bei diskreten Me"swerten
  w"are verst"andlicherweise keine Partitionierung notwendig. Bei unendlich vielen
  kontinuierlichen Me"swerten k"onnte man dagegen die integrale Form von
  \eqnref{redundancy2} verwenden. Hierbei gehen die Summen in Integrale und die
  Wahrscheinlichkeiten in die entsprechenden Dichten "uber. Ein Problem der integralen
  Form ist allerdings, da"s sie gegen Koordinatentransformationen nicht invariant ist.}.
Als Partitionen w"ahlen wir Intervalle $I^1_i=[\xi_i,\xi_{i+1}[$ bzw. Produkte von
Intervallen $I^2_{i,j}=[\xi_i,\xi_{i+1}[\times[\xi_j,\xi_{j+1}[$. Die Wahrscheinlichkeit
$\Prob_X(i)$ ist dann gleich dem Anteil der Punkte $N_i$, der in das Intervall $I^1_i$
f"allt, d.h.\ $\Prob_X(i)=N_i/N$. F"ur die Verbundwahrscheinlichkeiten gilt entsprechend
$\Prob_{x,\delay}(i,j)=N_{ij}/N$.

Offen ist noch wie die Grenzen der Intervalle $I^1_i$ gew"ahlt werden sollen. Es
existieren dazu verschiedene Ans"atze.
\begin{itemize}
\item Der erste benutzt einfach "aquidistante Grenzen $x_i$. Die L"ange der Intervalle
  wird auf einen bestimmten Teil der Varianz\footnote{Die Spannweite der Me"swerte w"are
    hier ein schlechtes Ma"s, da bei experimentellen Systemen "ofters \slang(Ausrei"ser)
    vorkommen, die die Spannweite start verbreitern, sonst aber nicht viel beitragen.} der
  Me"swerte festgelegt. In den hier benutzten Beispielen wird die Intervall"ange auf
  $1/50$ der Varianz der Daten eingestellt.
\item Eine weitere Methode arbeitet mit Intervallen $I^1_i$, die so festgelegt werden,
  da"s die Wahrscheinlichkeiten $\Prob_X(i)$ f"ur alle Intervalle nahezu gleich sind.
  Hierzu wird die Zeitreihe $x_i$ der Gr"o"se nach sortiert. Die sortierte Reihe werde mit
  $x_{(i)}$ bezeichnet.  Offensichtlich haben Intervalle $[x_{(i)},x_{(i+k)}[$ die
  Wahrscheinlichkeit $\Prob_X(x_{(i)}\leq x<x_{(i+k)})=k/N$. Da die Wahrscheinlichkeiten
  f"ur alle Intervalle gleich sein sollen, legen wir f"ur die Intervallgrenzen
  $\xi_i=x_{(\lfloor\frac{N}{n}(i-1)\rfloor+1)}$ fest, wobei $n$ die Anzahl der Intervalle
  ist.
\item Ein von \autor(Fraser) und \autor(Swinney) vorgeschlagenes Verfahren bestimmt die
  Intervalle so, da"s die Verbundwahrscheinlichkeiten $\Prob_{X,\delay}(i,j)$ f"ur alle
  $(i,j)$ ann"ahernd gleich werden \cite{fraser-swinney}. Gestartet wird mit einer
  Partitionierung, die aus einem einzigen Element
  $[\xi^0_1,\xi^0_2[\times[\xi^0_1,\xi^0_2[$ besteht. Dieses Element wird nun so in vier
  rechteckige Elemente zerlegt, da"s in jedem Element gleich viele Paare $(x_i,x_j)$ zu
  liegen kommen. Mit den durch diese Zerlegung enstandenen Elementen $\{
  [\xi^1_k,\xi^1_{k+1} [ \times [\xi^1_l, \xi^1_{l+1} [ \, \vert 1 \leq k , l \leq 4\}$
  wird nun in derselben Weise weiter verfahren. Nach $m$ Schritten erh"alt man so eine
  Partitionierung des Phasenraumes mit $4^m$ gleich wahrscheinlichen Elementen
  $\{[\xi^m_k,\xi^m_{k+1}[\times[\xi^m_l,\xi^m_{l+1}[\,\vert 1\leq k,l \leq 2^{m+1}\}$.
  Das Verfahren ist jedoch recht aufwendig und bringt meiner Einsch"atzung nach keine
  wesentlich besseren Ergebnisse.
\item \autor(K. Pawelzik) entwickelte ein Verfahren, um die Redundanz aus
  verallgemeinerten Korrelationsintegralen (s. Abschnitt \ref{chapcorrdim}) zu bestimmen.
  Es gilt $R(X,\delay,\eps)=2C_1(1,0,\eps)-C_1(2,\delay,\eps)$, wobei $C_q(d,\delay,\eps)$
  das verallgemeinerte Korrelationsintegral zur Einbettungsdimension $d$,
  Verz"ogerungszeit $\delay$ und $\eps$ ist.
\end{itemize}


Die Ergebnisse f"ur die Messung der Verz"ogerungszeit $\delay_R$ "uber das erste lokale
Minimum der Redundanz zeigt \psref{redresult}. W"ahrend sich das Ergebnis beim
R"osslerattraktor nicht wesentlich ge"andert hat, ist das Resultat beim Lorenzattraktor
deutlich besser als bei der Rekonstruktion mit $\delay_\ac$. Die Verz"ogerungszeiten sind
bei beiden Systemen kleiner geworden: beim R"osslersystem von $1.29$ auf $1.16$, beim
Lorenzsystem sogar von $2.50$ auf $0.162$.  Dies war auch zu erwarten, da lineare
Unabh"angigkeit immer auch generelle Unabh"angigkeit impliziert und daher die "uber die
Redundanzanalyse bestimmten Verz"ogerungszeiten $\delay_R$ immer kleiner oder gleich den
"uber den Nulldurchgang der Autokorrelationsfunktion bestimmten $\delay_\ac$ liegen
m"ussen.


\noafterpage{
  \epsfigfour{redundancy/roemut}{redundancy/lormut}{redundancy/roerec2}{redundancy/lorrec2}
  {Redundanz $\R(k)$ (oben) und die zum ersten lokalen Minimum von $\R(k_R)$
    rekonstruierten Attraktoren (unten) f"ur das R"ossler- (links, $\delay_R=1.16$) und
    das Lorenzsystem (rechts, $\delay_R=0.162$).  }{redresult}{-0.2cm} } \noafterpage{
  \epsfigdouble{autocorr/roecmp}{autocorr/lorcmp} {Vergleich der Autokorrelationsfunktion
    $\ac(\delay)$ mit der Redundanz $R(\delay)$. Die Verz"ogerungszeiten sind f"ur die
    Redundanz kleiner als die f"ur die Autkorrelationsfunktion.  }{acfigs3}{-0.2cm} }





\comment{

\subsubsection{Datenanforderungen}

\paragraph{Datenmenge}

\paragraph{Verrauschte Zeitreihen}

}

\newpage









%\subsection{Singular Value Decomposition}
\label{chapsvd}
Eine weitere Methode der Attraktorrekonstruktion stammt von \autor(Broomhead) und
\autor(King) \cite{Broomhead-king}. Diese sogenannte \begriff(Singular Value
Decomposition) (SVD) hat gegen"uber der Ver\-z"ogerungskoordinateneinbettung haupts"achlich
zwei Vorteile. Zum einen mu"s die Einbettungsdimension nicht vorher festgelegt oder "uber
andere Verfahren ermittelt werden. Zum anderen kann durch die SVD ein dem Signal
eingepr"agtes additives Rauschen vermindert werden.

%Grundidee
\comment{Die Grundidee ist folgende. Wir betten den Attraktor in einen hochdimensionalen Raum
$\R^\embed$ der Dimension $\embed$ ein. Dann spannt der so rekonstruierte Attraktor
in der Regel nur einen $\minembed$-dimensio\-nalen Unterraum
$\subspace=\mathop{\mathrm{span}}(\folge(\x,1,N))$  des
$\R^\embed$ auf\footnote{$N$ ist die Anzahl der Rekonstruktionsvektoren,
  und h"angt mit der Anzahl der Me"spunkte $\tilde N$ "uber $N=\tilde N-d+1$ zusammen. Im allgemeinen
  gilt: $d\ll N$.}, wobei die Dimension dieses Unterraumes im allgemeinen niedriger
als die des Einbettungsraumes ist: $d'<d$.  Dieser Unterraum $\subspace$ wird nun bestimmt und der
Attraktor hieraus in den niedrigdimensionalen $\R^\minembed$ projiziert.
}

Die Grundidee ist folgende: Der Attraktor wird in einen euklidischen Vektorraum
$\R^\embed$ eingebettet, dessen Dimension $d$ so hoch gew"ahlt wird, da"s man mit gro"ser
Sicherheit davon ausgehen kann, da"s das Einbettungstheorem 2 erf"ullt ist. Erwartet man
beispielsweise, da"s der Attraktor eine Kapazit"at $D_0\leq5$ hat, k"onnte man f"ur $d$
irgendeinen Wert gr"o"ser oder gleich 10 (z.B. $d=20$) w"ahlen. Durch die
Verz"ogerungskoordinatenabbildung erh"alt man aus der Zeitreihe, die aus den $\tilde N$
Me"swerten $\folge(x,1,\tilde N)$ bestehen m"oge, die $N=\tilde N-d+1$
Rekonstruktionsvektoren $\vec x_k = (\folge(x,k,k+d-1))^\Tr$.  Der so rekonstruierte
Attraktor spannt einen Unterraum $\subspace=\mathop{\mathrm{span}}(\folge(\x,1,N))$ des
$\R^\embed$ auf, dessen Dimension $\minembed=\mathop{\mathrm{dim}}\subspace$ kleiner oder
gleich $\embed$ ist. Man bestimmt eine Basis des Unterraumes $\subspace$, und kann nun
(durch Kenntnis dieser Basis) den in $\subspace$ liegenden Attraktor in den euklidischen
Vektorraum $\R^\minembed$ einbetten. Hierdurch verringert sich die Anzahl der f"ur die
Angabe eines Attraktorpunktes n"otigen Komponenten von $\embed$ auf $\minembed$.  Der
$\R^\minembed$ ist nach Konstruktion der Einbettungsraum mit der minimal ausreichenden
Einbettungsdimension.


%Neue Basis finden
Hauptbestandteil des Verfahrens ist es, eine neue Orthonormalbasis $\folge(\vec
c,1,\embed)$ des Einbettungsraumes $\R^\embed$ zu bestimmen. Diese soll so beschaffen sein,
da"s die ersten $\minembed$ Vektoren der Basis den Unterraum $\subspace$ aufspannen:
\eqn{\subspace = \mathop{\mathrm{span}}(\folge(\vec c,1,\minembed)).} 
Da $\subspace$ nach Voraussetzung auch von den Rekonstruktionsvektoren $\folge(\x,1,N)$
aufgespannt wird, lassen sich die $\vec c_i\forall(i,1,\minembed)$ offensichtlich als
Linearkombination der $\x_k$ darstellen:
\eqnl[svdkomp]{\vec c_i = \sum_{k=1}^N \frac{1}{\sigma_i\sqrt N} s_{ik} \x_k,} 
wobei die Darstellung wegen $\minembed\ll N$ allerdings nicht eindeutig ist.  Die $s_{ik}$
bilden eine $\minembed\times N$-Matrix, deren Zeilenvektoren durch
$\vec s_i^\Tr = (s_{i1},\dots,s_{iN})$ gegeben sind (die $\vec s_i$ sind also $N$-komponentige Spaltenvektoren).  Da die $\vec c_i\forall(i,1,\minembed)$
nach Voraussetzung orthonormal (und somit auch linear unabh"angig) sind, mu"s die
$s_{ik}$-Matrix mindestens 
  den Rang $\minembed$ haben; die $\vec s_i$ m"ussen daher auch
linear unabh"angig sein.  Wir nehmen weiterhin an, da"s die letzteren orthogonalisiert
sind. Durch den Faktor $1/\sigma_i\sqrt N$ wird erreicht, da"s der Betrag der $\vec c_i$
{\em und} der $\vec s_i$ auf eins normiert werden kann. Der Faktor $1/\sqrt N$ macht die
$\sigma_i$ unabh"angig von der Anzahl $N$ der Rekonstruktionsvektoren, wie sp"ater
einzusehen sein wird.  Wir definieren nun die als \begriff(Trajektorienmatrix) bezeichnete
$N\times \embed$-Matrix $\mat X$ durch:
\eqnl[svdtrdef]{\mat X = N^{-1/2}(\x_1, \dots, \x_N)^\Tr.}
\eqnref{svdkomp} kann dann in der folgenden Form geschrieben werden: 
\eqnl[svdbase]{\sigma_i \vec c_i = \tmat X \vec s_i.}

%Strukturmatrix
Daraus folgt unter Ausnutzung der vorausgesetzten Orthonormalit"at der $\vec c_i$:
\eqnl[svdbla1]{\sigma_i \sigma_j \delta_{ij} = \vec s_i^\Tr \mat X \tmat X \vec s_j.}
Die $N\times N$ Matrix $\gmat \Theta = \mat X \tmat X$, \comment{aufgrund ihres Aufbaus} auch
als \begriff(Strukturmatrix) bezeichnet, ist reell und symmetrisch. Ihre
Eigenvektoren bilden also eine vollst"andige, orthonormale Basis des $\R^N$. 
Eine L"osung der vorstehenden Gleichung erh"alt man, indem man f"ur $\vec s_i$ 
den $i$ten Eigenvektor von $\gmat \Theta$ sowie f"ur $\sigma_i$ die Wurzel des entsprechenden Eigenwerts w"ahlt:
\eqnl[svdeigen1]{\gmat \Theta \vec s_i = \sigma_i^2 \vec s_i.}
Dies best"atigt man leicht durch Einsetzen in \eqnref{svdbla1} und Ausnutzung der Orthogonalit"at der
Eigenvektoren symmetrischer Matrizen. Von den $N$ L"osungen von \eqnref{svdeigen1}
besitzen nur $\minembed$ von null verschiedene Eigenwerte, da der Rang von $\mat X$
und somit auch der Rang von $\gmat\Theta$ gleich $\minembed$ ist. \comment{Nur diese L"osungen
sind f"ur die Bestimmung der $\vec c_i\mraise{\rvert}{-0.2}_{i=1\dots\minembed}$ nach
\eqnref{svdbase} zu gebrauchen.}

\comment{Dies ist im allgemeinen  nicht die einzige L"osung 
  von \eqnref{svdbla1}. Zur Bestimmung des Unterraumes $\subspace$ reicht jedoch die
  Kenntnis {\em einer} L"osung. }

%Kovarianzmatrix
Die Bestimmung des Unterraumes $\subspace$ "uber \eqnref{svdeigen1} erfordert die
Diagonalisierung der Strukturmatrix $\gmat \Theta$. Da $\gmat \Theta$ eine $N\times
N$-Matrix ist, wird dies f"ur gro"se $N$ sehr rechenaufwendig. Die Diagonalisierung einer
$N\times N$-Matrix ben"otigt \order{N^3} Schritte.  Da von den $N$ Eigenwerten
$\sigma_i^2$ jedoch nur $\minembed$ nicht verschwinden, lohnt es sich nach einem
schnelleren Weg zu suchen.  Multiplizieren wir \eqnref{svdbase} von links mit der
\begriff(Kovarianzmatrix) $\gmat \Xi = \tmat X \mat X$, so erhalten wir mit
\eqnref{svdeigen1}:
\eqna{\gmat \Xi\sigma_i \vec c_i &=& \tmat X \mat X \tmat X \vec s_i \nonumber\\
&=& \tmat X \sigma_i^2 \vec s_i .}
F"ur $\sigma_i\neq 0$ ergibt sich dann mit \eqnref{svdbase}:
\eqnl[svdeigen2]{\gmat \Xi \vec c_i = \sigma_i^2 \vec c_i .}
Die neue Basis $\vec c_i$ erhalten wir  
also einfacher durch Diagonalisierung der Kovarianzmatrix $\gmat \Xi$. Da $\gmat \Xi$ eine
$\embed\times\embed$-Matrix und $\embed\ll N$ ist, kann \eqnref{svdeigen2} mit sehr viel
weniger Rechenaufwand als \eqnref{svdeigen1} gel"ost werden.

%Bedeutung der Eigenwerte
Der vom Attraktor aufgespannte Unterraum $\subspace$ ist durch die Eigenvektoren $\vec
c_i$ gegeben, deren zugeh"orige Eigenwerte $\sigma_i^2$ nicht verschwinden. Um die genaue
Bedeutung der $\sigma_i$ zu erhellen, m"ussen wir erst die Struktur des neuen
Einbettungsraumes $\R^\minembed$ analysieren. Dieser Raum, in den die
Rekonstruktionsvektoren aus dem Unterraum $\subspace$ abgebildet werden, habe die
Orthonormalbasis $\vec e'_j\mraise{\rvert}{-0.2}_{j=1\dots\minembed}$.  Die in den 
$\R^\minembed$ eingebetteten Rekonstruktionsvektoren seien mit $\x'_i$ bezeichnet. Damit
die Abbildung von $\subspace$ nach $\R^\minembed$ orthogonal ist, soll die Projektion eines Rekonstruktionsvektors $\vec x'_i$
auf einen Basisvektor $\vec e'_j$ genauso gro"s sein, wie die Projektion von $\x_i$ auf
$\vec c_j$.
\eqnl[svdrecvec]{\vec x'_i = \sum_{j=1}^{\minembed} (\vec x_i \cdot \vec c_j) \vec e'_j .} 

\comment{
Um die Bedeutung der Eigenwerte zu erhellen, m"ussen wir erst die Struktur des neuen
Phasenraumes analysieren. Die Basis des Rekonstruktionsraumes wird gebildet durch die
Eigenvektoren der Kovarianzmatrix $\vec c_i$. Die Darstellung der Rekonstruktionsvektoren
$\vec x'_i$ bez"uglich dieser neuen Basis erhalten wir "uber:
\eqnl[svdrecvecbla]{\vec x'_i = \sum_{j=1}^{\minembed} (\vec x_i \cdot \vec c_j) \vec e'_j .} 
}

Die mittlere Ausdehnung $\delta'_k$ des rekonstruierten Attraktors in Richtung eines Basisvektors
$\vec e'_k$ betr"agt
\eqna{ \delta'_k &=& \left( \frac1N \sum_{i=1}^N (\x'_i \cdot \vec e'_k)^2 \right)^{1/2} \nonumber\\
&=&  \left( \frac1N \sum_{i=1}^N (\x_i \cdot \vec c_k)^2 \right)^{1/2} .}
Mit der  \eqnref{svdtrdef} folgt daraus:
\eqna{ \delta'_k &=& \left( \mat X \vec c_k \cdot \mat X \vec c_k  \right)^{1/2}
\nonumber \\
&=& \left( \vec c_k \cdot \gmat \Xi \vec c_k  \right)^{1/2} \nonumber \\ 
&=& \sigma_k .}
%&=& \left( \sum_{i=1}^N (\x_i \cdot \vec c_k)^2 \right)^{1/2}  \\ 
%&=& \left( (\tmat X \vec c_k)^2 \right)^{1/2}  }
Die Eigenwerte $\sigma_k$ geben also an, wie weit sich der Attraktor im Mittel in Richtung
$\vec e'_k$ bzw.\ $\vec c_k$ erstreckt. Die Verteilung der Attraktorpunkte kann man sich
in etwa vorstellen als ein $\minembed$-Ellipsoid dessen Halbachsen durch die Eigenwerte
$\sigma_k$ der Kovarianzmatrix gegeben sind.

%Rauschen
Bevor wir auf die Anwendung des Verfahrens kommen, m"ussen wir uns mit dem Einflu"s von
Rauschen auf die Singular Value Decomposition besch"aftigen, da selbst in ``rauschfreien''
Systemen aufgrund endlicher Speicher- und Rechengenauigkeit Rauschen erzeugt wird. 
Wir betrachten eine Zeitreihe mit additivem, gau"sverteiltem Rauschen $x_i = \bar x_i +
\xi_i$. 
Ein "Uberstrich symbolisiert hier die deterministische Komponente. 
Wir m"ussen nun den Einflu"s des Rauschens auf die Kovarianzmatrix betrachten. 
Aus der Definition der Kovarianzmatrix folgt:
\def\sumin{{\sum\limits_{i=1}^N}}
\eqnl[svdkovdef]{\gmat \Xi = \frac{1}{N}\left( \begin{array}{cccc}
 \sumin x_i x_i & \sumin x_i x_{i+1} & \dots & \sumin x_i x_{i+\embed-1} \\
 \vdots         & \vdots             & \ddots & \vdots \\
 \sumin x_{i+\embed-1} x_i & \sumin x_{i+\embed-1} x_{i+1}& \dots & \sumin x_{i+\embed-1} x_{i+\embed-1}  
\end{array} \right) .}
\comment{Da die Rauschterme $\xi_i$ unkorreliert sind, ergibt sich f"ur die Elemente von $\gmat
\Xi$ bei gro"sen Datenmengen $N$}
Die $x_i$ ersetzen wir nun durch die Summe aus deterministischer Komponente $\bar x_i$ und 
Rauschkomponente $\xi_i$. Da die Rauschterme $\xi_i$ untereinander unkorreliert sind, 
ergibt die Summation "uber $\xi_i\xi_{i+k}$ f"ur $k\neq0$ bei gro"sen Datenmengen $N$ null. Da die $\xi_i$ und die $x_i$ auch
unkorreliert sind, verschwindet die Summation "uber $\xi_i x_{i+k}$ ebenfalls. Wir erhalten 
so f"ur die Elemente von $\gmat\Xi$:
\eqn{\gmat \Xi_{kl} = \frac{1}{N}\left( \underbrace{\sumin x_{i+k} x_{i+l}}_{= \overline{\gmat \Xi}_{kl}} + 
\underbrace{\sumin x_{i+k} \xi_{i+l}}_{\approx 0} + 
\underbrace{\sumin \xi_{i+k} x_{i+l}}_{\approx 0} + 
\underbrace{\sumin \xi_{i+k} \xi_{i+l}}_{\approx N <\xi^2>\delta_{kl}} \right), }
und somit:
\eqn{\gmat \Xi = \overline{\gmat \Xi}\; + <\!\xi^2\!>\unity_\embed .}
Die Eigenwerte der Kovarianzmatrix $\sigma_i^2$ werden also einheitlich um $<\!\xi^2\!>$
erh"oht. 
Die Wirkung additiven Rauschens besteht darin, die L"ange der Hauptachsen des oben beschriebenen
Ellipsoids gleichm"a"sig um den Betrag $\xi_0 = \sqrt{<\!\xi^2\!>}$ zu vergr"o"sern. Auch die Richtungen,
f"ur die ohne Rauschen $\sigma_i=0$ galt, werden nun von Trajektorien besucht. Es wird
somit der ganze $\embed$-dimensionale Phasenraum aufgespannt: $\subspace=\R^\embed$.

In Richtung der Achsen mit $\sigma_i\approx \xi_0$ wird die Dynamik allerdings
vollst"andig durch die Rauschterme dominiert. 
F"ur eine gute Rekonstruktion des Attraktors wird also nur der Teilraum
$\subspace=\mathop{\mathrm{span}}\{\vec c_1,\dots,\vec c_{\minembed}\}$ benutzt, f"ur den
die entsprechenden Eigenwerte gr"o"ser als die St"arke des Rauschens ist.

Wir k"onnen nun das Verfahren zusammenfassen.
\begin{enumerate}
\item Aus der Zeitreihe wird zu gegebenem $\embed$ die Kovarianzmatrix $\gmat \Xi$
  berechnet.  Aufgrund der Symmetrie dieser Matrix reicht es
  aus, nur die rechte obere H"alfte der Matrix inklusive der Diagonalen zu berechnen. Zus"atzlich ergibt sich
  \eqnref{svdkovdef} die Beziehung
  $\gmat\Xi_{i+1,j+1}=\gmat\Xi_{ij}+\frac{1}{N}(x_{N+i}x_{N+j}-x_ix_j)$. Hierdurch kann
  die Anzahl der ben"otigten Multiplikationen und Additionen betr"achtlich gesenkt werden.
\item Die Kovarianzmatrix wird diagonalisiert, und die erhaltenen Eigenwerte sowie
  Eigenvektoren werden nach absteigender Gr"o"se der Eigenwerte sortiert.  Hierf"ur stehen
  fertige Algorithmen in vielen Mathematikbibliotheken zur Verf"ugung (siehe z.B.~ \cite{Numerical-recipes}).
\item Die Gr"o"se des "uberlagerten Rauschens $\xi_0$ wird abgesch"atzt.  F"ur diese
  Absch"atzung existiert kein allgemeing"ultiges Verfahren;  sie erfolgt nach den
  besonderen Gegebenheiten und der Herkunft der Zeitreihe.  Hinweise auf die Gr"o"se von
  $\xi_0$ sind gegeben durch die interne Rechengenauigkeit und die Me"sgenauigkeit bei der
  Aufnahme der Zeitreihe (z.B. Digitalisierungsrauschen).
  
  Liegen hier"uber keine oder ungen"ugende Angaben vor, so kann $\xi_0$ auch anders
  abgesch"atzt werden.  $\sigma_i$ strebt f"ur gro"se $i$ im allgemeinen gegen einen Grenzwert
  $\sigma_\tmin$, der durch das Rauschen bestimmt ist (siehe \psref{svdvalnoise}). F"ur
  $\xi_0$ wird dann dieser Grenzwert benutzt.
\item Die Rekonstruktionsvektoren k"onnen nun gem"a"s \eqnref{svdrecvec} berechnet werden.
  Statt alle $\minembed$ Komponenten zu verwenden, kann alternativ auch nur die
  erste (nun gefilterte) Komponente $\x_i\cdot\vec c_1$ "uber die
  Verz"ogerungskoordinatenabbildung wieder eingebettet werden.  Auf diese Weise dient die
  SVD nur einer adaptiven Rauschfilterung, und herk"ommliche Zeitreihenanalyseverfahren
  k"onnen wieder angewandt werden.  Ein weitere M"oglichkeit liegt darin, die ersten
  $n\leq\minembed$ f"ur eine multivariante Verz"ogerungskoordinatenabbildung zu benutzen
  \cite{Fraedrich-wang}.  Darauf soll in diesem Rahmen jedoch nicht n"aher eingegangen
  werden.
\end{enumerate}

% \subsubsection{Anwendung des Verfahrens}
Das Verfahren soll nun am Beispiel des Lorenz-Attraktors demonstriert werden.  Nach
Integration der Differentialgleichungen wurde aus der $x$-Komponente des Zustandvektors
eine Zeitreihe gebildet.  Nach Berechnung der Kovarianzmatrix zu $\embed=7$ wurden die
Eigenwerte und Eigenvektoren ermittelt.  Die Ergebnisse der Rechnungen zeigen
\psref{svdval} und \psref{svdvec}.
\epsfigsingle{svd/simple/lorsvdval}
{Logarithmus der Eigenwerte $\sigma_i$ der Kovarianzmatrix $\gmat \Xi$ f"ur den
Lorenz-Attraktor mit der Einbettungsdimension $\embed=7$.
}
{svdval}{-0.2cm}
\epsfigtripletop{svd/simple/lorsvdvec1}{svd/simple/lorsvdvec2}{svd/simple/lorsvdvec3}
{Die ersten drei Eigenvektoren der Kovarianzmatrix f"ur den Lorenz-Attraktor
($\embed=7$). F"ur jeden Eigenvektor $\vec c_j$ ist die $k$-te Komponente
$\vec c_{j,k}$ "uber den Index $k$ aufgetragen.}
{svdvec}{-0.2cm}

In \psref{svdval} ist deutlich zu sehen, da"s ab $i\geq 4$ die Eigenwerte $\sigma_i$ auf
einem nahezu konstanten Wert bleiben. Dies liegt daran, da"s ab $i=4$ die Eigenwerte nur
noch durch Rauschen dominiert werden. Dieses ist allerdings nicht k"unstlich addiert
worden, sondern bedingt durch die Genauigkeit, mit der die Werte der Zeitreihe
zwischengespeichert wurden\footnote{Die Werte wurden mit einer Genauigkeit von 6 Stellen
zwischengespeichert, so da"s $\xi_0\simeq 10^{-6}$ und $\ln(\xi_0)\simeq -13,8$ ist.}. 
Der Rauschpegel kann durch $\xi_0\simeq e^{-14}\sigma_1$ abgesch"atzt werden.
Da nur die ersten drei Eigenwerte deutlich "uber $\xi_0$ liegen, sind auch nur die
ersten drei Eigenvektoren von $\gmat\Xi$ in \psref{svdvec}
dargestellt. 

F"ur die Berechnung der Rekonstruktionsvektoren $\x'_i$ im $\R^\minembed$ mu"s eine
Skalarmultiplikation der $\x_i$ mit den $\vec c_j$ durchgef"uhrt werden. Die $\x_i$
bestehen nach Konstruktion aus aufeinanderfolgenden Werten der Zeitreihe. Es kann nun
gezeigt werden, da"s die skalare Multiplikation mit $\vec c_1$ ungef"ahr einer Mittelung dieser 
Werte entspricht. Die skalare Multiplikation mit $\vec c_j$ entspricht i.a.\ einer gemittelten 
$(j-1)$-ten Ableitung. Um dies plausibel zu machen, sollen der Mittelwert und die 
numerischen Ableitungen f"ur $\embed=3$ explizit aufgef"uhrt werden:
\def\myvec(#1,#2,#3){\left( \begin{array}{c}#1\\#2\\#3\end{array}\right)}
\def\myvectr(#1,#2,#3){(#1,#2,#3)^\Tr}
\def\mysample{\sample}
%\def\mysample{}
\eqn{\begin{array}{lll} 
\bar x_i =  (x_{i-1} + x_{i} + x_{i+1} )/3  &=&
\myvec(x_{i-1},x_i,x_{i+1})\cdot\myvec(1/3,1/3,1/3) \\ 
{\bar{\dot x}}_i = (-x_{i-1} + x_{i+1})/2\mysample &=& \myvec(x_{i-1},x_{i},x_{i-1})\cdot\myvec(-1/2\mysample,0,1/2\mysample)\\ 
{\bar{\ddot x}}_i = ( x_{i-1} -2 x_{i} + x_{i+1} )/\mysample^2 &=& \myvec(x_{i-1},x_{i},x_{i-1})\cdot\myvec(1/\mysample^2,-2/\mysample^2,1/\mysample^2)
\end{array} .
}
Die Vektoren, die mit $\myvectr(x_{i-1},x_{i},x_{i-1})$ skalar multipliziert
werden, sind den $\vec c_j$ strukturell "ahnlich; auch f"ur $\embed>3$ wiesen die
entsprechenden Vektoren stets diese Form auf. Daraus wird ersichtlich, da"s die Komponente
$\x_i\cdot\vec c_j$ in etwa der zeitlichen Ableitung $\abls{^{(j-1)}x_{i+\lfloor d/2 \rfloor}}{t^{(j-1)}}$ 
entspricht.
Auf die hieraus folgenden Konsequenzen werden wir  jedoch sp"ater
genauer eingehen.

Wir wollen nun die Singular Value Decomposition bei Vorhandensein additiven Rauschens
untersuchen. 
Als Beispiel m"oge hier wieder eine Zeitreihe des Lorenz-Attraktors dienen. 
Zu dieses wurde Gau"ssches Rauschen in verschiedener St"arke addiert (Angaben in 
Prozent der Varianz von $x$). 
F"ur die so entstandenen verrauschten Zeitreihen wurden jeweils die
Kovarianzmatrix $\gmat \Xi$ und ihre Eigenwerte $\sigma_i^2$ bestimmt (siehe \psref{svdvalnoise}).
\noafterpage{
\epsfigsingle{svd/noise/lorsvd}
{Logarithmus der Eigenwerte der Kovarianzmatrix $\gmat \Xi$ f"ur verschiedene
Rauschpegel (0\%;0,2\%;0,5\%;1,0\%;2,0\%;5,0\% von unten nach oben)}
{svdvalnoise}{-0.2cm}
}

Man erkennt deutlich, da"s die Eigenwerte $\sigma_i$ ab $i=4$ nur durch die St"arke des
Rauschens bestimmt sind. 
Die Verh"altnisse der Eigenwerte sind durch die relative St"arke der
Rauschamplitude gegeben. 

Verschiedene Rekonstruktionen des Lorenz-Attraktors aus der Zeitreihe mit 5\% Rauschen sind
in \psref{svdrecnoise} abgebildet. Hierbei wurden unterschiedliche Verfahren angewendet.
In der oberen Abbildung wurde einfach durch die Verz"ogerungskoordinatenabbildung das 2-Tupel
$(x_i,x_{i+k_R})$ rekonstruiert, wobei die Verz"ogerung $k_R=19$ durch Redundanzanalyse
gewonnen wurde. Unten links wurden die ersten beiden Komponenten
$(x'_{i,1},x'_{i,2})$ der SVD-Rekonstruktion eingebettet, w"ahrend unten rechts die erste Komponente
$x'_{i,1}$ mit der um $k_R$ verz"ogerten Komponente $x'_{i+k_R,1}$ eingebettet wurde.
Man sieht deutlich, wie der Signal-Rausch-Abstand durch die SVD-Rekonstruktion verbessert
wird.

\afterpage{
\epsfigtriplebot{svd/noise/lorrec50_1}{svd/noise/lorsrec50_1}{svd/noise/lorrecs50_1}
{Rekonstruktionen des Lorenz-Attraktors bei 5\% Rauschen. 
Oben durch einfache Verz"ogerungskoordinatenabbildung, unten links durch normale
SVD-Rekonstruktion, unten rechts durch Verz"ogerungskoordinateneinbettung der ersten Komponente
der SVD-Rekonstruktion. 
}
{svdrecnoise}{-0.2cm}
}

Die Vorteile der Singular Value Decomposition zur Rauschverminderung sind klar
ersichtlich. Auf der anderen Seite kann die Rekonstruktion "uber Verz"ogerungskoordinaten
bei Verwendung der durch Redundanzanalyse gewonnenen Verz"ogerungszeit $\delay_R$
\naja(bessere) Einbettungen liefern. Dies zeigte \autor(Fraser), indem er ein
sogenanntes \begriff(Verzerrungsfunktional) (engl.: distortion functional)
einf"uhrte \cite{Fraser}. Dieses mi"st, wie gut die Lage eines Punktes im Originalphasenraum aus der
Kenntnis seines rekonstruierten Bildes bestimmt ist. Das Funktional mi"st
-- mit Frasers Worten --, wie \naja(diffeomorph) die Rekonstruktion ist bzw.\  welche
Rekonstruktion \naja(diffeomorpher) zum Original ist. Das Ergebnis dieses Vergleichs ist,
da"s die durch Redundanzanalyse gewonnenen Rekonstruktionen den SVD-Rekonstruktionen
"uberlegen sind. Wir werden also im folgenden das kombinierte Verfahren verwenden. Die
Zeitreihe wird durch SVD eingebettet, die erste, rauschgefilterte Komponente wird
extrahiert und "uber Redundanzanalyse wieder eingebettet. Dies steht auch in Einklang mit
den Ergebnissen von \autor(Mees \etal) , die zwar die rauschmindernden
Eigenschaften der SVD best"atigen, die M"oglichkeit, den vom Attraktor aufgespannten
Unterraum zu identifizieren, jedoch zweifelhaft erscheinen lassen\cite{Mees87}.




%\clearpage
%\section{Quantitative Charakterisierung seltsamer Attraktoren}
\section{Fraktale Dimensionen}
Bei der Untersuchung chaotischer Systems ist es wichtig, ein Ma"s f"ur die Komplexit"at der
Dynamik zu haben. Ein solches Komplexit"atsma"s ist z.B. durch die fraktale Dimension des
Attraktors gegeben. W"ahrend integrable Systeme stets Attraktoren mit ganzzahliger
Dimension besitzen, ist dies f"ur chaotische Systeme in der Regel nicht mehr der Fall.
Versucht man beispielsweise die Struktur des H\'enonattraktors zu beschreiben, so stellt
man fest, da"s er in gewissem Sinne \naja(mehr) ist als eine Kurve, jedoch \naja(weniger)
als eine Fl"ache. Seine Dimension liegt zwischen eins und zwei.  Im folgenden sollen
verschiedene Definitionen f"ur die Dimension solcher fraktalen Objekte sowie Verfahren zur
Berechnung selbiger beschrieben werden.  Das Hauptaugenmerk liegt hierbei auf der
\begriff(numerischen) Berechenbarkeit der Dimension.

\comment{Bei nichtdissipativen integrablen Systemen ist die Dimension des
Attraktors gleichzeitig auch die Anzahl der Freiheitsgrade. Bei dissipativen Systemen
hingegen wird die Anzahl der effektiven Freiheitsgrade durch die Dissipation stark
eingeschr"ankt. Im integrablen Fall ist

Bei chaotischen Systemen dagegen wird die Anzahl der Freiheitsgrade durch
Dissipation (s. Abschnitt \ref{chapdynsystems}) stark reduziert auf wenige effektive
Freiheitsgrade. Die Dimension des Attraktor ist i.allg.\  weit geringer als die Anzahl der
Freiheitsgrade. Dar"uber hinaus l"a"st sie sich  die Dimension seltsamer Attraktoren nicht
mehr durch ganze Zahlen 
beschreiben. Beispielsweise ist der Attraktor des H\'enonsystems \naja(mehr) als eine
Kurve, jedoch \naja(weniger) als eine Fl"ache. Seine Dimension liegt irgendwo zwischen
eins und zwei. Es sollen im folgenden mehrere Verfahren zur Berechnung der Dimension
solcher fraktalen Objekte beschrieben werden. Dabei liegt das Hauptaugenmerk auf der
numerischen Berechenbarkeit durch das Verfahren. 
}

\subsection{Hausdorffdimension}
Von den verschiedenen Dimensionsbegriffen soll als erster der grundlegendste, n"amlich der
der \begriff(Hausdorffdimension) eingef"uhrt werden. Die Definition kann "uber die
folgenden "Uberlegungen veranschaulicht werden. Wenn ein geometrisches Objekt,
beispielsweise eine Fl"ache, mit einem Ma"s niedrigerer Dimensionalit"at, z.B. durch
Geradenst"ucke, ausgemessen wird, ergibt sich ein unendlicher Wert bez"uglich dieses
Ma"ses, da wir unendlich viele Geradenst"ucke brauchen um die Fl"ache zu "uberdecken. Wird
dagegen mit einem Ma"s h"oherer Dimension gemessen, z.B. indem die Fl"ache durch W"urfel
ausgemessen wird, ergibt der Wert 0, da eine Fl"ache kein Volumen besitzt. Nur wenn mit
einem Ma"s der gleichen Dimension gemessen wird, resultiert ein endlicher
Wert\footnote{Vorausgesetzt die Menge ist kompakt. Dies ist im weiteren nicht wesentlich,
  da es nur darauf ankommt, da"s eine Sprungstelle von Unendlich auf 0 existiert.}. Aus
diesem Sprungverhalten kann auf die Dimension des Objekts geschlossen werden.

Nun m"ussen die oben benutzten Begriffe genauer mathematisch spezifiziert werden. 
Sei $\B$ eine beliebige, nichtleere Teilmenge des $\R^n$, deren Dimension wir messen m"ochten. Um das
$d$-dimensionale Ma"s  der Menge zu bestimmen, "uberdecken wir sie mit Teilmengen 
$\set C_{r,i}$ des $\R^n$, deren Durchmesser $\norm{\set C_{r,i}}$ kleiner als eine obere
Schranke $r$ sein soll. Ein solche "Uberdeckung bezeichnen wir mit $\mathcal
C_r$. Wir definieren nun:
\eqnl[hausdorffmass1]{\hdm(\B,d,r) = \inf_{\mathcal C_r} \sum_{\set C_{r,i} \in \mathcal C_r}\norm{\set C_{r,i}}^d .}
Auf diese Weise betrachten wir alle m"oglichen "Uberdeckungen von $\B$ durch Mengen,
deren Durchmesser h"ochstens $r$ ist, und \naja(versuchen) die Summe der $d$-ten Potenzen der
Durchmesser zu minimieren. Wenn $r$ kleiner wird, reduziert sich die Anzahl der
m"oglichen "Uberdeckungen, und $\hdm(\B,d,r)$ strebt einem Grenzwert entgegen:
\eqnl[hausdorffmass2]{\hdm(\B,d) = \lim_{r\to 0}\hdm(\B,d,r) .}
Der Grenzwert $\hdm(\B,d)$ existiert f"ur alle $\B$ und $d$ und hei"st das
$d$-dimensionale \begriff(Hausdorffma"s) der Menge $\B$. F"ur ganzzahlige $d$ entspricht
es, bis auf einen konstanten Faktor, dem $d$-dimensionalen Lebesgue-Ma"s der Menge
$\B$. Nach dem oben gesagten sollte nun eine Sprungstelle $D_H$ existieren, so
da"s:
\eqnl[hausdorffmass3]{\hdm(\B,d) = \left\{ \begin{array}{ll}
                                          \infty, & d<D_H\\
                                          0,      & d>D_H
                                          \end{array}\right.  .}
Diese Sprungstelle existiert immer\footnotemark. 
\footnotetext{Dies folgt aus den Skalierungseigenschaften von $\hdm(\B,d,r)$. Es gilt
$\hdm(\B,t,r)\leq r^{t-d}\hdm(\B,d,r)$. L"a"st man $r$ gegen 0 laufen, kann
mit \eqnref{hausdorffmass2} die Existenz der Sprungstelle gefolgert werden. } 
Das Hausdorffma"s $\hdm(\B,d)$ kann f"ur
$d=D_H$ einen unendlichen oder einen endlichen Wert gr"o"ser 0 annehmen\footnotemark. Wir definieren als
\begriff(Hausdorffdimension) der Menge $\B$:
\eqnl[hausdorffdim]{D_H(\B) = \inf\{d\vert \hdm(\B,d)=0 \} .}
 \footnotetext{Nicht jedoch den Wert 0, wie in manchen Publikationen f"alschlich
behauptet, au"ser im trivialen Fall $\B=\emptyset$. Hierf"ur ist die Hausdorffdimension
jedoch nicht definiert.}

Die Hausdorffdimension ist die mathematisch am \naja(saubersten) definierte. Ihr
Anwendungsbereich liegt vor allem in der Theorie. F"ur die Verwendung in
Computeralgorithmen ist sie dagegen schlecht geeignet. Dies liegt haupts"achlich an
der Einbeziehung beliebiger "Uberdeckungen, was mit Computermethoden nicht zu realisieren
ist. F"ur numerische Berechnungen ist es sinnvoller die Menge der m"oglichen
"Uberdeckungen einzuschr"anken. Eine dieser Einschr"ankungen f"uhrt uns auf den Begriff
der Kapazit"at.



\subsection{Kapazit"at}
\label{chapcapacity}
Bei der auf \autor(Kolmogorov) zur"uckgehenden \begriff(Kapazit"at) wird die Klasse der
$r$-"Uberdeckungen von $\B$ beschr"ankt auf "Uberdeckungen $\mathcal C^K_r$, die als Teilmengen nur
$n$-dimensionale W"urfel mit Durchmesser $r$ enthalten. Der Summand in
\eqnref{hausdorffmass1} ist nun konstant gleich $r^d$. Das auf die "Uberdeckungen $\mathcal C^K_r$
beschr"ankte Ma"s $\kpm$ ist also nur abh"angig von der Anzahl der Mengen, die zur "Uberdeckung
von $\B$ minimal gebraucht werden. Bezeichnen wir diese Anzahl mit $N(r)$, so gilt f"ur das
Ma"s $\kpm$:
\eqn{\kpm(\B,d,r)=N(r)r^d.}
Auch dieses Ma"s hat eine Sprungstelle bei einem bestimmten $d=D_K$. Diese l"a"st sich 
jedoch weit einfacher ermitteln als bei der Hausdorffdimension in
\eqnref{hausdorffdim}. Offensichtlich kann $\kpm(\B,d,r)$ f"ur $r\to0$ nur dann ein
endlichen Wert annehmen, wenn $N(r)$ mit $(1/r)^d$ skaliert. Daher definieren wir:
\eqnl[capacity]{D_K(\B) = \lim_{r\to0} \frac{\log N(r)}{\log(1/r)} .}
Dies ist die Kapazit"at der Menge $\B$. Da die Bestimmung der Kapazit"at nach
\eqnref{capacity} nur darauf beruht, da"s die \naja(K"astchen), die zur "Uberdeckung der
Menge $\B$ ben"otigt werden, gez"ahlt werden, spricht man auch von der
\begriff(Boxcounting-Dimension). F"ur typische Attraktoren
wird erwartet, da"s Kapazit"at und Hausdorffdimension "ubereinstimmen
\cite{Farmer-ott-yorke}. Es k"onnen jedoch Mengen konstruiert werden, f"ur die das nicht der 
Fall ist.\footnote{Beispielsweise hat die Menge $\B=\{1/i\vert i\in\N\}$ die 
Hausdorffdimension $D_H=0$ und die Kapazit"at $D_K=1/2$ \cite{Leven89}. Dies ist im "ubrigen einer der
Schwachpunkte der Kapazit"at, da"s sie abz"ahlbaren Mengen endliche Dimension zuordnen kann.}

K"astchenz"ahlalgorithmen sind auf Computern sehr einfach zu implementieren. Der
Phasenraum braucht nur in K"astchen der Kantenl"ange $r$ eingeteilt werden\footnotemark.
Dies geschieht auf dem Computer, indem ein \begriff(Array) $F$ mit $(L/r)^n$ Eintr"agen
angelegt wird, wobei $L$ die lineare Ausdehnung des Attraktors ist\footnote{Ein Array ist
  eine indizierte Menge von Variablen, wobei die Menge der Indizes beim Dimensionieren des
  Arrays vorher festgelegt wird. Der entsprechende deutsche Fachbegriff \naja(Feld) wird
  hier nicht verwendet, um Verwechselungen mit der physikalischen Bedeutung des Wortes
  auszuschlie"sen.}. Jedem dieser Eintr"age wird nun genau ein K"astchen des Phasenraumes
zugeordnet. F"ur jeden Rekonstruktionspunkt wird der Eintrag $i$ ermittelt und $F(i)$ um
eins erh"oht. Die Anzahl der Eintr"age, f"ur die $F(i)\neq 0$, entspricht (ungef"ahr) der
Anzahl $N(r)$. Hieraus kann dann "uber \eqnref{capacity} die Kapazit"at abgesch"atzt
werden. Dieses direkte Verfahren ist jedoch sehr Speicher- und Zeitaufwendig, da sowohl
Speicherbedarf als auch Rechendauer mit der Ordnung \order{(L/r)^n} anwachsen.
\footnotetext{Dies entspricht nicht genau dem bei der Definition der Kapazit"at gemachten
  Ansatz, da hier ein festes Gitter mit Gitterkonstante $r$, statt einer "Uberdeckung
  durch beliebige K"astchen der Kantenl"ange $r$ benutzt wird. Der Grenzwert f"ur $r\to0$
  ist jedoch bei beiden Ans"atzen gleich.  }


Ein wesentlich schnelleres und speicherschonenderes Verfahren soll im folgenden
vorgestellt werden \cite{Junglas}.
\comment{Das Verfahren, da"s sich aus der Definition der Kapazit"at f"ur eine computergest"utzte
Berechnung ableitet, ist sehr einfach und erfolgt  in folgenden Schritten \cite{Junglas}:}
\begin{itemize}
\item Jeder Punkt $\x$ aus der Menge\footnote{Da jede auf einem
Computer darstellbare Menge abz"ahlbar sein mu"s, gehen wir hier wie auch im folgenden bei 
rechnerischen Verfahren von abz"ahlbaren Mengen aus.} $\B\,\subset\,\R^n$ wird auf einen
Punkt $\y\in\Z^n$ des sogenannten \begriff(Indexraumes)  abgebildet. Hierzu wird jede
der Komponenten von $\x$ durch $r$ geteilt und der gebrochene Teil abgeschnitten,
d.h.\  $y_i=\lfloor x_i/r \rfloor$. Durch diese Abbildung wird jedem Element $\x$ der
Menge das $\x$ enthaltende K"astchen mit den Indizes $y_1,\dots,y_n$ zugeordnet.
\item Die Menge der $\y_i$ wird nun geordnet. Dazu ist es notwendig eine Ordnungsrelation
auf $\Z^n$ zu definieren. Die genaue Definition dieser Relation ist hier unwesentlich. Sie mu"s nur den
mathematischen Anforderungen an eine Ordnungsrelation entsprechen. Wir definieren: $\y_i$
ist genau dann kleiner als $\y_j$, wenn ein $k\in\{1,\dots,n\}$ existiert, so da"s
$\y_{i,m}=\y_{j,m}$ f"ur alle $m<k$ und $\y_{i,m}<\y_{j,m}$ f"ur $m=k$ gilt. 
\item Nach der Konstruktion im ersten Schritt ist $N(r)$ identisch mit der Anzahl
verschiedener $\y_i$, da dies genau der Anzahl von $\B$ belegter K"astchen
entspricht. Durch die Sortierung in Schritt zwei kann diese Anzahl sehr schnell abgez"ahlt 
werden. Division von $\log N(r)$ durch $\log(1/r)$ liefert eine Absch"atzung f"ur $D_0$.
\end{itemize}
Der zeitaufwendigste Teil ist die Sortierung der Punkte mit \order{n N\log N} Schritten
w"ahrend der erste und dritte Teil des Verfahrens nur \order{n N} Schritte ben"otigen. 
Die Berechnungszeit w"achst also, im Gegensatz zu manchen gegenteiligen Behauptungen,
nur linear mit $n$ und nicht exponentiell. Demgegen"uber w"achst jedoch die
ben"otigte Datenmenge $N$, wie wir sp"ater noch sehen werden, exponentiell oder
"uberexponentiell\footnote{In der Literatur existieren hier verschiedene Absch"atzungen
f"ur $N$.} mit $D_K$. Dies ist jedoch ein gemeinsames Charakteristikum
aller Dimensionsberechnungen.

Bei der Berechnung der Kapazit"at stellt sich allerdings ein anderes Problem. Aufgrund der
endlichen Datenmenge werden manche K"astchen nicht mitgez"ahlt, obwohl sie f"ur
$N\to\infty$ auch von Trajektorien besucht w"urden. Dies f"uhrt zu relativ gro"sen Fehlern 
bei der Anwendung dieses Verfahrens. 

Auf der anderen Seite wird die Tatsache, da"s manche K"astchen sehr oft besucht werden,
ignoriert.  Bei der Kapazit"at werden K"astchen entweder gez"ahlt oder nicht. Besser w"are
-- gerade bei endlichen Datenmengen -- den Beitrag der einzelnen K"astchen entsprechend
ihrer Wahrscheinlichkeit zu gewichten. Dies f"uhrt zu den sogenannten
\begriff(probabilistischen) Dimensionen, der Informations- und der Korrelationsdimension,
sowie den verallgemeinerten Dimensionen.



\subsection{Informationsdimension}
Die \begriff(Informationsdimension) w"ahlt einen v"ollig anderen Zugang zum Begriff der Dimension,
als die beiden vorangegangenen. Um einen Punkt in einem $n$-dimensionalen Raum festzulegen 
werden genau $n$ reelle Zahlen ben"otigt. Anstatt anzugeben, wieviele reelle Zahlen
hierzu ben"otigt werden, kann auch die Menge an Information angegeben
werden, die ben"otigt wird, um die Position des Punktes mit einer Genauigkeit $r$
festzulegen. Diese Information betr"agt $I(r)=-n\ln(r)$, wobei wir hier wieder die
\naja(physikalischere) Einheit der Information in $\nat's$ gew"ahlt haben. Ist nun bekannt, da"s
sich der Punkt in einer $D_I$-dimensionalen Teilmenge $\B$ des $\R^n$ befindet, reduziert 
sich die notwendige Information auf $I(r)=-D_I\ln(r)$. Andererseits bringt uns die 
Informationstheorie einen Ausdruck f"ur die mittlere Information, die die Messung eines
Punktes aus $\B$ liefert\footnote{Vorausgesetzt auf $\B$ ist ein Wahrscheinlichkeitsma"s
definiert.}. "Uberdecken wir die Menge mit K"astchen der Kantenl"ange $r$ und sei $\Prob_i$ 
das Wahrscheinlichkeitsma"s des $i$-ten K"astchens, dann gilt f"ur die mittlere
Information, die bei der Bestimmung, in welchem K"astchen der Punkt liegt, gewonnen wird $\bar I(r)=-\sum_i
\Prob_i\ln \Prob_i$. F"ur 
$r\to 0$ k"onnen wir beide Ausdr"ucke gleichsetzen und nach $D_I$ aufl"osen:
\eqnl[informationdim]{D_I(\B)=\lim_{r\to 0}\frac{\sum_i \Prob_i\ln \Prob_i}{\ln r} .}
Dies ist die Informationsdimension der Menge $B$. 

Die Informationsdimension hat in der
letzten Zeit gegen"uber der noch zu besprechenden Korrelationsdimension wieder vermehrt
Anwendung gefunden. Dies liegt daran, da"s sie durch die Einf"uhrung  sogenannter
\begriff(N"achst-Nachbar-Algorithmen) gut berechenbar wurde\cite{Badii85}. Sei
$\delta(k)$ der Abstand eines Referenzpunktes $\x$ zu seinem $k$-ten n"achsten Nachbarn, dann
gilt:
\eqn{D_I=-\frac{\log N}{<\log \delta(k)>} ,}
wobei $<\log \delta(k)>$ der Erwartungswert von $\delta(k)$ ist. Dieser wird durch
Mittelung "uber geeignete Referenzpunkte berechnet. F"ur die Vorteile dieser Methode
gegen"uber den Korrelationsintegralen sei auf \cite{Liebert91} verwiesen.



Bevor wir im n"achsten Abschnitt auf die Korrelationsdimension eingehen, sollen hier
vorher kurz die generalisierten Dimensionen $D_q$ eingef"uhrt werden. Diese sind definiert
durch:
\eqnl[gendim]{D_q = \frac1{q-1}\lim_{r\to0}\frac{\log\sum_i\Prob_i^q}{\log r}}
mit $q\in\R^{\geq0}$.
\autor(H. Hentschel) und \autor(I. Procaccia) zeigten \cite{Hentschel-procaccia}, da"s
f"ur $q\to0$ bzw.\ $q\to1$ die generalisierte Dimension $D_q$ in die Kapazit"at $D_K$
bzw.\\ die Informationsdimension $D_I$ "ubergeht:
\eqn{D_K=\lim_{q\to0}D_q   , \qquad  D_I=\lim_{q\to1}D_q .} 
Weiterhin konnten sie zeigen, da"s $D_q\leq D_{q'}$ f"ur $q>q'$, wobei das
Gleichheitszeichen genau dann gilt, wenn es sich bei dem betrachteten Objekt um ein
homogenes Fraktal handelt. Die Folge der $D_q$ ist also monoton fallend mit $D_0\geq
D_q\geq D_\infty$. 



%\subsection{Korrelationsdimension}
\label{chapcorrdim}


Die momentan g"angigste Methode der Dimensionsbestimmung ist die Berechnung der
\begriff(Korrelationsdimension) nach \autor(Grassberger) und \autor(Procaccia)
\cite{Grassberger-procaccia}. Die Korrelationsdimension ist definiert durch:
\eqnl[cdimdef]{D_C = \lim_{\eps\to 0} \frac{\log(C(\eps))}{\log(\eps)},}
wobei $C(\eps)$ das sogenannte \begriff(Korrelationsintegral) darstellt: 
\eqnl[cintdef]{C(\eps) = \frac{1}{N(N-1)}\sum_{i\neq j}\Theta(\eps-\norm{\x_i-\x_j}).}
$C(\eps)$ \naja(z"ahlt) wieviel Paare von Punkten existieren, die einen Abstand
$\norm{\x_i-\x_j}$ kleiner als $\eps$ haben. Wir wollen nun zeigen, da"s die "uber
\eqnref{cdimdef} definierte Korrelationsdimension mit der generalisierten Dimension $D_2$ 
"ubereinstimmt\footnote{Dies soll kein formaler Beweis werden, nur eine Beweisskizze.}.

Die verallgemeinerte Dimension $D_2$ ist gegeben durch:
\eqnl[d2def]{\corrdim =  \lim_{\eps\to 0} \frac{\log\left(\sum_i \Prob_i^2\right)}{\log(\eps)}.} 
$\Prob_i$ ist die Wahrscheinlichkeit einen Attraktorpunkt in der $i$-ten Box der
gew"ahlten Partitionierung zu finden. $\Prob_i^2$ ist somit die Wahrscheinlichkeit zwei
beliebige, aber verschiedene Punkte gleichzeitig in der Box~$i$ anzutreffen:
\eqn{\Prob_i^2 =  \frac{1}{N(N-1)}\sum_{j,k} I_i(\x_j) I_i(\x_k).}
Hierbei ist $I_i$ die Indikatorfunktion der Box~$i$.  Die Summation von $\Prob_i^2$ "uber
alle Boxen ergibt nun die Wahrscheinlichkeit, irgend zwei Punkte gleichzeitig in einer
beliebigen Box anzutreffen.  Diese kann angen"ahert werden durch die Wahrscheinlichkeit,
zwei Punkte in einer Entfernung kleiner als der Boxdurchmesser anzutreffen.
Diese ist nun gegeben durch Verh"altnis aller Punktepaare mit einem
Abstand kleiner als $\eps$ zur Gesamtanzahl aller Punktepaare, d.h.
\eqn{\sum_i \Prob_i^2 \simeq \frac{1}{N(N-1)} \, \# \left\{ (i,j) \, \vert \, i \neq j \land  \norm{\x_i-\x_j}\leq \eps \right\} .} 
Die linke Seite der Gleichung kann aber wieder durch das Korrelationsintegral
\eqnref{cintdef} ausgedr"uckt werden. Es kann nun weiterhin gezeigt werden, da"s die
gemachten N"aherungen f"ur $\eps\to0$ verschwinden, und wir erhalten somit:
\eqn{D_2=D_C.} 
Die Korrelationsdimension ist also gleich der verallgemeinerten Dimension $D_2$ und dient
uns so als untere Absch"atzung f"ur die  Kapazit"atsdimension. 

Aufgrund der Unabh"angigkeit von $D_q$ von der speziellen Form der Partitionierung
(siehe \eqnref{gendim}) ist das Korrelationsintegral (im Grenzfall $\eps\to0$) unabh"angig von
der Wahl der Norm. F"ur die numerische Berechnung des Korrelationsintegrals ist es daher
aus Gr"unden der Geschwindigkeit sinnvoll, statt der euklidischen Norm $\norm{\cdot}_2$ die
Maximumsnorm $\norm{\cdot}_\infty$ zu verwenden. Da die Definition des
Korrelationsintegrals in \eqnref{cintdef} symmetrisch bez"uglich $i$ und $j$ ist, gen"ugt
die Berechnung desselben f"ur $i<j$:
\eqnl[cintdef2]{C(\eps) = \frac{2}{N(N-1)}\sum_{i<j}\Theta(\eps-\norm{\x_i-\x_j}_\infty).}



%%%%%%%%%%%%%%%%%%%%%%%%%%%%%%%%%%%%%%%%%%%%%%%%%%%%%%%%%%
\subsubsection{Numerische Berechnung des Korrelationsintegrals}
Bedeutung erlangt hat die Korrelationsdimension vor allem wegen der schnellen und
relativ genauen Berechenbarkeit des Korrelationsintegrals. 
Die Berechnung des Korrelationsintegrals geschieht nun in den folgenden Schritten.
\begin{enumerate}
\item Der minimale und der maximale Abstand $\rmin$ und $\rmax$ zweier Punkte der
Zeitreihe werden bestimmt. Unter Verwendung der Maximumsnorm kann f"ur den maximalen
Abstand $\rmax=\max\limits_{i,j}{\lvert x_i-x_j \rvert}$ gew"ahlt werden bzw.\  f"ur den
minimalen $\rmin=\min\limits_{i,j}{\lvert x_i-x_j \rvert}$. 
\item Der Bereich $[\rmin,\rmax[$ wird in $m$ Intervalle $[r_l,r_{l+1}[$ mit
$r_0=\rmin$ und $r_{m-1}=\rmax$ aufgeteilt. Die Aufteilung sollte logarithmisch (d.h.\
$r_{l+1}/r_l=\rho=\const$) erfolgen, um dem Skalierungsverhalten des Korrelationsintegrals
Rechnung zu tragen.
\item Den Intervallen wird ein Array $K$ zugeordnet, so da"s dem Intervall $[r_l,r_{l+1}[$
der Wert $K_l$ entspricht. Das Array $K$ dient dazu, bei der sp"ateren Kalkulation die
Anzahl der Punktepaare, deren Abstand in dem entsprechenden Intervall liegt, aufzunehmen.
\item F"ur alle $i<j$ wird der Abstand $r_{ij}=\norm{\x_i-\x_j}_\infty$
berechnet. Dasjenige $K_l$ mit $r_l\leq r_{ij}\lt r_{l+1}$ wird um eins erh"oht.

Die Berechnung des Index $l$ aus dem Abstand $r_{ij}$ ist einer der zeitkritischen Teile 
des Algorithmus. Der Index $l$ kann "uber $l=\lfloor log(r_{ij} / \rmax ) / log(\rho)
\rfloor + m$ berechnet werden. Da die Berechnung des Logarithmus sehr langsam ist, wird
hier folgenderma"sen verfahren. Statt des nat"urlichen Logarithmus wird der
Zweierlogarithmus benutzt. W"ahlt man nun das Abstandsverh"altnis $\rho$, so da"s
$\rho=2^{1/k}$ mit $k\in\N$ gilt, kann die obige Formel umgeschrieben werden zu $l=\lfloor
log_2\left( (r_{ij} / \rmax )^k \right) \rfloor + m$. Die Funktion $\lfloor
log_2(\cdot) \rfloor$ kann aufgrund der internen Zahlendarstellung von Computern sehr
schnell berechnet werden.
\item Aus den $K_l$ wird "uber $C(r_l)=\sum\limits_{m=0}^{l-1} K_m$ das Korrelationsintegral bestimmt. 
\end{enumerate}
Am effektivesten ist das Verfahren, wenn das Korrelationsintegral in einem Durchlauf f"ur verschiedene 
Einbettungsdimensionen $d=1\dots d_\tmax$ bestimmt wird. In diesem Fall mu"s das
eindimensionale Array $K$ durch ein zweidimensionales ersetzt werden, dessen zweite
Komponente die Einbettungsdimension spezifiziert. Schritt 4 ist folgenderma"sen
abzu"andern, da"s zuerst der Abstand $r_{ij,1}=\abs{\x_{i,1}-\x_{j,1}}$ f"ur die
Einbettungsdimension $d=1$ berechnet wird. Die folgenden Abst"ande folgen dann bei Verwendung
der Maximumsnorm aus $r_{ij,d}=\max(r_{ij,d-1},\abs{\x_{i,d}-\x_{j,d}})$. Der Index $l$
mu"s nur dann neu berechnet werden, wenn $r_{ij,d}\gt r_{ij,d-1}$ ist.

\epsfigsingle{corrint/perfect/corrint700b}
{Korrelationsintegral f"ur eine Zeitreihe ($N=7\times 10^5$)des R"ossler-Attraktor mit $\rho=2^{1/10}$,
$m=102$ und $d=1\dots 8$ (von oben nach unten). Die Abst"ande sind einheitlich auf
$\rmax=1$ skaliert worden. Aufgrund des Skalierungsverhaltens des
Korrelationsintegrals erfolgt die Darstellung doppelt logarithmisch.} 
{corrintperf}{-0.2cm}

\psref{corrintperf} zeigt eine Berechnung des Korrelationsintegrals f"ur eine aus dem
R"ossler-System \cite{Roessler76} gewonnenen Zeitreihe mit $7\times 10^5$ Punkten 
f"ur Einbettungsdimensionen $d=1\dots 8$. In der Abbildung sind deutlich zwei Bereiche zu
unterscheiden. F"ur $\log r>-1$ geht $\log C(r)$ gegen 0. Der Grund liegt einfach
darin, da"s der Attraktor nur einen begrenzten Raumbereich aufspannt und f"ur hinreichend
gro"ses $r$ alle Paare von Rekonstruktionspunkte  in der Summe von \eqnref{cintdef2}
erfa"st sind. Dieser Bereich nennt sich auch \begriff(S"attigungsbereich) des
Korrelationsintegrals. Den Abstand, ab dem S"attigung auftritt, bezeichnen wir mit
$\rsaett$. F"ur $\log r<-1$ zeigt das Korrelationsintegral das 
 erwartete Skalierungsverhalten $C(r)\propto r^\nu$, wobei die Konstante
$\nu$ f"ur unterschiedliche Einbettungsdimensionen verschiedene Werte annimmt. Da der
R"ossler-Attraktor die Korrelationsdimension $\corrdim\sim 2.07>2$ besitzt, skaliert das $C(r)$
f"ur Einbettungsdimensionen $d\leq 2<\corrdim$ mit $r^d$, da in diesem Fall der gesamte Phasenraum
aufgespannt wird. F"ur Einbettungsdimensionen $d>\corrdim$ skaliert das Korrelationsintegral
mit $r^{\corrdim}$. Dieser Bereich, auf dessen Bestimmung bei der Korrelationsanalyse das
Hauptaugenmerk liegt, hei"st \begriff(Skalierungsbereich)
\footnote{F"ur die Dimensionsberechnung ben"otigen wir also nur $d>\corrdim$. Die Bedingung
$d>2D_H$ (nach Takens Theorem) ist hier nicht notwendig zu erf"ullen, da es bei
Dimensionsberechnungen nicht auf die Eineindeutigkeit von $\diffeo$ ankommt. Dieses Resultat geht
auch hervor aus dem Projektionssatz f"ur fraktale Mengen \cite{Falconer93}.}.

\subsubsection{Berechnung der Korrelationsdimension aus dem Korrelationsintegral}
\paragraph{Ableitung des Korrelationsintegrals}
F"ur die numerische Berechnung der Korrelationsdimension ist die Definition
\eqnref{cdimdef} schlecht geeignet, da die Gleichheit nur im Grenzfall $r\to 0$ gilt. F"ur 
endliche $r$ gilt wegen $C(r)=k r^{\corrdim}:$
\eqn{\mcorrdim(r)=\corrdim + \frac{\log k}{\log r}.}
Aufgrund des zweiten Summanden ist die Konvergenz gegen $\corrdim$ logarithmisch langsam.
Besser geeignet ist die Berechnung von $\corrdim$ "uber die Ableitung, da hier der Einflu"s
des Proportionalit"atsfaktors wegf"allt:
\eqn{\mcorrdim(r)=\abl{\log C(r)}{\log r}=\frac{\abls{C(r)}{r}}{C(r)/r} .}
Numerisch bestimmt wird die Ableitung an einer Stelle $r_i$, indem durch $2k+1$
Nachbarpunkte\footnote{D.h. die Punkte $(\log r_{i-k}, \log C(r_{i-k}),\dots,(\log r_i,
\log C(r_i),\dots,(\log r_{i+k}, \log C(r_{i+k})$. } von $r_i$ eine Regressionsgerade
gelegt und deren Steigung ermittelt wird. Einen Graphen von $\mcorrdim(r)$ "uber $\log r$
zeigt \psref{corrslpperf}.

\epsfigsingle{corrint/perfect/corrslp700b}
{Ableitung des Logarithmus des Korrelationsintegrals aus \psref{corrintperf}. In die
Berechnung der Steigung wurden f"unf Nachbarpunkte (d.h. $k=2$) mit einbezogen. Der
Skalierungsbereich liegt ungef"ahr zwischen $\log(\rmin)=-5.7$ und $\log(\rmax)=-3.3$.}
{corrslpperf}{-0.2cm}
Die Berechnung der Korrelationsdimension "uber die Ableitung hat jedoch einige
Nachteile. Zum einen schwankt der Wert von $\mcorrdim(r)$ relativ stark f"ur
unterschiedliche $r$. Zum anderen wird ein gro"ser Teil Informationen, die das
Korrelationsintegral liefert \naja(verschwendet), da nur ein sehr begrenzter Teil f"ur
die Berechnung verwendet wird. \psref{corrslpperf} liefert trotzdem wichtige Informationen zur 
Berechnung der Korrelationsdimension. Aus der Abbildung 
kann gut abgelesen werden in welchem Bereich die Steigung des
Korrelationsintegrals konstant bleibt. Wir k"onnen hier"uber die Grenzen des
Skalierungsbereiches $\rsmin$ und $\rsmax$ bestimmen. 

\paragraph{Steigung und lineare Regression}
Eine verl"a"slichere Sch"atzung der Korrelationsdimension erhalten wir, indem wir die
Steigung "uber den gesamten Skalierungsbereich ermitteln. Als Grenzen des
Skalierungsbereiches k"onnen beispielsweise die im vorigen Abschnitt gewonnenen Werte
$\rsmin$ und $\rsmax$ genommen werden. F"ur die Korrelationsdimension ergibt sich dann:
\eqn{\mcorrdim=\frac{\log C(\rsmax)-\log C(\rsmin)}{\log \rsmax-\log(\rsmin)} .}
Die Ermittelung von $\rsmin$ und $\rsmax$ aus der Slopekurve geschieht im allgemeinen
manuell. H"aufig ist dies auch sinnvoll, da die Daten zuerst mit dem Auge begutachtet
werden sollten, bevor eine Aussage "uber die Korrelationsdimension einer Zeitreihe gemacht
wird. 

Die Steigung des Korrelationsintegrals im Bereich $I=[\rmin,\rmax[$ kann nat"urlich auch
durch Bestimmung einer Regressionsgeraden berechnet werden. Die Steigung dieser Geraden
ist gegeben durch
\eqn{\mcorrdim = \frac{\sum\limits_{i\in I} (\log C(r_i)-\overline{\log C(r)}(\log r_i
-\overline{\log r_i}))}{\sum\limits_{i\in I}(\log r_i-\overline{\log r})^2} ,}
wobei $\overline{\log C(r)}$ und $\overline{\log r_i}$ die Mittelwerte von $\log C(r)$
bzw.\ $\log r$ im betrachteten Intervall $I$ sind. Es mu"s jedoch angemerkt werden, da"s
eine der Voraussetzungen f"ur die Durchf"uhrung einer linearen Regression hier nicht
erf"ullt ist. Die Werte von $\log C(r)$ sind f"ur verschiedene Werte von $r$ nicht
voneinander unabh"angig. Zudem hat die Absch"atzung des Fehlers bei einer solchen Methode
kleinster Quadrate meist wenig mit dem wirklichen Fehler bei der Dimensionsberechnung zu
tun. Wir werden sp"ater eine Methode entwickeln, die diese Schwachstellen nicht teilt (siehe
Abschnitt \ref{chaptakensest}).

\paragraph{Korrelationskoeffizient}
F"ur die Bestimmung der Korrelationsdimension vieler Zeitreihen ist jedoch ein Verfahren
zur automatischen Bestimmung des Skalierungsbereiches angebracht. Verfahren dieser Art
sind vielf"altig entwickelt worden, wir wollen jedoch nur eines, welches hier auch
Anwendung gefunden hat, besprechen. Der Skalierungsbereich ist derjenige, in der die
Auftragung $\log C(r)$ "uber $\log r$ mit konstanter Steigung $\mcorrdim$
verl"auft. Es ist also Aufgabe des Verfahrens, aus dem $\log C(r)$-$\log r$-Graph den
\naja(geradesten) Teil herauszufinden. 

Daf"ur ist es notwendig, ein Ma"s daf"ur zu finden, wie \naja(gerade) der Graph in einem
bestimmten Bereich verl"auft \cite{Raidl}. Ein solches Ma"s ist der 
\begriff(Korrelationskoeffizient)\footnote{Der (\begriff(Pearsonsche))
\begriff(Korrelationskoeffizient) $r(X,Y)=\mathrm{Cov}(X,Y)/\sigma_X \sigma_Y$ beschreibt die G"ute 
der linearen Vorhersagbarkeit der Zufallsvariable $Y$ durch die Zufallsvariable $X$. Er ist 
eng gekoppelt mit dem Optimierungsproblem $Y$ m"oglichst gut (linear) aus $X$
vorherzusagen, d.h.\  den Vorhersagefehler $E(Y-(a+bX))^2$ zu minimieren. Es gilt
$\min(E(Y-(a+bX))^2) = \sigma_Y(1-r^2(X,Y))$.} 
$r(X,Y)$, wobei $X$ und $Y$ zwei beliebige Zufallsvariablen sein m"ogen. Er wird $1$ 
bzw.\  $-1$, wenn zwischen $X$ und $Y$ ein exakt linearer Zusammenhang
besteht. Bei schw"acheren Zusammenh"angen wird er betragsm"a"sig kleiner bzw.\  $0$, falls 
gar keiner besteht. Betrachten wir nun $\log C(r)$ und $\log r$ in einem Intervall
$I=[r_1,r_2[$ als Zufallsvariablen, so erhalten wir mit $L(I)=\abs{ r(\log r, \log
C)\rvert_{I}}$ ein Ma"s f"ur die Linearit"at des Korrelationsintegrals in diesem
Bereich:
\eqn{L(I) = \left| \frac{\frac{1}{n-1} \sum\limits_{r\in I}\left( \log r - \overline{\log
r}\right)\left( \log C(r) - \overline{\log C(r)} \right) } {\sqrt{\left[ \frac{1}{n-1}
\sum\limits_{r\in I}\left( \log r - \overline{\log r}\right) \right] \left[ \frac{1}{n-1}
\sum\limits_{r\in I}\left( \log C(r) - \overline{\log C(r)} \right) \right ]}} \right| .} 
Hierbei bezeichnen $\overline{\log r}$ und $\overline{\log C(r)}$ jeweils die
im Intervall $I$ gemittelten Gr"o"sen. Um den Skalierungsbereich zu finden wird nun $L(I)$ 
f"ur alle m"oglichen Intervalle berechnet, und dasjenige, welches das maximale $L$
liefert, als Skalierungsbereich benutzt. Hier sind jedoch zwei Einschr"ankungen zu
beachten. Zum einen sollte eine Mindestl"ange f"ur die Intervalle vorgegeben
sein. Ansonsten tendiert der Algorithmus dazu, sehr kleine Intervalle auszuw"ahlen. Zum
anderen m"ussen Unter- und Obergrenzen f"ur $r$ vorgegeben werden, da das
Korrelationsintegral im Bereich des Digitalisierungsrauschens und der S"attigung  auch
einen linearen Bereich (mit Steigung 0) besitzt. F"ur das Korrelationsintegral aus \psref{corrintperf} 
liefert das Verfahren die Werte $\log \rmin = -5,7$ und $\log \rmax = -3,6$. Diese stimmen gut 
mit denen "uberein, die man auch aus \psref{corrslpperf} absch"atzen w"urde.

\paragraph{Takens' Sch"atzer}
\label{chaptakensest}
Die Berechnung der Korrelationsdimension "uber eine Regressionsgerade durch ein auf
irgendeine Weise festgelegtes Intervall hat jedoch zwei Nachteile. Zum einen bleiben
Informationen "uber den Verlauf von $C(r)$ unterhalb von $\rmin$ ungenutzt. Gerade kleine
Werte von $r$ sollten jedoch nach \eqnref{cdimdef} Informationen "uber die
Korrelationsdimension liefern. Zum anderen wird nicht ber"ucksichtigt, da"s
aufeinanderfolgende Werte von $C(r)$ nicht voneinander unabh"angig sind. Einen Ausweg
hieraus bietet der sogenannte \begriff(Takens' Sch"atzer) (Takens' estimator) \cite{Takens85a}. 

Wir m"ussen hier einen wahrscheinlichkeitstheoretischen Ansatz f"ur das
Korrelationsintegral machen. Betrachten wir die Wahrscheinlichkeit $\Prob$, da"s zwei
Attraktorpunkte $\x_i$ und $\x_j$ einen Abstand $\rij$ kleiner als $r$ haben, so erhalten wir:
\eqn{\Prob(\norm{\x_i-\x_j}<r)=\frac{\sum\limits_{i\neq
      j}\Theta(r-\norm{\x_i-\x_j})}{N(N-1)}=C(r) .}

Die betrachtete Wahrscheinlichkeit ist also genau gleich dem Korrelationsintegral $C(r)$.
Da uns nur das Skalierungsverhalten unterhalb des S"attigungsbereiches interessiert,
legen wir eine Obergrenze $\rmax<\rsaett$ f"ur $r$ fest. Alle Punktepaare mit Abstand
$r\geq\rmax$ werden ignoriert und wir betrachten die bedingte Wahrscheinlichkeit $\tilde
\Prob$, da"s zwei dieser Punktepaare einen Abstand kleiner $r$ haben m"ogen:
\eqn{\tilde \Prob(\rij<r)=\Prob(\rij<r \vert \rij < \rmax) = \frac{C(r)}{C(\rmax)}   .}

Da $C(r)$ im Skalierungsbereich idealerweise wie $kr^\corrdim$ skaliert, verh"alt sich
$\tilde \Prob(\rij<r)$ idealerweise wie $(r/\rmax)^\corrdim$. 

Unsere Aufgabe ist nun den freien Parameter $D_2$ der Wahrscheinlichkeitsverteilung
$\tilde\Prob$ abzusch"atzen.  Dies geschieht in der Statistik "ublicherweise durch
Anwendung der \begriff(Maximum-Likelyhood-Regel).  Sie besagt, da"s als Sch"atzwert f"ur
einen unbekannten Parameter einer Wahrscheinlichkeitsverteilung ein solcher Wert des
Parameters verwendet wird, bei dessen Vorliegen der konkreten Stichprobe eine m"oglichst
gro"se Wahrscheinlichkeit zukommt.


Die vorliegende konkrete Stichprobe
besteht aus den gemessenen Abst"anden $r_{ij}<\rmax$. Aus Gr"unden der Vereinfachung
"andern wir hier die Indizierung, und betrachten die Folge $\folge(r,1,m)$, welche alle
$r_{ij}$ beinhalten soll.  Die Wahrscheinlichkeit $L$ eine solche Stichprobe zu erhalten,
ist das Produkt der Wahrscheinlichkeiten f"ur die Messung jedes einzelnen Abstands:
\eqn{L_\mathrm{disk}(\folge(r,1,m);D_2) = \prod_{i=1}^m\tilde\Prob(r=r_i) .}
Die so definierte \begriff(Likelyhood-Funktion) ist jedoch nur f"ur diskrete
Wahrscheinlichkeitsverteilungen verwendbar.  Da die vorliegende Verteilung kontinuierlich
ist, ist die Wahrscheinlichkeit f"ur die exakte Gleichheit $\tilde\Prob(r=r_i)$ null, und
somit ist auch die Likelyhood-Funktion identisch null. Es kann nun gezeigt werden, da"s f"ur
kontinuierliche Verteilungen die Wahrscheinlichkeitsdichte ma"sgeblich ist\footnote{Man
  betrachtet statt den Wahrscheinlichkeiten $\Prob(r=r_i)$ die Wahrscheinlichkeiten $\Prob(r_i\leq r <r_i+
  dr_i) = f(r_i)dr_i$. Das Maximum der Likelyhood-Funktion ist unabh"angig von
  der Wahl der $dr_i$, so da"s direkt die Wahrscheinlichkeitsdichte $f$ benutzt werden kann.  }.
Wir erhalten als Likelyhood-Funktion somit:
\eqnl[maxlike1]{L(\folge(r,1,m);D_2) = \prod_{i=1}^m f(r_{i}) ,}
wobei die Wahrscheinlichkeitsdichte durch:
\eqn{f(r) = \abl{\tilde \Prob}{r} =   \corrdim  (r/\rmax)^{\corrdim-1}  }
gegeben ist.  Der Sch"atzwert f"ur $D_2$ ist nun derjenige, f"ur den $L$ ein Maximum
annimmt. 

Da die Ableitung des Produkts in \eqnref{maxlike1} sowie die Berechnung derer
Nullstellen recht kompliziert ist, verwendet man einen \naja(Standardtrick). Da der
Logarithmus eine streng monoton wachsende Funktion ist, besitzen $L$ und $\ln
L$ die gleichen Extremstellen. Das Produkt geht hierbei in eine Summation "uber, und die
Berechnung des Maximums vereinfacht sich erheblich:
\eqna{\abl{\ln L(\corrdim)}{\corrdim}&=& \abl{}{\corrdim}\left( m\ln\corrdim + (\corrdim-1)\sum_{i=1}^m\ln (r_i/\rmax) \right)\\
&=& \frac{m}{\corrdim}+\sum_{i=1}^m\log (r_i/\rmax).}


\comment{
Diese Aussage ist folgenderma"sen zu verstehen. Die erzeugte Verteilung hat in Bezug auf
die theoretische Verteilung eine bestimmte Wahrscheinlichkeit $L$, die vom Parameter
$\corrdim$ abh"angt. Wir bestimmen $\corrdim$ nun so, da"s $L(\corrdim)$ ein Maximum
annimmt. Welche Wahrscheinlichkeit hat nun die gemessene Verteilung? Zun"achst einmal
ordnen wir die $\rij$ so an, da"s wir einen Folge $r_1,\dots,r_m$ erhalten. Die
Wahrscheinlichkeit genau diese Folge zu messen, ist offensichtlich gleich:
\eqn{\Prob(r=r_1)\cdot \Prob(r=r_2)\cdot\dots\cdot \Prob(r=r_m).}
Diese ist jedoch 0, da wir es mit einer kontinuierlichen Verteilung zu tun haben und
somit die Einzelwahrscheinlichkeiten 0 sind. Wir suchen daher die Wahrscheinlichkeit
die Verteilung in den Intervallen $[r_i,r_i+\d r_i[$ gemessen zu haben, wobei die $\d r_i$ 
beliebig sind.\comment{ -- die Intervalle sollten sich jedoch nicht "uberschneiden.}  Wir erhalten
damit f"ur die gesuchte Wahrscheinlichkeit
\eqn{L(\corrdim)=\prod_{i=1}^m \Prob(r_i\leq r < r_i + \d r_i)}
$L(\corrdim)$ ist die sogenannte \begriff(Maximum-Likelyhood-Funktion).
Bei der theoretischen Verteilung $\Prob(r<\eps)=k\eps^\corrdim$ erhalten wir 
$\Prob(\eps<r<\eps+\d\eps)=k\corrdim\eps^{\corrdim-1}\d\eps$ und somit
\eqn{L(\corrdim)=\prod_{i=1}^m k \corrdim  r_i^{\corrdim-1}\d r_i}
Um das Maximum der Maximum-Likelyhood-Funktion zu finden, logarithmieren wir die obige
Gleichung und leiten nach $\corrdim$ ab
\eqna{\abl{\log L(\corrdim)}{\corrdim}&=& \abl{}{\corrdim}\left( m\log\corrdim + (\corrdim-1)\sum_{i=1}^m\log r_i +
\log k + \sum_{i=1}^m\d r_i\right)\nonumber\\
&=& \frac{m}{\corrdim}+\sum_{i=1}^m\log r_i}

Hier wird nun auch ersichtlich warum die spezielle Wahl der $\d r_i$ beliebig
war. 
}
Nullsetzen der Gleichung ergibt den Sch"atzwert f"ur $\corrdim$:
\eqn{\corrdim=-\frac{m}{\sum_{i=1}^m \ln(r_i/\rmax)}=-\frac{1}{<\ln (r_i/\rmax)>} .}

Dies ist der sogenannte \begriff(Takens' Sch"atzer). \autor(Takens) zeigte, da"s
diese Sch"atzung erwartungstreu ist, was bei Maximum-Likelyhood-Methoden
nicht immer der Fall ist\footnote{F"ur eine \begriff(erwartungstreue) Sch"atzung ist
  der Erwartungswert des Sch"atzwerts gleich dem zu sch"atzenden Wert. Beispielsweise ist
  die Sch"atzung $s^2=\frac1{N-1}\sum_{i=1}^N(x_i-\bar x)^2$ eine erwartungstreue
  Sch"atzung f"ur die Varianz $\sigma^2$ der Stichprobe $(\folge(x,1,N))$, da
  $<s^2>=\sigma^2$ gilt. Die "uber die Maximum-Likelyhood-Methode gewonnene Sch"atzung
  ${s^*}^2=\frac1{N}\sum_{i=1}^N(x_i-\bar x)^2$ ist nur \begriff(asymptotisch
  erwartungstreu), da nur f"ur den Grenzwert $\lim_{N\to\infty}{s^*}^2> = \sigma^2$ gilt.
  Maximum-Likelyhood-Sch"atzungen sind immer mindestens asymptotisch erwartungstreu. }.
Weiterhin ist diese Sch"atzung die \begriff(wirksamste), da die Varianz des
Sch"atzwertes f"ur diese Sch"atzung ein Minimum annimmt. 

Da das Korrelationsintegral, wie oben gesehen eine Wahrscheinlichkeitsverteilung f"ur die
$r_i$, darstellt, kann der Erwartungswert $<\ln (r_i/\rmax)>$ jetzt "uber $C(r)$ ausgedr"uckt
werden. Die Wahrscheinlichkeit, bei der gemessenen Verteilung einen Wert zwischen $r$ und
$r+\d r$ zu messen, betr"agt $\frac{\d  C(r)}{\d r}\frac{\d r}{C(\rmax)}$. Wir erhalten
also f"ur ein gegebenes $\rmax$:
\eqn{\corrdim(\rmax) = \int\limits_0^\rmax \abl{C(r)}{r}\frac{\ln (r/\rmax)}{C(\rmax)}\d r .}
Da wir das Korrelationsintegral jedoch nur f"ur diskrete Werte $r_i$ bestimmen k"onnen,
geht die obige Gleichung "uber in:
\eqn{\corrdim(r_n) = \sum\limits_{i=1}^{n-1} \frac{C(r_i+1)-C(r_i)}{C(r_n)}\ln(r_i/r_n), }
wobei $r_n=\rmax$ gesetzt wurde.
Ein Vergleich dieser Methode mit der im vorigen Abschnitt beschriebenen zeigt, da"s der
Takens' Sch"atzer bei bekannten Systemen im allgemeinen bessere Werte liefert (siehe \psref{corrdimcomp}).
%\inkorrektur(Fehlerabsch"atzung)

\epsfigdouble{corrint/perfect/corrdim700b}{corrint/perfect/takdim700b} { Vergleich der
  Bestimmung der Korrelationsdimension $\corrdim$ in Abh"angigkeit von der
  Einbettungsdimension $\embed$ durch Regressionsgeraden (links) bzw.\ Takens' Sch"atzer
  (rechts) f"ur das Korrelationsintegral aus \psref{corrintperf}. Die "uber Takens'
  Sch"atzer bestimmten Werte der Korrelationsdimension ($\corrdim=2,07\pm0.01$) $\corrdim$
  stimmen deutlich besser mit den theoretischen Werten "uberein.  }  {corrdimcomp}{-0.2cm}

\subsubsection{Fehler bei der Korrelationsanalyse}
Das Korrelationsintegral in \psref{corrintperf} ist nahezu optimal im Sinne des erwarteten 
theoretischen Verlaufs. Der Skalierungsbereich 
erstreckt sich "uber einige Gr"o"senordnungen ($\rmax/\rmin\simeq 54)$ und die Steigung innerhalb
dieses Bereichs ist nahezu konstant (siehe \psref{corrslpperf}). Der Grund liegt vor allem
darin, da"s f"ur die Berechnung eine gro"se Anzahl Datenpunkte ($7\times10^5$) vorlag und
mit nahezu rauschfreien Daten gearbeitet wurde. In experimentellen Situationen liegt
beides meist nicht vor. Die Fehler, die hieraus (und auch aus anderen Quellen) resultieren,
sollen im folgenden diskutiert werden.




\paragraph{Endliche Datenmenge}
Ein Faktor, der die verl"a"sliche Bestimmung der Korrelationdimension wesentlich
beeinflu"st, ist die Menge der verf"ugbaren Daten. Das Korrelationsintegral hat einen
Wertebereich von $2/N(N-1)$ bis $1$. Da $C(r)$ f"ur $r<\rmax$ mit $(r/\rmax)^\corrdim$ skaliert, tragen
bei Abst"anden der Gr"o"senordnung $\rmax(2/N^2)^{1/\corrdim}$ nur noch wenige Punktepaare 
zum Korrelationsintegral bei. Daraus resultieren in diesem Bereich starke statistische
Schwankungen von $\mcorrdim(r)$. Aus diesem Verhalten leiteten \autor(Eckmann und Ruelle)
\cite{Eckmann-ruelle2} eine Gleichung f"ur die minimal erforderliche Datenmenge f"ur
Dimensionsberechnungen ab. Bezeichnet man das Verh"altnis von gr"o"sten zu kleinsten
Skalen mit $\rho$, so ergibt sich f"ur die minimale Datenmenge:
\eqnl[corrdimnminer]{N_\tmin=\sqrt{2\rho^\corrdim}}
oder andererseits bei fester Datenmenge die maximal berechenbare Dimension:
\eqnl[corrmaxdim]{D_{2,\tmax}=\frac{2\log N}{\log \rho}.}
Nun mu"s f"ur eine vern"unftige Absch"atzung der Dimension das Skalenverh"altnis $\rho$
hinreichend gro"s sein. Eckmann und Ruelle geben hier ein minimales Verh"altnis von
$\rho_\tmin=10$ an, so da"s man etwa bei $N=1000$ Datenpunkten maximal eine
Korrelationsdimension $\corrdim\leq 6$ sinnvoll bestimmen kann. Diese Grenze scheint mir
allerdings sehr hoch, da sich bei $N=1000$ schon die Dimensionen von Attraktoren mit
$\corrdim\simeq 2$ nur sehr schlecht bestimmen lassen. Andererseits kann das Ergebnis gut
als wirkliche obere Grenze f"ur die Dimensionsberechnung angesehen werden. 




\paragraph{Kanten und endliche Ausdehnung des Attraktors}
Wie wir bereits in \psref{corrintperf} gesehen haben, geht das Korrelationsintegral ab einem
bestimmten Wert $\rsaett$ in einen S"attigungsbereich "uber. Dies ist eine Konsequenz der
endlichen Ausdehnung des Attraktors. Effekte, die aus dem S"attigungsverhalten resultieren, 
sind schon auf L"angen\-skalen weit unterhalb der linearen Ausdehnung des Attraktors
erkennbar. Sobald eine wesentliche Anzahl Punkte einen Abstand von weniger als $r$ vom
\naja(Rand) des Attraktors hat, weicht das Skalierungsverhalten des Korrelationsintegrals
stark vom theoretischen Verlauf ab.

Der Wert $r_s$, f"ur den das Skalierungsverhalten durch den \begriff(Kanteneffekt)
(engl.: edge effect) abbricht, h"angt sehr von
der Dimension und der Geometrie des Attraktors ab. 
F"ur einen $m$-dimensionalen Hyperkubus der Kantenl"ange 1 kann das 
Korrelationsintegral jedoch exakt berechnet werden\footnotemark:
\footnotetext{Wie bereits gezeigt wurde gilt f"ur das Korrelationsintegral
$C(r)=\Prob(\rij<r)$. Bei einem homogenen Hyperkubus sind die einzelnen Komponenten des
Abstandsvektors voneinander statistisch unabh"angig und es gilt bei Verwendung der
Maximumsnorm $\Prob(\rij)=\Prob((\vec r_{ij,1} < r) \land\dots\land (\vec r_{ij,m} < r)) = \Prob_1(\vec
r_{ij,1} < r)\cdot\dots\cdot \Prob_1(\vec r_{ij,m} < r)$. Da au"serdem die Verteilungen der
einzelnen Komponenten gleich sind folgt $\Prob(\rij)=\Prob_1(\vec r_{ij,1} < r)^m$. F"ur die
Verteilung $\Prob_1$ gilt nun $\Prob_1(\rij<r)=\int_0^1 \{\min(1,r'+r)-\max(0,r'-r)\}\d r'=
2r-r^2$. Somit folgt die Behauptung.
}
\eqnl[corrinthyper]{C(r)=(2r-r^2)^m.}
Unter der vereinfachenden Annahme, da"s sich der Attraktor "ahnlich einem Hyperkubus
der Dimension $\corrdim$ und Kantenl"ange $\rmax$ verh"alt, kann die Korrelationsdimension entsprechend
\eqnref{corrinthyper} durch Auftragung von $\log C(r)$ "uber $\log(2r/\rmax-(r(\rmax)^2)$
abgesch"atzt werden. Die sich hieraus ergebenden Korrekturen  
sind jedoch "au"serst gering (wenige Promille).

Die Eigenschaft des Korrelationsintegrals ab einer bestimmten Obergrenze in S"attigung zu
gehen wurde von \autor(Nerenberg und Essex) benutzt, um eine Untergrenze der
ben"otigten Datenmenge zur Berechnung der Korrelationsdimension seltsamer Attraktoren
 zu bestimmen \cite{Nerenberg-essex}. Ihre Berechnung beruht darauf, da"s der Skalierungsbereich
von unten durch die Menge der verf"ugbaren Daten beschr"ankt wird. Kennzeichnend hierf"ur
ist der charakteristische Abstand n"achster Nachbarn $r_n$. Von oben ist der Skalierungsbereich
beschr"ankt durch den durchschnittlichen Abstand der Punkte zum Rand des Attraktors
$r_s$. Fallen beide zusammen, verschwindet der Skalierungsbereich und eine
Dimensionsbestimmung ist nicht mehr m"oglich. Beide Werte werden nun wiederum f"ur einen
homogenen Hyperkubus der Dimension $m$ und Kantenl"ange 1 berechnet\footnotemark:
\eqna{r_n &=& \frac{1}{2(m+1)}\nonumber\\
r_s &=& N^{-1/m} .}
Gleichsetzen der beiden Grenzen $r_n$ und $r_s$ ergibt:
\eqnl[corrnminnea]{N_\tmin(m)=\{2(m+1)\}^m .}
Zur Berechnung der Dimension eines Attraktors der Korrelationsdimension $\corrdim$ sind nach
dieser Absch"atzung $N_\tmin(\lceil \corrdim \rceil)$ Datenpunkte n"otig\footnote{$\lceil x \rceil$
bedeutet hier, die n"achste ganze Zahl, gr"o"ser oder gleich $x$.}. Zum Vergleich mit der
weiter oben angegebenen Formel von \autor(Eckmann und Ruelle) setzen wir das Verh"altnis
von $r_s$ zu $r_n$ gleich $\rho$. Damit erhalten wir:
\eqnl[corrnminneb]{N_\tmin(m,\rho)=\{2(m+1)\rho\}^m.}
Dies ergibt beispielsweise f"ur Attraktoren der Dimension 2 eine minimale Datenmenge von
ca.\  4000 Punkten bzw.\  500000 Punkte f"ur die Dimension 3. Diese Absch"atzung deckt
sich eher mit meinen Erfahrungen bei Dimensionsberechnungen. 





\paragraph{Wei"ses und Digitalisierungsrauschen}
Der Skalierungsbereich des Korrelationsintegrals ist nach unten, au"ser durch die Menge der 
verf"ugbaren Daten, durch Rauschen beschr"ankt. Die Auswirkungen beider Arten von
Rauschen -- wei"ses und Digitalisierungsrauschen -- m"ussen zun"achst getrennt behandelt
werden. 

Wir betrachten dem Signal "uberlagertes, wei"ses Rauschen der St"arke $\xi$. F"ur Abst"ande 
$r>\xi$ spielt das Rauschen nur eine untergeordnete Rolle. Die Abst"ande zwischen den
Attraktorpunkten werden nicht wesentlich beeinflu"st und das Korrelationsintegral skaliert 
wie bei rauschfreien Daten. 
Im Bereich $r<\xi$ wird das Signal jedoch stark vom Rauschen dominiert. Die
Wahrscheinlichkeit in der Umgebung eines Attraktorpunktes einen weiteren Punkt zu finden
verh"alt sich also wie bei einem reinen Rauschsignal. Da Rauschen jedoch immer den ganzen
Phasenraum aufspannt, skaliert das Korrelationsintegral hier mit der Einbettungsdimension
$d$, d.h.\  $C(r)\propto r^d$. Dieser Effekt ist dargestellt in \psref{corrnoise} links.

Bei Digitalisierungsrauschen verh"alt sich die Sache etwas anders. Sei $\eps$ der
Diskretisierungslevel. Dann ist jeder Me"swert und somit auch jeder Abstand zwischen
Attraktorpunkten  ein ganzzahliges Vielfaches von $\eps$ \footnotemark. Das Korrelationsintegral
verl"auft daher nicht kontinuierlich sondern in Spr"ungen bei jedem $r=n\eps$ ($n\in\N$)
(siehe \psref{corrnoise} rechts). 
Zus"atzlich werden verschiedene Attraktorpunkte auf ein und denselben Punkt
abgebildet, so da"s Punkte mit $\rij=0$ mit endlicher Wahrscheinlichkeit auftreten. Ein
Algorithmus zur Berechnung der Korrelationsdimension mu"s mit diesen Punkten umgehen
k"onnen.
\footnotetext{Vorausgesetzt der Abstand wird "uber die Maximums- oder die Absolutnorm
bestimmt.}

Ein von \autor(Theiler) hergeleitetes Modell ergibt f"ur das Skalierungsverhalten des
Korrelationsintegrals eines $m$-dimensionalen Gitters ergibt $C(r)\propto(r+\eps/2)^m$. Er 
schl"agt daher vor bei einer diskretisierten Zeitreihe $\log C(r)$ gegen $\log(r+\eps/2)$
aufzutragen\footnotemark. Diese Korrektur hat, nach meiner Erfahrung, jedoch wenig
Einflu"s auf tats"achliche Dimensionsbestimmungen, sondern dient mehr dazu Singularit"aten 
bei der Berechnung von $\log r$ zu vermeiden.
\footnotetext{Sei $r_\eps$ das n"achste Vielfache von $\eps$ kleiner als $r$. Dann wird
$r$ mit 50 prozentiger Wahrscheinlichkeit auf $r_\eps$ bzw auf $r_\eps+\eps$ abgebildet. F"ur 
das Korrelationsintegral der diskretisierten Daten $C'(r)$ folgt daraus
$C'(r)=\{C(r_\eps)+C(r_\eps+\eps)\}/2$. N"aherung und ausnutzung des urspr"unglichen
Skalierungsverhaltens f"uhrt auf den oben angegebenen Term.}

Um Rauschen zu vermindern k"onnen verschiedene Filtertechniken angewandt werden. Die
einfachsten hiervon sind Tief- oder Bandpa"sfilter. Diese haben jedoch den Nachteil, da"s
ihre Anwendung dem System einen Freiheitsgrad hinzuf"ugt und somit auch die gemessene
Dimension erh"ohen kann (siehe Abschnitt \ref{chapcorrdimfiltered}). Zudem ist im
allgemeinen nicht direkt ersichtlich ab welcher Oberfrequenz gefiltert werden sollte. Das
Abschneiden aller Frequenzen oberhalb einer bestimmten Grenzfrequenz birgt auch die
Gefahr, da"s eventuell f"ur die Dynamik wesentliche Informationen verloren gehen.

Das Verfahren, welches hier zum Einsatz kommt beruht auf der in Abschnitt \korrektur(SVD)
beschriebenen Methode der Singular Value Decomposition. Hierbei verwenden wir allerding
nur die erste Komponente der Rekonstruktion, welche "uber die
Verz"ogerungskoordinatenabbildung dann wieder eingebettet wird (\begriff(Reembedding)).
Das Ergebnis des Verfahrens sieht man in \psref{corrfilter}. Die oben erw"ahnten Nachteile 
anderer Verfahren teilt diese Methode nicht, da sie sich automatisch dem System anpa"st
und das Problem der Dimensionserh"ohung hier prinzipiell nicht auftreten kann.

\epsfigdouble{corrint/errors/noise/corrint02b}{corrint/errors/discrete/corrint05b}
{Links das Korrelationsintegral zu einer Zeitreihe des Lorenz-Systems dem wei"ses
Rauschen "uberlagert ist. Die Varianz des Rauschsignals betr"agt 0,2 Prozent der Varianz des
Originalsignals. Rechts wurde dasselbe Signal diskretisiert in Schritten von 0,5 Prozent
der Varianz des Signals.
}{corrnoise}{-0.2cm}

\epsfigfour{corrint/errors/noise/svd/corrint10}{corrint/errors/noise/svd/corrint10sf}
{corrint/errors/noise/svd/corrslp10}{corrint/errors/noise/svd/corrslp10sf}
{Links das Korrelationsintegral und die Slopekurve des ungefilterten Signals aus
\psref{corrnoise}. Rechts die entsprechenden Kurven f"ur das SVD-gefilterte
Signal. Deutlich zu erkennen ist der gr"o"sere Skalierungsbereich im rechten Bild.
}{corrfilter}{-0.2cm}

\epsfigdouble{corrint/errors/discrete/corrint05b}{corrint/errors/discrete/corrint5bsf}
{Links das Korrelationsintegral der diskretisierten Zeitreihe aus \psref{corrnoise}. Auf
  der rechten Seite das Korrelationsintegral des SVD-gefilterten Signals. Die Aufweitung des
  Skalierungsbereiches ist betr"achtlich.
}{corrdiscrete}{-0.2cm}

\paragraph{Gefilterte Zeitreihen}
\label{chapcorrdimfiltered}
Wie im vorigen Abschnitt bereits erw"ahnt kann die Filterung von Zeitreihen "uber einfache 
\begriff(Tief-) oder \begriff(Bandpa"sfilter) die Dimension des Systems erh"ohen \cite{Badii-politi}. Wir wollen dazu einen
einfachen Tiefpa"s erster Ordnung betrachten. Sei $x$ das Originalsignal und $z$ das
gefilterte Signal. Dann kann die Zeitabh"angigkeit des gefilterten Signals durch die
folgende Differentialgleichung beschrieben werden:
\eqnl[corrlowpass]{\dot z(t) = -\eta z(t) + x(t).}
Hierbei ist $\eta$ die Grenzfrequenz des Filters. Offensichtlich wird die Zahl der
Freiheitsgrade des Systems durch Anwendung dieses Filters um eins erh"oht. Da"s durch den
Filter auch die Dimension des Systems erh"oht werden kann, zeigten \autor(Badii \etal)
(1987). Die Gleichung f"ugt dem System einen neuen Lyapunov-Exponenten
$\lambda_f=-\eta$ hinzu. Dies f"uhrt je nach Gr"o"se von $\lambda_f$ in Bezug auf die
anderen Lyapunov-Exponenten des Systems zu einer Erh"ohung der Lyapunov-Dimension
$D_L$. Unter Annahme der G"ultigkeit der  Kaplan-Yorke Vermutung $D_L=D_1$ f"uhrt dies
auch zu einer Zunahme der Informationsdimension $D_1$. Da"s auch die weiteren
verallgemeinerten Dimensionen erh"oht werden, kann hier nur vermutet werden, ist jedoch
wahrscheinlich.

Um dem entgegen zu wirken, ist von \autor(Chennaoui \etal) ein Verfahren entwickelt worden 
um den unbekannten Filterparameter $\eta$ zu bestimmen \cite{Chennaoui}. Durch Inversion der Gleichung
\eqnref{corrlowpass} kann dann, mit bekanntem $\eta$, die originale Zeitreihe wieder
extrahiert werden. Das Verfahren beruht jedoch auf der Berechnung der
Informationsdimension und funktioniert auch nur f"ur Filter der oben beschriebenen
Art. F"ur \begriff(akausale) Filter\footnote{Akausale Filter sind Filter, deren
Sprungantwort $h(t)$ bereits f"ur $t<0$ ungleich 0 ist. Der \begriff(ideale) Tiefpa"s
ist ein Beispiel f"ur solch ein Filter. Akausale Filter sind zwar in Echtzeit nicht zu
realisieren, ihrer Anwendung auf komplett vorliegende Zeitreihen steht jedoch nichts
entgegen.} konnte \autor(Mitschke) (1989) allerdings zeigen, da"s dieses Problem nicht 
auftaucht \cite{Mitschke}.

\paragraph{Autokorrelation}
\label{corrdimtheiler}
Aus deterministischen Systemen gewonnene Signale sind grunds"atzlich autokorreliert. Es
existiert eine Autokorrelationszeit $\tau_\ac$, so da"s f"ur Zeiten $\tau<\tau_\ac$ sind
$x(t)$ und $x(t+\tau)$ stark miteinander korreliert. Falls die Autokorrelationszeit gro"s 
gegen sie Sampling Time $\sample$ ist, kann im Korrelationsintegral eine anormale Stufe
auftreten. Nach einem Vorschlag von \autor(Theiler) l"a"st sich dies vermeiden, indem das
Korrelationsintegral nur "uber Punktepaare  gebildet wird, die zeitlich mindestens
$W\sample>\tau_\ac$ auseinanderliegen \cite{Theiler}. Das so korrigierte Korrelationsintegral lautet dann:
\eqnl[cintdefac]{C(\eps) = \frac{2}{(N-W+1)(N-W)}\sum_{i<=j-W}\Theta(\eps-\norm{\x_i-\x_j}_\infty).}
F"ur $W=1$ geht dies wieder in die urspr"ungliche Form von \eqnref{cintdef2} "uber. Bei
den hier untersuchten Zeitreihen trat dieser Effekt niemals deutlich auf. Da diese
Korrektur jedoch die Anzahl, der in die Berechnung eingehenden Punktepaare, nicht
wesentlich herabsetzt, wurde sie in allen Berechnungen des Korrelationsintegrals zur
Sicherheit vorgenommen.

\paragraph{Lakunarit"at}
\korrektur(Broggi)
Die fraktale Struktur einer Menge kann au"ser durch ihre Dimension auch durch die
sogenannte \begriff(Lakunarit"at) (lat.\  lacuna = Loch, H"ohle), einem von
\autor(B. Mandelbrot) \cite{Mandelbrot82} gepr"agten Begriff, charakterisiert werden. 
Bei Fraktalen gleicher Dimension ist dasjenige mit der h"oheren Lakunarit"at st"arker
texturiert und erscheint fraktal-"ahnlicher. Ein Beispiel f"ur zwei \begriff(Cantor-Mengen)
unterschiedlicher Lakunarit"at zeigt \psref{lacunarity}.
%
\epsfigdouble{corrint/errors/lacunarity/sevena}{corrint/errors/lacunarity/sevenb}
{Die ersten f"unf Konstruktionschritte zweier Cantor-Mengen. In jedem Schritt werden die
Intervalle in sieben gleich gro"se Teilintervalle unterteilt von denen im linken jeweils
das dritte, vierte und f"unfte und im rechten das zweite, vierte und sechste Teilintervall
gestrichen wird. Die Kapazit"at der entstehenden fraktalen Mengen ist 
gleich $(D^l_0=D^r_0=\ln4/\ln 7)$. Von beiden Fraktalen hat das linke jedoch eine h"ohere Lakunarit"at.
}{lacunarity}{-0.2cm}

Die Lakunarit"at eines Fraktals hat Auswirkungen auf die Dimensionsbestimmung. Die
Relation $C(r)\propto r^\corrdim$ ist hier falsch, da $C(r)$ eine stufenweise Funktion
(\begriff(Teufelstreppe)) ist \cite{Broggi88}. 
Die Auswirkungen sind Oszillationen im Korrelationsintegrals, d.h.\  $C(r)$ ist nun proportional zu
$k(r)r^\corrdim$, wobei $k(r)$ im allgemeinen periodisch in $\log r$ ist
(siehe \psref{corrintlac}). Der Einflu"s der  Lakunarit"at kann minimiert werden, wenn bei
der Dimensionbestimmung "uber volle Perioden von $k(r)$ gemittelt wird.
\autor(Grassberger) zeigte andererseits, da"s die Oszillationen im Grenzfall $r\to 0$
ausged"ampft werden\cite{Grassberger88}.
%
\epsfigfour{corrint/errors/lacunarity/canacint}{corrint/errors/lacunarity/canbcint}
{corrint/errors/lacunarity/canacslp}{corrint/errors/lacunarity/canbcslp}
{Korrelationsintegrale und Steigungskurven f"ur die Cantor-Mengen aus \psref{lacunarity}
(Die linke Menge ist durch \gpmarkb, die rechte durch \gpmarkd{} gekennzeichnet.). Die
Korrelationsdimension ergibt sich zu $\mcorrdim=0,710$ bzw.\  $\mcorrdim=0,718$, was sehr gut mit dem
theoretischen Wert $\corrdim=\ln 4 /\ln 7\simeq 0,712$ "ubereinstimmt\footnotemark.
}{corrintlac}{-0.2cm}
\footnotetext{Da es sich hier
um homogene Fraktale handelt gilt $D_q=\const$ also auch $\corrdim=D_0$.}.

\comment{
\epsfigsingle{bla}
{Alle Gnuplot Symbole \gpmarka, \gpmarkb, \gpmarkc, \gpmarkd, \gpmarke, \gpmarkf, \gpmarkg, \gpmarkh
}{gpsymbols}{-0.2cm}
}




\newpage \newpage
\section{Tests}
Eine der Fragen, die sich bei der Rekonstruktion von Attraktoren aus experimentellen
Zeitreihen stellte, war ob ob es sich hierbei wirklich um ein deterministisches System
handelt.  Man k"onnte hier m"oglicherweise versuchen, "uber die Dimension des
rekonstruierten Attraktors zu argumentieren. Die Dimension eines deterministischen,
nichtlinearen Systems hat immer einen endlichen Wert. Dagegen spannen die Rekonstruktionen
stochastischer Signale immer den ganzen Phasenraum auf. Die berechnete Dimension
konvergiert nicht mit steigender Einbettungsdimension. K"onnen also "uber die
Dimensionsberechnungen deterministische von stochastischen Systemen unterschieden werden?

Die Antwort ist leider \naja(Nein). Wie \autor(A. R. Osborne) und \autor(A. Provencale)
nachwiesen, k"onnen auch stochastische Systeme mit Leistungsspektren
$P(\omega)\propto\omega^{-\alpha}$, gegen eine endliche Korrelationsdimension
konvergieren\cite{Osborne89a}. F"ur $1<\alpha<3$ erhielten sie $D_2=2/(\alpha-1)$.  Dies
liegt an zeitlichen Korrelationen aufeinanderfolgender Werte in der Zeitreihe. Diese
k"onnen zwar durch die Methode von \autor(Theiler) (siehe Abschnitt \ref{corrdimtheiler})
vermieden werden, andererseits existieren noch andere Effekte, die eine Konvergenz der
Korrelationsdimension bei wei"sem oder farbigem Rauschen bewirken k"onnen.

\comment{
  \epsfigfour{surrogate/noise/noise}{surrogate/noise/fourier}{surrogate/noise/corrint}{surrogate/noise/corrdim}
  {Links oben Zeitreihe $1/f^2$ Rauschen. Rechts oben Fourierspektrum. Links unten
    Korrelationsintegral. Rechts unten Korrelationsdimension, s"attigt bei ca.\ $5,0\dots
    5,5$ }{einsfnoise}{-0.2cm} }

\subsection{Statitische Hypothesentests}
Es sind noch weitere M"oglichkeiten vorgeschlagen worden, Zeitreihen hinsichtlich eines
zugrundeliegenden deterministischen Systems zu untersuchen (beispielsweise der
Determinismustest von \autor(Kaplan) und \autor(Glass) \cite{kaplan-glass}).  Diese sind
jedoch in der Anwendung oft sehr beschr"ankt.  Die umfassendste und mathematisch
fundierteste M"oglichkeit diesem Problem zu begegnen, ergibt sich im Feld statistischer
Hypothesentests. Hiermit sind wir zugleich nicht mehr beschr"ankt auf die Frage nach dem
Determinismus, sondern haben ein Grundger"ust, um Fragen der verschiedensten Art an die
vorliegende Zeitreihen zu stellen. Beispielweise
\begin{itemize}
\item Sind die Daten nicht-gau"sverteilt ?
\item Gibt es zeitliche Korrelationen in der Zeitreihe ?
\item Existiert eine nichtlineare Struktur ?
\item Sind die Daten durch eine chaotische Dynamik erzeugt ?
\end{itemize}
Um eine dieser Frage zu beantworten, wird nun eine \begriff(Nullhypothese) $\nullhyp$
aufgestellt, welcher eine Verneinung eben dieser Frage entspricht.  Die Nullhypothese kann
nun weder bewiesen noch widerlegt werden. Man versucht hingegen, die Nullhypothese
abzulehnen, d.h. zu zeigen, da"s es unwahrscheinlich ist, da"s die Daten mit der Hypothese
in Einklang stehen.

Um genauer zu sein: der Nullhypothese $\nullhyp$ wird ein Proze"s bzw.\ eine Klasse von
Prozessen $\process$ zugeordnet, die mit $\nullhyp$ in Einklang stehen. Bei der ersten
Fragen w"are dies beispielsweise die Menge aller Prozesse, die gau"sverteilte Daten
erzeugen. Die Frage ist nun, ob die realen Daten durch einen Proze"s aus $\process$
erzeugt worden sein k"onnen. Hierzu wird eine \begriff(Teststatistik) $T$ berechnet. Liegt
der $T$-Wert der realen Daten au"serhalb des Bereichs, den man f"ur Prozesse aus
$\process$ erwarten kann, wird die Nullhypothese abgelehnt. Liegt der Wert innerhalb des
\begriff(Annahme-) oder \begriff(Akzeptanzbereichs) der Nullhypothese wird sie angenommen.
Man sagt hier auch, der Test h"atte versagt, die Nullhypothese abzulehnen, da das i.allg.\ 
das Ergebnis ist, das man haben m"ochte. Da die Teststatistik dazu dienen soll, die realen
Daten von mit der Nullhypothese konsistenten Prozessen $\process$ zu \naja(unterscheiden),
bezeichnet man $T$ auch als \begriff(Diskriminator).

\subsubsection{Fehler 1. und 2. Art}
Bei der Annahme oder Ablehnung einer Nullhypothese k"onnen jeweils Fehler auftreten. Der
erste Fehler ist, da"s die Nullhypothese abgelehnt wird, obwohl sie eigentlich wahr ist.
Man spricht hier von einem \begriff(Fehler 1. Art). Die Wahrscheinlichkeit $\alpha$, mit
der Fehler 1. Art auftreten, kann frei bestimmt werden. Dies geschieht, indem als
Annahmebereich des Tests das $(1-\alpha)$\begriff(-Konfidenzintervall) der Teststatistik
f"ur die betrachteten Prozesse gew"ahlt wird. Das $(1-\alpha)$-Konfidenzintervall ist der
Bereich der $T$-Werte, f"ur den mit Wahrscheinlichkeit $1-\alpha$ Realisierungen von
Prozessen aus $\process$ in diesem Bereich liegen\korrektur(einfacher formulieren).  Man
spricht bei einem Test mit vorgegebenem $\alpha$ auch von einem \begriff(Niveau
$\alpha$-Test) bzw.\ von einem \begriff(Test zum Signifikanzniveau $\alpha$). Anstatt die
Signifikanz von vorneherein festzulegen, wird ab und zu auch der $p$-Wert eines Tests
angegeben. Dies ist der kleinste Wert von $\alpha$, f"ur den die Nullhypothese gerade noch
abgelehnt w"urde.

Bei Annahme der Nullhypothese, wenn sie tats"achlich falsch ist, spricht man von einem
Fehler 2. Art.  Die Wahrscheinlichkeit f"ur das Auftreten solcher Fehler wird mit $\beta$
bezeichnet. Die komplement"are Wahrscheinlichkeit $1-\beta$ gibt an, wie \naja(gut) der
Test in der Lage ist, die Nullhypothese f"ur mit ihr inkonsistenten Daten abzulehnen. Man
bezeichnet $1-\beta$ daher auch als die \begriff(G"ute) des Tests.  Da die G"ute eines
Tests davon abh"angt, wie nicht-konsistent die wirklichen Daten mit der Nullhypothese
sind, kann $\beta$ im Gegensatz zu $\alpha$ nicht vorgegeben werden.  Allerdings h"angt
die G"ute des Tests von $\alpha$ ab -- je h"oher das Signifikanzniveau $\alpha$ des Test ist,
desto geringer ist $\beta$. Es ist andererseits nicht sinnvoll, um die G"ute $1-\beta$
gro"s zu machen, ein sehr hohes $\alpha$ zu w"ahlen. Man w"urde sich die h"ohere G"ute des Tests
mit einer geringeren Signifikanz, d.h.\ Aussagekraft, des Test erkaufen.

\begin{center}
\newcommand{\rb}[1]{\raisebox{1.5ex}[-1.5ex]{#1}}
\begin{tabular}{c|c|c}
 & $\nullhyp$ ist wahr & $\nullhyp$ ist falsch \\ \hline
& & \\
$\nullhyp$ wird & falsche Entscheidung &    \\
abgelehnt & Fehler 1. Art & \rb{richtige Entscheidung} \\ 
& & \\ 
\hline
 & & \\
$\nullhyp$ wird &   & falsche Entscheidung  \\
angenommen &  \rb{richtige Entscheidung} & Fehler 2. Art \\
& & \\
\end{tabular}
\end{center}


Hieraus wird auch ersichtlich, warum uns daran gelegen ist, die Nullhypothese abzulehnen.
Die Wahrscheinlichkeit, da"s die Nullhypothese abgelehnt, obwohl sie wahr ist, l"a"st sich
genau angeben. Es handelt sich dabei ja um einen Fehler 1. Art, der mit der
Wahrscheinlichkeit $\alpha$ auftritt. K"onnen wir die Nullhypothese dagegen nicht
ablehnen, so kann nichts dar"uber gesagt werden, mit welcher Wahrscheinlichkeit die
Annahmen der Nullhypothese korrekt ist. Die Wahrscheinlichkeit f"ur Fehler 2. Art h"angt
stark von den Daten selber ab, als auch vom Umfang der Daten. Man kann die G"ute i.allg.
nur f"ur den Test \rem{bestimmter} Daten gegen die Nullhypthese in Abh"angigkeit vom
Datenumfang angeben.


\subsubsection{Einfache Nullhypothesen}
Bei der Konstruktion eines Tests ist zu beachten, da"s zwei verschiedene Typen von
Nullhypothesen existieren: \begriff(einfache) und \begriff(zusammengesetzte). Bei
einfachen Nullhypothesen besteht die Menge der mit $\nullhyp$ konsistenten Prozesse
$\process$ aus nur einem Element. Ein Beispiel f"ur eine solche Nullhypothese w"are die,
da"s die Daten gau"sverteilt mit einem vorher festgelegten Mittelwert $\mu_0$ und
festgelegter Varianz $\sigma_0$ sind. Als Diskriminator $T$ k"onnten wir f"ur diese
Nullhypothese ein h"oheres Moment der Verteilung w"ahlen, sagen wir
\eqnl[teststatistik1]{T=\frac{1}{N}\sum_{i=1}^N x_i^4} 
Prinzipiell h"atten wir jede beliebige
Funktion der $N$ Argumente $X=(x_1,\dots,x_n) $ w"ahlen k"onnen. Allerdings h"angt die
G"ute des Tests stark von der Teststatistik $T$ ab\footnotemark.  \footnotetext{In der Tat
  ist das hier gew"ahlte $T$ nicht die optimale Wahl, da nicht-gau"sf"ormige Verteilungen
  existieren, die das gleiche vierte Moment wie eine Gau"sverteilung besitzen. Dies ist
  f"ur die folgenden Betrachtungen jedoch ohne Belang.}


Wir berechnen nun eine Anzahl $B$ von sogenannten \begriff(Surrogatdaten) oder kurz
\begriff(Surrogaten) $\{X_k, k=1,\dots,B\}$. Zur Erzeugung der Surrogatdaten nehmen wir
einen Gau"sproze"s $N(\mu_0,\sigma_0^2)$, der $N$ unabh"angige, gau"sverteilte Zufallszahlen mit
Mittelwert $\hat\mu\simeq \mu_0$ und empirischer Varianz $\hat\sigma^2\simeq \sigma_0^2$
erzeugt\footnotemark. Wir berechnen nun $T$ f"ur die Surrogatdaten als auch f"ur die
realen Daten. Die $T$-Werte seien mit $\{T_k,k=1,\dots,B\}$ bzw.\ $T_R$ bezeichnet. Wollen
dir die Nullhypothese auf dem $\alpha$-Signifikanzniveau ablehnen, mu"s $T_R$ unter den
$(B+1)/alpha/2$ kleinsten oder den $(B+1)\alpha/2$ gr"o"sten Werten der sortierten Liste
sein, die sowohl die $T_k$ als auch $T_R$ enth"alt\footnotemark. Zu beachten ist, da"s
$B+1$ mindestens gleich $2/\alpha$ sein mu"s. Im allgemeinen wird als ein Vielfaches von
$2/alpha$ gew"ahlt. F"ur ein "ubliches Signifikanzniveau von $\alpha=0,05$ w"are $B=39$,
die minimale Anzahl an verwendeten Surrogaten.  \footnotetext{Ein Gau"sproze"s
  $N(\mu,\sigma^2)$ erzeugt Zufallszahlen mit Erwartungswert $\mu$ und Standardabweichung
  $\sigma$. Bei der Realisierung $X$ eines solchen Prozesses k"onnen der Mittelwert
  $\hat\mu$ und die empirische Standardabweichung$\hat\sigma$ der erzeugten Daten hiervon abweichen.
  Die Werte f"ur die Realisierung eines solchen Prozesses sollen daher durch ein Dach
  "uber der Variablen unterschieden werden. F"ur sehr gro"se $N$ konvergieren $\hat\mu$ und
  $\hat\sigma$ gegen Erwartungswert $\mu$ und Standardabweichung $\sigma$.}  \footnotetext{Bei sehr
  gro"sem $B$ oder bekannter $T$-Verteilung h"atten wir auch den $\alpha$-Konfidenzbereich
  $[T_{\alpha,\tmin},T_{\alpha,\tmax}]$ berechnen k"onnen. Liegt $T_R$ au"serhalb des
  Kondidenzbereichs, kann die Nullhypothese abgelehnt werden.}

\subsubsection{Zusammengesetzte Nullhypothesen}
Dieses Beispiel ist nun -- au"ser zu Demonstrationszwecken -- reichlich uninteressant. Bei
vorliegenden Daten wollen wir die Nullhypothese pr"ufen, ob die Daten allgemein
gau"sverteilt mit unbekanntem Mittelwert $\mu$ und Varianz $\sigma^2$ sind. Dies ist
jedoch eine zusammengesetzte Nullhypothese. Die mit $\nullhyp$ konsistenten Prozesse, sind
alle Gau"sprozesse mit beliebigem $\mu$ und $\sigma^2$. Es w"are nun offensichtlich nicht
sinnvoll und auch nicht praktikabel, die Teststatistik \eqnref{teststatistik1} f"ur alle
m"oglichen Gau"sprozesse $N(\mu,\sigma^2)$ zu berechnen. Es existieren nun zwei verschiedene
Ans"atze, die wir im folgenden diskutieren und vergleichen wollen.

\paragraph{Typische Realisierungen}
Eine M"oglichkeit den Bereich der Modelle einzuengen besteht in der beschr"ankung auf
typische Realisierungen. Man berechnet hierzu $\hat\mu$ und $\hat\sigma$ der Originaldaten
und berechnet dann die Surrogatdaten durch Gau"sprozesse $N(\mu,\sigma^2)$ mit
$\mu=\hat\mu$ und $\sigma=\hat\sigma$. Ein Problem hierbei ist, da"s f"ur die
Surrogatdaten der empirische Mittelwert und die Standardabweichung im allgemeinen ungleich
$\hat\mu$ bzw.\ $\hat\sigma$ sind. Dies hat zur Folge, da"s die Teststatistik relativ
breit streut und die G"ute des Tests sehr schlecht wird (s.\psref{gurke}). Dem l"a"st sich
abhelfen, indem wir statt der Teststatitstik \eqnref{teststatistik1} ein
\begriff(zentrale) Teststatistik verwenden. Ein zentral Teststastitik ist eine, die f"ur
alle Realisierungen der betrachteten Prozesse die gleiche Verteilung aufweist. Dies w"are
in unserem Beispiel 
\eqnl[teststatistik2]{T'=\frac{1}{N}\sum_{i=1}^N \left(\frac{x_i-\hat\mu}{\hat\sigma} \right)^4} 

Die so definierte Teststatistik ist unabh"angig vom Mittelwert und der empirischen
Standardwabweichung der Surrogat- bzw.\ Originaldaten

\begin{itemize}
\item Wir benutzen eine \begriff(zentrale) Teststatistik. Bei einer zentralen
  Teststatistik ist die $T$-Verteilung f"ur alle Elemente von $\process$ gleich.
\item Wir berechnen $\hat\mu_R$ und $\hat\sigma_R$ f"ur die realen Daten und betrachten f"ur die
  Erzeugung der Surrogatdaten nur Gau"sprozesse mit $\mu=\hat\mu_R$ und $\sigma=\hat\sigma_R$,
  d.h.\ wir nehmen nur eine Teilmenge von $\process_0\subset\process$ mit
  $\process_0=\{N(\mu,\sigma),\mu=\hat\mu_R\land\sigma=\hat\sigma_R\}$. Man spricht hier von \begriff(typischen
  Realisierungen).
\item Wir benutzen nur Surrogatdaten f"ur das berechnete $\hat\mu$ und $\hat\sigma$ exakt gleich
  Mittelwert und Varianz der Originaldaten sind. Dieser als \begriff(eingeschr"ankte
  Realisierung) bezeichnete Ansatz, weicht von dem vorherigen leicht, aber doch
  signifikant ab.
\end{itemize}

Um aus der Teststatistik \eqnref{teststatistik1} eine zentrale Teststattistik zu machen definieren wir $T$
 nun folgenderma"sen
\eqnl[teststatistik2]{T'=\frac{1}{N}\sum_{i=1}^N \left( \frac{x_i-\hat\mu}{\hat\sigma} \right)^4} 
Durch die in der Gleichung vorgenommene Umskalierung hat den Effekt, da"s die Teststatistik
f"ur alle Gau"sprozesse $N(\mu,\sigma^2)$ die gleiche Verteilung aufweist. 


\paragraph{Zentrale Teststatistiken}




\paragraph{Eingeschr"ankte Realisierungen}



\begin{itemize}
\item Nullhypothesen: einfache und zusammengesetzte
\item Teststatistik bei zusammengesetzten: pivotal
\item andere M"oglichkeit: typische Realisierung, eingeschr"ankte (gezwungene, gez"ugelt) Realisierungen 
\item Realisierungen und Kofidenzintervalle "uber Monte-Karlo-Methoden
\item Beispiel der Gau"sverteilung: a) 1,0 Test b) pivotal c) typische Realisierung d) eingeschr"ankte Realisierung
\item Beispiel typische Realisierungen: ARMA-prozesse
\item Eingeschr"ankte Realisierungen: FT-Surrogate, AAFT-Surrogate
\end{itemize}


Bei Nullhyptothesen sind zwei Klassen zu unterscheiden: einfache und zusammengesetzte
Nullhypothesen.  Eine einfacher Nullhypothese

Bei zusammengesetzten Nullhypothesen, h"angt diese noch von einem oder mehreren freien
Parametern ab. Ein Beispiel f"ur eine solche, w"are die Nullhypothese, da"s die Daten
gau"sverteilt sind. Die freien Parameter w"aren hier der Mittelwert $\bar x$ und die
Varianz $\sigma_x$. Ein einfache Nullhypothese w"are dagegen, da"s die Daten gau"sverteilt
mit Mittelwert $0$ und Varianz $1$ sind. Ein Test f"ur einfache Nullhypothese ist
offensichtlich weitaus einfacher als f"ur eine zusammengesetzte. Jedoch sind die meisten
Nullhypothesen, die von Interesse sind, zusammengesetzt. Wie man f"ur diese vern"unftige
Tests und Modelle entwirft wird uns im folgenden besch"aftigen.






Eine bessere Methode \cite{prichard-theiler3}


\comment{
\subsection{Statistische Testverfahren}

\subsection{Anwendung in der Zeitreihenanalyse}

\subsection{Modelle}

\subsubsection{ARMA-Modelle}

\subsubsection{Surrogatdaten}

\paragraph{Phasenrandomisierung}

\paragraph{Amplitudenangepa"ste Phasenrandomisierung }

\subsection{Diskriminatoren}

\subsection{Beispiele}

\subsubsection{Wei"ses und farbiges Rauschen}

\subsubsection{Rauschfreie und verrauschte deterministische Systeme}
}









%\chapter{Anwendung}

Zitat siehe \autor(Theiler)
\begin{quote} \em
Es ist nicht schwierig einen Algorithmus zu entwickeln, der Zahlen liefert die als Dimension
bezeichnet werden k"onnen, aber es ist weit schwieriger, sicher zu sein, da"s diese Zahlen
wirklich die Dynamik des System repr"asentieren.
\end{quote}



\section{Physiologie Fr"uhgeborener}

\subsection{Atmungstypen}

\subsubsection{Regelm"a"sige Atmung}

\subsubsection{Periodische Atmung}

\section{Me"sverfahren}

\subsection{Torakale Impedanz (TI)}

\subsection{Volumenstrommessungen (Flow)}

\section{Analyse der Atmungsdaten}

\subsection{Spektren}

\subsubsection{Filterung}

\subsection{Bestimmung der Einbettungsparameter}

\subsection{Rekonstruktion}

\subsubsection{Singular Value Decomposition}

\subsection{Korrelationsdimension}

\subsubsection{Regelm"a"sige Atmung (TI)}

\subsubsection{Periodische Atmung (TI)}

\subsubsection{Regelm"a"sige Atmung (Flow)}

\subsection{Determinismustests}

\epsfigdouble{surrogate/aaft/regel/timeseries}{surrogate/aaft/regel/surronew}
{Links die Originalzeitreihe (Regelm"a"siges Atmen), rechts eine typische, durch AAFT
erzeugte Surrogat-Zeitreihe. Zur besseren Vergleichbarkeit  werden nur die ersten 30
Sekunden adrgestellt.}{surreg1}{-0.2cm}

\epsfigdouble{surrogate/aaft/regel/corrslp}{surrogate/aaft/regel/surcslp}
{Vergleich der Steigungen des Korrelationsintegrals $C(r)$ f"ur Original- (links) und eine
typische Surrogat-Zeitreihe (rechts).}{surreg2}{-0.2cm}

\epsfigdouble{surrogate/aaft/regel/cmpsurcdim}{surrogate/aaft/regel/signi}
{Links: Gemessene Korrelationsdimension f"ur Original- \captimes und Surrogatdaten
\capplus. Der Skalierungsbereich wurde anhand von \psref{surreg2} f"ur beide 
zu $-1.5\leq\ln r\leq\-0.9$ gew"ahlt. Rechts: Die Signifikanz $\Delta\nu/\sigma$ liegt
f"ur Einbettungsdimensionen $d\geq 2$ deutlich "uber $3$.}{surreg3}{-0.2cm} 


\newpage

\comment{
\epsfigdouble{surrogate/ft/regel/timeseries}{surrogate/ft/regel/surronew}
{Links die Originalzeitreihe (Regelm"a"siges Atmen), rechts eine typische, durch FT
erzeugte Surrogat-Zeitreihe. Zur besseren Vergleichbarkeit  werden nur die ersten 30
Sekunden adrgestellt.}{surareg1}{-0.2cm}
}

\epsfigdouble{surrogate/ft/regel/corrslp}{surrogate/ft/regel/surcslp}
{Vergleich der Steigungen des Korrelationsintegrals $C(r)$ f"ur Original- (links) und eine
typische Surrogat-Zeitreihe (rechts).}{surareg2}{-0.2cm}

\epsfigdouble{surrogate/ft/regel/cmpsurcdim}{surrogate/ft/regel/signi}
{Links: Gemessene Korrelationsdimension f"ur Original- \captimes und Surrogatdaten
\capplus. Der Skalierungsbereich wurde anhand von \psref{surareg2} f"ur beide 
zu $-1.5\leq\ln r\leq\-0.9$ gew"ahlt. Rechts: Die Signifikanz $\Delta\nu/\sigma$ liegt
f"ur Einbettungsdimensionen $d\geq 2$ deutlich "uber $3$.}{surareg3}{-0.2cm} 

\epsfigdouble{surrogate/ft/regel/B/cmpsurcdim}{surrogate/ft/regel/B/signi}
{Links: Gemessene Korrelationsdimension f"ur Original- \captimes und Surrogatdaten
\capplus. Der Skalierungsbereich wurde anhand von \psref{surareg2} f"ur beide 
zu $-1.5\leq\ln r\leq\-0.9$ gew"ahlt. Rechts: Die Signifikanz $\Delta\nu/\sigma$ liegt
f"ur Einbettungsdimensionen $d\geq 2$ deutlich "uber $3$.}{suraregb3}{-0.2cm} 













%\addchap{Schlu"s} 

In der vorliegenden Arbeit konnte gezeigt werden, da"s die bei Fr"uhgeborenen
haupt\-s"ach\-lich auftretenden Atemrhythmen (regelm"a"sige und periodische Atmung)  nicht durch stochastische Prozesse modellierbar
sind: Die Fr"uhgeborenenatmung besitzt mit sehr hoher Wahrscheinlichkeit einen
deterministischen Charakter. Trotz dieses zugrunde liegenden Determinismus konnte die
genaue Struktur der Dynamik nicht identifiziert werden. Eine verl"a"sliche Bestimmung der
Korrelationsdimension als Grundlage f"ur eine m"ogliche Beschreibung des Atmungssystems
durch einen seltsamen Attraktor erwies sich als  unm"oglich. Ferner ist die Korrelationdimension
als m"oglicher Indikator f"ur eine Vorhersage von Atemstillst"anden nach den Resultaten dieser Arbeit 
nicht geeignet. Als Gr"unde daf"ur, da"s  die verwendeten Methoden keinen Erfolg hatten, lassen sich mehrere
Punkte anf"uhren:

\begin{itemize}
\item Die Atmung, auf die reine Lungenaktivit"at reduziert, stellt kein autonomes System
  dar. In der Dynamik spielen noch weitere Kontrollparameter eine Rolle, die aus dem TI-Signal
  nicht rekonstruierbar sind. Insofern ist eine der Grundlagen f"ur die Verwendung der
  Verz"ogerungskoordinatenabbildung verletzt, da in diesem Fall selbst beliebig viele verz"ogerte
  Werte nicht den kompletten Systemzustand wiedergeben k"onnen.  Eventuell m"u"ste hier
  der Zustand der atemkompetenten Neuronengruppen in eine Analyse miteinbezogen werden.
  
\item In die gleiche Richtung wie der vorige Punkt geht die Vermutung, da"s
  weitere psychische und physische Einfl"usse existieren, die w"ahrend der Aufnahme der Zeitreihe variieren k"onnen.
  Dies "au"sert sich in zeitlichen "Anderungen der nat"urlichen Dichteverteilungen in den
  Phasenraumrekonstruktionen. Hierdurch sind Verfahren wie Dimensionsalgorithmen nicht
  mehr anwendbar, da sie auf station"aren Dichten beruhen.
  
\item Das gemessene Signal unterliegt einer Reihe weiterer nicht atemkorrelierter
  Einfl"usse, wie beispielweise der Herzt"atigkeit. Ein Signal, in dem solche Einfl"usse
  ausgeschlossen werden k"onnen (z.B. Volumenstrommessungen), mag bessere Ergebnisse
  liefern als die thorakale Impedanz.
  
\item Im Gegensatz zu anderen physiologischen Vorg"angen, wie der Herzt"atigkeit, kann die
  Atmung leicht bewu"st beeinflu"st werden. So k"onnen Vorg"ange, wie Schlucken,
  Laut"au"serungen oder Innehalten, die Aussagekraft des Signals bez"uglich der
  Atmungsdynamik stark beeintr"achtigen.
\end{itemize}

Trotz der negativen Ergebnisse bei dem Versuch einer Systemidentifikation der
Atmungsdynamik ist durch den
Nachweis ihres deterministischen Charakters ein wichtiger Grundstein gelegt. F"ur
weitergehende Analysen sind jedoch besser geeignete Signale und bessere Filtertechniken
vonn"oten (siehe beispielsweise \cite{Rao92}).  Eine weitere interessante M"oglichkeit ist
die Modellierung chaotischer Systeme "uber selbstlernende neuronale Netze
(Back\-pro\-pa\-ga\-tion-Netze), die -- zumindest f"ur einfache Systeme -- sehr gut in der 
Lage sind, die Dynamik aus kleinen, stark verrauschten Datenreihen zu extrahieren
\cite{Albano92}. Dieser gerade im Hinblick auf seine Robustheit 
gegen"uber Rauschen vielversprechende Ansatz k"onnte bei der Analyse der Atmungsdynamik
eine gro"se Hilfe sein.





%%\nocite{}
\nocite{Kantz97}
\nocite{Kaplan95}
\nocite{Schuster88}
\nocite{Lueke92}
\nocite{Ott94a}

%%%%%%%%%%%%%%%%%%%%%%%%%%%%%%%%%%%%%%%%%%%%%%%%%%
%% Anh"ange

\backmatter

\begin{appendix}
\chapter{Medizinische Fachbegriffe} 

Die meisten der hier erl"auterten medizinischen Fachbegriffe stammen aus ``Pschyrembel --
Klinisches W"orterbuch'' \cite{Pschyrembel}.  F"ur Fr"uhgeborene spezifische "Anderungen
oder Erweiterungen der Definitionenen sind aus \autor(Poets) (1993) \cite{Poets93} und
\autor(Hoch) und \autor(Bergmann) (1996) \cite{Hoch96} erg"anzt worden.

\begin{description}
\item[Apnoe:] Atemstillstand.
\item[Atmung, periodische:] Atmung mit Abwechselnd auftretenden mehreren tiefen Atemz"ugen
  und darauffolgender kurzer apnoischer Pause.
\item[Bradykardie:] langsame Schlagfolge des Herzens mit einer Pulsfrequenz unter 60/min,
  bei Fr"uhgeborenen schon ab unter 90-100/min.
\item[Epidemiologie:] Wissenschaftszweig, der sich mit der Verteilung von "ubertragbaren
  und nicht"ubertragbaren Krankheiten und deren physikalischen, chemischen, psychischen
  und sozialen Determinanten und Folgen in der Bev"olkerung befa"st.
\item[Gestationsalter:] Schwangerschaftsdauer, Reifezeichen des Neugeborenen.
\item[Hypox"amie:] niedriger Sauerstoffpartialdruck im arterielle Blut ($\mathrm{pO_2}<70$
  mmHg). Bei Neugeborenen ein Abfall auf unter 40 -- 45 mmHg bzw.\  unter 20\% des Basalwertes. 
\item[idiopathisch:] ohne erkennbare Ursache entstanden, Ursache nicht nachgewiesen.
\item[Konzeptionsalter:] Lebensalter mit Beginn der Empf"angnis.
\item[Neonatologie:] Teilgebiet der Kinderheilkunde, das sich mit Diagnose und Therapie von
  Erkrankungen des Neugeborenen befa"st.
\item[Pathophysiologie:] Lehre von den krankhaften Lebensvorg"angen im menschlichen
  Organismus.
\item[pathologisch:] krankhaft.
\item[Pr"avalenz:] Anzahl der Erkrankungsf"alle einer best.\  Erkrankung bzw.\
  H"aufigkeit eines best.\  Merkmals zu einem best.\  Zeitpunkt oder innerhalb einer
  best.\  Zeitperiode.
\item[Pulsoxymetrie:] transkutane (unblutige) Messung der arteriellen
  Sauerstoffs"attigung.
\item[QRS-Komplex:] Phase der Erregungsausbreitung in den Herzkammern.
\item[REM-Schlaf:] Abk"urzung f"ur engl.\  Rapid Eye Movement. Schlafphase mit raschen
  Augenbewegungen und erh"ohter Herz- und Atemfrequenz.
\item[thorakal:] zum Brustkorb geh"orig.
\item[Thorax:] Brustkorb.
\item[transkutan:] durch die Haut hindurch.
\item[Zyanose:] blau-rote F"arbung von Haut und Schleimh"auten infolge einer Abnahme des
  Sauerstoffgehalts im Blut.
\end{description}


\end{appendix}


%%%%%%%%%%%%%%%%%%%%%%%%%%%%%%%%%%%%%%%%%%%%%%%%%%
%% Literaturverzeichnis

\newpage
\addcontentsline{toc}{chapter}{Literaturverzeichnis}
%\selectlanguage{english}
\bibliography{bib/tsa}
%\bibliographystyle{myalpha}
%\bibliographystyle{plain}
\bibliographystyle{./bst/mysiam}

\end{document}















